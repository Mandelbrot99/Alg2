\newpage
\section{Kommutative Algebra}
\begin{center}
	\textbf{In diesem Kapitel sei $A$ stets ein kommutativer Ring (mit Eins)}
\end{center}
\setcounter{subsection}{10}
\subsection{Grundlagen}
\begin{df}\label{11.1}
	$A$ heißt \define{lokal\index{lokal}} $\defi A$ besitzt genau ein maximales Ideal $\mathfrak{m}$. In diesem Fall heißt $k= \QR{A}{\mathfrak{m}}$ der \define{Restklassenkörper\index{Restklassenkörper}} von $A$.
\end{df}
\begin{bem}\label{11.2}
	Sei $\mathfrak{m}\subseteq A$ ein maximales Ideal. Dann sind äquivalent:
	\begin{enumerate}[label= \roman*)]
		\item $A$ ist lokal mit einem maximalen Ideal $\mathfrak{m}$
		\item $A\backslash \m \subseteq A^*$
		\item $A \backslash \m = A^*$
		\item $1+ \m \subseteq A^*$
	\end{enumerate}
\end{bem}
\begin{proof}
	$i) \Ra ii)$ Sei $x\in A\backslash \m$. Falls $x\notin A^*$, dann existiert nach Algebra 1 ein maximales Ideal $\tilde{\m} \subseteq A$ mit $x\in \m$. Insbesondere ist $\tilde{\m} \neq \m$, d.h. $A$ ist nicht lokal.\\
	$ii) \Ra i)$ Es gelte $A\backslash \m \subseteq A^*$. Sei $\a\subsetneq A$ ein Ideal. Dann ist $\a \cap A^*= \emptyset$, also $\a \cap (A\backslash \m) = \emptyset$. Damit ist $\a \subseteq \m$. Somit ist $\m$ das einzige maximale Ideal in $A$.\\
	$ii) \Ra iii)$ klar: $x\in A^* \Ra x\in A \backslash \m$.\\
	$iii) \Ra iv)$ Sei $x\in 1+ \m$. Falls $x\in \m$, dann ist $1\in \m$, Widerspruch! Also $x\in A \backslash \m \overset{iii)}{=} A^*$.\\
	$iv) \Ra ii)$ Es gelte $1+ \m \subseteq A^*$. Sei $x\in A \backslash m$. Dann ist $Ax+ \m$ ein Ideal mit $Ax+ \m \supsetneq \m$, also $Ax+ \m =(1)$. Damit existiert ein $a\in A, \, y\in \m$ mit $ax+y =1$, also $ax=1-y\in 1+ \m \subseteq A^* \Ra x\in A^*$.
\end{proof}
\begin{df}\label{11.3}
	Sei $x\in A$. $x$ heißt \define{nilpotent\index{nilpotentes Element}} $\defi$ Es existiert ein $n\in \N$ mit $x^n=0$.
\end{df}
\begin{anm}
	Ist $A\neq 0$, dann ist jedes nilpotentes Element ein Nullteiler, die Umkehrung ist im Allgemeinen jedoch falsch.
\end{anm}
\begin{bem+df}
	$$\NN(A) := \{x\in A|\, x\text{ ist nilpotent}\}$$
	ist ein Ideal in $A$, das \define{Nilradikal\index{Nilradikal}} in $A$. Der Ring $\QR{A}{\NN(A)}$ hat keine nilpotenten Elemente $\neq 0$.
\end{bem+df}
\begin{proof}
	\begin{enumerate}
		\item $\NN(A)$ ist ein Ideal:
		\begin{itemize}
			\item $0\in \NN(A)$
			\item Seien $x,y\in \NN(A)$. Dann existiert ein $n\in \N$ mit $x^n=0$ und $y^n=0$, womit $(x+y)^{2n-1} =0$ (aus der binomischen Formel) folgt. Damit ist $x+y \in \NN(A)$.
			\item Sei $x\in \NN(A)\, a\in A$. Dann existiert ein $n\in \N$ mit $x^n=0$, also $a^nx^n = (ax)^n=0$, womit $ax \in \NN(A)$ gilt.
		\end{itemize}
	\item Sei $\bar x \in \QR{A}{\NN(A)}$ nilpotent. Dann existiert ein $n\in \N$ mit $\bar x^n=0$, also $x^n \in \NN(A)$, also existiert ein $m\in \N$ mit $(x^n)^m=0$, also $x^{nm} = 0$, woraus $x\in \NN(A)$, also $\bar x=0$ folgt.
	\end{enumerate}
Damit folgt die Aussage
\end{proof}
\begin{sa}\label{11.5}
	Es gilt
	$$\NN(A) = \bigcap_{\p \subseteq A\atop \text{Primideal}} \p$$
\end{sa}
\begin{proof}
	Wir setzen $\NN'(A)= \bigcap\limits_{\p \subseteq A\atop \text{Primideal}} \p$. Zeige, dass $\NN(A) = \NN'(A)$.\\
	"'$\subseteq$"' Sei $x\in \NN(A)$. Dann existiert ein $n\in \N$ mit $x^n=0$. Für jedes Primideal $\p\subseteq A$ ist dann $x^n\in \p$, also $x\in \p$. Damit ist auch $x\in \NN'(A)$.\\
	"'$\supseteq$"' Angenommen es existiert ein $x\in \NN'(A) \backslash \NN(A)$. Dann gilt $x^n\neq 0$ für alle $n\in \N$. Setze
	$$\Sigma:= \{\a\in A \text{ Ideal}|\, x^n\notin \a \text{ für alle }n\in \N\}$$
	Dann ist $\Sigma$ eine bezüglich Inklusion induktiv geordnete Menge $\neq \emptyset$ (Standardargument mit $\cup$). Nach dem Zornschen Lemma existiert ein maximales Element $\p$ in $\Sigma$. Diese $\p$ ist ein Primideal, denn: Seien $s,t\notin \p$. Dann ist $\p \subsetneq As+ \p$ und $\p\subsetneq At+\p$. Also ist $as+ \p, \, At + \p \notin\Sigma$. Damit existieren $m,n\in \N$ mit $x^n\in As+\p, \, x^m \in At+ \p$. Also liegt $x^{n+m} \in Ast+ \p$, weshalb $Ast + \p \notin\Sigma$. Falls $st \in \p$, dann wäre $Ast + \p = \p \in \Sigma$, Widersprch! Also ist $st \notin \p$, also ist $\p$ ein Primideal. Wegen $\p \in \Sigma$ folgt $x\notin \p$, also $x\notin \NN'(A)$. Widerpsruch!
\end{proof}
\begin{bem} \label{11.6}
	Seien $ \p_1, \dots \p_n $ Primideale in A, $\a \subseteq A $ Ideal mit $\a \subseteq \bigcup_{i = 0}^{n} \p_i$. Dann existiert ein $j \in \{1,\dots,n \} $ mit $ \a \subseteq \p_j$.
\end{bem}
\begin{proof}
	per Induktion nach n: $ \a \not \subseteq \p_i $ für $ i = 1,...,n  \Ra \a \not \subseteq \bigcup_{i = 0}^{n} \p_i $. \\
	n=1: trivial\\
	$n>1$: Sei $\a \not \subseteq \p_i $ für $i = 1,...,n$. Mit der Induktionsvoraussetzung folgt: $\a \not \subseteq \p_1 \cup ... \cup \p_{i-1} \cup \p_{i+1} \cup ... \cup \p_n $ für alle $i=1,...,n  \Ra $ Für alle $ i = 1,...,n $ existiert ein $ x_i \in \a $ mit $ x_i \notin \p_j $ für $ j \neq i $ \\
	1.Fall: Es existiert ein $ i \in \{1,...n\} $ mit $ x_i \notin \p_i.$ Dann $x \notin \bigcup_{j = 0}^{n} \p_j, $ fertig. \\
	2.Fall: $x_i \in \p_i $ für alle $i \in \{1,...n\} $. Setze $ y:= \sum_{j=1}^{n} x_1 \cdot ... \cdot x_{j-1} \cdot x_{j+1}  \cdot ... \cdot x_n$. Dann ist $y \in \a,\, y \notin \p_i $ für  alle $i \in \{1,...n\} $ ( "'Alle Summanden bis auf einen in $\p_i$"). Also $ \a \not \subseteq \bigcup_{i = 0}^{n} \p_i.$
\end{proof}
\begin{bem} \label{11.7}
	Seinen $ \a_1, \dots \a_n \subseteq A$ Ideale, $\p $ ein Primideal in $A$ mit $\p \supseteq \bigcap_{i = 0}^{n} \a_i$.  Dann existiert ein $j \in \{1,...n\} $ mit $\a_j  \subseteq \p$. Ist $\p =\bigcap_{i = 0}^{n} \a_i$, dann existiert ein $j \in \{1,...n\} $ mit $\a_j = \p$.
\end{bem}
\begin{proof}
	Angenommen  für alle $i \in \{1,...n\} $ gilt $ \a_i \not \subseteq \p. \Ra $ Für alle $i \in \{1,...n\} $ existiert ein $x_i \in \a_i, x_i \notin \p$. Dann ist  $x_1 \cdots x_n \notin \p$, da $ \p$ ein Primideal ist. Andererseits ist jedoch $x_1 \cdots x_n \in \bigcap_{i = 0}^{n} \a_i \subseteq \p.$, Widerspruch! Sei nun $ \p = \bigcap_{i = 0}^{n} \a_i $ Dann existiert ein $j \in \{1,...n\} $ mit $ \a_j \subseteq \p.  \\
	\Ra \p =\bigcap_{i = 0}^{n} \a_i \subseteq \a_j \subseteq \p \Ra \p = \a_j.$ 
\end{proof}
\begin{bem+df}
	Seinen $\a, \b \subseteq A $ Ideale, $a \in A $.
	\begin{itemize}
		\item[] $\a : \b := \{x \in A | \ x \b \subseteq \a \} $ heißt \define{Idealquotient\index{Idealquotients}} $\a $ durch $\b$. $\a : \b $ ist ein Ideal in $A$. 
		\item[] $\ann(\a) := (0) : \a = \{x \in A | \ x \a = 0\} $ heißt der \define{Annulator\index{Annulator eines Ideals}} von $\a$.
		\item[] $\ann(a) := \ann((a)) = \{x \in A| \ xa = 0\} $.
	\end{itemize}
\end{bem+df}
\begin{anm}
	\begin{itemize}
		\item $\a \b \subseteq \c \Lra \a \subseteq \c : \b $
		\item Die Menge der Nullteiler von $A$ ist gegeben durch $\bigcup_{ x \in A \backslash\{0\}} ann(x)$
	\end{itemize}
\end{anm}
\begin{bsp}
	$A = \Z, m,n \in \Z $ mit $(m,n) \neq (0) \Ra (m): (n) = \left(\frac{m}{ggt(m,n)}\right)$.
\end{bsp}
\begin{df}
	Sei $\a \subseteq A$ ein Ideal. 
	\begin{itemize}
		\item[] $\sqrt{\a} := \{x \in A | \ \text{Es existiert ein } n \in \N \ \text{mit } x^n \in \a \} $ heißt das \define{Radikal\index{Radikal eines Ideals}} von $\a $. 
	\end{itemize}
\end{df}
\begin{anm} 
	\begin{itemize}
		\item $\sqrt{(0)} = \NN(A)$
		\item  Ist $\pi: A \to \QR{A}{\a} $ die kanonische Projektion, dann ist:
		\begin{eqnarray*}
			\sqrt{\a} &=& \{x \in A | \ \text{Es existiert ein } n \in \N \ \text{mit} \ x^n \in \a \} = \{x \in A| \pi(x) \in \NN\left(\QR{A}{\a}\right)\}\\
		 &=& \pi^{-1}(\NN(\QR{A}{\a}))= \pi^{-1}\left(\bigcap_{\p \subseteq A/\a \atop \text{Primideal} } \p \right) = \bigcap_{\p \subseteq A \text{ PI}\atop \text{mit} \a \subseteq \p} \p
		\end{eqnarray*}
		Insbesondere ist $\sqrt{\a}$ ein Ideal.
	\end{itemize}
\end{anm}
\begin{df}
	Sei $B$ ein kommutativer Ring, $ f: A \to B $ ein Ringhomomorphimsus, $\a \subseteq A, \b \subseteq B $ Ideale. \\
	\begin{itemize}
		\item[] $\a^e := Bf(\a) = \{\sum_{endl.} b_i f(a_i)| b_i \in B, a_i \in \a \} $ heißt die \define{Erweiterung\index{Erweiterung von Idealen}} von $\a$ auf $B$.
		\item[] $\b^c := f^{-1}(\b) $ heißt die \define{Kontraktion\index{Kontraktion von Idealen}} von $\b$ auf A.
	\end{itemize}
\end{df}
\begin{anm}
	\begin{itemize}
		\item $\a^{e}, f^{c} $ sind Ideale in $B$ bzw. in $A$. 
		\item Wir können $f$ faktorisieren in $ A \xrightarrow{p} \im f \xhookrightarrow{\iota} B $. Die Situation für $p$ ist einfach die für $ \iota $ ist kompliziert. 
		\item $\q \in B $ Primideal $\Ra \q^{c} \subseteq A $ Primideal wegen $ \QR{A}{f^{-1}(\q)} \xhookrightarrow{} \underbrace{\QR{B}{\q}}_{\text{nullteilerfrei}}$ \\(beachte $f^{-1}(\q) = \q^c$)
		\item Ist $ \p \subseteq A $ ein Primideal, dann ist $ \p^e \subseteq B$ im Allgemeinen kein Primideal. (Übung: $p$ Primzahl mit $p \equiv 1 \mod  4$. Unter $ f: \Z \to \Z[i] $ ist $(p)^e$ ein Produkt zweier verschiedener Primideale.)
		 % hier auch
	\end{itemize}
\end{anm}
\begin{bem} \label{11.12}
	Sei $B$ ein kommutativer Ring, $f: A \to B $ ein Ringhomomorphismus, $\a \subseteq A$ Ideal, $\b \subseteq B $ Ideal. Dann gilt: 
	\begin{enumerate} [label= \alph*)]
		\item $\a \subseteq \a^{ec} $
		\item $\a^e = \a^{ece}$
		\item $\b^{ce} \subseteq \b $
		\item $\b^{c} = \b^{cec}$ 
	\end{enumerate}
\end{bem}
\begin{proof}
	(a),(c) klar.  \\
	(b) $\a^e \subseteq (\a^{ec})^e , (\a^e)^{ce} \subseteq \a^e$ \\
	(d) analog. 
\end{proof}
\begin{sa} \label{11.13}
	Sei $B$ ein kommutativer Ring, $f: A \to B $ ein Ringhomomorphismus. Setze
	$$C := \{ \a \subseteq A \ \text{Ideal} \ | \ \a \ \text{ist Kontraktion eines Ideals aus }B \} $$
	$$E:= \{\b \subseteq B \ \text{Ideal} \ | \ \b \ \text{ist Erweiterung eines Ideals aus } A\} $$. Dann gilt:
	\begin{enumerate} [label= \alph*)]
		\item $C=\{\a \subseteq A \ \text{Ideal} \ | \ \a^{ec} = \a \} $
		\item $E= \{\b \subseteq B \ \text{Ideal} \ |\ \b^{ce} = b\} $
		\item Die Abbildungen $$\Phi: C \to E, \quad \a \mapsto \a^e, \quad \quad \Psi: E \to C, \quad \b \mapsto \b^c $$ Sind zueinander inverserse, inklusionserhaltende Bijektionen.
	\end{enumerate}
\end{sa}
\begin{proof}
	\begin{enumerate} [label= \alph*)]
		\item "$\supseteq$" klar. "$\subseteq$" $\a \in C \Ra \text{es existiert ein } \b \subseteq B $ Ideal mit $ \a = \b^c \Ra \a^{ec} = \b^{cec} = \b^c \subseteq \a$ (letztes "$=$" per \ref{11.12}(d))
		\item analog 
		\item klar nach (a), (b).
	\end{enumerate}
\end{proof}
\begin{anm}
	Erinnerung an LA1: $T \in M(n \times n, A)$, dann existiert eine komplementäre Matrix $T^{\#} \in M(n \times n, A)$ zu $T$. Es ist $T^{\#}T = TT^{\#} = det(T)E_n.$ (LA1: Satz 17.20)
\end{anm}
\begin{sa}\label{11.14}
	Sei $M$ ein endlich erzeugter $A$-Modul, $\a \subseteq A $ ein Ideal, $\phi \in End_A(M)$ mit $ \phi(M) \subseteq \a M$. Dann existiert ein $ n \in N, a_0,...,a_{n-1} \in \a $ mit: 
	$$ \phi^{n} + a_{n-1}\phi^{n-1}+...+ a_1\phi + a_0id_M  = 0 $$
\end{sa} 
\begin{proof}
	Sei $x_1,...,x_n $ ein Erzeugendensystem von $M$. Dann ist $\phi(x_i) \in \a M = \{ \sum_{endl.} \alpha_i y_i | \alpha_i \in \a, y_i \in M \},$ insbesondere existieren $a_{i1},...,a_{in} \in \a $ mit $ \phi(x_i) = \sum_{j=1}^{n} a_{ij} x_j$ (stelle $y_i$ als Linearkombination von $x_1,...,x_n$ dar). Damit ist $$\sum_{j=1}^{n} ( \delta_{ij}\phi - a_{ij}id_M)(x_j) = 0, \quad \forall i=1,...,n $$ Betrachte $ A[\phi] = \{b_n \phi^n + b_{n-1}\phi^{n-1} +...+b_1\phi + b_0 id_M \ | \ n \in \N_0, b_i \in A \} $, ( was ein kommutativer Unterring von $End_A(M); $ ist mit der Konvention: $\phi^0 = id_M $). 
	Setze nun $$T:= ( \delta_{ij} \phi - a_{ij})_{ij} \in M(n \times n, A[\phi])$$
	$M $ wird via $ (\sum b_i \phi^i)x = \sum b_i \phi^i(x) $ zum $A[\phi]$-Modul. $$T \cdot \left( \begin{array}{c}x_1\\:\\ x_n\end{array} \right) = 0
	\Ra  0 = T^{\#}T \left( \begin{array}{c}x_1\\:\\ x_n\end{array} \right)= \det(T)\left( \begin{array}{c}x_1\\:\\ x_n\end{array} \right)$$
	Da $x_1,...,x_n $ ein Erzeugendensytem von $M$ ist folgt: $ det(T) x = 0 $ für alle $x \in M$, also $det(T) = 0 $. Andererseits gilt aber auch: $$det(T) = det(\delta_{ij} \phi - a_{ij})_{ij}) = \phi^{n} + a_{n-1}\phi^{n-1}+...+ a_1\phi + a_0id_M$$ mit $a_0,...,a_{n-1} \in \a $ nach Leibniz-Formel.
\end{proof}
\begin{fo} \label{11.15}
	Sei $M$ ein endlich erzeugter $A$-Modul, $\a \subseteq A $ ein Ideal, mit $\a M = M$. 
	Dann existiert ein $a \in A $ mit $ a = 1 \mod \a$ mit $ aM = 0. $
\end{fo}
\begin{proof}
	Mit $ \phi = id_M$ ist $\phi(M) = M = \a M $ Dann existieren  $a_0,..,a_{n-1} \in \a$, sodass $0 = id_M^{n} + a_{n-1}id_M^{n-1}+...+ a_1\phi + a_0id_M, $ das heißt: 
	$ 0 = x + a_{n-1}x +...+ a_1x+a_0x = \underbrace{(1+ a_{n-1}+...+a_1 +a_0)}_{:=a} x \Ra a \equiv 1 \mod \a,\, ax =0 $
\end{proof}
\begin{sa}[Nakayama-Lemma] \label{11.16}\index{Nakayama-Lemma}
	Sei $ A $ ein lokaler Ring mit maximalen Ideal $\m$, $M$ ein endlich erzeugter $A$-Modul, $ \QR{M}{\m M } = 0$. Dann ist $ M = 0$. 
\end{sa}
\begin{proof}
	 Aus $ \QR{M}{\m M } = 0$ folgt $M = \m M$. Nach \ref{11.15} folgt, dassein $a \in A, a \equiv 1 (\text{mod} \ \m) $ existiert mit $ aM = 0 $. Wegen  $a \equiv 1 \mod \m$ ist $ a \in A^{*} $. Mit \ref{11.2} folgt $M = 0 $. 
\end{proof}
\begin{fo} \label{11.17}
	Sei $A$ ein lokaler Ring mit maximalen Ideal $ \m$, $M$ ein endlich erzeugter $A$-Modul, $N \subseteq M $ Untermodul mit $ M = \m M + N $. Dann ist $M = N $.
\end{fo}
\begin{proof}
	Es ist $$\m(\QR{M}{N}) = \QR{(\m M + N)}{N} = \QR{M}{N}$$
	Mit dem Nakayama-Lemma folgt unmittelbar: $\QR{M}{N} = 0$, also $ M = N. $
\end{proof}
\begin{fo} \label{11.18}
	Sei $A$ ein lokaler Ring mit maximalen Ideal $\m$, $M$ ein endlcih erzeugter $A$-Modul, $x_1,...,x_n \in M $. Dann sind äquivalent: 
	\begin{enumerate} [label= \roman*)]
		\item $x_1, \dots ,x_n $ ist ein Erzeugendensystem von M 
		\item Die Bilder $ \bar{x_1},\dots ,\bar{x_n} $ von $x_1,\dots ,x_n$ in $\QR{M}{\m M } $ erzeugen den $\QR{A}{\m} $-VR \ $\QR{M}{\m M}$
	\end{enumerate}
\end{fo}
\begin{proof}
	(i)$\Ra $(ii)  klar. \\
	(ii)$\Ra$(i) Setze $N:= \sum_{i = 1}^{n} A x_i.$ Nach Voraussetzung ist $$ \QR{(N + \m M )}{\m M} = \QR{M}{\m M} \Longrightarrow N + \m M  = M$$
	Mit \ref{11.17} folgt: $N=M$.
\end{proof}
\begin{anm}
	Wichtig: $M$ endlich erzeugt ist eine Voraussetzung in \ref{11.18}.
\end{anm}
\newpage
\subsection{Lokalisierung}
\textbf{Erinnerung} (an Algebra 1): Sei $S\subseteq A$ ein Untermonoid bezüglich "'$\cdot$"' (d.h. $1\in S$ und $a,b\in S\Ra ab\in S$). Definiere eine Relation "'$\sim$"' auf $A\times S$ wie folgt:
$$(a_1, s_1) \sim (a_2,s_2) \defi \text{Es existiert ein }t\in S \text{ mit } ta_2s_1 = ta_1s_2$$
"'$\sim$"' ist eine Äquivalenzrelation, setze $S^{-1}A:= \QR{A\times S}{\sim}$, wobei $\frac{a}{s}$ die Äquivalenzklasse von $(a,s)\in A\times S$ ist. Dann ist $S^{-1}A$ ein kommutativer Ring via
\begin{align*}
\frac{a_1}{s_1} + \frac{a_2}{s_2} &:= \frac{a_1s_2+a_2s_1}{s_1s_2},&\frac{a_1}{s_1} \cdot \frac{a_2}{s_2} &:= \frac{a_1a_2}{s_1s_2}
\end{align*}
Ferner gibt es einen Ringhomomorphismus $\tau:A \to S^{-1}A, \; a \mapsto \frac{a}{1}$. Diese ist genau dann injektiv, wenn $S$ nur aus Nichtnullteilern besteht.\\
\begin{center}
	\textbf{Im Folgenden sei $S\subseteq A$ ein Untermonoid bezüglich "'$\cdot$"', Erweiterung und Kontraktion von Idealen sind bezüglich $\tau$ zu verstehen.}
\end{center}

~\

\begin{bem+df}\label{12.1}
	Sei $\a\subseteq A$ ein Ideal
	\begin{enumerate}
		\item[] $S^{-1}\a := a^e=\{\frac{a}{s}\ | \ a\in \a, \, s\in S\}\subseteq S^{-1}A$ ist ein Ideal.
	\end{enumerate}
	Ferner gilt: $S^{-1}\a= S^{-1}A\Lra \a\cap S \neq \emptyset$. 
\end{bem+df}
\begin{proof}
	"'$\Ra$"' $S^{-1}\a = S^{-1}A \Ra 1 = \frac{1}{1} \in S^{-1}\a \Ra $ Es existiert ein $a\in \a, \, s\in S$ mit $\frac{a}{s} = \frac{1}{1}$. Damit exitstiert ein $t\in S$ mit $ta = ts$, also ist $\a \cap S \neq \emptyset$.\\
	"'$\La$"' Sei $\a\cap S\neq \emptyset$. Dann existiert ein $s\in \a \cap S$ mit $\frac{s}{s} = \frac{1}{1}\in S^{-1}A$, also ist $S^{-1}\a = S^{-1}A$
\end{proof}
\begin{bem}\label{12.2}
	Sei $\p \subseteq A$ ein Primideal mit $\p\cap S=\emptyset$. Dann ist $S^{-1}\p$ ein Primideal in $S^{-1} A$.
\end{bem}
\begin{proof}
	Seien $\frac{a_1}{s_1}, \frac{a_2}{s_2}\in S^{-1} A$ mit $\frac{a_1}{s_1} \cdot\frac{a_2}{s_2}\in S^{-1} \p$. Dann existieren $b\in \p, \, s\in S$ mit $\frac{a_1a_2}{s_1s_2} = \frac{b}{s}$. Dann existiert ein $t\in S$ mit $tsa_1a_2 = tbs_1s_2 \in \p$, woebi $t,s\notin \p$. Da $\p$ Primideal, muss also $a_1a_2\in \p$ gelten und damit $a_1\in \p$ oder $a_2\in \p$. Damit folgt schließlich $\frac{a_1}{s_1}\in S^{-1} \p$ oder $\frac{a_2}{s_2}\in S^{-1}\p$.
\end{proof}
\begin{bem}\label{12.3}
	Es gilt:
	\begin{enumerate}[label=\alph*)]
		\item Für die Abbildungen
		$$\begin{tikzcd}[row sep = tiny]
		\{\text{Ideale in } A\} \arrow[yshift = 0.5ex]{r}{\Phi} & \{\text{Ideale in } S^{-1}A\} \arrow[yshift = -0.5ex]{l}{\Psi} \\
		\a \arrow[mapsto]{r}& a^e=S^{-1} \a \\
		\b^c=\tau^{-1}(\b) & b\arrow[mapsto]{l}
		\end{tikzcd}$$
		gilt: $\Phi \circ \Psi = \id_{\{\text{Ideale in } S^{-1}A\}}$, insbesondere ist $\Phi$ surjektiv und $\Psi $ injektiv. Beide Abbildungen sind inklusionserhaltend.
		\item Die Abbildungen
		$$\begin{tikzcd}[row sep = tiny]
		\{\text{Primideale in } A \text{mit } \p\cap S = \emptyset\} \arrow[yshift = 0.5ex]{r}{\Phi} & \{\text{Primideale in } S^{-1}A\} \arrow[yshift = -0.5ex]{l}{\Psi} \\
		\p \arrow[mapsto]{r}& p^e=S^{-1} \p \\
		q^c=\tau^{-1}(q) & q\arrow[mapsto]{l}
		\end{tikzcd}$$
		sind bijektiv und invers zueinander, beide sind inklusionserhaltend.
	\end{enumerate}
\end{bem}
\begin{proof}
	\begin{enumerate}[label= \alph*)]
		\item Es ist zu zeigen, dass $b^{ce} = b$ für alle Ideale $b$ in $S^{-1}A$.\\
		"'$\subseteq$"' gilt nach \ref{11.12}.\\
		"'$\supseteq$"' Sei $\frac{a}{s}\in \b$ mit $a\in A, \, s\in S$. Dann ist $\frac{a}{1} = \frac{s}{1} \cdot \frac{a}{s} \in \b$. Dann ist $a\in c^c$, also $\frac{a}{s} \in \b^{ce}$.
		\item \begin{itemize}
			\item $\Psi$ ist wohldefiniert: Für ein Primideal $\q\in S^{-1}A$ ist $\q^c$ ein Primideal (vgl. Übungen). Es ist $\q^c\cap S=\emptyset$, denn andernfalls würde ein $s\in S$ existieren mit $\frac{s}{1} \in\q $, also $1 = \frac{1}{s} \cdot \frac{s}{1}\in \q$, Widerpsruch!
			\item $\Phi$ ist wohldefiniert nach \ref{12.2}.
		\end{itemize}
	Nach $a)$ ist $\Phi \circ \Psi=\id_{\{\text{Primideale in } S^{-1}A\}}$, also genügt es zu zeigen, dass für alle Primideale $\p \subseteq A$ mit $\p \cap S= \emptyset$ ist $\p^{ec} = \p$.\\
	"'$\supseteq$"' aus \ref{11.12} $a)$\\
	"'$\supseteq$"' Sei $a\in \p^{ec}$. Dann existiert ein $b\in \p, \, s\in S$ mit $\frac{a}{1} = \frac{b}{s}$, also existiert ein $t\in S$ mit $tas = tb$, d.h. $sta=tb$. Da $st\notin \p$ und $\p$ Primideal, folgt $a\in \p$.
	\end{enumerate}
\end{proof}
\begin{bem+df}\label{12.4}
	Sei $\p\subseteq A$ ein Primideal, setze $S:= A\backslash \p$ (ist ein Untermonoid).
	\begin{enumerate}
		\item[] $A_\p:= S^{-1}A$ heißt die \define{Lokalisierung von $A$ bei $\p$\index{Lokalisiserung von $A$ bei $\p$}}.
	\end{enumerate}
	$A_\p$ ist ein lokaler Ring mit dem maximalen Ideal $S^{-1} \p$. Erweiterung und Kontraktion liefern inklusionserhaltende Bijektionen zwischen der Menge der Primideale in $A$, die $\p$ enthalten und der Menge der Primideale in $A_\p$.
\end{bem+df}
\begin{proof}
	folgt aus \ref{12.3} $b)$ (beachte: $\q \cap S= \emptyset \Lra \q \subseteq \p$).
\end{proof}
\begin{bsp}
	Betrachte $A=\Z, \, \p = (p)$ für eine Primzahl $p$. Dann ist $$\Z_{(p)} = \left\{ \frac{m}{n} \in \Q|\, m,n \in \Z, \, \text{ggT}(m,n) = 1, \, p\nmid n\right\}$$
	lokal mit dem maximalen Ideal $$p\Z_{(p)} = \left\{\frac{m}{n} \in \Q| \, m,n\in \Z, \, \text{ggT}(m,n) = 1, \, p |m,\,, p\nmid n\right\}$$
\end{bsp}
\begin{bem+df}\label{12.6}
	Sei $M$ ein $A$-Modul. Wir definieren eine Relation "'$\sim$"' auf $S\times M$ wie folgt:
	$$(s_1,m_1) \sim (s_2, m_2) \defi \text{Es existiert ein } t\in S \text{ mit } ts_2m_1 = ts_1m_2$$
	"'$\sim$"' ist ein Äquivalenzrelation und wir setzen $S^{-1}M := \QR{(S\times M)}{\sim}$; $\frac{m}{s}$ bezeichne die Äquivalenzklasse on $(s,m) \in S\times M$. Dann ist $S^{-1}M$ ein $S^{-1}A$-Modul vermöge
	\begin{align*}
	\frac{m_1}{s_1}+ \frac{m_2}{s_2}&:= \frac{s_2m_1+s_1m_2}{s_1s_2}, & \frac{a}{s} \cdot \frac{m}{t}&:= \frac{am}{st}
	\end{align*}
	für $m_1,m_2,m \in M, \, s_1,s_2,s,t\in S$. $S^{-1}M$ heißt der \define{Quotientenmodul\index{Quotientenmodul}} von $M$ nach $S$. Es gibt eine natürliche Abbildung $\tau:M \to S^{-1} M, \, m \mapsto \frac{m}{1}$.
\end{bem+df}
\begin{proof}
	nachrechnen.
\end{proof}
\begin{anm}
	$S^{-1}M$ ist auch ein $A$-Modul vermöge $a\cdot \frac{m}{s} := \frac{a}{1} \cdot \frac{m}{s}= \frac{am}{s}$. $\tau$ ist dann ein Homomorphismus von $A$-Moduln.
\end{anm}
\begin{sa}
	Seien $M,N$ $A$-Moduln, $\phi:M \to N$ $A$-linear, $\tau_M:M \to S^{-1}M$,\\
	\begin{minipage}[t]{0.6\textwidth}
		 $\tau_N:N \to S^{-1}N$ die kanonischen Abbildungen.
		 Dann gibt es genau eine $S^{-1}A$-lineare Abbildung
		$$S^{-1}\phi:S^{-1}M \longrightarrow S^{-1}N $$
		mit $S^{-1}\phi \circ \tau_M = \tau_N \circ \phi$
	\end{minipage}
	\begin{minipage}[t]{0.4\textwidth} 
		$$\begin{tikzcd}
		M \arrow{r}{\phi} \arrow[swap]{d}{\tau_M} & N \arrow{d}{\tau_N} \\
		S^{-1}M \arrow[dashed, red]{r}{S^{-1}\phi}& S^{-1}N
		\end{tikzcd}$$
	\end{minipage}
	Auf diese Weise wird $S^{-1}\_:A$-Mod$\longrightarrow S^{-1}A$-Mod zu einem additiven Funktor.
\end{sa}
\begin{proof}
	\begin{enumerate}
		\item Existenz: Setze $S^{-1}\phi:S^{-1}M \to S^{-1}N, \; \frac{m}{s} \mapsto \frac{\phi(m)}{s}$.
		\begin{itemize}
			\item $S^{-1}\phi$ ist wohldefiniert: Sei $(s_1,m_1) \sim (s_2, m_2)$. Dann existiert ein $t\in S$ mit $ts_2m_1 = ts_1m_2$, also 
			$ts_2\phi(m_1) = ts_1\phi(m_2)\Ra (s_1, \phi(m_1)) \sim (s_2\phi(m_2)) \Ra \frac{\phi(m_1)}{s_1} = \frac{\phi(m_2)}{s_2}$. 
			\item $S^{-1}\phi$ ist $S^{-1}A$-linear.
			\item Das Diagramm kommutiert, denn 
			$$(S^{-1}\phi)(\tau_M(m)) = (S^{-1}\phi)\left(\frac{m}{1}\right) = \frac{\phi(m)}{1} = \tau_N(\phi(m))$$
		\end{itemize}
		\item Eindeutigkeit: Aus der Kommutativität des Diagramms und der $S^{-1}A$-Linearität vo $S^{-1}\phi$ folgt:
		\begin{eqnarray*}
			(S^{-1}\phi)\left(\frac{m}{s}\right) &=& S^{-1}\phi\left(\frac{1}{s} \cdot \frac{m}{1}\right) = \frac{1}{s} S^{-1}\phi\left(\frac{m}{1}\right) = \frac{1}{s}S^{-1}\phi(\tau_M(m)) = \frac{1}{s} (\tau_N(\phi(m)))\\
			&=& \frac{1}{s} \cdot \frac{\phi(m)}{1} = \frac{\phi(m)}{2}
		\end{eqnarray*}
		\item Die Funktorialität und Additivität von $S^{-1}\_$ ist leicht nachzurechnen.
	\end{enumerate}
\end{proof}
\begin{sa}\label{12.8}
	$S^{-1}\_:A$-Mod $\longrightarrow S^{-1}A$-Mod ist ein exakter Funktor.
\end{sa}
\begin{proof}
	Sei $\begin{tikzcd}
	M' \arrow{r}{f} & M \arrow{r}{f} & M''
	\end{tikzcd}$ eine exakte Folge von $A$-Moduln. Nach den Übungen genügt es zu zeigen, dass 
	$$\begin{tikzcd}
	S^{-1}M' \arrow{r}{S^{-1}f} & S^{-1}M \arrow{r}{S^{-1}g} & S^{-1}M''
\end{tikzcd}$$
exakt ist, d.h. $\im(S^{-1}f) = \ker(S^{-1}g)$.\\
"'$\subseteq$"' Es ist $S^{-1}g\circ S^{-1}f = S^{-1}(\underbrace{g\circ f}_{=0}) = 0$\\
"'$\supseteq$"' Sei $\frac{m}{s} \in \ker(S^{-1}g)$. Dann ist $\frac{g(m)}{s} = 0 = \frac{0}{1}$, also existiert ein $t\in S$ mit $tg(m) = 0$, d.h. $g(tm) = 0 \Ra tm\in \ker g = \im f$, also existiert ein $m'\in M$ mit $f(m') = tm$. Damit folgt schließlich
$$S^{-1}f\left(\frac{f(m')}{st} \right) = \frac{f(m')}{st} = \frac{tm}{st} = \frac{m}{s}$$
also $\frac{m}{s} \in \im \left(S^{-1}f\right)$.
\end{proof}
\begin{fo} \label{12.9}
Sei $M$ ein $A$-Modul, $N\subseteq M$ ein Untermodul. Dann ist $S^{-1}N$ auf natürliche Weise ein Untermodul von $S^{-1}M$ und es gilt: 
$$S^{-1} \left(\QR{M}{N}\right) \cong \QR{S^{-1}M}{S^{-1}N}$$
(wir identifizieren diese Moduln im Folgenden mit einander.)
\end{fo}
\begin{proof}
Wende $S^{-1}\_$ auf die exakte Folge 
$$\begin{tikzcd}
0 \arrow{r} & N \arrow{r} & M \arrow{r} & \QR{M}{N} \arrow{r} & 0
\end{tikzcd}$$
an. Dann olgt die Behauptung aus \ref{12.8}
\end{proof}
\begin{bem}\label{12.10}
Seien $M,N$ $A$-Moduln, $\phi:M \to N$ $A$-linear. Dann gilt:
\begin{enumerate}[label= \alph*)]
	\item $\ker(S^{-1}\phi) = S^{-1}(\ker \phi)$
	\item $\coker(S^{-1}\phi) = S^{-1}(\coker\phi)$
	\item $\im(S^{-1}\phi) = S^{-1}(\im \phi)$
\end{enumerate}
\end{bem}
\begin{proof}
\begin{enumerate}[label= \alph*]
	\item Wende $S^{-1}$ auf die exakte Folge $\begin{tikzcd}
	0 \arrow{r} & \ker \phi \arrow{r} & M \arrow{r}{\phi}& N
	\end{tikzcd}$ an.
	\item Wende $S^{-1}$ auf die exakte Folge $\begin{tikzcd}
	M \arrow{r} & M  \arrow{r} & \coker \phi \arrow{r}& 0
	\end{tikzcd}$ an.
	\item Wende $S^{-1}$ auf die exakte Folge $\begin{tikzcd}
	0 \arrow{r} & \im \phi \arrow{r} & N \arrow{r}& \coker \phi \arrow{r} & 0
	\end{tikzcd}$ an.
\end{enumerate}
\end{proof}
\begin{bem}\label{12.11}
Für die Abbildungen:
$$\begin{tikzcd}[row sep = tiny]
\{A\text{-Untermoduln von } M\} \arrow[yshift = 0.5ex]{r}{\Phi} & \{S^{-1}A\text{-Untermoduln von } S^{-1}M\} \arrow[yshift = -0.5ex]{l}{\Psi} \\
N \arrow[mapsto]{r}& S^{-1}N \\
\tau^{-1}(P) & P\arrow[mapsto]{l} 
\end{tikzcd}$$
gilt $\Phi \circ \Psi = \id_{\{S^{-1}A\text{-Untermoduln von } S^{-1}M\}}$, insbesondere ist $\Phi$ surjektiv und $\Psi$ injektiv. Beide Abbildungen sind inklusionserhaltend.
\end{bem}
\begin{proof}
	nachrechnen.
\end{proof}
\begin{fo} \label{12.12}
	Es gilt: 
	\begin{enumerate} [label= \alph*)]
		\item Ist $M$ ein endlich erzeugter $A$-Modul, dann folgt: $S^{-1}M $ ist ein endlich erzeugter $S^{-1}A$-Modul
		\item Ist $M$ ein  noetherscher $A$-Modul, dann ist auch $S^{-1}M$ ein noetherscher $S^{-1}A$-Modul. 
	\end{enumerate}
\end{fo}
\begin{proof}
	\begin{enumerate}
		\item Sei $x_1,\dots,x_r $ ein Erzeugendensystem von $M$ über $A$, dann ist $ \frac{x_1}{1}, \dots, \frac{x_r}{1} $ ein Erzeugendensystem von $S^{-1}M$ über $S^{-1}A$.
		\item Sei $M$ ein noetherscher $A$-Modul, $L \subseteq S^{-1}M $ ein $S^{-1}A$-Untermodul. Mit \ref{12.11} folgt die Existenz eines Untermoduls $N$ in $M$ mit $ L = S^{-1}N $, da $M$ noethersch ist $N$ endlich erzeugt. Mit (a) folgt $L = S^{-1}N$ ist endlich erzeugt über $S^{-1}A$.
	\end{enumerate}
\end{proof}
\begin{df} \label{12.13}
	Sei $M$ ein $A$-Modul, $\p \subseteq A $ ein Primideal. Wir setzen $S := A \backslash \p$.
	\begin{enumerate}
		\item[] $M_{\p} := S^{-1}M $ heißt die \define{Lokalisierung\index{Lokalisierung}} von $M$ bei $\p$.
	\end{enumerate}  Für einen Homomorphismus $\phi: M \to N $ von $A$-Moduln ist entsprechend $\phi_{\p} = S^{-1}\phi: M_{\p} \to N_{\p} $ definiert. 
\end{df}
\begin{anm}
	Eine Eigenschaft $(E)$ eines $A$-Moduls $M$ nennt man eine \define{lokale Eigenschaft\index{lokale Eigenschaft eines Moduls}}, wenn gilt: $M$ erfüllt $(E) \Lra M_{\p} $ erfüllt $(E)$ für jedes Primideal $ \p \subseteq A $. 
\end{anm}
\begin{sa} \label{12.14}
	Sei $M$ ein $A$-Modul. Dann sind äquivalent: 
	\begin{enumerate} [label= \roman*)]
		\item $M = 0$
		\item $M_{\p} = 0 $ für alle Primideale $\p \subseteq A$ 
		\item $M_{\m} = 0 $ für alle maximalen Ideale $\m \subseteq A $
	\end{enumerate}
\end{sa}
\begin{proof}
	$(i) \Ra (ii) \Ra (iii) $ trivial \\
	$(iii) \Ra (i) $ Sei $M \neq 0.$ Dann existiert ein $x\in M, x\neq 0.$ Es ist $\ann_{A}(x) \subseteq A $ ein Ideal mit $\ann_{A}(x )\neq A.$ Wegen $1 \notin \ann_A(x)$, folgt dass ein maximales Ideal $\m \subseteq A $ mit $ann_A(x)\subseteq \m $. Für alle $s \in A\backslash \m \subseteq A \backslash \ann_A(x) $ ist $sx \neq 0$ ist $ \frac{x}{1} \neq \frac{0}{1} $ in $M_{\m} $. Daraus folgt $M_{\m} \neq 0. $
\end{proof}
\begin{fo} \label{12.15}
	Sei  $\begin{tikzcd}
	 M^{'} \arrow{r}{f} & M \arrow{r}{g} & M^{''} 
	\end{tikzcd}$ eine Sequenz vn $A$-Moduln. Dann sind äquivalent: 
	\begin{enumerate} [label= \roman*)]
		\item $\begin{tikzcd} M^{'} \arrow{r}{f} & M \arrow{r}{g} & M^{''} \end{tikzcd}$ ist exakt 
		\item $\begin{tikzcd} M_{\p}^{'} \arrow{r}{f_{\p}} & M_{\p} \arrow{r}{g_{\p}} & M_{\p}^{''}  \end{tikzcd}$ ist exakt für alle Primideale $\p \subseteq A$. 
		\item $\begin{tikzcd} M_{\m}^{'} \arrow{r}{f_{\m}} & M_{\m} \arrow{r}{g_{\m}} & M_{\m}^{''}  \end{tikzcd}$ ist exakt für alle maximalen Ideale $\m \subseteq A$.
	\end{enumerate}
\end{fo}
\begin{proof}
	$(i) \Ra (ii) $ aus \ref{12.8}, $(ii) \Ra (iii)$ trivial. \\
	$(iii) \Ra (i) $ zu zeigen ist: $ \im f = \ker g $: \\
	"'$\subseteq$"' Für jedes maximale Ideal $\m \subseteq A $ ist $$(\im (g \circ f))_{\m} = \im((g \circ f)_{\m}) = \im (g_{\m} \circ f_{\m}) = 0$$ Mit \ref{12.14} folgt $\im(g \circ f) = 0 \Ra g \circ f = 0$. \\
	"'$\supseteq$"' Setze $ N := \QR{\ker g }{\im f} $ mit \ref{12.9} folgt: $$N_{\m} = \QR{(\ker g)_{\m} }{(\im f)_{\m}} = \QR{\ker g_{\m} }{\im f_{\m}} = 0 $$ für alle maximalen Ideale $\m \subseteq A$. Mit \ref{12.14} folgt $\ker g = \im f $.
\end{proof}
\begin{fo} \label{12.16}
	Seien $M,N$ $A$-Moduln, $f: M \to N $ $A$-Modulhomomorphismen. Dann gilt: 
	\begin{enumerate} [label= \alph*)]
		\item $f$ injektiv $\Lra f_{\m} $ injektiv für alle maximalen Ideale $\m \subseteq A  \Lra f_{\p} $ injektiv für alle Primideale $\p \subseteq A $. 
		\item $f$ surjektiv $ \Lra f_{\m} $ surjektiv für alle maximalen Ideale $\m \subseteq A  \Lra f_{\p} $ surjektiv für alle Primideale $\p \subseteq A $. 
		\item $f = 0  \Lra f_{\m} = 0 $ für alle maximalen Ideale $\m \subseteq A  \Lra f_{\p} = 0 $ für alle Primideale $\p \subseteq A $. 
	\end{enumerate}
\end{fo}
\begin{proof}
	\begin{enumerate} [label= \alph*)]
		\item Wende \ref{12.15} an auf $\begin{tikzcd}	0 \arrow{r} & M \arrow{r}{f} & N
		\end{tikzcd}$ an.
		\item Wende \ref{12.15} an auf $\begin{tikzcd}	M \arrow{r}{f} & N \arrow{r} & 0
		\end{tikzcd}$ an.
		\item Wende \ref{12.14} auf $\im f $ an.
	\end{enumerate}
\end{proof}
\begin{bem} \label{12.17}
	Sei $A$ nullteilerfrei, $K = \Quot(A) $. Die natürliche Abbildung $A \to K $ beziehungsweise $A_{\p} \to K $, $\p$ Primideal in $A$, sind alle injektiv, fasse also $A$ bzw $A_{\p} $ als Unterringe von $K$ auf. Dann gilt: $$ A = \bigcap_{\p \subseteq A\atop \text{Primideal}} A_{\p} = \bigcap_{\m \subseteq A\atop \text{max. Ideal}} A_{\m}$$.
\end{bem}
\begin{proof}[Für Primideale]
	"$\subseteq" $klar. \\
	"$\supseteq"$ Sei $ x \in K $ mit $x \in A_{\p} $ für alle Primideale $ \p \subseteq A $. Setze $$\phi: A \to \QR{K}{A}, \quad a \mapsto ax +A$$ (Homomorphismus von $A$-Moduln) dann ist
	$$\phi_{\p}: A_{\p} \to \left(\QR{K}{A}\right)_{\p} = \QR{k_{\p}}{A_{\p}} , \quad \frac{a}{s} \mapsto \frac{ax}{s} + A_\p$$
	die Nullabbildung für alle Primideale $\p\subseteq A$ wegen $x\in A_\p$. Nach \ref{12.16} ist $\phi=0$, also insbesondere $\phi(1) = 0=x+ A=A$, also ist $x\in A$. Analog für maximale Ideale.
\end{proof}
\newpage
\subsection{ Tensorprodukt und flache Moduln}
\begin{df}\label{13.1}
	Seien $L,M,N$ $A$-Moduln, $\phi: M \times N \to L. $. $ \phi$  heißt \define{A-bilinear\index{bilineare Abbildung}} $\defi$ für jedes $n \in \N $ ist die Abbildung $M \to L \quad m \mapsto \phi(m,n) $ $A$-linear und für jedes $m \in M $ ist die Abbildung $ N \to L \quad n \mapsto \phi(m,n)  $ $A$-linear.
\end{df}
\begin{df} \label{13.2}
	Seien $M,N$ $A$-Moduln, Ein \define{Tensorprodukt\index{Tensorprodukt}} von $M$ und $N$ über $A$ ist ein $A$-Modul $T$ zusammen mit einer $A$-bilinearen Abbildung $\tau: M \times N \to T $ sodass folgende Universelle Eigenschaft erfüllt ist: Für jeden $A$-Modul $L$ und jede $A$-lineare Abbildung $\phi: M \times N \to L $ gibt es genau einen $A$-Modulhomomorphismus $\alpha:  T \to L $, sodass $\alpha \circ \tau = \phi$ gilt:
	$$\begin{tikzcd}
	M \times N  \arrow{rr} \arrow{dr}{\tau} && L \\
	& T \arrow[dashed, red]{ur}{\alpha}&
	\end{tikzcd}$$
\end{df}
\begin{sa} \label{13.3}
	Seien $M,N $ $A$-Moduln. Dann gilt: 
	\begin{enumerate} [label= \alph*)]
		\item Es gibt ein Tensorprodukt von $M,N$ über $A$.
		\item Sind $T,T'$ Tensorprodukte von $M,N$ über $A$ mit $ A$-bilinearen Abbildungen $\tau: M \times N \to T$, $\tau^{'}: M \times N \to T^{'}$, dann existiert genau ein $A$-Modulhomomorphismus $\alpha: T \to T^{'} $ mit $\alpha \circ \tau = \tau^{'}. $ $ \alpha $ ist ein Isomorphismus. Mit anderen Worten: Das Tensorprodukt von $M,N$ ist eindeutig bestimmt bis auf einen eindeutigen Isomorphismus. 
		\item Ist $T$ ein Tensorprodukt von $M,N$ über $A$ mit einer $A$-bilinearen Abbildung $\tau: M \times N \to T $ und setzen wir für $m \in M, n \in N$: 
		$$ m \otimes n  := \tau(m,n)$$
		Dann wird $T$ erzeugt von den Elementen $m \otimes n $, $m \in M, n \in N $ das heißt jedes Element von $T$ ist von der Form $\sum_{i=1}^{r} a_i (m_i \otimes n_i) $ mit $ a_i \in A, m_i \in M , n_i \in N $. 
		Hierbei gilt:
		 \begin{align*}
		 (m +m^{'}) \otimes n &= m \otimes n + m^{'} \otimes n,&
		  m \otimes (n+n^{'}) &= M \otimes n + m \otimes n^{'},\end{align*}
		 $$(am) \otimes n = a ( m \otimes n ) = m \otimes (an) $$
	   für alle $m,m^{'} \in M, n,n^{'} \in N, a\in A $. Notation für Tensorprodukt von $M$ und $N$ über $A$: $M \otimes_A N $
	\end{enumerate}
\end{sa}
\begin{proof} 
	\begin{enumerate} [label= \alph*)]
		\item 1. Es gibt eine kanonische Inklusion:
		$$\begin{tikzcd}[row sep = tiny]
			\iota: M \times N\arrow[hook]{r} & A^{(M \times N)} = \bigoplus_{k\in M \times N} A	\\
			(m,n) \arrow[mapsto]{r} & [m,n] = ([m,n]_k)_{k\in M\times N}	
		\end{tikzcd}$$
		(was kein $A$-Modulhomomorphismus ist) wobei 
		$$[m,n]_k:= \begin{cases}
		1, & \text{falls } k=(m,n) \\0 & \text{sonst}
		\end{cases}$$
		Die Elemente $[m,n] $ bilden offenbar eine $A$-Basis von $A^{M \times N } $. Sei $U$ ein Untermodul von $A^{M \times N }$, der von allen Elementen der folgenden Gestalt erzeugt wird: 
		\begin{align*}
		 &[m + m^{'}, n] - [m,n] - [m^{'}, n], &[m, n + n^{'}] - [m,n] -[m,n^{'}], \\
		 &[am,n] -a[m,n],& [m,an] -a[m,n]
		\end{align*} mit $m,m^{'} \in M, n,n^{'} \in N, a\in A.$ Setze $$T:= \QR{A^{M \times N}}{U}, \quad \tau: M \times N \overset{\iota}{\hookrightarrow} A^{M \times N} \overset{\pi}{\to} T,\quad  (m,n) \mapsto \bar{[m,n]}$$ 
		2. T ist zusammen mit $\tau$ ein Tensorprodukt von $M,N$ über $A$: 
		\begin{itemize}
			\item T ist $A$- bilinear nach Konstruktion von U. ( zum Beispiel: $\tau(m +m^{'},n) = \bar{[m +m^{'},n]} = \bar{[m,n]} + \bar{[m^{'},n]} = \tau(m,n) + \tau(m^{'},n)$)
			\item $T$ erfüllt die Universelle Eigenschaft: Sei $L$ ein $A$-Modul $\phi: M \times N \to L $ $A$-bilinear. Wir erhalten eine $A$-lineare Abbildung $$\psi: A^{M \times N } \to L, \quad \psi([m,n]) := \phi(m,n)$$ Wegen $\phi$ bilinear ist $\psi(U) = 0:$
			\begin{eqnarray*} \psi( [m + m', n] - [m,n] - [m', n]) &=& \psi([m + m', n]) -\psi([m,n]) - \psi([m', n])\\
				& =& \phi(m + m', n) -\phi(m,n) -\phi(m', n) \\
				&=& 0
			 \end{eqnarray*}. Daraus erhalten wir eine $A$-lineare Abbildung $$ \alpha: T = \QR{A^{M \times N}}{U} \to L $$ mit $\alpha([m,n]) = \psi([m,n]) = \phi(m,n)$ also $\alpha(\tau(m,n)) = \phi(m,n)$, das heißt $\alpha \circ \tau = \phi $. $\alpha $ ist durch die Vorgabe $\alpha \circ \tau = \phi $ eindeutig bestimmt, da die Elemente $ \bar{[m,n]} = \tau(m,n) $ den $A$-Modul $T$ erzeugen. 
		\end{itemize}
		\item mit Standardargumenten 
		\item aus der Konstruktion im Beweis von (a)
	\end{enumerate}
\end{proof}
\begin{anm}
	Für $m \in M $ ist stets $m \otimes 0 =0 $ denn  $ m \otimes 0 = m \otimes (0+0) = m \otimes 0 + m \otimes 0 $ analog $ 0 \otimes n = 0 $ für $n \in N$. 
\end{anm}
\begin{bsp}
	\begin{enumerate}
		\item $\QR{\Q}{\Z} \otimes_{\Z} \QR{\Q}{\Z} = 0 $ Denn: Seien $a,b \in \QR{\Q}{\Z}$ so existiert ein $ n \in N $ mit $ na = 0 $, es existiert ein $b^{'} \in \QR{\Q}{\Z} $ mit $nb^{'} = b \Ra a \otimes b = a \otimes (nb^{'}) = (na) \otimes b^{'} = 0 \otimes b^{'}$
		\item $\QR{\Z}{2\Z} \otimes_{\Z} \QR{\Z}{3\Z} = 0 $. Denn: Für $a \in \QR{\Z}{2\Z}, b \in \QR{\Z}{3\Z} $ ist $a \otimes b = (3a) \otimes b = a \otimes (3b) = a \otimes 0 = 0$
	\end{enumerate}
\end{bsp}
\begin{bem}\label{13.5}
	Seien $M,M', N,N'$ $A$-Moduln, $f:M \to M'$, $g:N \to N'$ $A$-Modulhomomorphismen. Dann gibt es genau einen $A$-Modulhomomorphismus 
	$$f\otimes g: M \otimes N \longrightarrow M' \otimes N'$$
	mit $(f \otimes g)(m \otimes n) = f(m) \otimes g(n)$ für alle $m\in M, \, n\in N$.
\end{bem}
\begin{proof}
	Die Abbildung $f\times g:M \times N \to M' \otimes N', \; (m,n) \mapsto f(m) \otimes g(n)$ ist $A$-bilinear. Dann folgt die Behauptung aus der Universellen Eigenschaft des Tensorprodukts.
\end{proof}
\begin{fo}
	Seien $M,N$ $A$-Moduln. Dann sind 
	$$M \otimes_A - : A\text{-Mod} \to A\text{-Mod} \quad \text{und} \quad -\otimes_A N: A\text{-Mod} \to A\text{-Mod}$$
	additive Funktoren. hierbei setzen wir für $N_1, N_2\in A$-Mod, $\phi \in \Hom_A(N_1, N_2)$
	$$(M \otimes_A -)(\phi) : =\id_M\otimes\, \phi : M \otimes N_1 \to M \otimes N_2, \quad m\otimes n \mapsto m \otimes \phi(n)$$
	(analog für $-\otimes_A N$)
\end{fo}\label{13.6}
\begin{proof}
	Offenbar ist $(M \otimes_A -)(\id_N) = \id_M\otimes \id_N = \id_{M \otimes N}$ und für $\phi \in \Hom_A(N_1, N_2)$, $\psi \in \Hom_A(N_2,N_3)$ ist 
	$$(M \otimes_A -)(\psi\circ \phi) = \id_M \otimes(\psi \circ \phi) = (\id_M \otimes \psi)\otimes(\id_M\otimes \phi) = (M \otimes_A-)(\psi) \circ (M \otimes_A -)(\phi)$$
\end{proof}
\begin{bem}\label{13.7}
	Seien $L,M,N$ $A$-Moduln, $(N_i)_{i\in I}$ eine Familie von $A$-Moduln. Dann gibt es natürliche Isomorphismen 
	\begin{enumerate}[label= \alph*)]
		\item $M \otimes_A A \cong A \cong A \otimes_A N$
		\item $N\otimes_A N\cong N \otimes_A M$
		\item $(L\otimes_A M)\otimes_A N \cong L \otimes_A(M \otimes_A N)$
		\item $M \otimes_A \left(\bigoplus_{i\in I} N_i\right) \cong \bigoplus_{i\in I}(M \otimes N_i)$
	\end{enumerate}
\end{bem}
\begin{proof}
	\begin{enumerate}[label= \alph*)]
		\item Die Abbildung $\phi:M \times A \to M, \, (m,a) \mapsto am$ ist $A$-bilinear. Aus der Universellen Eigenschaft erhalten wir einen $A$-Modulhomommorphismus $\alpha:M \otimes_A A \to M$ mit $\alpha(m \otimes a) = am$. Die Abbildung $\beta:M \to M \otimes_A A, \, m \mapsto m \otimes 1$ ist $A$-linear. Es ist 
		\begin{eqnarray*}
		(\beta \circ \alpha)(m \otimes a)& =& \beta(am) = (am) \otimes 1 = m \otimes a\\
		(\alpha \circ \beta)(m) &=& \alpha(m \otimes 1) = 1m = m
		\end{eqnarray*}
	d.h. $\alpha, \beta$ sind invers zueinander, also $M \otimes_AA\cong A$. Für $A$-Modulhomomorphismen $M_1,M_2, \, \phi\in \Hom_A(M_1, M_2)$ kommutiert das Diagramm
	$$\begin{tikzcd}
	M_1 \otimes A \arrow[swap]{d}{\phi\otimes \id_A}\arrow{r}{\alpha_1, \sim}& M_1\arrow{d}{\phi} \\
	M_2 \otimes A \arrow{r}{\alpha_2, \sim} & M_2
	\end{tikzcd}$$
	denn: 
	$$\alpha(\phi\otimes \id_A) (m \otimes a) = \alpha_2(\phi(m) \otimes a) = a\phi(m) = \phi(am) = \phi(\alpha_1(m \otimes a))$$
	\item Die Abbildung $\phi:M \times N \to N \otimes_A M, \, (m,n) \mapsto n \otimes m$ ist $A$-bilinear. Nach der Universellen Eigenschaft erhalten wir einen $A$-Modulhomomorphismus $\alpha:M \otimes_A N \to N \otimes_A M$ mit $\alpha(m \otimes n) = n \otimes m$ für alle $m\in M, \, n\in N$. Analog erhlalten wir einen $A$-Modulhomomorphismus $\beta:M \otimes_A M \to M \otimes_A N$ mit $\beta(n \otimes m) = m \otimes n$ für alle $m\in M, \, n\in N$. Diese sind offenbar invers zueinander. Die Natürlichkeit folgt aus der Kommutativität des Diagramms
	$$\begin{tikzcd}
	M_1 \otimes N_1 \arrow{r}{\alpha_1} \arrow[swap]{d}{\phi \otimes \psi} & N_1 \otimes_A M_1 \arrow{d}{\psi \circ \phi}\\
	M_2 \otimes_A N_2 \arrow{r}{\alpha_2} & N_2  \otimes M_2
	\end{tikzcd}$$
	Rest Übung.
	\end{enumerate}
\end{proof}
\begin{anm}
	Das Tensorprodukt kommutiert im Allgemeinen nicht mit Produkten (Übung).
\end{anm}
\begin{fo}\label{13.8}
	Seien $M,N$ freie $A$-Moduln. Dann ist $M\otimes_A N$ ein freier $A$-Modul.
\end{fo}
\begin{proof}
	Nach Voraussetzung ist $M \cong R^{(I)}  =\bigoplus_{i\in I} A, \, N \cong A^{(J)}$ für geeignete $I,J$. Dann ist 
	$$M \otimes_A N \cong A^{(I)} \otimes_A A^{(J)} = \left( \bigoplus_{i\in I} A\right)\otimes \left(\bigoplus_{j\in J}A\right) = \bigoplus_{I\times J}(A \otimes_A A) \cong \bigoplus_{I\times J}A = A^{(I \times J)}$$
\end{proof}
\begin{bem}\label{13.9}
	Sei $B$ ein kommutativer Ring. $f:A\to B$ ein Ringhomomorphismus, $M$ ein $A$-Modul. Dann wird $B$ ein $A-$Modul vermöge $A\times B \to M, \, (a,b) \mapsto f(a) b$ und $M \otimes_AB$ ist ein $B$-Modul vermöge 
	$$B \times (M \otimes_A B)\to M \otimes_A B\quad(b, \sum_{i=1}^r m_i \otimes b_i)\mapsto \sum_{i=1}^r m_i \otimes b b_i$$
	
\end{bem}
\begin{bem}\label{13.10}
	Sei $B$ ein kommutativer Ring, $M$ ein $A$-Modul, $L$ ein $B$-Modul, $N$ $A$- und $B$-Modul mit $a(b x) = b(ax)$ für alle $a\in A, \, b\in B, \, x\in N$ ("'\define{$N$ ist ein $(A,B)$-Bimodul\index{Bimodul}}"'). Dann ist $M \otimes_AN$ in natürlicher Weise ein $B$-Modul, $M\otimes_BL$ ein $A$-Modul und es ists 
	$$(M \otimes_A N) \otimes_B L \cong M \otimes_A(N \otimes_B L)$$
	ein Isomomorphismus von $A$- und $B$-Moduln.
\end{bem}
\begin{proof}
	Übung.
\end{proof}
\begin{bem}\label{13.11}
	Sei $M$ ein $A$-Modul, $S\subseteq A$ ein Untermonoid bezüglich "'$\cdot$"'. Dann gibt es einen natürlichen Isomorphismus (von $A$- und $S^{-1}A$-Moduln)
	$$S^{-1}A \otimes_AM\cong S^{-1}M$$
\end{bem}
\begin{proof}
	Setze $$\phi:S^{-1}A \times M \to S^{-1} M, \quad  \left(\frac{a}{s}, m\right) \mapsto \frac{am}{s}$$
	$\phi$ ist wohldefniert und $A$-bilinear, d.h. wir erhalten einen $A$-Modulhomomorphismus
	$$\psi: S^{-1}A \otimes_A M \longrightarrow S^{-1} M \quad \text{mit} \quad \psi\left(\frac{a}{s} \otimes m \right) =\frac{am}{s}$$
	für alle $a\in A, \, s\in S,\, m \in M$. Definiere $\delta: S^{-1}M \to S^{-1}A \otimes_AM, \, \frac{m}{s} \mapsto \frac{1}{s} \otimes me $. Dann ist $\delta$ wohldefiniert, denn sei $\frac{m_1}{s_1} =\frac{m_2}{s_2}$. Dann existiert ein $t\in S$ mit $ts_2m_1 = ts_1m_2$, also ist 
	$$\frac{1}{s_1} \otimes m_1 = \frac{1}{ts_2s_1} \otimes ts_2 m_1 = \frac{1}{ts_2s_1} \otimes ts_1 m_2 = \frac{1}{s_2} \otimes m_2$$
	Es ist $\psi \circ \delta = \id_{S^{-1}M}$ und 
	$$(\delta\circ \psi)\left(\frac{a}{s} \otimes m\right)= \delta \left(\frac{am}{s}\right)= \frac{1}{s}\otimes (am) = \frac{a}{s}\otimes m$$
	d.h. $\delta \circ \psi = \id_{S^{-1}A \otimes_A M}$. $\psi$ ist auch ein Homomorphismus von $S^{-1}A$-Moduln. Die Natürlichkeit n $M$ rechnet man leicht nach.	
\end{proof}
\begin{bem}\label{13.12}
	Seiene $M,N$ $A$-Moduln, $S\subseteq A$ ein Untermonoid bezüglich "'$\cdot$"'. Dann gibt es einen natürlichen Isomorphismus von $S^{-1}A$-Moduln 
	$$S^{-1} M \otimes_{S^{-1}M} S^{-1}N \cong S^{-1} (M \otimes_A N)$$
\end{bem}
\begin{proof}
	Übung.
\end{proof}
\begin{bem}\label{13.13}
	Seien $L,M,N$ $A$-Moduln. Dann gilt: 
	$$\Hom_A(M \otimes_A N, L) \cong \Hom_A(M, \Hom_A(N,L))$$
	auf natürliche Weise. Insbesondere ist $-\otimes_AN$ \rotatebox{90}{$\perp$} $\Hom_A(N,-)$. 
\end{bem}
\begin{proof}
	Setze $\text{Bil}_A(M\times N,L):= \{f:M \times N \to L \ | \ f \text{ ist }A\text{-bilinear}\}$. Dann ist $\text{Bil}_A(M \times N,L)$ auf natürliche Weise ein $A$-Modul. Nach Definition des Tensorprodukts ist 
	$$\Hom_A(M \otimes_AN,L) \cong \text{Bil}_A(M \times N,L)$$
	ein Isomorphismus von $A$-Moduln. Definiere 
	\begin{eqnarray*}
		\Phi: \text{Bil}_A(M \times N, L) &\longrightarrow &\Hom_A(M, \Hom_A(N,L))\\
		 \phi &\mapsto& \Phi(\phi): M \to \Hom_A(N,L), \, m \mapsto (N \to L, n \mapsto \phi(m,n))
 	\end{eqnarray*}
 	und 
 	\begin{eqnarray*}
 	\Psi: \Hom_A(M, \Hom_A(N,L)) & \longrightarrow & \text{Bil}_A(M \times N,L) \\
 	\psi &\mapsto& (M \times N \to L, \, (m,n) \mapsto \psi(m)(n) ) 
 	\end{eqnarray*}
 	$\Phi, \Psi$ sind zueinander inverse $A$-Modulhomomorphismen. Also folgt
 	$$\Hom_A(M \otimes_A N,L) \cong \text{Bil}_A(M \times N,L) \cong \Hom_A(M,\Hom_A(N,L))$$
\end{proof}
\begin{fo}\label{13.14}
	Seien $M,N$ $A$-Moduln. Dann sind die Funktoren $M \otimes_A -$ und $-\otimes_A N$ rechtsexakt.
\end{fo}
\begin{proof}
	Wegen $-\otimes_A$ \rotatebox{90}{$\perp$} $\Hom_A(N,-)$ ist $-\otimes_A N$ rechtsexakt nach \ref{5.33}. Nach \ref{13.7} $b)$ ist auch $M \otimes_A -$ rechtsexakt.
\end{proof}
\begin{bsp}
	$-\otimes_A N$ ist im Allgemeinen nicht exakt: Sei $A= \Z$, $N = \QR{\Z}{2 \Z}$. Wir betrachten die Folge 
	$$\begin{tikzcd}
	0 \arrow{r} & \Z \arrow{r}{f} & \Z\arrow{r}{\pi} & \QR{\Z}{2\Z} \arrow{r} & 0
	\end{tikzcd}$$
	wobei $f: \Z \to \Z, \, x\mapsto 2x$ und $\pi$ der kanonischen Projektion.
	Es ist $\Z\otimes_\Z\QR{\Z}{2\Z} \cong \QR{\Z}{2\Z} \neq 0$ und die Abbildung
	$$f\otimes_A \id_N : \Z \otimes_\Z \QR{\Z}{2\Z} \longrightarrow \Z \otimes_\Z \QR{\Z}{2\Z}$$
	ist die Nullabbildung, denn für $x\in \Z, \, y\in \QR{\Z}{2\Z}$ ist 
	$$(f \otimes \id_N)(x \otimes y) = f(x) \otimes y = 2x \otimes y = x \otimes 2y = x \otimes 0 = 0$$
\end{bsp}
\begin{bem}\label{13.16}
	Sei $M$ ein $A$-Modul, $\a\subseteq A$ ein Ideal. Dann gilt:
	$$\QR{A}{\a} \otimes_A M \cong \QR{M}{\a M}$$
\end{bem}
\begin{proof}
	Wir betrachten die exakte Sequenz
	$$\begin{tikzcd}
	0 \arrow{r} & \a \arrow{r}{\iota} & A \arrow{r} & \QR{A}{\a} \arrow{r} & 0 
	\end{tikzcd}$$
	von $A$-Moduln. Anwendung von $-\otimes_AM$ liefert nach \ref{13.14} die exakte Folge 
	$$\begin{tikzcd}
	\a \otimes_A M \arrow{r}{\iota \otimes \id_M} & A\arrow{d}{\psi, \sim} \otimes_AM\arrow{r} & \QR{A }{\a} \otimes_A M \arrow{r} & 0\\
	& M & & 
	\end{tikzcd}$$
	d.h. $\QR{A}{\a} \otimes_AM\cong \coker(\iota\otimes_A \id_M)$. Nach \ref{13.7} $a)$ ist die Abbildung $\psi:A \otimes_A M \to M,\, a \otimes m \mapsto am$ ein $A$-Modulisomorphismus. Es folgt 
	$$\QR{A}{\a} \otimes_A M \cong \coker(\iota\otimes_A \id_M) \cong \coker(\psi\circ (\iota \otimes_A \id_M))$$
	wobei $\psi \circ(\iota\otimes_A \id_M) : \a \otimes_A M \to M, \, \sum a_i \otimes m_i \mapsto \sum a_i m_i$, insbesondere ist $\im(\psi \circ (\iota \otimes_A \id_M)) = \a M$, d.h. $\coker(\psi \circ (\iota \otimes \id_M)) = \QR{M}{\a M}$.
\end{proof}
\begin{df}
	Sei $M$ ein $A$-Modul. Dann heißt $M$  \define{flach} $\defi$ $-\otimes_A M$ ist exakt $\Lra M \otimes_A -$ ist exakt.
\end{df} 
\begin{bem} \label{13.18}
	Sei $P$ ein projektiver $A$-Modul. Dann ist $P$ flach. 
\end{bem}
\begin{proof}
	Nach \ref{6.3} existiert eine Menge $I$, ein $A$-Modul $Q$ mit $ P \oplus Q \cong A^{(I)}$. Nach \ref{13.14} genügt es zu zeigen: $\begin{tikzcd}
	0 \arrow{r} & M' \arrow{r}{f} & M \end{tikzcd}$ exakte Folge von $A$-Moduln $\Ra \begin{tikzcd}
	0 \arrow{r} & M^{'}\otimes_A P \arrow{r}{f \otimes id_P} & M \otimes_A P \end{tikzcd}$ ist exakt. \\
	Sei $\begin{tikzcd} 0 \arrow{r} & M' \arrow{r}{f} & M \end{tikzcd}$ exakt. Wir erhalten kommutatives Diagramm mit exakten Zeilen.
		$$\begin{tikzcd}[column sep = huge]
	0 \arrow{r} & \bigoplus_{i \in I} M^{'} \arrow[swap]{d}{\cong}\arrow{r}{\oplus f}& \bigoplus_{i \in I} M \arrow{d}{\cong} \\
	0 \arrow{r} & M^{'} \otimes_A \bigoplus_{i \in I} A \arrow[swap]{d}{\cong} \arrow{r}{ f \otimes id_{A^{(I)}}} & 	M \otimes_A \bigoplus_{i \in I} A \arrow[swap]{d}{\cong} \\
	0 \arrow{r} & M^{'} \otimes_A P \oplus Q \arrow[swap]{d}{\cong} \arrow{r}{ f \otimes (id_p \oplus id_Q) } & 	M \otimes_A P \oplus Q \arrow[swap]{d}{\cong}\\
	0 \arrow{r} & (M^{'} \otimes_A P) \oplus ( M^{'} \otimes_A Q) \arrow{r}{ (f \otimes id_p) \oplus (f \otimes id_Q) } & 	(M \otimes_A P) \oplus (M \otimes Q) \\
	\end{tikzcd}$$
	Woraus folgt dass $\begin{tikzcd}
	0 \arrow{r} & M^{'}\otimes_A P \arrow{r}{f \otimes id_P} & M \otimes_A P \end{tikzcd}$ exakt ist.
\end{proof}
\begin{bem} \label{13.19}
	Seien $M,N $ flache $A$-Moduln. Dann ist $ M \otimes_A N $ flach.
\end{bem}
\begin{proof}
	Sei $\begin{tikzcd}
	0 \arrow{r} & L' \arrow{r} & L \end{tikzcd}$ eine exakte Sequenz von $A$-Moduln \\
	Da $M$ flach ist folgt: $\begin{tikzcd}	0 \arrow{r} & L' \otimes_A M \arrow{r} & L \otimes_A M \end{tikzcd}$ ist exakt \\
	Da $N$ flach ist, folgt: 
	$\begin{tikzcd}	0 \arrow{r} & (L' \otimes_A M) \otimes_A N\arrow{r} & (L \otimes_A M) \otimes_A N \end{tikzcd}$ ist exakt \\
	Und mit \ref{13.7} schließlich dass
	$\begin{tikzcd}	0 \arrow{r} & L' \otimes_A (M \otimes_A N) \arrow{r} & L \otimes_A( M \otimes_A N) \end{tikzcd}$ exakt ist.
	
\end{proof}
\begin{bem} \label{13.20}
	Sei $(M_i)_{i \in I } $eine Familie von $A$-Moduln. Dann sind äquivalent: 
	\begin{enumerate} [label= \roman*)]
		\item $\bigoplus_{i \in I} M_i $ flach
		\item $M_i $ flach für alle $i \in I$.
	\end{enumerate}
\end{bem}
\begin{proof}
	Sei $\begin{tikzcd} 0 \arrow{r} & N' \arrow{r} & N \end{tikzcd}$. Dann ist \\
	\begin{eqnarray*}
	& &	\begin{tikzcd}
	0 \arrow{r} & N^{'}\otimes_A (\bigoplus_{i \in I} M_i) \arrow{r} & N \otimes_A (\bigoplus_{i \in I} M_i) \end{tikzcd} \text{ exakt} \\
	&\overset{13.7}{\Lra}& \begin{tikzcd}
	0 \arrow{r} & \bigoplus_{i \in I} (N^{'}\otimes_A M_i) \arrow{r} & \bigoplus_{i \in I} (N^{}\otimes_A M_i) \end{tikzcd} \text{ exakt}\\
	&\Lra& \begin{tikzcd} 0 \arrow{r} & N' \otimes_A M_i \arrow{r} & N \otimes_A M_i \end{tikzcd} \text{ exakt für alle }i \in I
\end{eqnarray*}
\end{proof}
\begin{bem} \label{13.21}
	Sei $B$ ein kommutativer Ring, $f: A \to B $ ein Ringhomomorphismus, $M$ ein flacher $A$-Modul. Dann ist $ B \otimes_A M $ ein flacher $B$-Modul. 
\end{bem}
\begin{proof}
	Übungsaufgabe.
\end{proof}
\begin{bem} \label{13.22}
	Sei $S \subseteq A $ ein Untermonoid bezüglich $ "\cdot " $. Dann ist $S^{-1}A $ ein flacher $A$-Modul. 
\end{bem}
\begin{proof}
	Sei $\begin{tikzcd} 0 \arrow{r} & M' \arrow{r} & M \end{tikzcd}$ eine exakte Folge von $A$-Moduln. Wir erhalten kommutatives Diagramm mit exakten Zeilen: 
	$$\begin{tikzcd}
	0 \arrow{r} &S^{-1}M' \arrow[swap]{d}{\sim}\arrow{r} &  S^{-1}M \arrow{d}{\sim} \\
0 \arrow{r} & 	S^{-1}A \otimes_A M^{'} \arrow{r} & S^{-1}A \otimes_A M
	\end{tikzcd}$$
	(Beachte: $S^{-1}-$ ist exakt nach \ref{12.8})
\end{proof}
\begin{anm}
	$A$ nullteilerfrei $\Ra \Quot A $ flacher Modul.
\end{anm}
\begin{bem} \label{13.23}
	Sei $M$ ein $A$-Modul. Dann sind äquivalent: 
	\begin{enumerate} [label= \roman*)]
		\item $M$ ist ein flacher $A$-Modul 
		\item $M_{\p} $ ist ein flacher $A_{\p}$-Modul für alle Primideale $\p \subseteq A$
		\item $M_{\m}$ ist ein flacher $A_{\m}$-Modul für alle maximalen Ideale $\m \subseteq A$. 
		
	\end{enumerate}
\end{bem}
\begin{proof}
	$(i) \Ra(ii) $ folgt wegen $M_{\p} \cong M \otimes_A A_{\p} $ aus \ref{13.21} \\
	$(ii) \Ra(iii) $ trivial.\\
	$(iii) \Ra(i) $ Sei $\begin{tikzcd} 0 \arrow{r} & N' \arrow{r} & N \end{tikzcd}$ eine exakte Folge von $A$-Moduln. Mit \ref{12.15} folgt  $\begin{tikzcd} 0 \arrow{r} & N_{\m}^{'} \arrow{r} & N_{\m} \end{tikzcd}$ ist eine exakte Sequenz von $A_{\m}$-Moduln für alle maximalen Ideale $\m \subseteq A $. mit $(iii)$ folgt  $\begin{tikzcd} 0 \arrow{r} & N_{\m}^{'} \otimes_A M_{\m} \arrow{r} & N_{\m} \otimes_A M_{\m} \end{tikzcd}$
	ist eine exakte Sequenz von $A_{\m}$-Moduln für alle maximalen Ideale $\m \subseteq A $. Das heißt $\begin{tikzcd} 0 \arrow{r} & (N^{'} \otimes_A M)_{\m} \arrow{r} & ( N \otimes_A M)_{\m} \end{tikzcd}$ ist eine exakte Sequenz von $A_{\m}$-Moduln für alle maximalen Ideale $\m \subseteq A $. Mit \ref{12.15} ist dann $\begin{tikzcd} 0 \arrow{r} & N^{'} \otimes_A M \arrow{r} & N \otimes_A M \end{tikzcd}$ eine exakte folge von $A$-Moduln. 
\end{proof}
\begin{anm}
	Erinnerung an LA2 (29.15): Sei $M$ ein $A$-Modul. $$ T(M) := \{ x \in M \ | \ \exists a \in A, a \text{ kein Nullteiler mit } ax = 0 \} $$ ist das \define{Torsionsmodul von $M$\index{Torsionsmodul}}.	$M$ heißt \define{torsionsfrei\index{torsionsfrei}} $\defi T(M) = \{0\}$
\end{anm}
\begin{bem} \label{13.24}
	Sei $A$ nullteilerfrei, $M$ ein flacher $A$-Modul. Dann ist $M$ torsionsfrei. 
\end{bem}
\begin{proof}
	Setze $K := \Quot A$ Dann folgt $\begin{tikzcd} 0 \arrow{r} & A \arrow{r} & K \end{tikzcd}$ ist eine exakte Folge von $A$-Moduln. Aus der Flachheit von $M$ folgt $\begin{tikzcd} 0 \arrow{r} & A  \otimes_A M \arrow{r} & K \otimes_A M \end{tikzcd}$ exakt. $K \otimes_A M $ ist ein $K$-Vektorraum und somit torsionsfrei als $K$-Vektorraum und auch als $A$-Modul. Daraus folgt, dass $M$ torsionsfrei als $A$-Modul ist.
\end{proof}
\begin{bem+df} \label{13.25}
	Sei $M$ ein $A$-Modul. Dann sind äquivalent: 
	\begin{enumerate} [label= \roman*)]
		\item Für jede Sequenz $\begin{tikzcd} N^{'} \arrow{r} & N \arrow{r} & N^{''} \end{tikzcd}$ von $A$-Moduln gilt: \\
		$$ \begin{tikzcd}[column sep = small] N^{'} \arrow{r} & N \arrow{r} & N^{''} \end{tikzcd} \text{exakt} \Lra \begin{tikzcd}[column sep = small] N^{'} \otimes_A M \arrow{r} & N \otimes_A M \arrow{r} & N^{''} \otimes_A M \end{tikzcd} \text{exakt} $$
		\item $M$ ist flach und für alle maximalen Ideale $\m \subseteq A $ ist $\QR{M}{\m M} \neq 0 $.
		\item $M$ ist flach und für alle $A$-Moduln $N$ gilt: $N \otimes_A M = 0 \Ra N = 0 $.
		\item $M$ ist flach und für alle $A$-Modulhomomorphismen $\phi: N_1 \to N_2 $ gilt: 
		$$ \phi \otimes_A id_M : N_1 \otimes_A M \to N_2 \otimes_A M = 0 \Ra \phi = 0 $$ 
		Ist eine dieser äquivalenten Bedingungen erfüllt so heißt $M$ ein \define{treuflacher\index{treuflacher Modul}} $A$-Modul. 
	\end{enumerate}
\end{bem+df}
\begin{proof}
	$(i) \Ra (ii) $	Dass $M$ flach ist, ist trivial. Sei also $\m \subseteq A $ ein maximales Ideal, dann erhält man die exakte Sequenz $\begin{tikzcd} 0 \arrow{r} & \m \arrow{r} & A \arrow{r} & \QR{A}{\m} \arrow{r} & 0 \end{tikzcd}$ von $A$-Moduln. mit (i) und unter Anwendung von $- \otimes_A M $ erhalten wir dass $$\begin{tikzcd} 0 \arrow{r} & \m \otimes_A M \arrow{r} & A \otimes_A M \arrow{r} & \QR{A}{\m} \otimes_A M \arrow{r} & 0 \end{tikzcd}$$ exakt ist. 
	Nach \ref{13.16} gilt außerdem $\QR{A}{\m} \otimes_A M \cong \QR{M}{\m M}$. Angenommen $\QR{M}{\m M} = 0$. Dann wäre $$\begin{tikzcd} 0 \arrow{r} & \m \otimes_A M \arrow{r} & A \otimes_A M \arrow{r} & 0 \end{tikzcd}$$
	exakt und mit (i) wäre dann auch $\begin{tikzcd} 0 \arrow{r} & \m \arrow{r} & A \arrow{r} &  0 \end{tikzcd}$ exakt, Widerspruch!  \\
	$(ii) \Ra (iii) $ Sei $N$ ein $A$-Modul, $N \neq 0$. Dann existiert ein $x \in N,\, x \neq 0$, das heißt $\ann_A(x) \subsetneq A $. Dann existiert ein maximales Ideal $\m \in A $  mit $\ann_A(x) \subseteq \m $. Man erhält einen Epimorphismus von $A$-Moduln 
	$$f: Ax \longrightarrow \QR{A}{\ann_A(x)} \longrightarrow \QR{A}{\m}$$
	woraus $$ f \otimes_A id_M: Ax \otimes_A M \to  \QR{A}{\m} \otimes_A M \cong \QR{M}{\m M } \neq 0$$ folgt. Damit ist dann $Ax \otimes_A M \neq 0 $. Die Sequenz $\begin{tikzcd} 0 \arrow{r} & Ax \arrow{r} & N   \end{tikzcd}$ ist exakt, da wegen $M$ flach auch $\begin{tikzcd} 0 \arrow{r} & Ax \otimes_A M  \arrow{r} & N \otimes_A M \end{tikzcd}$ exakt ist. Somit: $N \otimes_a M \neq 0 $. \\
	$(iii) \Ra (iv) $ Sei $\phi: N_1 \to N_2 $ $A$-linear und $\phi \otimes \id_M: N_1 \otimes_A M \to N_2 \otimes_A M $ sei die Nullabbildung. Setze $U := \im \phi $. So erhalten wie die Faktorisierung $\phi: N_1 \to U \to N_2$. Dann ist $\phi|^{U} \otimes_A id_M $ ist surjektiv, $\iota \otimes_A id_M $ ist injektiv, da $M$ flach ist. Außerdem ist
	$$(\iota \otimes_A id_M) \circ (\phi|^{U} \otimes_A id_M) = \phi \otimes id_M = 0$$
	Aus der Injektivität von $\iota \otimes_A id_M $  folgt $\phi|^{U} \otimes_A id_M = 0 $ und die Surjektivität von $\phi|^{U} \otimes_A id_M = 0 $ impliziert $U \otimes_A M =  0 $ und mit (iii) folgt $U = 0 $, das heißt $\phi = 0.$ \\
	$(iv) \Ra (iii) $ Sei $N$ ein $A$-Modul mit $N \otimes_A M = 0 \Ra id_N \otimes_A id_M: N \otimes_AM \to N \otimes_A M $ ist die Nullabildung. Dann folgt mit (iv) $id_N = 0 \Ra N = 0 $. \\
	$(iii) \Ra (i) $ Sei $\begin{tikzcd} N^{'} \arrow{r} & N \arrow{r} & N^{''} \end{tikzcd}$ eine Sequenz von $A$-Moduln. Zu zeigen ist: 
	$$\begin{tikzcd} N^{'} \arrow{r}{f} & N \arrow{r}{g} & N^{''} \end{tikzcd} \text{ exakt }\Lra \begin{tikzcd} N^{'} \otimes_A M  \arrow{r}{f \otimes id_M} & N \otimes_A M  \arrow{r}{g \otimes id_M} & N^{''} \otimes_A M  \end{tikzcd} \text {exakt}$$
	$"\Ra "$ folgt aus der Flachheit von $M$. \\
	Für $"\Leftarrow "$ ist zu zeigen $ \im f = \ker g $ \\
	$"\subseteq " $ Es ist $(g \circ f) \otimes_A id_M = (g \otimes_A id_M) \circ (f \otimes_A id_M ) = 0$  mit (iv) folgt $g \circ f = 0 $. \\
	$"\supseteq " $ Wir betrachten die exakten Sequenzen \begin{eqnarray*}
		&&\begin{tikzcd} N^{'} \arrow{r}{f|^{\im f}} & \im f \arrow{r} & 0 \end{tikzcd}\\
		&&\begin{tikzcd} 0 \arrow{r}  & \im f \arrow{r}{\iota_1} & \ker g \arrow{r} & \QR{\ker g}{\im f} \arrow{r} & 0 \end{tikzcd} \\
		& &\begin{tikzcd} 0 \arrow{r}  & \ker g \arrow{r}{\iota_2} & N \arrow{r}{g|^{\im g}} & \im g \arrow{r} & 0 \end{tikzcd}
	\end{eqnarray*}
	Es ist \begin{eqnarray*}
		\im ((\iota_a \otimes_A id_M) \circ (\iota_i \otimes_A id_M) \circ (f|^{\im f} \otimes_A id_M )) &=& \im(f \otimes_A id_M) = \ker(g \otimes_A id_M) \\
		&=& \ker(g|^{\im g} \otimes_A  id_M) = \im(\iota_2 \otimes_A id_M)
	\end{eqnarray*}
	wobei die letze Gleichheit aus der Flachheit von $M$ folgt. Wegen $M$ flach, ist $\iota_2 \otimes id_M $ injektiv, woraus $$\im ((\iota_1 \otimes id_M) \circ (f|^{\im f} \otimes id_M)) = \ker g \otimes_A M$$ folgt, also 
	$(\iota_1 \otimes id_M ) \circ (f|^{\im f} \otimes id_M) $ surjektiv.
	Damit ist auch $\iota_1 \otimes id_M $ surjektiv. Da $M$ flach ist, erhalten wir die exakte Sequenz $$ \begin{tikzcd} 0 \arrow{r}  & \im f \otimes_A M \arrow{r}{\iota_1 \otimes id_M } & \ker g \otimes_A M \arrow{r} & (\QR{\ker g}{\im f}) \otimes_A M \arrow{r} & 0 \end{tikzcd}$$
	 Daraus resultiert schließlich
	 $$\left(\QR{\ker g}{\im f}\right) \otimes_A M = 0$$ und mit (ii) folgt, dass $\QR{\ker g}{\im f} = 0$, d.h. $\ker g = \im f $.
\end{proof}
\begin{bsp}
	\begin{enumerate} [label= \alph*)]
		\item $\Q$ ist flacher $\Z$-Modul nach \ref{13.22}, aber kein treuflacher $\Z$-Modul, denn $\Q \otimes_{\Z} \QR{\Z}{2\Z} = 0 $
		\item $A^{(I)}$ ist ein treuflacher $A$-Modul für $I \neq \emptyset $.Denn: \\
		$A^{(I)} $ ist flach, da freier $A$-Modul \\
		Sei $N$ $A$-Modul mit $ N \otimes_A A^{(I)} = 0 \Ra \bigoplus_{i\in I} N = 0 \Ra N = 0$ \\
		Insbesondere ist $A[X] \cong A^{(\N_0)} $ ein treuflacher $A$-Modul. 
	\end{enumerate}
\end{bsp}
\newpage
\subsection{Tor}
\begin{df}\label{14.1}
	Seien $M,N$ $A$-Moduln. Wir setzen $$\text{Tor}_n^A(M,N)\index{Tor} := L_n(M \otimes_A -)(N) \quad \text{für} \quad n\geq 0$$
	Explizit: Wähle eine projektive Auflösung $Q_ \bullet \to N$, dann ist $\Tor_n^A(M,N) = \HH_n(M \otimes_A Q_\bullet)$
\end{df}
\begin{sa}\label{14.2}
	Sei $M$ ein $A$-Modul. Dann ist $(\Tor_n^A(M,-))_{n\geq 0}$ ein universeller $\delta$-Funktor, d.h.
	\begin{itemize}
		\item $\Tor_n^A(M,-): A$-Mod $\to A$-Mod sind additive Funktoren für alle $n\geq 0$
		\item Für jede exakte Folge $\begin{tikzcd}
		0 \arrow{r} & N' \arrow{r} & N \arrow{r} & N'' \arrow{r} & 0
		\end{tikzcd}$ gibt es Verbindungshomomorphismen $\delta:\Tor_{n+1}^A(M,N'') \to \Tor_n^A(M,N)$, sodass die lange Folge 
		$$\begin{tikzcd}[column sep = small]
		 \ldots \arrow{r} & \Tor_{n+1}^A(M,N'') \arrow{r}{\delta} & \Tor_n^A(M,N')\arrow[d, phantom, ""{coordinate, name=Z}] \arrow{r} & \Tor_n^A(M,N) \arrow[dll,
		 "\delta",
		 rounded corners,
		 to path=
		 { -- ([xshift=2ex]\tikztostart.east)
		 	|- (Z) [near end]\tikztonodes
		 	-| ([xshift=-2ex]\tikztotarget.west)
		 	-- (\tikztotarget)}
		 ]\\
	 & \Tor_n^A(M,N'') \arrow{r}{\delta} & \Tor_{n-1}^A(M,N') \arrow{r} & \ldots 
		\end{tikzcd}$$

		exakt ist, $\delta$ funktoriell (vgl. \ref{9.1}).
		\item Für jeden homologischen $\delta$-Funktor $H'=(H_n')_{n\geq 0}: A$-Mod $ \to A$-Mod setzt sich jede natürliche Transformation $f_0 : M \otimes_A - \Ra H_0'$ eindeutig zu einem Homomorphismus von homologichen $\delta$-Funktoren fortsetzt.
	\end{itemize}
\end{sa}
\begin{proof}
	Analog zu Kapitel 2.
\end{proof}
\begin{sa}\label{14.3}
	Seien $M,N$ $A$-Moduln. Dann gibt es einen kanonischen Isomorphimus 
	$$\Tor_n^A(M,N) \cong L_n(- \otimes_A N)(M) \quad \text{für alle} \quad n\geq 0$$
	insbesondere kann $\Tor_n^A(M,N)$ auch über eine projektive Auflösung $P_\bullet\to M$ von $M$ berechnet werden via $\Tor_n^A(M,N) = \HH_n (P_\bullet \otimes_A N)$.
\end{sa}
\begin{proof}
	Analog zum Beweis von \ref{10.2}, verwende hierbei, dass projektive Moduln flach sind.
\end{proof}
\begin{fo}\label{14.4}
	Sei $M$ ein flacher $A$-Modul, $N$ ein $A$-Modul. Dann ist $M$ $(-\otimes_A N)$-azyklisch.
\end{fo}
\begin{proof}
	Es ist $$L_n(- \otimes_A N)(M) \underset{\ref{14.3}}{\cong} \Tor_n^A(M,N) = L_n(M \otimes_A -)(N)$$
	Da $M$ flach ist, ist $M \otimes_A -$ exakt, d.h. $L_n(M \otimes_A -) = 0$ für $n\geq 1$, also $L_n(- \otimes_AN)(M) = 0$ für $n\geq 1$.
\end{proof}
\begin{fo}\label{14.5}
	Seien $M,N$ $A$-Moduln. Dann gilt: 
	$$\Tor_n^A(M,N) \cong \Tor_n^A(N,M) \quad \text{für alle}\quad n\geq 0$$
\end{fo}
\begin{proof}
	Sei $P_\bullet\to M$ eine projektive Auflösung von $M$. Es ist 
	\begin{eqnarray*}
	\Tor_n^A(M,N) &=& L_n(M \otimes_A-)(N) \underset{\ref{14.3}}{\cong} L_n(- \otimes_A N)(M) = \HH_n(P_\bullet \otimes_AN) = \HH_n(N \otimes_AP_\bullet )\\
	&=& L_N(N \otimes_A-)(M) = \Tor_n^A(N,M)
	\end{eqnarray*}
\end{proof}
\begin{bem}\label{14.6}
	Sei $\begin{tikzcd}
	0 \arrow{r} & M' \arrow{r} & M \arrow{r} & M'' \arrow{r} & 0
	\end{tikzcd}$ eine exakte Folge von $A$-Moduln, $M''$ flach, $N$ ein $A$-Modul. Dann ist die Folge
	$$\begin{tikzcd}
	0 \arrow{r} & M' \otimes_AN\arrow{r} & M \otimes_AN\arrow{r} & M'' \otimes_AN\arrow{r} & 0
	\end{tikzcd}$$
	exakt.
\end{bem}
\begin{proof}
	Nach \ref{14.2} erhalten wir eine exakte Folge 
	$$\begin{tikzcd}
	\Tor_n^A(M'',N) \arrow{r} & M' \otimes_A N \arrow{r} & M \otimes_A N \arrow{r} & M'' \otimes_AN\arrow{r} & 0
	\end{tikzcd}$$
	Da $M''$ flach ist, ist $\Tor_1^A(M'', N) = 0$ nach \ref{14.4}.
\end{proof}
\begin{sa}\label{14.7}
	Sei $A$ ein lokaler Ring mit maximalem Ideal $\m$ und Restklassenkörper $\QR{A}{\m}$, $M$ ein endlich erzeugter $A$-Modul. Dann sind äquivalent:
	\begin{enumerate}[label= \roman*)]
		\item $M$ ist frei
		\item $M$ ist projektiv
		\item $M$ ist flach
		\item $\Tor_1^A(\QR{A}{\a}, M) =0$ für jedes Ideal $\a\subseteq A$.
	\end{enumerate}
Ist $A$ noethersch, dann sind $i) -iv)$ äquivalent zu:
\begin{enumerate}
	\item[v)] $\Tor_1^A(k,M) = 0$ 
\end{enumerate}
\end{sa}
\begin{proof}
	$i)\Ra i))$ klar, $ii) \Ra iii)$ aus \ref{13.18}, $iii)\Ra iv)$ aus \ref{14.4}, $iv) \Ra v)$ klar.\\
	$iv)\Ra i)$ (bzw. $v) \Ra i)$)
	\begin{enumerate}
		\item Seien $x_1, \ldots, x_n\in M$, sodass die Restklassen $\bar x_1 , \ldots, \bar x_n$ in $k$ eine $k$-Basis von $\QR{M}{\m M}$ bilden. Das Nakayama-Lemma liefert, dass $x_1, \ldots, x_n\in M$ bereits ein Erzeugendensystem von $M$ sind. Setze $F:= A^n, \, \phi: F\to M, \, e_i \mapsto x_i$. $E:= \ker \phi$, erhalte also ein exakte Sequenz
		\begin{equation}\tag{$\ast$}
		\begin{tikzcd}
		0\arrow{r} & E\arrow{r} & F \arrow{r}{\phi} & M \arrow{r} & 0
		\end{tikzcd}
		\end{equation}
		Anwendung von $-\otimes_A k$ liefert die exakte Folge 
		$$\begin{tikzcd}
		0 \overset{iv)}{=} \Tor_1^A(k,M) \arrow{r} & k \otimes_A E \arrow{r} & k \otimes_A F \arrow{r}{\id_k\otimes \phi} & k \otimes_A M \arrow{r} & 0
		\end{tikzcd}$$
		Ferner gilt:
		\begin{eqnarray*}
		\dim_k(k\otimes_A M) &=& \dim_k\left(\QR{A}{\m} \otimes_AM\right) = \dim_k\left(\QR{M}{\m M}\right) = n = \dim_k(k^n)\\
		& = &\dim_k(k \otimes_A A^n) = \dim_k(k \otimes_A F)
		\end{eqnarray*}
		Also ist $\id_k \otimes \phi$ ein Isomorphismus, d.h. $k\otimes_A E = 0$, also $\QR{A}{\m}\otimes_A E = 0$ und damit $\QR{E}{\m E} =0$. Ist $A$ noethersch, dann ist $E$ endlich erzeugt und nach dem Nakayama-Lemma folgt, dass $E=0$, also $M\cong F$, d.h. $M$ ist frei, woraus $v) \Ra i)$ folgt.
		\item Sei $x= a_1 e_a + \ldots + a_n e_n\in E\subseteq F$. Setze $\a_x:= a_1 A + \ldots + a_n A$ (ist ein endlich erzeugtes Ideal in $A$.) Wir wollen einsehen. das $\a_x=0$ ist. Tensoriere $(\ast)$ mit $\QR{A}{\a_x}$ und erhalte die exakte Folge:
		$$\begin{tikzcd}[column sep = small]
		0 \underset{iv)}{=} \Tor_1^A\left( \QR{A}{\a_x}, M\right)\arrow{r} & \QR{A}{\a_x}\otimes_A E \arrow{d}{\sim} \arrow{r} & \QR{A}{\a_x} \otimes_A F \arrow{r}  \arrow{d}{\sim} & \QR{A}{\a_x} \otimes_M M \arrow{r} & 0\\
		& \QR{E}{\a_x E} \arrow[hook]{r} & \QR{F}{\a_x F} & &
		\end{tikzcd}$$
		Da $x\in E \cap F$ ist $x\in \a_XE= \a_x \m E \subseteq \a_x \m F$. Damit ist $x= \sum a_i' b_i$ mit $a_i' \in \a_x \m,\, b_i \in F$, also ist $x= \sum_i \tilde{a}_i e_i$ mit $\tilde{a}_i \in \a_x \m$. Also lieft $a_i \in \a_x \m $ für alle $i=1,\ldots, $, das heißt $\a_x \subseteq \a_x \m \subseteq \a_x$, woraus schließlich $\a_x = \a_x \m$ folgt. Anwendung des Nakayama-Lemmas liefert also $\a_x=0$, also $x=0$, womit $E = \ker \phi =0$, also $M \cong F$, also ist $M$ frei.
	\end{enumerate}
\end{proof}
\begin{fo}\label{14.8}
	Sei $M$ ein endlich erzeugter $A$-Modul. Dann sind äquivalent:
	\begin{enumerate}[label = \roman*)]
		\item $M$ ist ein flacher $A$-Modul
		\item $M_\p$ ist ein freier $A_\p$-Modul für alle Primideale $\p\subseteq A$.
		\item $M_\m$ ist ein freier $A_\m$-Modul für alle maximalen Ideale $\m\subseteq A$.
	\end{enumerate}
\end{fo}
\begin{proof}
	Flachheit ist eine lokale Eigenschaft nach \ref{13.24}. Damit folgt die Behauptung aus \ref{14.7}.
\end{proof}
\begin{sa}\label{14.9}
	Sei $A$ ein Hauptidealring, $M,N$ seien $A$-Moduln. Dann ist $\Tor_n^A(M,N) = 0$ für $n\geq 2$.
\end{sa}
\begin{proof}
	Es existiert ein freier $A$-Modul $F_0$ und ein Epimorphismus $\epsilon:F_0 \twoheadrightarrow N$. Setze $F_1:= \ker \epsilon \subseteq F_0$. Da $A$ ein Hauptidealring ist, ist $F_1$ frei und 
	$$\begin{tikzcd}
	(0 \arrow{r} & F_1 \arrow{r} & F_0) \arrow{r}{\epsilon} & N
	\end{tikzcd}$$
	eine projektive Auflösung von $N$. Für $\Tor$ folgt damit:
	$$\Tor_n^A(M,N) = \HH_n(M \otimes F_0) = 0\quad \text{für} \quad n\geq 2$$
\end{proof}
\begin{bem}\label{14.10}
	Sei $M$ ein $A$-Modul, $a\in A$ kein Nullteiler. Dann gilt:
	$$\Tor_1^A\left( \QR{A}{(a)} , M \right) \cong \{x\in M \ | \ ax=0\}$$
\end{bem}
\begin{proof}
	Setze $\phi_a:A \to A, \, b \mapsto ab$. Da $a$ kein Nullteiler ist, ist $\phi_a$ injekitv. Erhalte also eine exakte Folge
	$$\begin{tikzcd}
	0 \arrow{r} & A \arrow{r}{\phi_a} & A \arrow{r} & \QR{A}{(a)} \arrow{r} & 0
	\end{tikzcd}$$
	Tensorieren mit $M$ liefert die exakte Folge
	$$\begin{tikzcd}[column sep = small]
	0=\Tor_1^A(A,M) \arrow{r} & \Tor_1^A\left( \QR{A}{(a)}, M\right)\arrow{r} & A \otimes_A M \arrow{d}{\sim} \arrow{r}{\phi_a \otimes \id_M} & \arrow{d}{\sim} A \otimes_A M \arrow{r} & \QR{A}{(a)} \otimes_AM\arrow{r} & 0\\
	& & M \arrow{r}{x\mapsto ax} & M & &
	\end{tikzcd}$$
	Also ist $\Tor_1^A\left(\QR{A}{(a)}, M\right)\cong \ker(\phi_a \otimes \id_M) \cong \{x\in M \ | \ ax=0 \}$.
\end{proof}
\newpage
\subsection{Ganze Ringerweiterungen und Dimension}
\begin{center}
	\textbf{In diesem Abschnitt bedeute "'Ringerweiterung"' stets Erweiterung kommutativer Ringe}
\end{center}
\begin{df}
	Sei $B|A$ eine Ringerweiterung.
	\begin{enumerate}
		\item[] $B|A$ heißt \emph{endlich\index{endliche Ringerweiterung}} $\defi B$ ist endlich erzeugt als $A$-Modul
		\item[] $b\in B$ heißt \emph{ganz\index{ganze Ringerweiterung}} über $A \defi A[b]|A$ ist endlich
		\item[] $B|A$ heißt \emph{ganz} $\defi$ Alle $b\in B$ sind ganz über $A$.
	\end{enumerate}
\end{df}
\begin{anm}
	Seien $B|A, \, C|B$ endliche Ringerweiterungen. Dann ist $C|A$ eine endliche Ringerweiterung, denn ist $(b_i)_{i=1, \ldots, n}$ ein Erzeugendensystem von $B$ als $A$-Modul, $(c_j)_{j=1, \ldots, m}$ ein Erzeugendensystem von $C$ als $B$-Modul. Dann ist $(b_ic_j)_{i=1, \ldots, n\atop j=1,\ldots, m}$ ein Erzeugendensystem von $C$ als $A$-Modul.
\end{anm}
\begin{sa}\label{15.2}
	Sei $B|A$ eine Ringerweiterung, $b\in B$. Dann sind äquivalent:
	\begin{enumerate}[label= \roman*)]
		\item $b$ ist ganz über $A$
		\item ES gibt ein $n\in \N$, $a_{n-1}, \ldots, a_0\in A$ mit 
		\begin{equation}\tag{$\ast$}
		b^n + a_{n-1}b^{n-1} + \ldots + a_1 b + a_0 =0
		\end{equation}
		$(\ast)$ heißt eine \define{Ganzheitsgleichung\index{Ganzheitsgleichung}} für $b$.
		\item Es gibt eine Zwischenring $A\subseteq Z\subseteq B$, sodass $b\in Z$ ist und $Z|A$ endlich ist.
		\item Es gibt einen $A[b]$-Modul $M$ mit $\ann_{A[b]} M =0$, der als $A$-Modul endlich erzeugt ist.
	\end{enumerate}
\end{sa}
\begin{proof}
	$i)\Ra iii)$ Ist $b$ ganz über $A$. Dann ist $A[b]|A$ endlich. Setze $Z:= A[b]$.\\
	$iii)\Ra iv)$ Setze $M:= Z$. Dann ist $M$ in natürlicher Weise ein $A[b] \subseteq Z$-Modul. Es gilt: $x\in \ann_{A[b]}Z \Ra xZ =0$ und wegen $1\in Z$ ist $x=0$, also $\ann_{A[b]} M =0$. Wegen $Z|A$ endlich, ist $M$ endlich erzeugt als $A$-Modul.\\
	$iv)\Ra ii)$ Setze $\phi:M \to M,\, x\mapsto bx$. Dann ist $\phi$ offenbar ein $A$-Modulhomomorphismus und wegen \ref{11.14} mit $\a = A$ folgt die Existenz von $a_0, \ldots, a_{n-1}\in A$ mit 
	$$\phi^n + a_{n-1}\phi^{n-1} + \ldots + a_1 \phi + a_0 \id_M= 0$$
	Für alle $x\in M$ ist dann 
	$$(\underbrace{b^n + a_{n-1} b^{n-1} + \ldots + a_1 b + a_0}_{\in \ann_{A[b]}M=0})x=0$$
	Als ist $b^n + \ldots + a_0 =0$.\\
	$ii)\Ra i)$ Sei $b^n + a_{n-1}b^{n-1} + \ldots + a_0=0$. Dann ist $1, b,\ldots, b^{n-1}$ ein Erzeugendensystem von $A[b]$ als $A$-Modul, denn: Sei $c\in A[b]$, etwa $c= \alpha_0 + \alpha_1 b + \ldots + \alpha_sb^s$ mit $\alpha_0, \ldots, \alpha_s\in A$. Setze $h:= \sum_{i=0}^s \alpha_i T^i\in A[T]$, $f:=\sum_{i=0}^n a_i T^i$ ($a_n:= 1$). Da $f$ normiert ist, existieren $q,r\in A[T]$ mit $h=qf+r$ und $\deg r < \deg f$. Dann ist 
	$$c=h(b) = q(b) \underbrace{f(b)}_{=0} + r(b) = r(b) \in \sum_{i=0}^{n-1} Ab^i$$
	
\end{proof}
\begin{anm}
	Ist $B|A$ eine Körpererweiterung, dann gilt:
	\begin{eqnarray*}
	b\in B \text{ ist ganz über } A & \Lra & b\text{ ist algebraisch über } A\\
	B|A \text{ ist ganz} & \Lra & B|A \text{ ist algebraisch}
	\end{eqnarray*}
\end{anm}
\begin{fo}\label{15.3}
	Es gilt:
	\begin{enumerate}[label= \alph*)]
		\item $B|A$ endlich $\Ra B|A$ ganz
		\item $C|B|A$ Ringerweiterungen, $c\in C$ ganz über $A$. Dann ist $c$ gnaz über $B$.
	\end{enumerate}
\end{fo}
\begin{proof}
	$b)$ Sei $b\in B$. Dann folgt die Behauptung aus \ref{15.2} $iii)\Ra i)$ mit $Z=B$.\\
	$a)$ aus \ref{15.2} $i)\Ra ii)$
\end{proof}
\begin{bsp}
	$\Z[i]|\Z$ ist ganz, denn $\Z[i]|\Z$ ist endlich, da $1,i$ ein Erzeugendensystem von $\Z[i]$ als $\Z$-Modul bilden.
\end{bsp}
\begin{sa}\label{15.5}
	Seien $C|B|A$ Ringerweiterungen. Dann sind äquivalent:
	\begin{enumerate}[label= \roman*)]
		\item $B|A$ ist ganz und $C|B$ ist ganz
		\item $C|A$ ist ganz
	\end{enumerate}
\end{sa}
\begin{proof}
	$i)\Ra ii)$ Sei $c\in C$. Dann existieren $b_0, \ldots, b_{n-1}\in B $ mit \begin{equation}\tag{$\ast$}
	c^n + b_{n-}c^{n-1} + \ldots + b_1 c+ b_0 =0
	\end{equation}
	Dann ist $A[b_0, \ldots, b_{n-1}, c]|A$ endlich (also ist $c$ ganz über $A$ nach \ref{15.2} mit $Z = A[b_0, \ldots, b_{n-1}, c]$), denn:
	\begin{itemize}
		\item $A[b_0]|A$ ist endlich wegen $b_0\in B$ und $B|A$ ist ganz.
		\item $(A[b_0])[b_1]|A[b_0]$ ist endlich, wegen $b_1\in B$, also ist $b_1$ ganz über $A$. Somit ist $b_1$ ganz über $A[b_0]$, usw. Dann ist $(A[b_0, \ldots, b_{n-1}])[c]|A[b_0, \ldots, b_{n-1}]$ endlich, da $c$ gegenüber $A[b_0, \ldots, b_{n-1}]$ ganz ist wegen $(\ast)$.
	\end{itemize}
	$ii)\Ra i)$ Sei $C|A$ ganz. Dann ist offensichtlich $B|A$ ganz und $C|B$ ist ganz nach \ref{15.3} $b)$.
\end{proof}
\begin{bem}\label{15.6}
	Sei $B|A$ eine ganze Ringerweiterung, $\b\subseteq B$ ein Ideal. Dann ist $\QR{B}{\b} |\QR{A}{\b \cap A}$ eine ganze Ringerweiterung. 
\end{bem}
\begin{proof}
	Wir haben eine natürliche Inklusion $\QR{A}{\b \cap A} \hookrightarrow \QR{B}{\b}$. Sei $\bar b\in \QR{B}{\b}$. Da $b$ ganz über $A$ ist, existieren $a_{n-1}, \ldots, a_0\in A$ mit $b^n + \ldots + a_1 b + a_0 =0$, d.h. 
	$$\bar b + \ldots + \bar a_1 \bar b + \bar a_0 =\bar 0$$
	Damit ist $\bar b$ ganz über $\QR{A}{\b \cap A}$.
\end{proof}
\begin{bem}\label{15.7}
	Sei $B|A$ eine ganze Ringerweiterung, $S\subseteq A$ ein Untermonois bezüglich "'$\cdot$"'. Dann ist $S^{-1} B |S^{-1} A$ eine ganze Ringerweiterung.
\end{bem}
\begin{proof}
	Wir haben eine natürliche Inklusion $S^{-1}A \hookrightarrow S^{-1}B$. Sei $b\in B, \, s\in S$. Da $B|A$ ganz ist, existieren $a_{n-1}, \ldots, a_0\in A$ mit $b^n + \ldots + a_0 =0$. Also ist 
	$$\left(\frac{b}{s}\right)^n + \frac{a_{n-1}}{s} \left(\frac{b}{s}\right)^{n-1} + \ldots + \frac{a_1}{s^{n-1}}\left(\frac{b}{s}\right)+ \frac{a_0}{s^n} =0$$
	also ist $\frac{b}{s}$ ganz über $S^{-1}A$.
\end{proof}
\begin{bem+df}\label{15.8}
	Sei $B|A$ eine Ringerweiterung. Dann gilt:
	$$\bar A^B:= \{b\in B \ | \ b\text{ ist ganz über } A\}$$
	ist ein Unterring von $B$ mit $A\subseteq \bar A^B$, der \define{ganze Abschluss\index{ganzer Abschluss einer Ringerweiterung}}. $A$ heißt \define{ganzabgeschlossen in $B$ \index{ganzabgeschlossene Ringerweiterung}} $\defi \bar A^B =A$. $\bar A^B|A$ ist eine ganze Ringerweiterung und $\bar A^B$ ist ganzabgeschlossen in $B$. 
\end{bem+df}
\begin{proof}
	Seien $b_1, b_2\in \bar A^B$. Setze $Z:= A[b_1,b_2]$. Es ist $A[b_1]|A$ endlich, da $b_1$ ganz über $A$ ist. Da $b_2$ ganz über $A$ ist, ist $b_2$auch ganz über $A[b_1]$ nach \ref{15.3}. Dann ist $Z= (A[b_1])[b_2]|A[b_1]$ endlich und $Z|A$ ist endlich. Es sind $b_1-b_2, \, b_1\cdot b_2\in Z$ und nach \ref{15.2} sind $b_1-b_2, \, b_1\cdot b_2$ ganz über $A$. Außerdem sind $0,1\in \bar A^B$. Somit ist $\bar A^B$ ein Unterring von $B$. $\bar A^B|A$ ist ganz nach Konstruktion und $\bar A^B$ ist ganzabgeschlossen in $B$, denn:
	$$\overline{(\bar A^B)}^B = \{b\in B \ | \ b\text{ ist ganz über } \bar A^B\} = \{b\in B \ | \ b\text{ ist ganz über }A \} = \bar A^B$$
\end{proof}
\begin{bem}\label{15.9}
	Sei $B|A$ eine Ringerweiterung, $S\subseteq A$ ein Untermonoid bezüglich "'$\cdot$"'. Dann gilt: $$\overline{S^{-1}A}^{S^{-1}B} = S^{-1}( \bar A^B)$$ Insbesondere gilt: Ist $A$ ganzabgeschlossen in $B$, dann ist $S^{-1}A$ ganzabgeschlossen in $S^{-1}B$.
\end{bem}
\begin{proof}
	"'$\subseteq$"' Die Erweiterung $S^{-1}(\bar A^B)|S^{-1}A$ ist ganz, da $\bar A^B|A$ ganz ist und nach \ref{15.7} ist $S^{-1}(\bar A^B) \subseteq \overline{S^{-1}A}^{S^{-1}B}$.\\
	"'$\supseteq$"' Sei $b\in B$, $s\in S$ mit $\frac{b}{s}$ ganz über $S^{-1}A$. Dann existieren $a_{n-1}, \ldots, a_0\in A, \, s_{n-1}, \ldots, s_0\in S$ mit
	$$\left(\frac{b}{s}\right)^n + \frac{a_{n-1}}{s_{n-1}}\left( \frac{b}{s}\right)^{n-1} + \ldots + \frac{a_1}{s_1} \frac{b}{s} + \frac{a_0}{s_0} =0$$
	Setze $t:= s_0 \cdots s_{n-1}\in S$. Mutlipliziere die Gleichung mit $(st)^n$:
	$$(bt)^n + \frac{a_{n-1}st}{s_{n-1}}(bt)^{n-1}+ \ldots =0$$
	d.h. wir erhalten eine Ganzheitsgleichung für $bt$ über $A$. Dann ist $bt\in \bar A^B$, d.h. $\frac{b}{s} = \frac{bt}{st} \in S^{-1}( \bar A^B)$
\end{proof}
\begin{df}
	Sei $A$ nullteilerfrei. Der \define{ganze Abschluss\index{ganzer Abschluss eines Rings}} von $A$ ist der ganze Abschluss von $A$ in $\Quot(A)$. $A$ heißt \define{normal (ganzabgeschlossen)\index{normaler Ring}} $\defi A$ stimmt mit seinem ganzen Abschluss überein.
\end{df}
\begin{bem}\label{15.11}
	Sei $A$ faktoriell. Dann ist $A$ normal.
\end{bem}
\begin{proof}
	Sei $b\in \Quot(A)$ ganz über $A$. Es ist zu zeigen, dass $b\in A$ liegt. Es existieren $a_{n-1}, \ldots, a_0\in A$ mit 
	$$b^n + a_{n-1} b^{n-1} + \ldots + a_1 b + a_0 =0$$
	Sei $b=\frac{p}{q}$ mit $p,q\in A$ teilerfremd. Dann ist
	$$\frac{p^n}{q^n} + a_{n-1} \frac{p^{n-1}}{q^{n-1}} + \ldots + a_1 \frac{p}{q} + a_0=0$$
	Multiplikation mit $q^n$ liefert
	$$p^n = a_{n-1}qp^{n-1} - \ldots - a_1 q^{n-1} p - a_0 q^n$$
	Angenommen $q\notin A^*$. Da $A$ faktoriell, existieren Primteiler $\pi$ von $a$, sodass $\pi |p^n$, also $\pi|p$ Widerspruch zu $p,q$ teilerfremd! Alst liegt $q\in A^*$, d.h. $b\in A$.
\end{proof}
\begin{bsp}
	Es ist $\bar \Z^{\Q(i)} = \Z[i]$, denn: $\Z[i]|\Z$ ist ganz, $\Z[i]$ ist normal, da faktoriell.
\end{bsp}
\begin{bem}\label{15.13}
	Sei $A$ normal, $S\subseteq A$ ein Untermonoid bezüglich "'$\cdot$"'. Dann ist $S^{-1}A$ normal.
\end{bem}
\begin{proof}
	Sei $A$ ganzabgeschlossen in $\Quot(A)$. Nach \ref{15.9} ist $S^{-1}A$ ganzabgeschlossen in $S^{-1}\Quot(A)= \Quot(A) = \Quot(S^{-1}A)$.
\end{proof}
\begin{bem}\label{15.14}
	Sei $A$ nullteilerfrei. Dann sind äquivalent:
	\begin{enumerate}[label= \roman*)]
		\item $A$ ist normal
		\item $A_\p$ ist normal für alle Primideale $\p\subseteq A$.
		\item $A_\m$ ist normal für alle maximalen Ideal $\m\subseteq A$.
	\end{enumerate}
\end{bem}
\begin{proof}
	$i)\Ra ii)$ aus \ref{15.13}\\
	$ii)\Ra iii)$ klar\\
	$iii)\Ra i)$ Sei $x\in \Quot(A)$ ganz über $A$. Dann ist $x$ ganz über $A_\m$ für alle maximalen Ideale $\m$ in $A$. Nach $iii)$ liegt 
	$$x\in \bigcap_{\m\subseteq A \atop \text{max. Ideal}} A_\m \underset{\ref{12.17}}{=} A$$
\end{proof}
\begin{bem}\label{15.15}
	Sei $B|A$ eine ganze Ringerweiterung, $B$ nullteilerfrei. Dann sind äquivalent:
	\begin{enumerate}[label= \roman*)]
		\item $A$ ist ein Körper
		\item $B$ ist ein Körper
	\end{enumerate}
\end{bem}
\begin{proof}
	$i)\Ra ii)$ Sei $0\neq b\in B$. Dann ist. da $B|A$ ganz, $b\in \Quot(B)$ algebraisch über $A$. Dann ist $A[b]$ ein Körper. Also liegt $b^{-1}\in A[b]\subseteq B$.\\
	$ii)\Ra i)$ Sei $0\neq a\in A$. Dann ist $a^{-1}\in B$, da $B$ ein Körper ist. Da $B|A$ ganz, existieren $c_{n-1}, \ldots, c_0\in A$ mit $(a^{-1})^n + c_{n-1} (a^{-1})^{n-1} + \ldots + c_0 =0$ und Multiplikation mit $a^{n-1}$ liefert
	$$a^{-1} = - c_{n-1} - \ldots - c_0 a^{n-1}\in A$$
\end{proof}
\begin{df}
	Sei $B|A$ eine Ringerweiterung, $\p\subseteq B$, $\p'\subseteq A$ Primideale.
	\begin{enumerate}
		\item[] $\p$ \define{liegt über\index{Primideal liegt über Primideal}} $\p' \defi \p' = \p \cap A$ 
	\end{enumerate}
\end{df}
\begin{sa}\label{15.17}
	Sei $B|A$ eine ganze Ringerweiterung. Dann gilt:
	\begin{enumerate}[label= \alph*)]
		\item ("'\define{Lying over\index{Lying Over}}"') Zu jedem Primideal $\p'\subseteq A$ existiert ein Primideal $\p\subseteq B$, so dass $\p$ über $\p'$ liegt. 
		\item Sind $\p \subseteq \q$ Primideale in $B$, die über demselben Primideale $\p'\subseteq A$ liegen, dann ist $\p=\q$.
		\item Liegt das Primideal $\p$ von $B$ über dem Primideal $\p'$ von $A$, dann gilt:
		$$\p \text{ maximales Ideal} \Lra \p' \text{ maximales Ideal}$$
	\end{enumerate}
\end{sa}
\begin{proof}
	$c)$ Nach \ref{15.6} ist $\QR{B}{\p}| \QR{A}{\p'}$ ganz und nach \ref{15.15} ist $\p$ maximal $\Lra$ $\p'$ maximal.\\
	$b)$ Seien $\p\subseteq \q$ Primideale in $B$, die über dem Primideale $\p'$ von $A$ liegen. Wir setzen $S:= A\backslash \p'$ und betrachten die nach \ref{15.7} ganze Ringerweiterung $S^{-1}B |S^{-1} A = A_{\p'}$. $S^{-1}\p'$ ist das eindeutig bestimmte maximale Ideal im lokalen Ring $A_{\p'}$. $S^{-1}\p, S^{-1}\q$ sind wegen 
	$\p\cap S\subseteq \q\cap S= \q \cap ( A \backslash \p') \underset{\q\cap A = \p'}{=} \emptyset$ Primideale in $S^{-1}B$. Es ist
	$$S^{-1}\q\cap A_{\p'}=S^{-1}\q \cap S^{-1}A = S^{-1}(\q \cap A) = S^{-1}\p'= S^{-1}\p \cap A_{\p'}$$
	Nach $c)$ sind dan $S^{-1}\q \supseteq S^{-1}\p$ maximale Ideale in $S^{-1}B$, also folgt $S^{-1} \q = S^{-1}\p$, also nach \ref{12.3} $b)$ $\q = \p$.\\
	$a)$ Sei $\p' \subseteq A$ ein Primideal. Wir setzen $S:= A \backslash \p'$. Dann ist nach \ref{15.7} $S^{-1}B|S^{-1}A= A_{\p'}$ ganz. Sei $\m\subseteq S^{-1}B$ ein maximales Ideal. Dann st $\m \cap A_{\p'}$ ein maximales Ideal in $A_{\p'}$, also ist $\m \cap A_{\p'} = S^{-1}\p'$. Wir betrachten das kommutative Diagramm
	$$\begin{tikzcd}
	A\arrow[hook]{r} \arrow[swap]{d}{\tau_S^A} & B\arrow{d}{\tau_S^B}\supseteq \p\\
	S^{-1}A \arrow[hook]{r} & S^{-1}B \supseteq \m
	\end{tikzcd}$$
	Setze $\p:= (\tau_S^B)^{-1}(\m)$. Dann ist $\p$ ein Primideal in $B$ und $$\p \cap A=(\tau_S^B)^{-1}(\m) \cap A = (\tau_S^A)^{-1}( \m \cap A_{\p'})= (\tau_S^A)^{-1}(S^{-1} \p') = \p'$$
\end{proof}
\begin{fo}\label{15.18}
	Sei $B|A$ eine ganze Ringerweiterung, $\p_0\subsetneq \p_1 \subsetneq \ldots \subsetneq \p_r$ eine Primideal-Kette in $B$. Dann ist 
	$$\p_0\cap A \subsetneq \p_1 \cap A \subsetneq \ldots \subsetneq \p_r \cap A$$
	eine Primideal-Kette in $A$.
\end{fo}
\begin{proof}
	aus \ref{15.17} $b)$
\end{proof}
\begin{fo}[Going-Up\index{Going-Up}]
	Sei $B|A$ eine ganze Ringerweiterung, $\p_0'\subsetneq \p_1'\subsetneq \ldots \subsetneq \p_r'$ eine Primideal-Kette in $A$, $\p_0$ ein Primideal in $B$ mit $\p_0\cap A = \p_0'$. Dann existiert eine Primideal-Kette $$\p_0\subsetneq \p_1 \subsetneq \ldots \subsetneq \p_r \text{ in }B \quad\text{mit} \quad \p_i \cap A = \p_i'$$ für $i=1, \ldots, r$.
\end{fo}
\begin{proof}
	Offenbar genügt es den Fall $r=1$ zu betrachten, Rest induktiv. Es ist $\QR{B}{\p_0}| \QR{A}{\p_0'}$ eine ganze Ringerweiterung nach \ref{15.6}. Außerdem ist $\QR{\p_1'}{\p_0'}$ ein Primideal in $\QR{A}{\p_0'}$ und nach Lying-Over existiert ein Primideal $\q\subseteq \QR{B}{\p_0}$ über dem Primideal $\QR{\p_1'}{\p_0'}$ von $\QR{A}{\p_0'}$. Dann existiert ein Primideal $\p_1\subseteq B$ mit $\q = \QR{\p_1}{\p_0}$. Wegen $\p_1' \supsetneq \p_0'$ ist $\q \neq 0$ und $\p_1 \supsetneq \p_0$. Außerdem ist $\p_1\cap A = \p_1'$, denn betrachte das kommutative Diagramm
	$$\begin{tikzcd}
	A \arrow[hook]{r}{i} \arrow{d}{\pi_A} & B \arrow{d}{\pi_B} \\
	\QR{A}{\p_0'} \arrow{r}{j} & \QR{B}{\p_0} 
	\end{tikzcd}$$
	(wobei $i,j$ Inklusionen, $\pi_A, \pi_B$ die kanonischen Projektionen seien). Es ist 
	$$\p_1\cap A = i^{-1}(\p_1) = i^{-1}\left(\pi_B^{-1}\left(\QR{\p_1}{\p_0}\right)\right) = \pi_A^{-1}\left(j^{-1}\left(\QR{\p_1}{\p_0}\right)\right) = \pi_A^{-1}\left(\QR{\p_1'}{\p_0'}\right)= \p_1'$$
\end{proof}
\begin{df}
	Sei $A\neq 0$. Eine endliche Kette von $n+1$ Primidealen $\p_0\supsetneq \p_1 \supsetneq \ldots \supsetneq \p_n$ heißt eine \define{Primidealkette der Länge $n$\index{Primidealkette}} in $A$. Für ein Primideal $\p\subseteq A$ heißt 
	$$\text{ht}(\p):= \sup\{n\in \N_0 \ | \ \p = \p_0 \supsetneq \p_1 \supsetneq \ldots \supsetneq \p_n \text{ ist eine PI-Kette der Länge } n\text{ in } A\}$$
	die \define{Höhe\index{Höhe eines Primideals}}. 
	$$\dim(A) := \sup\{\text{ht}(\p) \ | \ \p \text{ Primideal in }A \}$$
	heißt die \define{(Krull-)Dimension\index{(Krull-)Dimension}} von $A$.
\end{df}
\begin{bsp}
	\begin{enumerate}[label = \alph*)]
		\item Sei $K$ ein Körper. Dann ist $\dim K=0$, denn $(0)$ ist das einzige Primideal in $K$.
		\item Sei $A$ ein Hauptidealring, der kein Körper ist. Dann ist $\dim A=1$, denn es existiert ein Primideal $\neq (0)$ in $A$, da ein maximales Ideal $\neq (0)$ in $A$ existiert. Sei $\p$ ein Primideal in $A$, $\p\neq 0$. Dann existiert ein Primelement $\pi \in A$ mit $\p = (\pi)$. Es ist $\text{ht}(\p) =1$, denn ist $(0) \neq \q \subseteq \p $ ein Primideal, dann existiert ein $\tilde{\pi} \in A$ mit $\q=(\tilde{\pi})$, also $\pi |\tilde{\pi}$, d.h. $\pi$ ist assoziiert zu $\tilde{\pi}$ und damit $\q = \p$, also $\text{ht}(\p) =1$.
	\end{enumerate}
\end{bsp}
\begin{sa}
	Sei $B|A$ eine ganze Ringerweiterung, $\p'\subseteq A$ ein Primideal. Dann gilt:
	\begin{enumerate}[label= \alph*)]
		\item $\dim B = \dim A$.
		\item Für jedes Primideal $\p\subseteq B$ über $\p'$ ist $\dim \QR{A}{\p'} = \dim \QR{B}{\p}$ und $\text{ht}(\p) \leq \text{ht}(\p')$.
		\item Falls $\text{ht}(\p')< \infty$, dann existiert ein Primideal $\p$ von $B$ über $\p'$ mit $\text{ht}(\p) = \text{ht}(\p')$
	\end{enumerate}
\end{sa}
\begin{proof}
	$a)$ Aus \ref{15.18} folgt $\dim A \geq \dim B$ und aus dem Going-Up (\ref{15.19}) folgt $\dim A \leq \dim B. $\\
	$b)$ Es ist $\QR{B}{\p} | \QR{A}{\p'}$ ganz nach \ref{15.6}, also ist nach $a)$ $\dim \QR{A}{\p'} = \dim \QR{B}{\p}$. Aus \ref{15.18} folgt $\text{ht}(\p) \leq \text{ht}(\p')$.\\
	$c)$ Sei $\text{ht}(\p')=r < \infty$, $\p_0'\subsetneq \p_1' \subsetneq \ldots \subsetneq \p_r' = \p'$ eine Primidealkette in $A$. Nach Lying-Over existiert ein Primideal $\p_0$ in $B$ über $\p_0'$ und nach Going-Up existiert ein Primidealkette 
	$$\p_0\subseteq \p_1 \subsetneq \ldots \subsetneq \p_r \quad \text{mit} \quad \p_i \cap A = \p_i'$$
	für $i=1, \ldots, r$. Es folgt $\text{ht}(\p_r)\geq r$. Nach $b)$ ist $\text{ht}(\p_r) \leq \text{ht}(\p_r \cap A) =\text{ht}(\p_r') = \text{ht}(\p') =r$. Somit ist $\text{ht}(\p_r) = r$. 
\end{proof}
\begin{anm}
	Im Allgemeinen ist $\text{ht}(\p) \neq \text{ht}(\p')$
\end{anm}
\begin{sa}[Going-Down]
	Sei $B|A$ eine ganze Erweiterung nullteilerfreier Ringe, $A$ normal. Sei $\p_0'\supsetneq \p_1'\supsetneq \ldots \supsetneq \p_r'$ eine Primidealkette in $A$ und $\p_0$ sei ein Primideal in $B$ über $\p_0'$. Dann existiert eine Primidealkette
	$$\p_0\supsetneq \p_1\supsetneq \ldots \supsetneq \p_r \quad \text{in }B \quad \text{mit} \quad \p_i \cap A = \p_i'$$
	für $i=1, \ldots, r$. Insbesondere gilt: Ist $\p'$ ein Primideal in $A$ und $\p$ ein Primideal in $B$ über $A$, dann ist $\text{ht}(\p) = \text{ht}(\p')$.
\end{sa}
\begin{proof}
	siehe Atiyah-MacDonald, Thm 5.16.
\end{proof}
\newpage
\subsection{Direkte und projektive Limiten}
\begin{df}\label{16.1}
	Sei $I$ eine Menge, $\leq$ eine Halbordnung auf $I$. $(I, \leq)$ heißt \define{gerichtet\index{gerichtete Menge}} $\defi$ Für alle $a,b\in I$ existiert ein $c\in I$ mit $a\leq c$ und $b\leq c$.
\end{df}
\begin{center}
	\textbf{Für den Rest des Abschnitts sei $(I, \leq)$ stets eine gerichtete, halbgeordnete Menge}
\end{center}
\begin{df}\label{16.2}
	Ein über $I$ indiziertes \define{direktes System\index{direktes System} (induktives System)} von $A$-Moduln besteht aus
	\begin{itemize}
		\item einer Familie $(M_i)_{i\in I}$ von $A$-Moduln
		\item einer Familie $(\phi_{ij})_{i,j\in I, i\leq j}$ von $A$-Modulhomomorphismen $\phi_{ij}:M_i \to M_j$ (\define{Übergangsabbildungen \index{Übergangsabbildungen}}), sodass gilt:
		\begin{itemize}
			\item $\phi_{ii}= \id_{M_i}$ für alle $i\in I$
			\item $\phi_{ik}= \phi_{jk} \circ \phi_{ij}$ für alle $i,j,k\in I$ mit $i\leq j \leq k$
		\end{itemize}
	\end{itemize} 
	Im Folgenden schreiben wir dafür kurz $(M_i, \phi_{ij})_I$
\end{df}
\begin{bsp}\label{16.3}
	\begin{enumerate}[label = \alph*)]
		\item $I=\N$ mit "'$\leq$"', einen $A$-Modul $M$ und $M_1 \subseteq M_2 \subseteq \ldots $ einer Folge von Untermoduln von $M$, $\phi_{ij}:M_i \hookrightarrow M_j$ der Inklusion für $i\leq j$ liefert ein direktes System von $A$-Moduln
		\item Sei $M$ ein $A$-Modul, $I$ einer Menge von Untermoduln von $M$, die gerichtet bezüglich "'$\subseteq$"' sei. Setze $M_i:= i$ für $i\in I$. Dann ist $(M_i)_{i\in I}$ ein direktes System von $A$-Moduln mit den Inklusionen als Übergangsabbildungen.
		\item Sei $M$ ein $A$-Modul, $I$ die Menge der endlich erzeugten Untermoduln von $M$ ist gerichtet bezüglich "'$\subseteq$"' (mit $M_1, M_2\subseteq M$ endlich erzeugt ist auch $M_1+M_2$ endlich erzeugt), dies liefert ein Beispiel für $b)$.
		\item $I=\N$ mit "'$\big| $"'-Halbordnung, $M_i= \QR{\Z}{i\Z}$, 
		$$\phi_{ij}:\QR{\Z}{i\Z} \longrightarrow \QR{\Z}{j\Z}, \quad a+ i\Z\mapsto \frac{j}{i}a + j\Z\quad \text{für } i|j$$
		liefert ein direktes System von $\Z$-Moduln.
	\end{enumerate}
\end{bsp}
\begin{bem+df}\label{16.4}
	Ein Homomorphismus von direkten Systemen $(M_i, \phi_{ij}^M)_I$ ins direkte System $(N_i, \phi_{ij}^N)_I$ ist eine Familie $(f_i)_{i\in I}$ von $A$-Modulhomomorphismen $f_i:M_i \to N_i$, sodass $\phi_{ij}^N \circ f_i = f_j \circ \phi_{ij}^M$ für alle $i,j\in I$ mit $i\leq j$ gilt:
	$$\begin{tikzcd}
	M_i \arrow{r}{f_i} \arrow[swap]{d}{\phi_{ij}^M} & N_i \arrow{d}{\phi_{ij}^N}\\
	M_j \arrow{r}{f_j} & N_j
	\end{tikzcd}$$
	Die über $I$ indizierten direkten Systeme von $A$-Moduln bilden zusammen mit obigem Homomorphismus eine abelsche Kategorie (alles komponentenweise), welche mit $I$-Dir-$A$-Mod bezeichnet wird.
\end{bem+df}
\begin{df}\label{16.5}
	Ein \define{direkter Limes\index{direkter Limes}} des direkten Systems $(M_i, \phi_{ij})_I$ von $A$-Moduln (indirekter Limes, Kolimes) ist ein $A$-Modul $M$ zusammen mit einer Familie $(\phi_i)_{i\in I}$ von $A$-Modulhomomorphismen $\phi_i:M_i \to M$, sodass $\phi_i = \phi_j \circ \phi_{ij}$ für alle $i,j\in I$ mit $i\leq j$ ist, sodass folgende universelle Eigenschaft erfüllt ist: Für jeden $A$-Modul $N$ und jede Familie $(\psi_i)_{i\in I}$ von $A$-Modulhomomorphismen $\psi_i :M_i \to N$ mit $\psi_i = \psi_j \circ \phi_{ij}$ für alle $i\leq j$ existiert ein eindeutig bestimmter $A$-Modulhomomorphismus $\psi:M \to N$ mit $\psi_i = \psi \circ \phi_i$ für alle $i\in I$:
	$$\begin{tikzcd}
	& N & \\
	& M \arrow[dashed]{u}{\psi}& \\
	M_i \arrow{ur}{\phi_i} \arrow[bend left]{uur}{\psi_i} \arrow{rr}{\phi_{ij}} & & M_j \arrow[swap]{ul}{\phi_j} \arrow[bend right, swap]{uul}{\psi_j}
	\end{tikzcd}$$
\end{df}
\begin{sa}\label{16.6}
	Sei $(M_i, \phi_{ij})_I$ ein direktes System von $A$-Moduln. Dann gilt:
	\begin{enumerate}[label= \alph*)]
		\item Setzt man $L:= \overset{.}{\bigcup\limits_{i\in I}} M_i$ und für $x,y\in L$, $x= m_i \in M_i$, $y= m_j \in M_j$ 
		$$x\sim y \defi \text{Es existiert ein }k\in I \text{ mit } i\leq k, j \leq k \text{ mit } \phi_{ik}(x) = \phi_{jk}(y)$$
		dann ist "'$\sim$"' eine Äquivalenzrelation auf $L$.  \\
		Hierbei ist $\overset{.}{\bigcup\limits_{i\in I}} M_i:= \bigcup_{i\in I}(M_i \times \{i\})$, man identifiziert $M_i\times \{i\}$ mit $M_i$
		\item Setzt man $M:= \QR{L}{\sim}$, dann wird $M$ auf naürliche Weise zu einem $A$-Modul und die Abbildung $\phi_i : M_i \to M, \, x \mapsto \bar x$ sind $A$-Modulhomomorphismen mit $\phi_j = \phi_j \circ \phi_{ij}$ für alle $i,j\in I $ mit $i\leq j$. Für jedes $m\in M$ existiert ein $m_i \in M_i$ mit $m= \phi_i(m_i)$.
		\item $(M,(\phi_i)_{i\in I})$ ist ein direkter Limes von $(M_i, \phi_{ij})_I$
		\item Ist $(M', (\phi_i')_{i\in I})$ ein weiterer direkter Limes des obigen direkten System, dann existiert ein eindeutig bestimmter Isomorphismus $\gamma:M \to M'$ mit $\gamma \circ \phi_i = \phi_i'$ für alle $i\in I$
	\end{enumerate}
Notation: $M = \varinjlim\limits_{i\in I} M_i$
\end{sa}
\begin{proof}
	\begin{enumerate}[label= \alph*)]
		\item Offenbar ist "'$\sim$"' reflexiv und symmetrisch. "'$\sim$"' ist transitiv, denn: Sei $x\sim y, \, y\sim z$, $x= m_i \in M_i, \, y= m_j \in M_j, \, z=m_k \in M_k$. Dann existieren $l\in I$ mit $i\leq l, \, j \leq l$, sodass $\phi_{il}(x) = \phi_{jl}(y)$, $m\in I$ mit $j\leq n, \, k \leq n$, sodass $\phi_{jm}(y) = \phi_{km}(z)$. Da $I$ gerichtet ist, existiert ferner ein $n\in I$ mit $l,m\leq n$. Also ist 
		\begin{eqnarray*}
		\phi_{in}(x) &=& \phi_{ln}(\phi_{il}(x)) = \phi_{ln}(\phi_{jl}(y)) = \phi_{jn}(y) = \phi_{mn}(\phi_{jm}(y)) = \phi_{mn}(\phi_{km}(z)) \\
		&=& \phi_{kn}(z)
		\end{eqnarray*}
	Also ist $x\sim z$.
	\item Seien $m,n\in M$. Dann existieren$i\in I, \, m_i \in M_i$ mit $m=\bar m_i = \phi_i(m_i)$, $j\in I, \, n_j \in M_j$ mit $n= \bar n_j = \phi_j(n_j)$. Da $I$ gerichtet ist, existiert ein $k\in K$ mit $i\leq k, \, j\leq k$. Setze $m_k:= \phi_{ik}(m_i), \, n_k:= \phi_{jk}(n_j)$
	$$m+n:= \bar{m_k + n_k} = \phi_k(m_k + n_k)$$
	Man rechnet nach, dass dies wohldefiniert ist. Ist $a\in A$, setze $am:= \bar{a m_i}= \phi_i(a m_i)$. Erneut rechne man nach, dass dies wohldefiniert ist und $M$ dadurch zum $A$-Modul wird. Die $\phi_i$, $i\in I$ sind dann $A$-Modulhomomorphismen nach Konstruktion:
	$$\phi_i(m_i + m_i ') = \bar{m_i + m_i'} = \bar n_i + \bar m_i' = \phi_i(m_i) + \phi_i(n_i')$$
	\item Sei $N$ ein $A$-Modul und $(\psi_i)_{i\in I}$ eine Familie von $A$-Modulhomomorphismen $\psi_i:M_i \to N$ mit $\psi_i = \psi_j \circ \phi_{ij}$ für alle $i,j\in I$ mit $i\leq j$
		$$\begin{tikzcd}
	& N & \\
	& M \arrow[dashed]{u}& \\
	M_i \arrow{ur}{\phi_i} \arrow[bend left]{uur}{\psi_i} \arrow{rr}{\phi_{ij}} & & M_j \arrow[swap]{ul}{\phi_j} \arrow[bend right, swap]{uul}{\psi_j}
	\end{tikzcd}$$
	Sei $m\in M$. Wähle ein $i\in I$, $m_i \in M_i$ mit $\phi_i(m_i) = m$. Setze $\phi(m) := \psi_i(m_i)$. Dann ist $\psi$ wohldefiniert, denn sei $\phi_i(m_i) = m = \phi_j(m_j)$. Dann ist $m_i \sim m_j$. Dann existiert ein $k\in I$ mit $i,j\leq k$ und $\phi_{ik}(m_i) = \phi_{jk}(m_j)$. Dann ist
	$$\psi(m_i) = \psi_k(\phi_{ik}(m_i)) = \psi_k(\phi_{jk}(m_j)) = \psi_j(m_j)$$
	$\psi$ ist ein $A$-Modulhomomorphismus, denn sei $m= \phi_i(m_i), \, n= \phi_j(n_j) $ mit $m_i = M_i, \, n_j = M_j$. Dann existiert ein $k\in I$ mit $i\leq k, \, j \le k$, es ist 
	$$m+n = \phi_i(m_i) + \phi_j(n_j) = \phi_k(\phi_{ik}(m_i)) + \phi_l(\phi_{jk}(n_j)) = \phi_k(\phi_{ik}(m_i) + \phi_j(n_j))$$
	Also ist 
	$$\psi(m+n) = \psi_k(\phi_{ik}(m_i) + \phi_{jk}(n_j)) = \psi_i(m_i) + \psi_{j}(n_j) = \psi(m) + \psi(n)$$
	Für $a\in A$ ist $\phi_i(am_i) = am$, also $\psi(am) = \psi_i(am_i) = a \psi_i(m_i) = a \psi(m)$. Es ist $\psi \circ \phi_i = \psi_i$ für alle $i\in I$, da $\psi(\phi_i(m_i)) = \psi_i(m_i)$ nach Definition von $\psi$. Durch die Vorgabe $\psi \circ \phi_i = \psi_i$ für alle $i\in I$ ist $\psi$ eindeutig bestimmt.
	\item mit den Standardargumenten.
	\end{enumerate}
\end{proof}
\begin{bsp} \label{16.7}
		vergleiche \ref{16.3}
	\begin{enumerate}[label= \alph*)]
		\item Betrachte $M_1 \subseteq M_2 \subseteq \ldots \subseteq M$ mit den Inklusionen als Übergangsabbildungen. Dann ist 
		$$\varinjlim_{i\in \N} M_i = \bigcup_{i\in \N} M_i \subseteq M, \quad \text{wobei }\phi_i :M_i \hookrightarrow \bigcup_{i\in \N}M_i \text{ Inklusion}$$
			$$\begin{tikzcd}[row sep = large]
		& N & \\
		& \bigcup\limits_{i\in \N}M_i \arrow[dashed]{u}& \\
		M_i \arrow[hook]{ur}{\phi_i} \arrow[bend left]{uur}{\psi_i} \arrow{rr}{\phi_{ij}} & & M_j \arrow[swap, hook]{ul}{\phi_j} \arrow[bend right, swap]{uul}{\psi_j}
		\end{tikzcd}\qquad \psi_j \circ \phi_{ij} = \psi_i \text{ für } i\leq j \text{ d.h. } \psi_j \big|_{M_i} = \psi_i$$
		\item Sei $(M_i)_{i\in I}$ eine bezüglich "'$\subseteq$"' gerichtete Familie von Untermoduln von $M$, indiziert über sich selbst. Dann ist $\varinjlim_{i\in I} M_i = \bigcup_{i\in I} M_i$
		\item Sei $(M_i)_{i\in I}$ die Familie der endlich erzeugten Untermoduln von $M$. Dann ist $\varinjlim_{i\in I} M_i = \bigcup_{i\in I} M_i = M$, denn jedes Element aus $M$ liegt in einem endlich erzeugten Untermodul von $M$.
	\end{enumerate}
\end{bsp}
\begin{bem}\label{16.8}
	Sei $(f_i)_{i\in I}:(M_i, \phi_{ij}^M)_I \to (N_i, \phi_{ij}^N)_I$ ein Homomorphismus der direkten Systeme von $A$-Moduln. Dann existiert ein eindeutig bestimmter Homomorphismus 
	$$\varinjlim_{i\in I} f_i: \varinjlim_{i\in I} M_i \longrightarrow \varinjlim_{i\in I} N_i$$
	mit $(\varinjlim f_i) \circ \phi_i^M = \phi_i^N \circ f_i $ für alle $i\in I$:
	$$\begin{tikzcd}[row sep = large]
	M_i \arrow{r}{f_i} \arrow{d}{\phi_i^M} & N_i \arrow{d}{\phi_i^N} \\
	\varinjlim_{i\in I} M_i \arrow[swap]{r}{\varinjlim f_i} & \varinjlim_{i\in I} N_i
	\end{tikzcd}$$
\end{bem}
\begin{proof}
	Wende die Universelle Eigenschaft von $\varinjlim_{i\in I} M_i$ an:
	$$\begin{tikzcd}[column sep = large]
	& \varinjlim\limits_{i\in I} N_i & \\
	& \varinjlim\limits_{i\in I} M_i \arrow[dashed]{u}{\varinjlim f_i}& \\
	M_i \arrow[hook]{ur}{\phi_i^M} \arrow[bend left]{uur}{\phi_i^N \circ f_i} \arrow{rr}{\phi_{ij}^M} & & M_j \arrow[swap, hook]{ul}{\phi_j^M} \arrow[bend right, swap]{uul}{\phi_j^N \circ f_j}
	\end{tikzcd}$$
	(beachte: $\phi_j^N \circ f_j \circ \phi_{ij}^M = \phi_j^N \circ \phi_{ij}^N \circ f_i = \phi_i^N \circ f_i$ für $i\leq j$)
\end{proof}
\begin{fo} \label{16.9}
	$\varinjlim_{i\in I}-:I$-Dir-$A$-Mod$\to A$-Mod ist ein additiver Funktor
\end{fo}
\begin{proof}
	Die Funktorialität und Additivität von $\varinjlim_{i\in I}-$ ergeben sich aus der Charakterisierung von $\varinjlim f_i$ in \ref{16.8}.
\end{proof}
\begin{bsp} \label{16.10}
	(vgl. Beispiel \ref{16.3}(d)) $I=\N$ mit "'$\big | $"'-Halbordnung, $M_i = \QR{\Z}{i\Z}$,$$\phi_{ij}: \QR{\Z}{i\Z} \to \QR{\Z}{j\Z}, \quad  a+i\Z \mapsto \frac{j}{i}a + j \Z \quad \text{für} \ i|j$$\\
	Setze $$f_i: \QR{\Z}{i\Z} \overset{\sim}{\longrightarrow} \QR{(\frac{1}{i} \Z)}{\Z} \subseteq \QR{\Q}{\Z}, \quad a+i \Z \mapsto \frac{a}{i} + \Z $$ sowie $$ \psi_{ij}: \QR{(\frac{1}{i} \Z)}{\Z} \hookrightarrow \QR{(\frac{1}{j} \Z)}{\Z}, \quad \frac{a}{i} + \Z \mapsto \frac{a}{i} + \Z  \quad \text{für} \ i|j$$ 
	Dann ist $ (\QR{(\frac{1}{i} \Z)}{\Z}, \psi_{ij})_{\N} $ ist ein direktes System von $\Z$-Moduln und $$ (f_i)_{i \in \N} : (\QR{\Z}{i \Z}, \phi_{ij})_{\N} \to (\QR{(\frac{1}{i} \Z)}{\Z}, \psi_{ij})_{\N} $$ ist ein Isomorphismus direkter Systeme( beachte: $\psi_{ij} \circ f_i = f_j \circ \phi_{ij} $ für $i|j$, denn: \\
	$(\psi_{ij} \circ f_i)(a + i\Z) = \psi_{ij}(\frac{a}{i} + \Z) = \frac{a}{i} + \Z$,\\
	$(f_j \circ \phi_{ij})(a + i\Z) = f_j(\frac{j}{i} a + j\Z) = \frac{ja}{ij} + \Z = \frac{a}{i} + \Z$\\
	Mit \ref{16.9} folgt $\varinjlim_{i\in \N} \QR{\Z}{i \Z} \cong \varinjlim_{i\in \N} \QR{(\frac{1}{i} \Z)}{\Z} \underset{\ref{16.7}(b)}{=} \bigcup_{i\in \N} \QR{(\frac{1}{i} \Z)}{\Z} = \QR{\Q}{\Z}$.
\end{bsp}
\begin{bem} \label{16.11}
	$(M_i, \phi_{ij})_I $ direktes System von $A$-Moduln, $N$ ein $A$-Modul. Dann gibt es einen natürlichen Isomorphismus $$\varinjlim_{i\in I}(M_i \otimes_A N) \cong (\varinjlim_{i\in I} M_i) \otimes_A N $$
\end{bem}
\begin{proof}
	Es genügt nachzurechnen, dass die rechte Seite die Universelle Eigenschaft von $\varinjlim$ erfüllt.
\end{proof}
\begin{sa} \label{16.12}
	Der Funktor $\varinjlim_{i\in I} - :$ $I$-Dir-$A$-Mod $\to A$-Mod  ist exakt. 
\end{sa}
\begin{proof}
	Sei $\begin{tikzcd}
	(K_i, \phi_{ij}^K)_I \arrow{r}{(f_i)_I} & (M_i, \phi_{ij}^M)_I  \arrow{r}{(g_i)_I} & (N_i, \phi_{ij}^N)_I \end{tikzcd}$ eine exakte Sequenz in $I$-Dir-$A$-Mod (insebsondere ist $\begin{tikzcd} K_i \arrow{r}{f_i} & M_i \arrow{r}{g_i} & N_i \end{tikzcd}$ exakt für alle $i \in I$). \\
	 Die Limiten der Systeme seien durch $(\varinjlim K_i, \phi_i^K),(\varinjlim M_i, \phi_i^M), (\varinjlim N_i, \phi_i^N) $ gegeben. \\
	Mit Folgerung \ref{16.9} ist $\varinjlim g_i \circ \varinjlim f_i = \varinjlim (g_i \circ f_i) = 0$. \\
	Sei $m \in \varinjlim M_i $ mit $ (\varinjlim g_i)(m) = 0$. Nach \ref{16.6}(b) existieren $ i \in I, m_i \in M_i $ mit $m=\phi_i^M(m_i) $. Wir haben das kommutative Diagramm: \\
	$$\begin{tikzcd}[row sep = large]
	m_i \in \arrow[mapsto]{d} & M_i \arrow{r}{g_i} \arrow{d}{\phi_i^M} & N_i \arrow{d}{\phi_i^N} \\
	m \in & \varinjlim M_i \arrow[swap]{r}{\varinjlim g_i} & \varinjlim N_i
	\end{tikzcd}$$
	Dann ist $ \phi_i^N(g_i(m_i)) = (\varinjlim g_i)(\underbrace{\phi_i^M(m_i)}_{=m}) = (\varinjlim g_i)(m) = 0.$  \\
	Es folgt, dass ein $j \in J, j \geq i $ existiert, mit $\phi_{ij}^N(\underbrace{g_i(m_i)}_{g_j(\phi_{ij}^M(m_i))})= 0.$ \\
 	Somit $ \phi_{ij}^M(m_i) \in \ker(g_j) = \im(f_j) $, woraus resultiert, dass ein $k_j \in K_j $ existiert, mit $ f_j(k_j) = \phi_{ij}^M(m_i)$\\
	Setze $k:= \phi_j^K(k_j) \in \varinjlim K_i $ und betrachte folgendes kommutatives  Diagramm: \\
	$$\begin{tikzcd}[row sep = large]
	k_j \in \arrow[mapsto]{d} & K_j \arrow{r}{f_j} \arrow{d}{\phi_j^K} & M_j \arrow{d}{\phi_j^M} \\
	k \in & \varinjlim K_i \arrow[swap]{r}{\varinjlim f_i} & \varinjlim M_i
	\end{tikzcd}$$
	So gilt:\\
	 $(\varinjlim f_i)(k) = (\varinjlim f_i)(\phi_j^K(k_j)) = \phi_j^M(f_j(k_j)) =\phi_j^M(\phi_{ij}^M(m_i))=\phi_i^M(m_i)=m$ \\
	 Das heißt $ m \in \im(\varinjlim f_i) $
\end{proof}
\begin{anm}
	$\varinjlim_{i\in I} -$  \ ist linksadjungiert zum "konstanten System-Funktor": $$ I-\text{const}: A\text{-Mod} \to I-\text{Dir-}A\text{-Mod}, \quad M \mapsto (M,id_M)_I $$
	denn: $Hom_{A\text{-Mod}}(\varinjlim_{i\in I} M_i,N) \cong Hom_{I-\text{Dir-}A\text{-Mod}}((M_i,\phi_{ij})_I, I\text{-const(N)}) $
\end{anm}
\begin{fo} \label{16.13}
	Sei $ (M_i, \phi_{ij})_I $ ein direktes System flacher $A$-Moduln. Dann ist $\varinjlim_{i\in I} M_i $ flach. 
\end{fo}
\begin{proof}
	Sei  $\begin{tikzcd} 0 \arrow{r} & N' \arrow{r}{f} & N \end{tikzcd}$  eine exakte Sequenz von $A$-Moduln. Da $M_i$ flach für alle $i \in I $ folgt dass $\begin{tikzcd} 0 \arrow{r} & (M_i \otimes_A N',\phi_{ij}\otimes id_{N'})_I \arrow{r} & (M_i \otimes_A N, \phi_{ij} \otimes id_N)_I \end{tikzcd}$  eine exakte Sequenz von direkten Systemen.\\
	 Mit \ref{16.12} folgt die Exaktheit von $\begin{tikzcd} 0 \arrow{r} & \varinjlim_{i\in I}(M_i \otimes_A N')  \arrow{r} & \varinjlim_{i\in I}(M_i \otimes N) \end{tikzcd}$ und mit  \ref{16.11} die Exaktheit von  $\begin{tikzcd} 0 \arrow{r} & (\varinjlim_{i\in I}M_i) \otimes_A N' \arrow{r} & (\varinjlim_{i\in I}M_i) \otimes N \end{tikzcd}$ \\
	 Damit ist $ \varinjlim_{i\in I} M_i $ flach
\end{proof}
\begin{fo} \label{16.14}
	Sei $M$ ein $A$-Modul, sodass jeder endlich erzeugte Untermodul von $M$ flach ist. Dann ist M flach. 
\end{fo}
\begin{proof}
	Ist $(M_i)_{i \in I } $ die Familie der endlich erzeugten Untermoduln von $M$, dann ist $M = \varinjlim_{i\in I} M_i $ (vergleiche Beispiel \ref{16.7}(c)). Die Behauptung folgt somit aus \ref{16.13}.
\end{proof}
\begin{df}
	$J \subseteq I $ heißt \define{kofinal} $\Leftrightarrow $ für jedes $ i \in I $ existiert ein $j \in J $ mit $ i \leq j$.
\end{df}
\begin{anm}
	\begin{itemize}
		\item Ist $J \subseteq I $ kofinal, dann ist $J $ gerichtet, denn  $I$ ist  gerichtet und damit folgt für  $i,j \in J  $, dass ein $k \in I$ existiert, mit $i,j \leq k $ und es existiert ein $l \in J $ mit $k \leq l $. Das heißt $i,j \leq l .$
		\item Ist $(M_i, \phi_{ij})_I $ ein direktes System, dann ist auch $(M_i, \phi_{ij})_J $ ein direktes System und es gibt einen eindeutig bestimmten Homomorphimus: $$ \iota: \varinjlim_{i\in J} M_i \to \varinjlim_{i\in I} M_i $$ 
		Mit $ \iota \circ \phi_i^J = \phi_i^I $ für alle $ i \in J $: 
		$$\begin{tikzcd}[column sep = large]
		& \varinjlim\limits_{i\in I} M_i & \\
		& \varinjlim\limits_{i\in J} M_i \arrow[dashed, red]{u}{\iota}& \\
		M_i \arrow{ur}{\phi_i^J} \arrow[bend left]{uur}{\phi_i^I} \arrow{rr}{\phi_{ij}} & & M_j \arrow[swap]{ul}{\phi_j^J} \arrow[bend right, swap]{uul}{\phi_j^I}
		\end{tikzcd}$$
	\end{itemize}
\end{anm}
\begin{bem} \label{16.16}
	Sei $ J \subseteq I $ kofinal, $(M_i, \phi_{ij})_I $ ein direktes System von $A$-Moduln. Dann ist der natürliche Homomorphismus $$ \iota: \varinjlim_{i\in J} M_i \to \varinjlim_{i\in I} M_i $$ ein Isomorphismus. 
\end{bem}
\begin{proof}
	Situation: 
	$$\begin{tikzcd}[row sep = large]
	M_i \arrow{r}{id_{M_i}} \arrow{d}{\phi_i^J} & M_i \arrow{d}{\phi_i^I} \\
	\varinjlim_{j \in J} M_j \arrow[swap]{r}{\iota} & \varinjlim_{i\in I} M_i
	\end{tikzcd}$$
	\begin{enumerate}
		\item[] Injektivität von $\iota $: \\
		Sei $m \in \varinjlim_{j \in J} M_j $ mit $ \iota(m)=0$. Es existieren  $i \in J, m_i \in M_i $ mit $ m = \phi_i^J(m_i) \Ra \iota(\phi_i^J(m_i)) =0 \Ra \phi_i^I(m_i) = 0  \Ra $ Es existiert ein $k \in I, k \geq i $ mit $\phi_{ik}(m_i) =0$.Wegen $J \subseteq I $ kofinal, existiert ein $j\in J, j \geq k $. Es ist $\phi_{ij}(m_i)= \phi_{kj}(\underbrace{\phi_{ik}(m_i)}_{=0}) = 0 \Ra m = \phi_i^J(m_i)=0. $
		\item[] Surjektivität von $\iota$: \\
		Sei $m \in \varinjlim_{i\in I} M_i \Ra $ Es existieren $i \in I, m_i \in M_i $ mit $ m = \phi_i^I(m_i)$. Wegen $J \subseteq I $ kofinal, existiert  ein $j \in J $ mit $ j \geq i $. Setze $m_j := \phi_{ij}(m_i) \in M_j$, dann ist $m = \phi_i^I(m_i) = \phi_j^I(\phi_{ij}(m_i)) = \phi_j^I(m_j)=\iota(\phi_j^J(m_j)) \in \im\iota $.
	\end{enumerate}
\end{proof}
\begin{df} \label{16.17}
	Ein über $I$ indiziertes \define{projektives System\index{projektives System}} von $A$-Moduln besteht aus
	\begin{itemize}
		\item einer Familie $(M_i)_{i\in I}$ von $A$-Moduln
		\item einer Familie $(\phi_{ij})_{i,j\in I, i\leq j}$ von $A$-Modulhomomorphismen $\phi_{ij}:M_j \to M_i$ (\define{Übergangsabbildungen \index{Übergangsabbildungen}}), sodass gilt:
		\begin{itemize}
			\item $\phi_{ii}= \id_{M_i}$ für alle $i\in I$
			\item $\phi_{ik}= \phi_{ij} \circ \phi_{jk}$ für alle $i,j,k\in I$ mit $i\leq j \leq k$
		\end{itemize}
	\end{itemize} 
	Im Folgenden schreiben wir dafür kurz $(M_i, \phi_{ij})_I$
\end{df}
\begin{bsp} \label{16.18}
	\begin{enumerate} [label = \alph*)]
		\item $I=\N$ mit "'$\leq$"', einem $A$-Modul $M$ und $M_1 \subseteq M_2 \subseteq \ldots $ einer Folge von Untermoduln von $M$, $\phi_{ij}:M_j \hookrightarrow M_i$ der Inklusion für $i\leq j$ liefert ein projektives System von $A$-Moduln
		\item Sei $M$ ein $A$-Modul, $I$ einer Menge von Untermoduln von $M$, die gerichtet bezüglich "'$\subseteq$"' sei. Setze $M_i:= i$ für $i\in I$. Dann ist $(M_i)_{i\in I}$ ein projektives System von $A$-Moduln mit den Inklusionen als Übergangsabbildungen.
		\item $I=\N$ mit "'$\big| $"'-Halbordnung, $M_i= \QR{\Z}{i\Z}$, 
		$$\phi_{ij}:\QR{\Z}{j\Z} \longrightarrow \QR{\Z}{i\Z}, \quad a+ j\Z \mapsto a + i\Z \quad \text{für } i|j$$ liefert ein projektives System von $\Z$-Moduln.
		\item $I=\N$ mit "' $\leq $"'-Halbordnung, $p$ Primzahl, $M_i= \QR{\Z}{p^{i}\Z}$, 
		$$\phi_{ij}:\QR{\Z}{p^{j}\Z} \longrightarrow \QR{\Z}{p^{i}\Z}, \quad a+ p^{j}\Z \mapsto a + p^{i}\Z \quad \text{für } i \leq j$$ liefert ein projektives System von $\Z$-Moduln.
	\end{enumerate}
\end{bsp}
\begin{bem+df} \label{16.19}
	Ein Homomorphismus von projektiven Systemen $(M_i, \phi_{ij}^M)_I$ ins projektive System $(N_i, \phi_{ij}^N)_I$ ist eine Familie $(f_i)_{i\in I}$ von $A$-Modulhomomorphismen $f_i:M_i \to N_i$, sodass $\phi_{ij}^N \circ f_j = f_i \circ \phi_{ij}^M$ für alle $i,j\in I$ mit $i\leq j$ gilt:
	$$\begin{tikzcd}
	M_i \arrow{r}{f_i} & N_i \\
	M_j \arrow{r}{f_j} \arrow{u}{\phi_{ij}^M} & N_j \arrow{u}{\phi_{ij}^N}
	\end{tikzcd}$$
	Die über $I$ indizierten projektiven Systeme von $A$-Moduln bilden zusammen mit obigem Homomorphismus eine abelsche Kategorie (alles komponentenweise), welche mit $I$-Pro-$A$-Mod bezeichnet wird.
\end{bem+df}
\begin{df} \label{16.20}
		Ein \define{projektiver Limes\index{projektiver Limes}} des projektiven Systems $(M_i, \phi_{ij})_I$ von $A$-Moduln ist ein $A$-Modul $M$ zusammen mit einer Familie $(\phi_i)_{i\in I}$ von $A$- Modulhomomorphismen $\phi_i:M \to M_i$, sodass $\phi_i = \phi_{ij} \circ \phi_j$ für alle $i,j\in I$ mit $i\leq j$ ist, sodass folgende universelle Eigenschaft erfüllt ist: Für jeden $A$-Modul $N$ und jede Familie $(\psi_i)_{i\in I}$ von $A$-Modulhomomorphismen $\psi_i :N \to M_i$ mit $\psi_i = \phi_{ij} \circ \psi_j$ für alle $i\leq j$ existiert ein eindeutig bestimmter $A$-Modulhomomorphismus $\psi:N \to M$ mit $\psi_i = \phi_i \circ \psi$ für alle $i\in I$:
	$$\begin{tikzcd}
	& N \arrow[dashed, red]{d}{\psi} \arrow[bend right]{ddl}{\psi_i} \arrow[bend left, swap]{ddr}{\psi_j}& \\ 
	& M \arrow{dl}{\phi_i} \arrow{dr}{\phi_j}& \\
	M_i  & & M_j  \arrow{ll}{\phi_{ij}}
	\end{tikzcd}$$
\end{df}
\begin{sa} \label{16.21}
	Sei $(M_i,\phi_{ij})_I $ ein projektives System von $A$-Moduln. Dann gilt: 
	\begin{enumerate} [label = \alph*)]
		\item Setzt man $$ M:= \{(m_i)_{i \in I} \in \prod_{i\in I}M_i| \ \phi_{ij}(m_j)=m_i \ \text{für alle} \ i \leq j\} $$ $$\phi_i: M \to M_i, \quad (m_j)_{j\in I} \mapsto m_i $$
		Dann ist $(M, (\phi_i)_{i \in I }) $ ein projektiver Limes von $(M_i,(\phi_{ij})_I$.
		\item Ist $(M',(\phi'_i)_{i \in I})$ ein weiterer projektiver Limes des obigen Systems, dann existiert ein eindeutig bestimmter Isomorphismus $\gamma: M' \to M $ mit $\phi_i \circ \gamma = \phi_i'$ für alle $i \in I $.
	\end{enumerate}
	Notation: $M = \varprojlim_{i\in I} M_i$ 
\end{sa}
\begin{proof}
	\begin{enumerate} [label = \alph*)]
		\item $M$ ist trivialerweise $A$-Modul. Weiterhin sind $\phi_i, i \in I $ $A$ -Modulhomo- morphismen mit $\phi_{ij} \circ \phi_j = \phi_i $ für $i \leq j$: \\
		$$ (\phi_{ij} \circ \phi_j)((m_k)_{k \in I}) = \phi_{ij}(m_j) = m_i= \phi_i((m_k)_{k \in I})  \quad \quad  \text{für} \ (m_k)_{k \in I } \in M $$ Sei $N$ ein $A$-Modul und $(\psi_i)_{i \in I} $ eine Familie von $A$-Modulhomomorphismen $\psi_i:N \to M_i $ mit $ \psi_i = \phi_{ij} \circ \psi_j $ für alle $i,j \in I $ mit $i \leq j$, Setze:  $$ \delta: N \to \prod_{i\in I} M_i, \quad n \mapsto (\psi_i(n))_{i \in I} $$ 
		Wegen $\psi_i = \phi_{ij} \circ \psi_j $ für $i \leq j $ ist $\im\delta \subseteq M $. Setze $\psi:= \delta \big|^M : N \to M$.\\ 
		$ \psi $ ist ein $A$-Modulhomomorphismus mit $\phi_i(\psi(n)) = \psi_i(n) $ für alle $ n \in N, i \in I $, das heißt $\phi_i \circ \psi = \psi_i$. Durch die Vorgabe $\phi_i \circ \psi = \psi_i $ ist $\psi $ eindeutig bestimmt. 
		\item mit Standardargumenten.
	\end{enumerate}
\end{proof}
\begin{bsp} \label{16.22} (vergleiche Beispiel \ref{16.18})
	\begin{enumerate}[label= \alph*)]
		\item Betrachte $M \supseteq M_1 \supseteq M_2 \supseteq \dots $ mit den Inklusionen als Übergangsabbildungen. Dann ist 
		$$\varprojlim_{i\in \N} M_i = \bigcap_{i\in \N} M_i \subseteq M \quad  (\text{mit} \ \phi_i: \bigcap_{\in I} M_i \hookrightarrow M_i \  \text{Inklusionen}) $$
		$$ \begin{tikzcd}[row sep = large]
		& N \arrow[dashed]{d} \arrow[bend right]{ddl}{\psi_i}  \arrow[bend left, swap]{ddr}{\psi_j}& \\
		& \bigcap_{i\in I} M_i \arrow[hook]{dl}{\phi_i} \arrow[swap, hook]{dr}{\phi_j}& \\
	    M_i   & & M_j  \arrow{ll}{\phi_{ij}}
		\end{tikzcd}  $$
		alternativ: $$ \bigcap_{i \in \N} M_i \longrightarrow \{ (m_i)_{i \in \N} \in \prod_{i \in \N} M_i | \ m_j = m_i \ \text{für alle} \ i \leq j\}  \qquad m \mapsto (m)_{i \in \N}$$
		\item Sei $(M_i)_{i\in I}$ eine bezüglich "'$\supseteq$"' gerichtete Familie von Untermoduln von $M$, indiziert über sich selbst. Dann ist $\varprojlim_{i\in I} M_i = \bigcap_{i\in I} M_i$
		\item $\varprojlim_{i\in \N} \QR{\Z}{i \Z} $ bezeichnet man mit $\hat{\Z}$
		\item $\varprojlim_{i\in \N} \QR{\Z}{p^{i} \Z} $ bezeichnet man als $\Z_{p}$
	\end{enumerate}
\end{bsp}
\begin{bem} \label{16.23}
	Sei $(f_i)_{i\in I}:(M_i, \phi_{ij}^M)_I \to (N_i, \phi_{ij}^N)_I$ ein Homomorphismus projektiver  Systeme von $A$-Moduln. Dann existiert ein eindeutig bestimmter Homomorphismus 
	$$\varprojlim_{i\in I} f_i: \varprojlim_{i\in I} M_i \longrightarrow \varprojlim_{i\in I} N_i$$
	mit $\phi_i^N \circ (\varprojlim_{i \in I} f_i) =  f_i \circ \phi_i^M $ für alle $i\in I$ (Wobei $\phi_i^N$ beziehungsweise $\phi_i^M $ die Strukturmorphismen zu $\varprojlim N_i $ beziehungsweise $\varprojlim M_i $ sind ):
	$$\begin{tikzcd}[row sep = large]
	\varprojlim_{i\in I}M_i \arrow{r}{\varprojlim f_i} \arrow{d}{\phi_i^M} & \varprojlim_{i\in I} N_i \arrow{d}{\phi_i^N} \\
	M_i \arrow[swap]{r}{f_i} & N_i
	\end{tikzcd}  \qquad
	\text{Explizit:} \ (\varprojlim f_i)((m_i)_{i \in I }) = (f_i(m_i))_{i \in I}$$
\end{bem}
\begin{proof}
		Wende die Universelle Eigenschaft von $\varprojlim N_i$ an:
	$$\begin{tikzcd}[column sep = large]
	& \varprojlim M_i \arrow[dashed, red]{d}{\varprojlim f_i} \arrow[bend right]{ddl}{f_i \circ \phi_i^M } \arrow[bend left, swap]{ddr}{f_j \circ \phi_j^M } &\\
	& \varprojlim N_i\arrow{dl}{\phi_i^N} \arrow[swap, hook]{dr}{\phi_j^N}& \\
	N_i & & N_j \arrow{ll}{\phi_{ij}^N} 
	\end{tikzcd}$$
	beachte: $\phi_{ij}^N \circ f_j \circ \phi_j^M = f_i \circ \phi_{ij}^M \circ \phi_j^M = f_i \circ \phi_i^M $ 
\end{proof}
\begin{fo} \label{16.24}
	$\varprojlim_{i\in I}-:I$-Pro-$A$-Mod$\to A$-Mod ist ein additiver Funktor
\end{fo}
\begin{proof}
Die Funktorialität und Additivität von $\varprojlim_{i\in I}-$ ergeben sich aus der Charakterisierung von $\varprojlim f_i$ in \ref{16.23}.
\end{proof}
\begin{sa} \label{16.25}
	Der Funktor $\varprojlim_{i\in I}-:I$-Pro-$A$-Mod$\to A$-Mod ist linksexakt.
\end{sa}
\begin{proof}
	 $\varprojlim_{i\in I}-$ ist rechtsadjungiert zum Funktor $I$-const: $A$-Mod $\to I$-Pro-$A$-Mod, $\quad M \mapsto (M, id_M)_I $ 
	 $$ Hom_{I\text{-Pro-}A\text{-Mod}}(I\text{-const(N)}, (M_i,\phi_{ij})_I) \cong Hom_{A\text{-Mod}}(N, \varprojlim_{i \in I} M_i ) \quad \text{(vergleiche UE)} $$
	 Damit ist $ \varprojlim_{i \in I }- $ ist linksexakt 
\end{proof}
\begin{anm}
	$\varprojlim_{i\in I}-$ ist im Allgemeinen nicht rechtsexakt.
\end{anm}
\begin{bsp} \label{16.26}
	Wir betrachten die exakte Sequenz projektiver Systeme von $Z$-Moduln über $ I = \N $: 
	$$\begin{tikzcd}  0 \arrow{r} & (\Z, \cdot p )_{\N} \arrow{r}{(\cdot p^n)_{n \in \N}} & (\Z, id_{\Z})_{\N} \arrow{r} & (\QR{\Z}{p^n \Z}, \text{Proj.Abb.}) \arrow{r} &  0 \end{tikzcd}$$
	
	$$\begin{tikzcd}
	n+1: & 0 \arrow{r} & \Z \arrow{r}{\cdot p^{n+1}} \arrow{d}{\cdot p}& \Z \arrow{r}{\text{proj.}} \arrow{d}{id_{\Z}} & \QR{\Z}{p^{n+1}\Z} \arrow{r} \arrow{d}{\text{proj.}} & 0 \\
	n:  &0 \arrow{r}& \Z \arrow{r}{\cdot p^n} & \Z \arrow{r}{\text{proj.}} & \QR{\Z}{p^n \Z} \arrow{r} & 0
	\end{tikzcd}$$
	Das projektive System $(\Z, \cdot p )_{\N}$ ist via 
	$$ \begin{tikzcd}
	n+1: & \Z \arrow{r}{\cdot p^{n+1}} \arrow{d}{\cdot p}& p^{n+1}\Z  \arrow[hook]{d} &  \\
	n:  & \Z \arrow{r}{\cdot p^n} & p^n \Z 
	\end{tikzcd}$$
	isomorph zum System $p\Z \supseteq p^2 \Z \supseteq \dots $ von $\Z$-Untermoduln von $\Z$, das heißt, der projektive Limes ist isomorph zu $\varprojlim_{i\in \N} p^n \Z = \bigcap_{n \in \N} p^n \Z = 0 $ \\
	Somit erhält man im projektiven Limes die exakte Sequenz 
	$$ \begin{tikzcd}
	0 \arrow{r} & 0 \arrow{r} &\Z \arrow{r}{f} & \Z_{p} \qquad  \text{mit} \  f: \Z \to \Z_{p}, \ x \mapsto (x+p^n \Z )_{n \in \N} \end{tikzcd}$$
	$f$ ist nicht surjektiv, denn es existiert kein $x \in \Z $ mit $x \equiv 1 +p + \dots + p^{n-1} (mod p^n) $ für alle $n \in \N$ ($p\neq 2$) \\
	(alternativ: $\Z_{p} $ überabzählbar).
\end{bsp}
\begin{df} \label{16.27}
	Sei $(M_i, \phi_{ij})_{\N}$ ein (bzgl "$\leq$") projektives System von $A$-Moduln. Das System erfüllt die \define{Mittag-Leffler-Bedingung (ML)} \defi Für jedes $i \in \N $ wird die Sequenz 
	$$ M_i = \phi_{ii}(M_i) \supseteq \phi_{i,i+1}(M_{i+1}) \supseteq \phi_{i,i+2}(M_{i+2}) \supseteq \dots $$ stationär. 
\end{df}
\begin{anm}
	Sind die Homomorphismen $\phi_{ij} $ alle surjektiv oder sind alle $M_i$ endlich, so ist die (ML) erfüllt. Das System $(\Z, \cdot p) $ von der linken Seite der Sequenz in Beispiel \ref{16.26} erfüllt die (ML) nicht: 
	$$ p\Z \supseteq p^2 \Z \supseteq \dots $$
\end{anm}
\begin{bem} \label{16.28}
	Sei 
	$$\begin{tikzcd}  0 \arrow{r} & (K_i, \phi_{ij}^K)_{\N} \arrow{r}{(f_i)_{i \in \N}} & (M_i, \phi_{ij}^M)_{\N} \arrow{r}{(g_i)_{i \in \N}} & (N_i, \phi_{ij}^N)_{\N} \arrow{r} &  0 \end{tikzcd}$$
	eine exakte Sequenz in $\N$-Pro-$A$-Mod und $(K_i, \phi_{ij}^K)_{\N} $ erfülle die (ML). Dann ist die Sequenz 
	$$ \begin{tikzcd}  0 \arrow{r} & \varprojlim_{i\in \N} K_i \arrow{r} & \varprojlim_{i\in \N} M_i  \arrow{r} & \varprojlim_{i\in \N} N_i \arrow{r} &  0 \end{tikzcd}$$
	exakt. 
\end{bem}
\begin{proof}
	Siehe Stacks-Poject (O2MY)
\end{proof}
\begin{bem} \label{16.29}
	Sei $J \subseteq I $ kofinal, $(M_i, \phi_{ij})_I $ ein projektives System von $A$-Moduln. Dann ist der natürliche Homomorphismus
	$$ \epsilon: \varprojlim_{i \in I } M_i \longrightarrow \varprojlim_{i \in J} M_i, \qquad (m_i)_{i \in I } \longmapsto (m_j)_{j \in J } $$ 
	ein Isomorphismus. 
\end{bem}
\begin{proof}
	\begin{enumerate} 
		\item[] Injektivität von $\epsilon$: \\ Sei $(m_i)_{i \in I} \in \varprojlim M_i $ mit $ m_j = 0 $ für alle $j \in J $. Sei $i \in I$. Da $J \subseteq I $ kofinal, folgt, dass ein $j \in J $ existiert mit $ i \leq j \Ra m_i = \phi_{ij}(m_j)= \phi_{ij}(0) = 0$.
		\item[] Surjektivität von $\epsilon$: \\ Sei $(m_i)_{i \in J} \in \varprojlim_{i\in J} M_i$. Sei $i \in I $. Dann existiert ein $ j \in J $ mit $j \geq i$. Setze $m_i:= \phi_{ij}(m_j) \in M_i$. Dies ist unabhängig von der Wahl von $j$, denn: Ist $j' \in J $ mit $ j' \geq i $ , dann existiert (wegen I gerichtet) ein $k \in I $ mit $k \geq j',j $ und somit exisitert ein $l \in J $ mit $l \geq k \geq j,j' $ \\
		Dann ist $ \phi_{ij}(m_j) = \phi_{ij}(\phi_{jl}(m_l)) = \phi_{il}(m_l)=\phi_{ij}'(\phi_{j'l}(m_l)) = \phi_{i'j}(m_j') $ Es ist $(m_i)_{i \in I} \in \varprojlim_{i \in I } M_i$, denn: Sind $i,k \in I $ mit $k\geq i $, dann existiert ein $j \in J $ mit $j \geq k \geq i $ \\
		$\Ra \phi_{ik}(m_k) = \phi_{ik}(\phi_{kj}(m_j)) = \phi_{ij}(m_j) = m_i$. Somit $(m_i)_{i \in I } \in \im \epsilon $.
	\end{enumerate}
\end{proof}
\newpage 
\subsection{Diskrete Beweruntgsringe}
\begin{df} \label{17.1}
	Sei $K$ ein Körper, $v: K \to \Z \cup \{\infty\}$. $v$ heißt \define{diskrete Bewertung} auf $K$, wenn gilt: 
	\begin{enumerate}
		\item[(DB1)] $ v(x) = \infty \Leftrightarrow x = 0 $
		\item[(DB2)] $v(xy) = v(x) + v(y) $
		\item[(DB3)] $v(x+y) \geq min\{v(x),v(y)\}$
	\end{enumerate}
	für alle $x,y \in K $.In diesem Fall heißt $v$ \define{triviale Bewertung} \defi \ $v(K) = \{0,\infty\} $ \\
	\define{normierte Bewertung} \defi \ $v$ surjektiv \\
	Letzteres ist für $v(x^n) = nv(x) $ äquivalent dazu, dass ein $x \in K $ mit $v(x) = 1 $ existiert. 
\end{df}
\begin{anm}
	$v(K^{*})$ ist eine Untergruppe von $\Z$, denn $ v \big|_{K^{*}} : K^{*} \to \Z $ ist ein Gruppenhomomorphismus. Somit ist $v(K^{*}) = m\Z $ für ein $ m \in \N_0$. Es gilt dann: 
	\begin{itemize}
		\item $v$ trivial $\Leftrightarrow m = 0$
		\item Ist $v$ nichttrivial, so ist durch $v': K \to \Z \cup \{\infty\}, \quad x \mapsto \begin{cases}
		\frac{1}{m}v(x), &\text{falls} \ x \neq 0 \\ \infty, & \text{falls} \ x = 0 
		\end{cases} $
		eine normierte diskrete Bewertung gegeben.
	\end{itemize}
\end{anm}
\begin{bsp} \label{17.2}
	Sei $A$ ein faktorieller Ring, $p$ ein Primelement in $A$. Jedes  $x \in \Quot(A), x \neq 0 $, lässt sich eindeutig schreiben als
	$$ x = p^r\frac{a}{b}, \qquad \text{mit} \ p \nmid a, p\nmid b, r \in \Z $$
	Setze $v_p(x) := r, v_p(0) := \infty$, dann ist $v_p$ eine normierte diskrete Bewertung auf $\Quot(A)$, denn: 
	\begin{itemize}
		\item (DB1),(DB2) klar.
		\item (DB3): Seien $x,y \in K $. Falls $x=0$ oer $y=0 $, dann ist (DB3) klar. Falls $x,y \neq 0 $, dann $ x = p^r \frac{a}{b}, y= p^s\frac{c}{d}, \text{mit } r,s \in \Z, p\nmid a, p\nmid b, p\nmid c, p\nmid d, \OE \ r \geq s $. \\
		$\Ra x+y = p^r \frac{a}{b} + p^s\frac{c}{d} = p^s(p^{r-s}\frac{a}{b}+\frac{c}{d}) = p^s(\frac{p^{r-s}ad+bc}{bd})$. \\
		$\Ra v_p(x+y) = \underbrace{v_p(p^s)}_{=s} +\underbrace{v_p(\frac{p^{r-s}ad+bc}{bd})}_{\geq 0} \geq s$
		\item $v_p$ normiert, da $v_p(p) = 1$.
	\end{itemize}
\end{bsp}
\begin{sa} \label{17.3}
	Sei $K$ ein Körper, $v$ eine diskrete Bewertung auf $K$. Dann gilt: 
	\begin{enumerate} [label= \alph*)]
		\item $\br := \{x \in K| \ v(x) \geq 0\}$ ist ein nullteilerfreier Ring mit $\Quot(\br) = K $
		\item $\br^{*} = \{ x \in K | \ v(x) =0 \} $
		\item $\br$ ist ein lokaler Ring mit maximalen Ideal $$ \m_v := \{x \in K | \ v(x) > 0\} $$ 
		\item $\br$ ist ein Hauptidealring
		\item $\br$ ist ein Körper $\Leftrightarrow v$ ist trivial 
		\item Ist $v$ normiert,dann gilt: $$ p \in \br \ \text{ist Primelemt in } \ \br \quad \Leftrightarrow \quad v(p) =1 $$
		Die Primelemente von $ \br $ sind alle zueinander assoziiert und jedes Primelemt erzeugt $\m_v$.
	\end{enumerate}
\end{sa}
\begin{proof}
	\begin{enumerate} [label= \alph*)]
		\item Seien $x,y \in \br \Ra v(x) \geq 0, v(y) \geq 0$. Damit folgt: $v(xy) = v(x) +v(y) \geq 0, v(x+y) \geq min\{v(x),v(y)\} \geq 0$.\\
		 Es ist $v(1)=v(1 \cdot 1) = v(1) +v(1)$, das heißt $v(1)=0 $, also $1 \in \br$. $0 \in \br$ klar. \\
		Außerdem: $0 = v(1) =v((-1)(-1)) = v(-1) +v(-1)$, das heißt $v(-1) = 0 $. \\
		$\Ra v(-x) = v((-1)x) = v(-1)+v(x) =v(x) \geq 0 $ Womit folgt, dass $\br$ ein Ring ist und  wegen $\br \subseteq K $ ist $\br $ nullteilerfrei. \\
		Sei $x \in K, x \neq 0 $. Wegen $0 = v(1)= v(xx^{-1}) = v(x) + v(x^{-1})$ ist $v(x) \geq 0 $ oder $v(x^{-1}) \geq 0 $.\\
		$\Ra x \in \br $ oder $ x^{-1} \in \br \Ra x \in \Quot(\br)$. Somit folgt $K \subseteq \Quot(\br) \subseteq K $. 
		\item Sei $x \in K, x \neq 0 $. \\ 
		Dann ist $ x \in \br^{*} \Leftrightarrow v(x) \geq 0$ und $ v(x^{-1}) \geq 0$.
		Wegen $0 = v(1) = v(xx^{-1})$ ist $v(x^{-1}) = - v(x) $ \\
		Das heißt $x \in \br^{*} \Leftrightarrow v(x) = 0 $.
		\item $\m_v = \br \backslash \br^{*} $ ist ein Ideal in $\br $,denn: 
		\begin{itemize}
			\item $0 \in \m_v $ klar 
			\item $x,y \in \m_v \Ra v(x) > 0, v(y) > 0 \Ra v(x+y) \geq \text{min}{v(x),v(y)} > 0$
			\item $ x \in \m_v, y\in \br \Ra v(x) >0, v(y) \geq 0 \Ra v(yx) = \underbrace{v(y)}_{\geq 0} + \underbrace{v(x)}_{ > 0} >0 $
		\end{itemize}
		$\Ra \m_v$  ist einziges maximales Ideal in $\br$.
		\item $\br$ ist nullteilerfrei nach (a). Sei $\a \in \br $ Ideal, $\a \neq (0), \Ra v(\a) \subseteq \N_0 \cup \{\infty\}$, insbesondere hat $v(\a) $ ein minimales Element. Sei $a \in \a$ mit $ v(a) $ minimal (insbesondere ist $a\neq 0 $). \\
		Behauptung: $ \a = (a) $, denn: 
		\begin{enumerate}
			\item[$"\supseteq"$] klar 
			\item[$"\subseteq"$] Sei $b \in \a \Ra b = a \frac{b}{a} $ in $K$, es ist $v(\frac{b}{a}) = v(ba^{-1}) = v(b) -v(a) \geq 0$, wegen $v(a) $ minimal. \\
			$ \Ra \frac{b}{a} \in \br \Ra b \in (a)$.
		\end{enumerate}
		\item $\br$ Körper $\Leftrightarrow \br \backslash \br^{*} = \{0\} \Leftrightarrow \{x \in K | \ v(x) > 0 \} = \{0\} \Leftrightarrow v(K^{*}) = \{0\} \Leftrightarrow v$ ist trivial.
		\item Sei $p \in \br$. \\
		Ist $v(p) = 1$, dann ist $p$ nach Beweis in (d) ein Erzeuger von $\m_v$ und somit ein Primelement. \\
		Ist $v(p) = n \geq 1 $, dann ist $p$ kein Primelement in $\br$. \\
		denn: Nach Voraussetzung ist $v$ normiert, das heißt es existiert ein $q \in K $ mit $v(q) =1$.$q$ ist dann dein Erzeuger von $\m_v$ und somit ein Primelement in $\br$. Es ist $v(pq^{-n}) = v(p) -nv(q) = n -n = 0$, das heißt $pq^{-n} \in \br^{*} $\\
		$ \Ra p $ ist assoziiert zu $q^n$ und da $\br$ faktoriell folgt, dass p kein Primelement ist.
	\end{enumerate}
\end{proof}
\begin{df} \label{17.4}
	$A$ heißt \define{diskreter Bewertungsring} (DBR) \defi $A$ ist ein lokaler Hauptidealring, der kein Körper ist. 
\end{df}
\begin{sa} \label{17.5}
	Sei $A$ ein diskreter Bewertungsring, $p$ ein Primelement von A. Dann gilt: 
	\begin{enumerate} [label= \alph*)]
		\item Jedes Element $x \in \Quot(A), x\neq 0$, lässt sich eindeutig darstellen als 
		$$ x = up^n $$ 
		mit $u \in A^{*}, n \in \Z$.Hierbei ist $n$ unabhängig von der Wahl von $p$. 
		\item Die Abbildung 
		$$ v = v_A: \Quot(A) \longrightarrow  \Z \cup \{\infty\}, \quad x \mapsto \begin{cases}
		n, & \text{falls} \ x = up^n, u \in A^{*} \\ \infty, & \text{falls} x = 0 
		\end{cases} $$
		ist eine normierte diskrete Bewertung auf $\Quot(A) $ mit $\br = A $. 
	\end{enumerate}
\end{sa}
\begin{proof}
	\begin{enumerate} [label= \alph*)]
		\item Da $A$ ein lokaler  Hauptidealring ist, erzeugt $p$ das maximale Ideal $\m$ von $A$ (vergleiche Algebra1, Satz 6.3), daraus folgt, dass sich jedes $x \in A, x \neq 0$ als $x = up^n $ mit $ u \in A^{*}, n \in \N_0 $ schreiben lässt. (beachte: Jedes Primelement aus $A$ ist assoziiert zu $p$) \\
		Es folgt auch, dass sich jedes $x \in \Quot(A), x \neq 0 $ als $ x = up^n$ mit $ u \in A^{*}, n \in \Z$ schreiben lässt. Dies ist eindeutig, denn: $ up^n = vp^m $ mit $ u,v \in A^{*}, n,m \in \Z $ und $ \OE \ n \geq m $. \\ 
		$\Ra uv^{-1}p^{n-m} = 1 \Ra n=m \Ra u=v$. \\
		Die Darstellung ist außerdem unabhängig von der Wahl von $p$, da jedes Primelement von $A$ assoziiert zu $p$ ist.
		\item $v_A $ ist normierte diskrete Bewertung auf $\Quot(A) $ nach Beispiel \ref{17.2}. Es ist $x \in \br \Leftrightarrow x = up^n $ mit $ u\in A^{*}, n \geq 0 \Leftrightarrow x \in A $.
	\end{enumerate}
\end{proof}
\begin{fo} \label{17.6}
	Sei $K$ ein Körper. Dann sind die Abbildungen 
	$$\begin{tikzcd}[row sep = tiny]
	\{\text{normierte diskrete BR} \ v \ \text{auf} \ K \} \arrow[yshift = 0.5ex]{r} & \{ \text{Unterringe} \ A \ \text{von} \ K | \ A \ \text{ist DBR mit } \ \Quot(A) = K \} \arrow[yshift = -0.5ex]{l} \\
	v \arrow[mapsto]{r}& \br \\
	v_A &  A \arrow[mapsto]{l} 
	\end{tikzcd}$$
	bijjektiv und invers zueinander.
\end{fo}
\begin{proof}
	Die Abbildungen sind wohldefiniert nach \ref{17.3} und \ref{17.5}. \\
	Sei $v$ eine normierte diskrete Bewertung auf $K \Ra v_{\br} = v $, denn: Sei $x \in K, x \neq 0$, dann folgt, dass ein Primelement $p$ in $\br$ und $u \in \br^{*}$ existieren mit $ x = up^n$ (folgt, da $\br$ ein DBR mit $\Quot(\br) = K$ ) \\
	$\Ra v(x) = \underbrace{v(u) }_{=0} +\underbrace{nv(p)}_{=1 \ \text{nach} \ \ref{17.3}(e)} = n = v_{\br}(x)$. \\
	Sei $A$ ein Unterring von $K$ mit $\Quot(A) = K$, welcher ein DBR ist \\
	$\Ra \mathcal{O}_{v_A} =A $ nach \ref{17.5}(b).
\end{proof}
\begin{bem} \label{17.7}
	Sei $A$ ein lokaler, noetherscher, nullteilerfreier Ring mit $dim(A) =1 $, $\m$ das maximale Ideal von $A$, $\a \subseteq A $ Ideal, $\A \neq 0 $. Dann exisitiert ein $n \in \N$ mit $\m^n \subseteq \a $. 
\end{bem}
\begin{proof}
	Die Behauptung ist klar für $\a = A$, im Folgenden sei $\a \neq A $. 
	\begin{enumerate}
		\item Sei $x \in \m $. Behauptung: Es existiert ein $n \in \N $ mit $x^n \in \a $. \\
		Annahme: Für alle $n \in \N $ ist $x^n \notin \a $. Setze $\bar{A}:= \QR{A}{\a}, S:= \{\bar{1}, \bar{x}, \bar{x}^2, \dots \} \subseteq \bar{A}$. \\
		In $S^{-1}\bar{A} $ ist $ \frac{\bar{1}}{\bar{1}} \neq \frac{\bar{0}}{\bar{1}}$, denn sonst existiert ein $n \in \N_0 $ mit $ \bar{x}^n = \bar{0}$, das heißt mit $x^n \in \a $ (Widerspruch) \\
		Insbesondere ist $S^{-1}\bar{A} \neq 0 \Ra$ in $S^{-1}\bar{A} $ existiert ein Primideal $\q$. Ist $\tau: \bar{A} \to S^{-1}\bar{A} $ die kanonische Abbildung, dann ist $\p := \tau^{-1}(\q) $ ein Primideal in $\bar{A}$. Es ist $\bar{x} \notin \p $, andernfalls wäre $ \underbrace{\frac{\bar{x}}{\bar{1}}}_{\text{Einheit in } \ S^{-1}\bar{A}}\in \q $.\\
		Ist $\pi: A \to \bar{A} $ die kanonische Projektion, dann ist $x \notin \phi^{-1}(\p) =: \tilde{\p}$, und $\tilde{\p} $ ist ein Primideal in $A$ mit $\tilde{\p } \supseteq \a $. Nach Voraussetzung sind $(0), \m $ die einzigen Primideale in $A$ (denn: $A$ ist eindimensonaler, lokaler, nullteilerfreier Ring). Aus $\tilde{\p} \supseteq \a $ folgt $ \tilde{\p} =\m$ und damit $ x \notin \m $ (Widerspruch)
		\item Da $A$ noethersch, exisiteren $x_1,\dots,x_r \in \m $ mit $ m = \sum_{i = 1}^{r} Ax_i$. Wegen 1. exisiteren $n_i \in\N, i=1,...,r $ mit $x_i,n_i \in \a $. Setze $n:= \sum_{i = 1}^{r} n_i $, dann wird $\m^n $ erzeugt von Elementen der Form $ x_1^{e_1} \cdot ... \cdot x_r^{e_r} $ mit $\sum_{i = 1}^{r} e_i =n = \sum_{i = 1}^{r} n_i$, das heißt es existieren stets ein $i \in\{1,..,r\} $ mit $ e_i \geq n_i $, insbesondere $x_1^{e_1} \cdot ... \cdot  x_r^{e_r} \in \a  \Ra \m^n \subseteq \a $. 
	\end{enumerate}
\end{proof}
\begin{sa} \label{17.8}
	Sei $A$ ein noetherscher, lokaler, nullteilerfreier Ring mit maximalem Ideal $\m$. Dann sind äquivalent: 
	\begin{enumerate} [label= \roman*)]
		\item $A$ ist ein diskreter Bewertungsring
		\item $dim(A) =1 $ und $A$ ist normal 
		\item $\m$ ist ein Hauptideal $\neq (0)$
		\item $A$ ist faktoriell und besitzt bis auf Assoziiertheit genau ein Primelement.
	\end{enumerate}
\end{sa}
\begin{proof}
	\begin{enumerate}
		\item[] (i)$\Ra$ (ii) Sei $A$ ein diskreter Bewertungsring, dann folgt mit \ref{15.21}(b), dass $dim(A)=1$. $A$ ist normal nach \ref{15.11}, da $A$ faktoriell als Hauptidealring.
		\item[](ii) $\Ra $ (iii) Da $A$ kein Körper ist, wegen $dim(A)=1 $ folgt: $\m \neq 0$. $\m $ ist ein Hauptideal, denn: \\
		Sei $a \in \m $ mit $a \neq 0$. Nach \ref{17.7} existiert ein $n \in \N $ mit $\m^n \subseteq (a) $. Es ist $A=\m^0 \not\subseteq (a) $, sondt$ a\in A^{*}$ (Widerspruch). \\
		Wähle ein $n \in \N $ minimal mit $\m^n \subseteq (a) $ (insbesondere $\m^{n-1} \not \subseteq (a)$ ). Sei $b \in \m^{n-1} \backslash (a)$, setze $x := \frac{a}{b} \in \Quot(A) $. Weil $b \notin (a) $ ist $x^{-1} = \frac{b}{a} \notin A$, da $A$ normal, folgt, dass $x^{-1} $ nicht ganz über $A$ ist. \\
		Wegen $\m^n \subseteq (a), b \in \m^{n-1} $ ist $x^{-1}\m = \frac{b}{a}\m \subseteq A$ (da $b\m \subseteq \m^n$ ), $x^{-1}\m $ ist ein Ideal in A. Falls $x^{-1}\m = A$, dann ist $\m = (x) $, fertig. \\
		Annahme: $x^{-1}\m \neq A$. Dann ist $x^{-1}\m \subseteq \m \Ra \m$ ist $ A[x^{-1}]$-Modul. Da A noethersch, ist $\m $ endlich erzeugt als $A$-Modul. Es ist $\ann_{A[x^{-1}]}\m = 0$,  denn $A[x^{-1}],\m \subseteq \Quot(A)$. Somit folgt aus der Charakterisierung (iv) in Satz \ref{15.2}, dass $x^{-1} $ ganz über $A$. Im Widerspruch zu $A$ normal und $x^{-1} \notin A $. 
		\item[] (iii) $\Ra $ (iv) Sei $\m = (p) , p \in A, p \neq 0$, sei $x\in A, x\neq 0$.
		\begin{enumerate} 
			\item[1.] Behauptung: Es existiert ein $n \in \N $ mit $p^n \nmid x. $ \\
			Annahme: Für alle $ n \in \N $ existiert ein $a_n \in A, a_n \neq 04 mit x = a_np^n$\\
			$\Ra a_{n+1}p^{n+1} = x =a_np^n \Ra a_{n+1}p^{n+1} -a_np^n = 0 \Ra p^n(a_{n+1}p-a_n) = 0 $,da $A$ nullteilerfrei, folgt$ a_{n+1p}-a_n = 0 \Ra a_n = a_{n+1} p \Ra (a_n) \subseteq (a_{n+1}) $ für alle $n \in \N$.\\
			Da $A$ noethersch, existiert ein $n\in \N $ mit $(a_n) = (a_{n+1}) \Ra$ es existiert ein$ b \in A $ mit $ a_{n+1} = ba_n \Ra a_{n+1} =ba_n = ba_{n+1}p \Ra a_{n+1}(1-bp) = 0 \Ra bp = 1 \Ra p \in A^{*} $(Widerspruch)
			\item[2.] Wegen 1. existiert ein $n \in \N_0 $ mit $ p^n | x$, aber $p^{n+1} \nmid x$. Daraus folgt: es existiert ein $u \in A $ mit $ x = up^n, p \nmid u$. Insbesondere ist $u \notin \m $, also $u \in A^{*}$. Diese Darstellung von $x$ ist eindeutig, denn: Ist $x = vp^m$ mit $v \in A^{*}, m \in \N_0$, dann ist $up^n=vp^m$ und für $\OE n \geq M $ folgt: $uv^{-1}p^{n-m} =1 \Ra n =m \Ra u =v$. Ist $q$ ein Primelement von $A$, dann exisitert ein $u \in A^{*}, n\in \N_0$ mit $q =up^n$, da $q$ ein Primelement, folgt $q|p$, sowie $p|q \Ra p$ ist assoziiert zu $q$.
		\end{enumerate}
		\item[](iv)$\Ra$ (i) Da $A$ ein Primelement, ist $A$ kein Körper. Sei $p$ ein Primelement von $A$. Nach Beispiel \ref{17.2} erhalten wir die normierte diskrete Bewertung $v_p$ auf $\Quot(A), A = \mathcal{O}_{v_p} $ ist ein diskreter Bewertungsring nach \ref{17.6}
	\end{enumerate}
\end{proof}
