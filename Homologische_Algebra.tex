\newpage
\section{Homologische Algebra}
\begin{center}
	In diesem Kapitel sei R stets ein Ring
\end{center}
\setcounter{subsection}{3}
\subsection{Kategorien}
\begin{df}
	Eine Kategorie $ \mathcal{C} $ besteht aus 
	\begin{itemize}
		\item einer Klasse $\ObC$ von "'Objekten"'
			einer Menge $Mor_{\mathcal{C}}(A,B) $ von "'Morphismen"' für alle $A,B \in \ObC$
		\item einer Verknüpfung $\circ : \Mor_{\mathcal{C}}(B,C) \times \Mor_{\mathcal{C}}(A,B) \to Mor_{\mathcal{C}}(A,C) $ für alle $A,B,C \in \ObC$
	\end{itemize}
	wobei folgende Axiome gelten:
	\begin{enumerate}
		\item[$(K1)$] $Mor_{\mathcal{C}}(A,B) \cap Mor_{\mathcal{C}}(A',B') = \emptyset$, falls $A \neq A'$ oder $B \neq B'$
		\item[$(K2)$] Für alle $A,B,C,D \in \ObC, f \in Mor_{\mathcal{C}}(A,B), g \in Mor_{\mathcal{C}}(B,C), h \in Mor_{\mathcal{C}}(C,D)$ gilt:
		$$ h \circ (g \circ f) = (h \circ g) \circ f \qquad (\text{Assoziativität})$$
		\item[$(K3)$] für jedes $ A \in \ObC$ existiert ein Morphismus $id_A \in Mor_{\mathcal{C}}(A,A)$, sodass für alle $B \in \ObC, f \in Mor_{\mathcal{C}}(A,B), g \in Mor_{\mathcal{C}}(B,A)$ gilt: $$f \circ id_A = f, \ id_A \circ g = g $$.
	\end{enumerate}
\end{df}
\begin{anm}
	\begin{itemize}
		\item Man sagt "Klasse"' statt Menge, um Paradoxien, wie " die Menge aller Mengen" zu vermeiden.
		\item Trotzdem schreiben wir $A \in \ObC$ um zu sagen dass $A$ zu $\ObC$ gehört (und werden $\ObC$ im Folgenden wie eine Menge behandeln).
		\item In den folgenden Abschnitten werden wir mengentheoretische Probleme ignorieren und häufig von Mengen sprechen auch wenn es sich nur um Klassen handelt.
		\item Für $f \in \Mor_{\mathcal{C}}(A,B)$ schreiben wir auch $f: A \to B $. $A$ heißt "'Quelle"' und $B$ heißt "'Ziel"' von $f$; wegen $(K1)$ sind diese eindeutig bestimmt.
		\item für $A \in \ObC$ ist $id_A$ eindeutig bestimmt (analoges Argument wie bei Monoiden: $id_A = id_A^{'} \circ id_A = id_A^{'}$)
	\end{itemize}
\end{anm}
\begin{bsp}
	\begin{itemize}
		\item Mengen: Kategorie der Mengen mit Abbildungen von Mengen als Morphismen
		\item Ringe: Kategorie der Ringe mit Ringhomomorphismen als Morphismen
		\item $R$-Mod: Kategorie der $R$-(Links)-Moduln mit $R$-Modulhomomorphismen als Morphismen
		\item Top: Kategorie der topologischen Räume mit stetigen Abbildungen als Morphismen
		\item $\ObC=\{\ast\}, Mor_{\mathcal{C}}(\ast,\ast) := M $, wobei $M$ Monoid, $\circ = $ Verknüpfung in $M$.
	\end{itemize}
\end{bsp}
\begin{df}
	Sei $\mathcal{C}$  eine Kategorie. Die zu $\mathcal{C}$ "duale Kategorie'" $\mathcal{C}^{op}$"' ist die Kategorie mit: 
	\begin{itemize}
		\item $\text{Ob}(\mathcal{C}^{op}) = \ObC$, $Mor_{\mathcal{C}^{op}}(A,B) := Mor_{\mathcal{C}}(B,A)$ für $A,B \in \text{Ob}(\mathcal{C}^{op}) = \ObC$
		\item $\circ_{op}: Mor_{\mathcal{C}^{op}}(A,B) \times Mor_{\mathcal{C}^{op}}(B,C) \to Mor_{\mathcal{C}^{op}}(A,C) $ mit $(f,g) \mapsto f \circ g $ für $A,B,C \in \ObC$
	\end{itemize}
\end{df}
\begin{anm}
	\begin{itemize}
		\item  Anschaulich: Übergang von $\mathcal{C}$ zu  $\mathcal{C}^{op} \ \is $ Pfeile umdrehen
		\item $(\mathcal{C}^{op})^{op} = \mathcal{C}$
	\end{itemize}
\end{anm}
\begin{df}
	Seien $\mathcal{C}, \mathcal{D} $ Kategorien. Ein "'(kovarianter) Funktor"' $F: \mathcal{C} \to \mathcal{D} $ besteht aus einer Abbildung $$ \ObC \to \text{Ob}(\mathcal{D}), \quad A \mapsto FA$$
	und Abbildungen: $$ Mor_{\mathcal{C}}(A,B) \to Mor_{\mathcal{D}}(FA,FB), \quad f \mapsto F(f) $$ für alle $A,B \in \ObC$, sodass gilt:
	\begin{enumerate}
		\item[(F1)] $F(g \circ f) = F(g) \circ F(f) $ für alle $ f \in Mor_{\mathcal{C}}(A,B), g\in Mor_{\mathcal{C}}(B,C), \ A,B,C \in \ObC$
		\item[(F2)] $F(id_A) =id_{FA} $ für alle $A \in \ObC.$
	\end{enumerate}
\end{df}
\begin{bsp}
	\begin{enumerate} [label= \alph*)]
		\item Vergiss-Funktoren, zum Beispiel: $R$-Mod $\to $ Mengen, $R$-Mod $\to \Z$-Mod, ...
		\item Sei $\mathcal{C}$ eine Kategorie $\Ra$ Jedes Objekt $X \in \ObC$ induziert einen Funktor $$Mor_{\mathcal{C}}(X,-): \mathcal{C} \to \text{Mengen}, \quad A \mapsto Mor_{\mathcal{C}}(X,A) $$
		Für $f \in Mor_{\mathcal{C}}(A,B) $ ist hierbei $f_{\ast}^{X} := Mor_{\mathcal{C}}(X,-)(f)$ gegeben durch $$ f_{\ast}^{X}: Mor_{\mathcal{C}}(X,A) \to Mor_{\mathcal{C}}(X,B), \quad g \mapsto f \circ g\qquad \begin{tikzcd}
		X \arrow{r}{g} \arrow[swap]{rd}{f_*^X(g)} & A \arrow{d}{f} \\ & B
		\end{tikzcd} $$
		\item Sei $M \in R$-Mod $\Ra Hom_R(M,-): R$-Mod $\to \Z$-Mod, $N \mapsto Hom_R(M,N) $ ist ein Funktor.
	\end{enumerate}
\end{bsp}
\begin{df}
	Seien $\mathcal{C}, \mathcal{D}$ Kategorien. Ein "'(kontavarianter) Funktor"' $F$ von $\mathcal{C}$ nach $\mathcal{D}$ ist ein Funktor $F: \mathcal{C}^{op} \to \mathcal{D} $, das heißt besteht aus einer Abbildung $$ \ObC \to \text{Ob}(\mathcal{D}), \quad A \mapsto FA$$
	und Abbildungen: $$ Mor_{\mathcal{C}}(A,B) \to Mor_{\mathcal{D}}(FB,FA), \quad f \mapsto F(f) $$ für alle $A,B \in \ObC$, sodass gilt:
	\begin{enumerate}
		\item[(F1')] $F(g \circ f) = F(f) \circ F(g) $ für alle $ f \in Mor_{\mathcal{C}}(A,B), g\in Mor_{\mathcal{C}}(B,C), \ A,B,C \in Ob\mathcal{C}$
		\item[(F2')] $F(id_A) =id_{FA} $ für alle $A \in \ObC.$
	\end{enumerate}
\end{df}
\begin{bsp}
		\begin{enumerate} [label= \alph*)]
		\item Sei $\mathcal{C}$ eine Kategorie $\Ra$ Jedes Objekt $Y \in \ObC$ induziert einen kontravarianten Funktor $$Mor_{\mathcal{C}}(-,Y): \mathcal{C} \to \text{Mengen}, \quad A \mapsto Mor_{\mathcal{C}}(A,Y) $$
		Für $f \in Mor_{\mathcal{C}}(A,B) $ ist hierbei $f_{Y}^{\ast} := Mor_{\mathcal{C}}(-,Y)(f)$ gegeben durch $$ f_{Y}^{\ast}: Mor_{\mathcal{C}}(B,Y) \to Mor_{\mathcal{C}}(A,Y), \quad g \mapsto g \circ f\qquad \begin{tikzcd}
		A \arrow{r}{f_Y^*(g)} \arrow[swap]{d}{f} & Y \\B \arrow[swap]{ur}{g}
		\end{tikzcd}$$
		\item Sei $N \in R$-Mod $\Ra Hom_R(-,N): R$-Mod $\to \Z$-Mod, $M \mapsto Hom_R(M,N) $ ist ein kontavarianter Funktor.
	\end{enumerate}
\end{bsp}
\begin{anm}
	\begin{itemize}
		\item Sind $F: \mathcal{C} \to \mathcal{D}, G: \mathcal{D} \to \mathcal{E} $ Funktoren, so ist auf naheliegende Weise der Funktor $G \circ F : \mathcal{C} \to \mathcal{E}$
		\item Unter Funktoren werden kommutative Diagramme auf kommutative Diagramme abgebildet.
	\end{itemize}
\end{anm}
\begin{df}
	Seien $\mathcal{C},\mathcal{D} $ Kategorien. "'Das Produkt"' $\mathcal{C} \times \mathcal{D} $ ist diejenige Kategorie mit $ Ob(\mathcal{C} \times \mathcal{D}) = Ob(\mathcal{C}) \times \text{Ob}(\mathcal{D}) $ und $ Mor_{\mathcal{C} \times \mathcal{D}}((A_1,B_1),(A_2,B_2)) = Mor_{\mathcal{C}}(A_1,A_2) \times Mor_{\mathcal{D}}(B_1,B_2)$ und "komponentenweisen $\circ"$.
\end{df}
\begin{df}
	Seien $\mathcal{C},\mathcal{D}, \mathcal{E}$ Kategorien. Ein "Bifunktor" \ $F$ "von $\mathcal{C}$ kreuz $\mathcal{D} $ nach $\mathcal{E}$ "' ist ein Funktor $F: \mathcal{C} \times \mathcal{D} \to \mathcal{E}$
\end{df}
\begin{bsp}
	\begin{enumerate}  [label= \alph*)]
		\item $\bigoplus$: $R$-Mod $\times R$-Mod $\to R$-Mod, $(M,N) \to M \bigoplus N $ ist ein Bifunktor
		\item Sei $\mathcal{C} $ eine Kategorie $\Ra \mathcal{C}^{op} \times \mathcal{C} \to \text{Mengen}, (M,N) \mapsto  Mor_{\mathcal{C}}(M,N) $ ist ein Bifunktor.
	\end{enumerate}
\end{bsp}
\begin{df}
	Sei $\mathcal{C}$ eine Kategorie, $ A,B \in \ObC, f: A \to B $ $f$ heißt
	\begin{enumerate}
		\item[] "Monomorphismus" $\defi$ Für alle $C \in \ObC,\, g_1,g_2: C \to A $ gilt: $f \circ g_1 = f \circ g_2 \Ra g_1 = g_2$ $\Lra$ Für alle $C \in \ObC$ ist $f_{\ast}^{C}: Mor_{\mathcal{C}}(C,A) \to Mor_{\mathcal{C}}(C,B)$ injektiv.
		\item[] "'Epimorphismus" $\defi$ Für alle $C \in Ob\mathcal{C}, g_1,g_2: B \to C $ gilt: $g_1 \circ f = g_2 \circ f \Ra g_1 = g_2$ $\Lra$ Für alle $C \in Ob\mathcal{C} ist f_{C}^{\ast}: Mor_{\mathcal{C}}(B,C) \to Mor_{\mathcal{C}}(A,C)$ injektiv.
		\item[] "'Isomorphismus" $\defi$ Es existiert ein $g:b \to A $ mit $ f\circ g = id_B $ und $ g \circ f = id_A.$
	\end{enumerate}
\end{df}
\begin{anm}
	In der Situation von 4.11 gilt:
	\begin{itemize}
		\item $f$ Monomorphismus in $\mathcal{C}$ $\Lra$ $f$ Epimorphismus in $\mathcal{C}^{op}.$
		\item $f$ Isomorphismus in $\mathcal{C} \Lra  f $ ist Isomorphismus in $\mathcal{C}^{op}.$
		\item Ist $f$ ein Isomorphismus und $g:B \to A $ mit $ f \circ g = id_B $ und $g \circ f = id_A$, dann ist $g$ ein eindeutig bestimmt (und wird mit $f^{-1} $ bezeichnet). Denn: $g_1,g_2: B \to A$ mit dieser Eigenschaft $Ra g_1 = g_1 \circ id_B = g_1 \circ (f \circ g_2) =(g_1 \circ f) \circ g_2 = id_A \circ g_2 = g_2.$
		\item In Mengen ist $f$ Monomorphismus $\Lra f $ injektv, $f$ Epimorphismus $Lra f$ surjektiv, $f$ Isomorphismus $\Lra f $ bijektiv. Im Allgemeinen ist dies für Kategorien, in denen die Morphismen Abbildungen sind, jedoch falsch (vgl. Bsp. 4.13)
	\end{itemize}
\end{anm}
\begin{bem}
	Sei $\mathcal{C}$ eine Kategorie, $A,B \in \ObC, f:A \to B $ ein Isomorphismus. Dann ist $f$ ein Monomorphismus und Ein Epimorphismus.
	\begin{proof}
		Seien $ C \in \ObC , g_1,g_2:C \to A $ mit $ f \circ g_1 = f \circ g_2 \Ra f^{-1} \circ (f \circ g_1) = f^{-1} \circ (f \circ g_2) \Ra (f^{-1} \circ f) \circ g_1 = (f^{-1} \circ f) \circ g_2 \Ra g_1=g_2 \Ra f $ Monomorphimus. Analog wird gezeigt dass $f$ ein Epimorphimus.
		%hier fehlen die geschweiften klammern 
	\end{proof}
\end{bem}
\begin{anm}
	Die Umkehrung von 4.12 ist im Allgemeinen falsch, siehe nächstes Beispiel.
\end{anm}
\begin{bsp}
	\begin{enumerate} [label= \alph*)]
		\item  Sei $\mathcal{C} = Top  $ die Kategorie der Topologischen Räume mit stetigen Abbildungen. Wir betrachten $id: (\R, \text{diskrete Topologie}) \to (\R, \text{Standardtopologie}).$ Diese ist eine stetige Abbildung, ein Monomorphismus sowie ein Epimorphismus, jedoch kein Isomorphismus (Nicht hömöomorph, da kein stetiges Inverses)
		\item Sei $\mathcal{C} = Ringe, f:\Z \to \Q $ Inklusion. $f$ ist ein  Monomorphismus und ein Epimorphimus (Achtung, denn: Für $g_1,g_2: \Q \to R$ Ringhomomorphismus ist ein Ring $R$ mit $g_1 \circ f = g_2 \circ f, $ das heißt $ g_1\big|_{\Z} = g_2\big|_{\Z} $ folgt $ g_1 = g_2 $
		 \begin{minipage}[t]{0.7\textwidth}
		 	wegen der Universellen Eigenschaft von $\Q$ als
		 	Quotientenkörper von $\Z$), aber kein Isomorphismus. 
		Insbesondere ist ein Epimorphismus in $\mathcal{C} $ im obigen Sinne  ("kategorieller Epimorphismus") nicht dasselbe wie ein surjektiver Ringhomomorphismus.
		\end{minipage}
		\begin{minipage}[t]{0.3\textwidth} 
			$$\begin{tikzcd}
			\Z \arrow[hook]{r}{f} & \Q \arrow[xshift = 0.7ex]{d}{g_2} \arrow[xshift = -0.7ex, swap]{d}{g_1}\\ & R
			\end{tikzcd}$$
		\end{minipage}
	\end{enumerate}
\end{bsp}




