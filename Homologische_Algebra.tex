\newpage
\section{Homologische Algebra}
\begin{center}
	In diesem Kapitel sei R stets ein Ring
\end{center}
\setcounter{subsection}{3}
\subsection{Kategorien}
\begin{df}\label{2.4}
	Eine \define{Kategorie\index{Kategorie}} $ \mathcal{C} $ besteht aus 
	\begin{itemize}
		\item einer Klasse $\ObC$ von \define{Objekten\index{Objekt}}
			einer Menge $Mor_{\mathcal{C}}(A,B) $ von \define{Morphismen\index{Morphismus}} für alle $A,B \in \ObC$
		\item einer Verknüpfung $\circ : \Mor_{\mathcal{C}}(B,C) \times \Mor_{\mathcal{C}}(A,B) \to Mor_{\mathcal{C}}(A,C) $ für alle $A,B,C \in \ObC$
	\end{itemize}
	wobei folgende Axiome gelten:
	\begin{enumerate}
		\item[$(K1)$] $Mor_{\mathcal{C}}(A,B) \cap Mor_{\mathcal{C}}(A',B') = \emptyset$, falls $A \neq A'$ oder $B \neq B'$
		\item[$(K2)$] Für alle $A,B,C,D \in \ObC, f \in Mor_{\mathcal{C}}(A,B), g \in Mor_{\mathcal{C}}(B,C), h \in Mor_{\mathcal{C}}(C,D)$ gilt:
		\begin{enumerate}
			\item[] $h \circ (g \circ f) = (h \circ g) \circ f$ \hfill \emph{(Assoziativität)}
		\end{enumerate}
		\item[$(K3)$] für jedes $ A \in \ObC$ existiert ein Morphismus $id_A \in Mor_{\mathcal{C}}(A,A)$, sodass für alle $B \in \ObC, f \in Mor_{\mathcal{C}}(A,B), g \in Mor_{\mathcal{C}}(B,A)$ gilt: $$f \circ id_A = f, \ id_A \circ g = g $$.
	\end{enumerate}
\end{df}
\begin{anm}
	\begin{itemize}
		\item Man sagt \define{Klasse\index{Klasse}} statt Menge, um Paradoxa, wie "'die Menge aller Mengen"' zu vermeiden.
		\item Trotzdem schreiben wir $A \in \ObC$ um zu sagen dass $A$ zu $\ObC$ gehört (und werden $\ObC$ im Folgenden wie eine Menge behandeln).
		\item In den folgenden Abschnitten werden wir mengentheoretische Probleme ignorieren und häufig von Mengen sprechen auch wenn es sich nur um Klassen handelt.
		\item Für $f \in \Mor_{\mathcal{C}}(A,B)$ schreiben wir auch $f: A \to B $. $A$ heißt \emph{Quelle} und $B$ heißt \emph{Ziel} von $f$; wegen $(K1)$ sind diese eindeutig bestimmt.
		\item für $A \in \ObC$ ist $\id_A$ eindeutig bestimmt (analoges Argument wie bei Monoiden: $\id_A = \id_A^{'} \circ \id_A = \id_A^{'}$)
	\end{itemize}
\end{anm}
\begin{bsp}
	\begin{itemize}
		\item Mengen: Kategorie der Mengen mit Abbildungen von Mengen als Morphismen
		\item Ringe: Kategorie der Ringe mit Ringhomomorphismen als Morphismen
		\item $R$-Mod: Kategorie der $R$-(Links)-Moduln mit $R$-Modulhomomorphismen als Morphismen
		\item Top: Kategorie der topologischen Räume mit stetigen Abbildungen als Morphismen
		\item $\ObC=\{\ast\}, Mor_{\mathcal{C}}(\ast,\ast) := M $, wobei $M$ Monoid, $\circ = $ Verknüpfung in $M$.
	\end{itemize}
\end{bsp}
\begin{df}\label{4.3}
	Sei $\mathcal{C}$  eine Kategorie. Die zu $\mathcal{C}$ \define{duale Kategorie\index{Duale Kategorie}}  ($\mathcal{C}^{op}$) ist die Kategorie mit: 
	\begin{itemize}
		\item $\text{Ob}(\mathcal{C}^{op}) = \ObC$, $Mor_{\mathcal{C}^{op}}(A,B) := Mor_{\mathcal{C}}(B,A)$ für $A,B \in \text{Ob}(\mathcal{C}^{op}) = \ObC$
		\item $\circ_{op}: Mor_{\mathcal{C}^{op}}(A,B) \times Mor_{\mathcal{C}^{op}}(B,C) \to Mor_{\mathcal{C}^{op}}(A,C) $ mit $(f,g) \mapsto f \circ g $ für $A,B,C \in \ObC$
	\end{itemize}
\end{df}
\begin{anm}
	\begin{itemize}
		\item  Anschaulich: Übergang von $\mathcal{C}$ zu  $\mathcal{C}^{op} \ \is $ Pfeile umdrehen
		\item $(\mathcal{C}^{op})^{op} = \mathcal{C}$
	\end{itemize}
\end{anm}
\begin{df}\label{4.4}
	Seien $\mathcal{C}, \mathcal{D} $ Kategorien. Ein \define{(kovarianter) Funktor\index{kovarianter Funktor}} $F: \mathcal{C} \to \mathcal{D} $ besteht aus einer Abbildung $$ \ObC \to \text{Ob}(\mathcal{D}), \quad A \mapsto FA$$
	und Abbildungen: $$ Mor_{\mathcal{C}}(A,B) \to Mor_{\mathcal{D}}(FA,FB), \quad f \mapsto F(f) $$ für alle $A,B \in \ObC$, sodass gilt:
	\begin{enumerate}
		\item[(F1)] $F(g \circ f) = F(g) \circ F(f) $ für alle $ f \in Mor_{\mathcal{C}}(A,B), g\in Mor_{\mathcal{C}}(B,C), \ A,B,C \in \ObC$
		\item[(F2)] $F(id_A) =id_{FA} $ für alle $A \in \ObC.$
	\end{enumerate}
\end{df}
\begin{bsp}
	\begin{enumerate} [label= \alph*)]
		\item Vergiss-Funktoren, zum Beispiel: $R$-Mod $\to $ Mengen, $R$-Mod $\to \Z$-Mod, ...
		\item Sei $\mathcal{C}$ eine Kategorie $\Ra$ Jedes Objekt $X \in \ObC$ induziert einen Funktor $$Mor_{\mathcal{C}}(X,-): \mathcal{C} \to \text{Mengen}, \quad A \mapsto Mor_{\mathcal{C}}(X,A) $$
		Für $f \in Mor_{\mathcal{C}}(A,B) $ ist hierbei $f_{\ast}^{X} := Mor_{\mathcal{C}}(X,-)(f)$ gegeben durch $$ f_{\ast}^{X}: Mor_{\mathcal{C}}(X,A) \to Mor_{\mathcal{C}}(X,B), \quad g \mapsto f \circ g\qquad \begin{tikzcd}
		X \arrow{r}{g} \arrow[swap]{rd}{f_*^X(g)} & A \arrow{d}{f} \\ & B
		\end{tikzcd} $$
		\item Sei $M \in R$-Mod $\Ra Hom_R(M,-): R$-Mod $\to \Z$-Mod, $N \mapsto Hom_R(M,N) $ ist ein Funktor.
	\end{enumerate}
\end{bsp}
\begin{df}\label{4.6}
	Seien $\mathcal{C}, \mathcal{D}$ Kategorien. Ein \define{(kontavarianter) Funktor\index{kontravarianter Funktor}} $F$ von $\mathcal{C}$ nach $\mathcal{D}$ ist ein Funktor $F: \mathcal{C}^{op} \to \mathcal{D} $, das heißt besteht aus einer Abbildung $$ \ObC \to \text{Ob}(\mathcal{D}), \quad A \mapsto FA$$
	und Abbildungen: $$ Mor_{\mathcal{C}}(A,B) \to Mor_{\mathcal{D}}(FB,FA), \quad f \mapsto F(f) $$ für alle $A,B \in \ObC$, sodass gilt:
	\begin{enumerate}
		\item[(F1')] $F(g \circ f) = F(f) \circ F(g) $ für alle $ f \in Mor_{\mathcal{C}}(A,B), g\in Mor_{\mathcal{C}}(B,C), \ A,B,C \in \ObC$
		\item[(F2')] $F(id_A) =id_{FA} $ für alle $A \in \ObC.$
	\end{enumerate}
\end{df}
\begin{bsp}
		\begin{enumerate} [label= \alph*)]
		\item Sei $\mathcal{C}$ eine Kategorie $\Ra$ Jedes Objekt $Y \in \ObC$ induziert einen kontravarianten Funktor $$Mor_{\mathcal{C}}(-,Y): \mathcal{C} \to \text{Mengen}, \quad A \mapsto Mor_{\mathcal{C}}(A,Y) $$
		Für $f \in Mor_{\mathcal{C}}(A,B) $ ist hierbei $f_{Y}^{\ast} := Mor_{\mathcal{C}}(-,Y)(f)$ gegeben durch $$ f_{Y}^{\ast}: Mor_{\mathcal{C}}(B,Y) \to Mor_{\mathcal{C}}(A,Y), \quad g \mapsto g \circ f\qquad \begin{tikzcd}
		A \arrow{r}{f_Y^*(g)} \arrow[swap]{d}{f} & Y \\B \arrow[swap]{ur}{g}
		\end{tikzcd}$$
		\item Sei $N \in R$-Mod $\Ra Hom_R(-,N): R$-Mod $\to \Z$-Mod, $M \mapsto Hom_R(M,N) $ ist ein kontavarianter Funktor.
	\end{enumerate}
\end{bsp}
\begin{anm}
	\begin{itemize}
		\item Sind $F: \mathcal{C} \to \mathcal{D}, G: \mathcal{D} \to \mathcal{E} $ Funktoren, so ist auf naheliegende Weise der Funktor $G \circ F : \mathcal{C} \to \mathcal{E}$ definiert.
		\item Unter Funktoren werden kommutative Diagramme auf kommutative Diagramme abgebildet.
	\end{itemize}
\end{anm}
\begin{df}\label{4.8}
	Seien $\mathcal{C},\mathcal{D} $ Kategorien. Das \define{Produkt\index{Produkt von Kategorien}} \,$\mathcal{C} \times \mathcal{D} $ ist diejenige Kategorie mit $ Ob(\mathcal{C} \times \mathcal{D}) = Ob(\mathcal{C}) \times \text{Ob}(\mathcal{D}) $ und $ Mor_{\mathcal{C} \times \mathcal{D}}((A_1,B_1),(A_2,B_2)) = Mor_{\mathcal{C}}(A_1,A_2) \times Mor_{\mathcal{D}}(B_1,B_2)$ und "komponentenweisen $\circ"$.
\end{df}
\begin{df}\label{4.9}
	Seien $\mathcal{C},\mathcal{D}, \mathcal{E}$ Kategorien. Ein \define{Bifunktor} \ $F$ "'von $\mathcal{C}$ kreuz $\mathcal{D} $ nach $\mathcal{E}$ "' ist ein Funktor $F: \mathcal{C} \times \mathcal{D} \to \mathcal{E}$
\end{df}
\begin{bsp}
	\begin{enumerate}  [label= \alph*)]
		\item $\bigoplus$: $R$-Mod $\times R$-Mod $\to R$-Mod, $(M,N) \to M \bigoplus N $ ist ein Bifunktor
		\item Sei $\mathcal{C} $ eine Kategorie $\Ra \mathcal{C}^{op} \times \mathcal{C} \to \text{Mengen}, (M,N) \mapsto  Mor_{\mathcal{C}}(M,N) $ ist ein Bifunktor.
	\end{enumerate}
\end{bsp}
\begin{df}\label{4.11}
	Sei $\mathcal{C}$ eine Kategorie, $ A,B \in \ObC, f: A \to B $ $f$ heißt
	\begin{enumerate}
		\item[] \define{Monomorphismus\index{kategorieller Monomorphismus}} $\defi$ Für alle $C \in \ObC,\, g_1,g_2: C \to A $ gilt: $f \circ g_1 = f \circ g_2 \Ra g_1 = g_2$ $\Lra$ Für alle $C \in \ObC$ ist $f_{\ast}^{C}: Mor_{\mathcal{C}}(C,A) \to Mor_{\mathcal{C}}(C,B)$ injektiv.
		\item[] \define{Epimorphismus\index{kategorieller Epimorphismus}} $\defi$ Für alle $C \in \ObC, \,g_1,g_2: B \to C $ gilt: $g_1 \circ f = g_2 \circ f \Ra g_1 = g_2$ $\Lra$ Für alle $C \in \ObC$ ist $f_{C}^{\ast}: Mor_{\mathcal{C}}(B,C) \to Mor_{\mathcal{C}}(A,C)$ injektiv.
		\item[] \define{Isomorphismus\index{kategorieller Isomorphismus}} $\defi$ Es existiert ein $g:B \to A $ mit $ f\circ g = id_B $ und $ g \circ f = id_A.$
	\end{enumerate}
\end{df}
\begin{anm}
	In der Situation von \ref{4.11} gilt:
	\begin{itemize}
		\item $f$ Monomorphismus in $\mathcal{C}$ $\Lra$ $f$ Epimorphismus in $\mathcal{C}^{op}.$
		\item $f$ Isomorphismus in $\mathcal{C} \Lra  f $ ist Isomorphismus in $\mathcal{C}^{op}.$
		\item Ist $f$ ein Isomorphismus und $g:B \to A $ mit $ f \circ g = id_B $ und $g \circ f = id_A$, dann ist $g$ ein eindeutig bestimmt (und wird mit $f^{-1} $ bezeichnet), denn sind $g_1,g_2: B \to A$ mit dieser Eigenschaft $\Ra g_1 = g_1 \circ id_B = g_1 \circ (f \circ g_2) =(g_1 \circ f) \circ g_2 = id_A \circ g_2 = g_2.$
		\item In Mengen ist $f$ Monomorphismus $\Lra f $ injektv, $f$ Epimorphismus $Lra f$ surjektiv, $f$ Isomorphismus $\Lra f $ bijektiv. Im Allgemeinen ist dies für Kategorien, in denen die Morphismen Abbildungen sind, jedoch falsch (vgl. Bsp. 4.13)
	\end{itemize}
\end{anm}
\begin{bem}\label{4.12}
	Sei $\mathcal{C}$ eine Kategorie, $A,B \in \ObC, f:A \to B $ ein Isomorphismus. Dann ist $f$ ein Monomorphismus und ein Epimorphismus.
	\begin{proof}
		Seien $ C \in \ObC , g_1,g_2:C \to A $ mit $ f \circ g_1 = f \circ g_2 \Ra f^{-1} \circ (f \circ g_1) = f^{-1} \circ (f \circ g_2) \Ra (f^{-1} \circ f) \circ g_1 = (f^{-1} \circ f) \circ g_2 \Ra g_1=g_2 \Ra f $ Monomorphimus. Analog wird gezeigt dass $f$ ein Epimorphimus. 
	\end{proof}
\end{bem}
\begin{anm}
	Die Umkehrung von \ref{4.12} ist im Allgemeinen falsch, siehe nächstes Beispiel.
\end{anm}
\begin{bsp}
	\begin{enumerate} [label= \alph*)]
		\item  Sei $\mathcal{C} = Top  $ die Kategorie der Topologischen Räume mit stetigen Abbildungen. Wir betrachten $$\id: (\R, \text{diskrete Topologie}) \to (\R, \text{Standardtopologie})$$ Diese ist eine stetige Abbildung, ein Monomorphismus sowie ein Epimorphismus, jedoch kein Isomorphismus (Nicht hömöomorph, da kein stetiges Inverses)
		\item Sei $\mathcal{C} = Ringe, f:\Z \to \Q $ Inklusion. $f$ ist ein  Monomorphismus und ein Epimorphimus (Achtung, denn: Für $g_1,g_2: \Q \to R$ Ringhomomorphismus ist ein Ring $R$ mit $g_1 \circ f = g_2 \circ f, $ das heißt $ g_1\big|_{\Z} = g_2\big|_{\Z} $ folgt $ g_1 = g_2 $
		 \begin{minipage}[t]{0.7\textwidth}
		 	wegen der Universellen Eigenschaft von $\Q$ als
		 	Quotientenkörper von $\Z$), aber kein Isomorphismus. 
		Insbesondere ist ein Epimorphismus in $\mathcal{C} $ im obigen Sinne  ("kategorieller Epimorphismus") nicht dasselbe wie ein surjektiver Ringhomomorphismus.
		\end{minipage}
		\begin{minipage}[t]{0.3\textwidth} 
			$$\begin{tikzcd}
			\Z \arrow[hook]{r}{f} & \Q \arrow[xshift = 0.7ex]{d}{g_2} \arrow[xshift = -0.7ex, swap]{d}{g_1}\\ & R
			\end{tikzcd}$$
		\end{minipage}
	\end{enumerate}
\end{bsp}
\begin{df}\label{4.14}
	Seien $\mathcal{C}, \mathcal{D}$ Kategorien, $F,G:\mathcal{C} \to \mathcal{D}$ Funktoren. Eine \define{natürliche Transformation\index{natürliche Transformation}} $t$ von $F$ nach $G$ ($t: F \Ra G$) ist eine Familie $(t_A)_{A\in \ObC}$ von Morphismen $t_A\in \Mor_{\mathcal{D}}(FA, GA)$, sodass 
	$$\begin{tikzcd}
	FA \arrow{r}{t_a} \arrow[swap]{d}{F(f)}& GA\arrow{d}{G(f)}\\
	FB \arrow{r}{t_B} & GB
	\end{tikzcd}$$
	für alle $A,B \in \ObC, \, f:A\to B$ kommutiert. Man sagt häufig auch $t_A: FA \to GA$ ist natürlich in $A$.
\end{df}
\begin{bsp}
	\begin{enumerate}[label= \alph*)]
		\item Sei $\mathcal{C}$ eine Kategorie, $A,B\in \ObC$, $f:A\to B$. Dann ist 
		$$f^* = (f^*_Y)_{Y\in \ObC} : \Mor_{\mathcal{C}}(B, -) \Ra \Mor_{\mathcal{C}}(A, -)$$
		eine natürliche Transformation von Funktoren $\mathcal{C} \to \text{Mengen}$, denn für $Y_1, Y_2\in \ObC$, $g:Y_1 \to Y_2$ kommutiert das Diagramm:
		$$\begin{tikzcd}
		\Mor_{\mathcal{C}}(B, Y_1) \arrow{r}{f_{Y_1}^*}\arrow[swap]{d}{g_*^B} & \Mor_{\mathcal{C}}(A, Y_1)\arrow{d}{g_*^A} \\
		\Mor_{\mathcal{C}}(B, Y_2) \arrow{r}{f_{Y_2}^*} & \Mor_{\mathcal{C}}(B, Y_2)
		\end{tikzcd}$$
		denn: Für $\phi:B\to Y_1$ ist 
		$$(g_*^A \circ f_{Y_1}^*)(\phi) = g_*^A(\phi \circ f) = g \circ \phi \circ f = f^*_{Y_2}(g\circ \phi) = (f_{Y_2}^* \circ g_*^B) (\phi)$$
		\item Sei $K$-VR die Kategorie der $K$-Vektorräume über einem festen Körper $K$ (mit linearen Abbildungen als Morphismen). Für $V\in K$-VR sei $V^*:= \Hom_K(V,K)$ der Dualraum. Die kanonische Abbildung $\phi_v:V \to V^{**}, \; w \mapsto \phi_v(w):V^* \to K, \, \psi \mapsto \psi(w)$ ist natürlich in $V$, denn für $V,W \in K$-VR, eine lineare Abbildung $f:V \to W$ kommutiert das Diagramm
		$$\begin{tikzcd}
		V \arrow{r}{\phi_v}\arrow{d}{f} & V^{**}  \arrow{d}{f^{**}} \\W \arrow{r}{\phi_w} & W^{**}
		\end{tikzcd}$$
		mit $f^{**}: V^{**} \to W^{**}, \, (\phi: V^* \to K) \mapsto f^{**}(\phi) : W^* \to K, \, \psi \mapsto \phi(\underbrace{\psi \circ f}_{\in V^*})$, d.h. $\phi:id_V \Ra \_^{**}$ ist eine natürliche Transformation von $id:K$-VR$ \to K$-VR nach $\_^{**}:K$-VR$ \to K$-VR.
	\end{enumerate}
\end{bsp}
\begin{df}\label{df4.16}
	Seien $\mathcal{C}, \mathcal{D}$ Kategorien, $F,G: \mathcal{C} \to \mathcal{D}$ Funktoren, $t:F\Ra G$ eine natürliche Transformation. $t$ heißt \define{natürliche Äquivalenz\index{natürliche Äquivalenz}} $\defi$ Für alle $A\in \ObC$ ist $t_A:FA \to GA$ ein Isomorphismus. (Notation $t: F\overset{\sim}{\Ra} G$)
\end{df}
\begin{anm}
	Ist $t:F\to G$ eine natürliche Äquivalenz, dann existiert eine natürliche Äquivalenz $t^{-1}: G \overset{\sim}{\Ra} F$ via $t_A^{-1} = (t_A)^{-1} : GA \to FA$ 
\end{anm}
\begin{bsp}
	Bezeichne $K$-VR$_{<\infty}$ die Kategorie der endlichdiimensionalen $K$-VR. Dann ist die natürliche Transformation $\phi:\id \Ra \_^{**}$ aus Beispiel 4.15 eine natürliche Äquivalenz.
\end{bsp}
\begin{df}\label{4.18}
	Seien $\mathcal{C}, \mathcal{D}$ Kategorien, $F:\mathcal{C} \to \mathcal{D}$ ein Funktor. $F$ heißt \define{Kategorienäquivalenz\index{Kategorienäquivalenz}} $\defi$ Es existiert ein Funktor $G: \mathcal{D} \to \mathcal{C}$ und natürliche Äquivalenzen $F \circ G \overset{\sim}{\Ra} \id_{\mathcal{D}}$, $G\circ F \overset{\sim}{\Ra} \id_{\mathcal{C}}$
\end{df}
\begin{bsp}
	Der Funktor $\_^*: K$-VR$_{<\infty} \to (K$-VR$_{<\infty})^\text{op}, \; v \mapsto V^*$ ist eine Kategorienäquivalenz, denn mit $\_^{\overset{\sim}{*}}: (K$-VR$_{<\infty})^\text{op} \to K$-VR$_{<\infty}, \; W \mapsto W^*$ gilt offenbar $\_^{\overset{\sim}{*}} \circ \_^* = \_^{**}$, und $\phi:\id \overset{\sim}{\Ra}  \_^{**}$ ist eine natürliche Äquivalenz, analog andersherum (d.h. die Kategorie $K$-VR$_{< \infty}$ ist selbstdual).
\end{bsp}
\begin{sa}[Yoneda-Lemma]\label{4.20}\index{Yoneda-Lemma}
	Sei  $\mathcal{C}$ eine Kategorie, $A\in \ObC$, $F:\mathcal{C} \to \text{Mengen}$ ein Funktor. Dann gibt es eine Bijektion 
	\begin{eqnarray*}
	\Phi: \{\text{natürliche Transformationen }t:\Mor_{\mathcal{C}}(A, -) \Ra F\} & \to & F(A) \\
	t & \mapsto & t_a(\id_A)
	\end{eqnarray*}
\end{sa}
\begin{proof}
	\begin{enumerate}
		\item Sei $a\in F(A)$. Wir definieren $S^a: \Mor_{\mathcal{C}}(A,-) \Ra F$ als $s^a= (s^a_B)_{B\in \ObC}$ mit 
		$$s_B^a := F(\phi)(a) \quad \text{für } \phi \in \Mor_{\mathcal{C}}(A,B)$$
		$s^a$ ist eine natürliche Transformation, denn für $B,C\in \ObC, \, f:B \to C$ kommutiert
		$$\begin{tikzcd}
		\Mor_{\mathcal{C}}(A,B) \arrow{r}{s_B^a}\arrow[swap]{d}{f_*^A} & F(B) \arrow{d}{F(f)}\\
		\Mor_{\mathcal{C}}(A,C) \arrow{r}{s_C^a} & F(C)
		\end{tikzcd}$$
		denn:
		\begin{eqnarray*}(F(f) \circ s_B^a)(\phi) &=& F(f) (s_B^a(\phi)) = F(f)(F(\phi)(a)) = F(f\circ \phi)(a) \\
			&=& F(f_*^A(\phi))(a) = s_C^a(f_*^A(\phi))
	\end{eqnarray*}
	\item Setze \begin{eqnarray*}
	\Psi:F(A) &\to& \{\text{natürliche Transformationen }t:\Mor_{\mathcal{C}}(A, -) \Ra F\}\\
	a & \mapsto & s^a
	\end{eqnarray*}
	Dann sind $\Phi, \Psi$ invers zueinander, denn: Für $a\in F(A)$, $t:\Mor_{\mathcal{C}}(A,-) \Ra F$ gilt
	$$(\Phi \circ \Psi)(a) = \Phi(s^a) = s_A^a(\id_A) = F(\id_A)(a) = \id_{FA}(a) = a$$
	und 
	$$(\Psi \circ \Phi)(t) = \Psi(t_A(\id_A))$$
	und für $B\in \ObC, \, \phi \in \Mor_{\mathcal{C}}(A,B)$ gilt wegen der Kommutativität von 
	$$\begin{tikzcd}
	\Mor_{\mathcal{C}}(A,A) \arrow{r}{t_A}\arrow[swap]{d}{\phi_*^A} & F(A) \arrow{d}{F(\phi)}\\
	\Mor_{\mathcal{C}}(A,B) \arrow{r}{t_B} & F(B)
	\end{tikzcd}$$
	$$(\Psi(t_A(\id_A)))_B(\phi) = s^{t_A(\id_A)}_B(\phi)= F(\phi)(t_A(\id_A)) = t_B(\phi_*^A(\id_A)) = t_B(\phi) $$
	d.h. $(\Psi \circ \Phi)(t) = t$
	\end{enumerate}
\end{proof}
\begin{fo}\label{4.21}	Sei $\mathcal{C}$ eine Kategorie, $A,B \in \ObC$. Dann ist die Abbildung 
	\begin{eqnarray*}
	 \Psi: \Mor_{\mathcal{C}}(B,A) & \longrightarrow & \{\text{natürliche Transformationen } \Mor_{\mathcal{C}}(A,-) \Ra \Mor_{\mathcal{C}}(B,-) \}\\
	 \psi:B \to A & \mapsto & \psi^*: \Mor_{\mathcal{C}}(A,-) \to \Mor_{\mathcal{C}}(B,-)
 	\end{eqnarray*}
 bijektiv.
\end{fo}
\begin{proof}
	Wende \ref{4.20} auf $F= \Mor_{\mathcal{C}}(B,-)$ and. In der Notation des Beweises von 4.20 ist $\Psi(\psi) = s^\psi = (s_C^\psi)_{C\in \ObC}$, wobei für $C\in \ObC, \, \phi \in \Mor_{\mathcal{C}}(A,C)$ gilt: 
	$$(s_C^\psi)(\phi) = \Mor_{\mathcal{C}}(B,-)(\phi)(\psi) = \phi_*^B(\psi) = \phi \circ \psi = \psi^*_C(\phi)$$
	d.h. $\Psi(\psi) = \psi^*$.
\end{proof}
\begin{anm}
	\begin{itemize}
		\item Folgerung \ref{4.21} liefert einen sogenannten volltreuen Funktor $\mathcal{C}^\text{op} \to \text{Funk}(\mathcal{C}, \text{Mengen})$, wobei $A\mapsto \Mor_{\mathcal{C}}(A,-)$, wobei $\text{Funk}(\mathcal{C}, \text{Mengen})$ die Funktorkategorie von $\mathcal{C}$ nach Mengen bezeichnet (Objekte sind Funktoren: $\mathcal{C} \to $ Mengen, und Morphismen die natürlichen Transformationen) (\define{Yoneda-Einbettung\index{Yoneda-Einbettung}})
		\item Folgerung \ref{4.21} liefert insbesondere eine Verallgemeinerung des Satzes von Caley aus der Gruppentheorie: Für eine Gruppe $G$ ist $G \hookrightarrow S(G), \; g \mapsto \tau_G$ (Linkstranslation mit $g\in G$) ein injektiver Gruppehomomorphismus. Wende \ref{4.21} an auf: \begin{itemize}
			\item $\mathcal{C} = $ Kategorie mit $\ObC = \{ \cdot  \}$, $\Mor_{\mathcal{C}}( \cdot , \cdot ) = G$
			\item $A=B= \cdot $
		\end{itemize}
	und erhalte eine Bijektion 
	\begin{eqnarray*}
		G= \Mor_{\mathcal{C}}( \cdot, \cdot )  & \longrightarrow & \{\text{natürliche Transformationen } \Mor_{\mathcal{C}}( \cdot, -)  \Ra \Mor_{\mathcal{C}}(\cdot, -) \}\\
		g & \mapsto & g^*: \Mor_{\mathcal{C}}( \cdot, -) \Ra \Mor_{\mathcal{C}}( \cdot, -) \is \tau_G
	\end{eqnarray*}
	\end{itemize}
\end{anm}
\newpage
\subsection{Abelsche Kategorien}
\begin{df}\label{5.1}
	Sei $\mathcal{C}$ eine Kategorie, $A\in \ObC$. $A$ heißt
	\begin{enumerate}
		\item[] \define{Anfangsobjekt\index{Anfangsobjekt}} $\defi$ Für alle $M\in \ObC$ ist $\Mor_{\mathcal{C}}(A,M)$ einelementig
		\item[] \define{Endobjekt\index{Endobjekt}} $\defi$ Für alle $M\in \ObC$ ist $\Mor_{\mathcal{C}}(M,A)$ einelementig
	\end{enumerate}
\end{df}
\begin{anm}
	Falls sie existieren, sind Anfangs- und Endobjekte eindeutig bestimmt bis auf eindeutigen Isomorphismus (denn: Sind $A_1, A_2$ Anfangsobjekte, dann ist $\Mor_{\mathcal{C}}(A_1, A_2) = \{\alpha\}$, $\Mor_{\mathcal{C}}(A_2, A_1) = \{\beta\}$, $\Mor_{\mathcal{C}}(A_1, A_1) = \{\id_{A_1}\}$ und analog $\Mor_{\mathcal{C}}(A_2, A_2) = \{\id_{A_2}\}$, insbesondere ist $\beta \circ \alpha = \id_{A_1}, \, \alpha \circ \beta = \id_{A_2}$).
\end{anm}
\begin{df}\label{5.2}
	Sei $\mathcal{C}$ eine Kategorie. $0\in \ObC$ heißt \define{Nullobjekt\index{Nullobjekt}} $\defi 0$ ist sowohl Anfangs- als auch Endobjekt. Existiert in $\mathcal{C}$ ein Nullobjekt $0$, so enthält $\Mor_{\mathcal{C}}(A,B)$ für alle $A,B \in \ObC$ einen ausgezeichnetes Element, den "'Nullmorphismus"' $ A \to 0 \to B$
\end{df}
\begin{anm}
	Der Nullmorphismus in $\Mor_{\mathcal{C}}(A,B)$ ist unabhängig von der Wahl des Nullobjekts:
	$$\begin{tikzcd}
	A \arrow{r} \arrow{dr} & 0\arrow{d} \arrow{r} & B \\
	& \tilde{0} \arrow{ur} &
	\end{tikzcd}$$
\end{anm}
\begin{bsp}\label{5.3}
	\begin{enumerate}[label=\alph*)]
		\item In Mengen ist $\emptyset$ ein Anfangsobjekt, jede einelementige Menge ist ein Endobjekt, insbesondere existiert in Mengen kein Nullobjekt
		\item in Ringe ist $\Z$ ein Anfangsobjekt, und der Nullring ist ein Endobjekt. In Ringe existiert ebenfalls ein Nullobjekt
		\item In $R$-Mod ist der Nullmodul ein Nullobjekt.
	\end{enumerate}
\end{bsp}
\begin{df}\label{5.4}
	Sei $\mathcal{C}$ eine Kategorie. $(A_i)_{i \in I} $ eine Familie von Objekten aus $\mathcal{C}$. Ein \define{Produkt\index{Produkt}} $(A, (p_i)_{i \in I})$ von $(A_i)_{i \in I}$ ist ein Objekt $A\in \mathcal{C} $ zusammen mit Morphismen $p_i:A \to A_i $, sodass für alle $B \in \ObC $die Abbildung $$ Mor_C(B,A) \to \prod_{i \in I} Mor_C(B,A_i) , \quad f \mapsto (p_i \circ f) _{i \in I} $$ bijektiv ist, das heißt für jede Familie $(f_i)_{i \in I}$ von Morphismen $f_i: B \to A_i $ existiert ein eindeutig bestimmtes $f:B \to A $ mit $ f_i = p_i \circ f $ für alle $ i \in I$.
\end{df}
\begin{bem}\label{5.5}
	Sei $\mathcal{C}$ eine Kategorie, $(A_i)_{i \in I}$ eine Familie von Objekten aus $\mathcal{C}$, $(A, (p_i)_{i \in I}), (A', (p_i')_{i \in I}),$ Produkte von $(A_i)_{i \in I}. $Dann existiert ein eindeutig bestimmter Isomorphismus $ f:A \to A' $, sodass für alle $i \in I $ gilt: $p_i' \circ f = p_i$
	$$\begin{tikzcd}
	A \arrow{rr}{f} \arrow[swap]{dr}{p_i} & & A' \arrow{dl}{p_i'	}\\
	&A_i& 
	\end{tikzcd}$$
	(kurz: $A,A'$ sind kanonisch isomorph. Wir sprechen daher oft von "'dem Produkt"' und schreiben $ A = \prod_{i \in I} A_i $
\end{bem}
\begin{proof}
	\begin{enumerate} 
		\item Wir wenden die Universelle Eigenschaft auf das Produkt $(A', (p_i')_{i \in I}), B = A, f_i = p_i \Ra $ Wir erhalten einen eindeutig bestimmten Morphismus $ f: A \to A' $ mit $ p_i' \circ f = p_i $ für alle $ i \in I$. Analog: Wende die Universelle Eigenschaft auf das Produkt $(A, (p_i)_{i \in I}), B = A', f_i= p_i' \Ra $ Es existiert genau ein $ g: A' \to A $ mit $p_i \circ g = p_i'$ für alle $i \in I $.
		\item Es gilt $ g \circ f = id_A, f \circ g = id_{A'} $ (d.h $f$ ist ein Isomorphismus), denn: Für alle $ i \in I $ ist $p_i \circ ( g \circ f ) = ( p_i \circ g) \circ f =  p_i' \circ f = p_i $. Wende die Universelle Eigenschaft auf das Produkt$ (A, (p_i)_{i \in I}), B= A, f_i = p_i $ an:
		Es existiert genau ein $ h: A \to A $ mit $ p_i \circ h = p_i $ für alle $ i \in I $ ( nämlich $ h = id_A $) Somit ist $ id_A = g \circ f.$ Analog:  $ f \circ g = id_{A'}$.
	\end{enumerate}
\end{proof}
\begin{bsp}
	\begin{enumerate} [label=\alph*)]
		\item In Mengen ist das Produkt das kartesische Produkt.
		\item In $R$-Mod ist das Produkt das direkte Produkt.
		\item In der Kategorie der endlichen abelschen Gruppen existiert kein Produkt der Familie $\left( \QR{\Z}{n\Z}\right)_{n \in \N}$ (Übung)
	\end{enumerate}
\end{bsp}
\begin{bem+df}\label{5.7}
	Sei $\mathcal{C} $ eine Kategorie, $ (A_i)_{i \in I}$ eine Familie von Objekten  aus $ \mathcal{C}$. Ein \emph{Koprodukt\index{kategorielles Koprodukt}} $(A, (q_i)_{i \in I})$ von $(A_i)_{i \in I}$ ist ein Objekt $ A \in \ObC $ zusammen mit Morphismen $ q_i: A_i \to A$ , sodass $(A, (q_i)_{i \in I})$ Ein Produkt von $(A_i)_{i \in I } $ in $ \mathcal{C}^{op} $ ist, das heißt für alle $ B \in \ObC $ ist die Abbildung $$ Mor_C(A,B) \to \prod_{i \in I} Mor_C(A_i,B) , \quad f \mapsto (f \circ q_i) _{i \in I} $$ bijektiv. Falls es existiert ist ein Koprodukt von $(A_i)_{i \in I }$ eindeutig bestimmt bis auf Isomorphie (analog zu \ref{5.5}). Wir sprechen dan von dem \define{Koprodukt\index{Koprodukt}} und schreiben $ A = \bigoplus_{i \in I} A_i $ ($=\coprod_{i\in I} A_i$)
\end{bem+df}
\begin{bsp}
	\begin{enumerate} [label=\alph*)]
		\item In Mengen ist das Koprodukt die disjunkte Vereinigung.
		\item in $R$-Mod ist das Koprodukt die direkte Summe.
		\item In der Kategorie der Gruppen existiert ein Koprodukt, das sogenannte freie Produkt ( siehe Zettel Algebra 1)
	\end{enumerate}
\end{bsp}
\begin{df}\label{5.9}
	Sei $\mathcal{A} $ eine Kategorie. $\mathcal{A} $ heißt \define{additiv\index{additive Kategorie}}, wenn gilt:
	\begin{enumerate}
		\item[($K1$)] $\mathcal{A}$ hat ein Nullobjekt,
		\item[($K2$)] In $\mathcal{A}$ existieren endliche Produkte
		\item[($K3$)] Für alle $A,B \in \ObA$ trägt $Mor_{\mathcal{A}}(A,B) $ die Struktur einer abelschen Gruppe mit dem Nullmorphismus als neutrales Element, sodass für alle $ A,B,C \in \ObA $ die Verknüpfung:
		 $$ Mor_{\mathcal{A}}(B,C) \times  Mor_{\mathcal{A}}(A,B) \overset{\circ}{\longrightarrow}Mor_{\mathcal{A}}(A,C)$$ bilinear ist.
	\end{enumerate}
\end{df}
\begin{anm}
	In einer additiven Kategorie $\mathcal{A} $ schreiben wir auch $Hom_{\mathcal{A}}$ für $Mor_{\mathcal{A}}. $
\end{anm}
\begin{bsp}
	\begin{enumerate} [label=\alph*)]
		\item $R$-Mod ist eine additive Kategorie
		\item Ringe sind keine additive Kategorie (kein Nullobjekt, vgl \ref{5.3}(b)).
	\end{enumerate}
\end{bsp}
\begin{sa}\label{5.11}
	Sei $\mathcal{A} $ eine additive Kategorie, $ A_1,A_2 \in \ObA$ ,$(A_1 \times A_2, (p_1,p_2)) $ Produkt von $ A_1 \times A_2 $, $ i_1: A_1 \to A_1 \times A_2 $ sei via der Univesellen Eigenschaft gegeben durch $id_{A_1}: A_1 \to A_1 , 0: A_1 \to A_2$. Analog sei $ i_2: A_2 \to A_1 \times A_2 $ sei via der Univesellen Eigenschaft gegeben durch $ 0: A_2 \to A_1, id: A_2 \to A_2$. Dann ist $(A_1 \times A_2, (i_1,i_2)) $ ein Koprodukt von $A_1,A_2$ in $\mathcal{A}$.
\end{sa}
\begin{proof}
	\begin{enumerate}
		\item Behauptung: $\iota_1 \circ p_1 + \iota_2 \circ p_2: A_1 \times A_2 \to A_1 \times A_2 $ stimmt mit $id_{A_1 \times A_2} $ überein. Denn: Es ist
		$$p_1 \circ (\iota_1 \circ p_1 + \iota_2 \circ p_2) = \underbrace{p_1 \circ \iota_1}_{=\id_{A_1}} \circ p_1+ \underbrace{p_1 \circ \iota_2}_{=0:A_2\to A_1} \circ p_2 = p_1 = p_1 \circ id_{A_1 \times A_2}$$ 
			%Klammern drunter 
		Analog: $$p_2 \circ (\iota_1 \circ p_1 + \iota_2 \circ p_2) = p_2 = p_2 \circ id_{A_1 \times A_2} \overset{UE}{\Ra} \iota_1 \circ p_1 + \iota_2 \circ p_2 = id_{A_1 \times A_2}$$
		\item Universelle Eigenschaft des Koprodukts:
		 Sei $ B \in \ObA $, $ f_1: A_1 \to B, f_2: A_2 \to B $\\
		 Existenz: Wir setzen $f:= f_1 \circ p_1 + f_2 \circ p_2: A_1 \times A_2 \to B$. DAnn ist 
		 $$f \circ \iota_1 = f_1 \circ \underbrace{p_1 \circ \iota_1}_{=\id_{A_1}} + f_2 \circ \underbrace{p_2 \circ \iota_1}_{=0:A_1 \to A_2} = f_1.$$
		 % KLammern 
		Analog: $f \circ i_2 = f_2.$\\
		Eindeutigkeit: Sei $f': A_1 \times A_2 \to B $ mit $f' \circ \iota_1 = f_1 , f' \circ \iota_1 = f_2.$. Dann folgt 
		$$f' = f' \circ (\iota_1 \circ p_1 + \iota_2 \circ p_2 ) = \underbrace{f' \circ \iota_1}_{=f_1}  \circ p_1 + \underbrace{f' \circ  \iota_2}_{=f_2} \circ p_2 = f_1 \circ p_1 + f_2 \circ p_2 = f $$
	\end{enumerate}
\end{proof}
\begin{fo}\label{5.12}
	Sei $\mathcal{A} $ eine Additive Kategorie. Dann existieren in $\mathcal{A} $  endliche Koprodukte. 
\end{fo}
\begin{df}\label{5.13}
	Seien $ \mathcal{A}, \mathcal{B} $ additve Kategorien, $F: \mathcal{A} \to \mathcal{B}$ Funktor.
	 $F$ heißt \define{additiv\index{additive Kategorie}} "' $\defi $ für alle $ A, A' \in \ObA $ ist eine Abbildung: $$ Mor_{\mathcal{A}}(A,A') \to  Mor_{\mathcal{B}}(FA,FA'), \quad f \mapsto F(f) $$ ein Homomorphismus abelscher Gruppen.
\end{df}
\begin{anm}
	$F$ additiv $\Ra  F(A \oplus A' ) = F(A) \oplus F(A') $ (Übungen)
\end{anm}
\begin{bem+df}\label{5.14}
	Sei $ \mathcal{A}$  eine additive Kategorie, $ A,A' \in \ObA, f: A \to A' $. Ein \define{Kern\index{kategorieller Kern}} $(B,\iota) $ von $f$ ist ein Objekt $ B \in \ObA$ zusammen mit einem Morphismus $ \iota: B \to A $, sodass $ f \circ \iota = 0 $ ist und für alle $ C \in$Ob$\mathcal{A}$  die Abbildung: $$Hom_{\mathcal{A}}(C,B) \longrightarrow \{g \in Hom_{\mathcal{A}}(C,A) | f \circ g = 0 \}, \quad h \mapsto \iota \circ h $$ bijektiv ist, das heißt für alle $g:C \to A $ mit $ f \circ g = 0 $ existiert ein eindeutig bestimmter Morphismus $ h: C \to B $ mit $ g = \iota \circ h$:
	$$\begin{tikzcd}
	B \arrow{r}{\iota}  & A \arrow{r}{f} & A' \\
	C \arrow[swap]{ur}{g}\arrow[dashed]{u}{h}
	\end{tikzcd}$$
	Ist $(B',\iota') $ ein weiterer Kern von f, dann existiert ein eindeutig bestimmte Isomorphismus $ \alpha: B \to B' $ mit $ \iota = \iota' \circ \alpha$:
	$$\begin{tikzcd}
	B\arrow{rr}{\alpha} \arrow{dr}{\iota} & & B'\arrow{dl}{\iota'}\\
	& A &
	\end{tikzcd}$$
	
	Wir nennen $(B,\iota) $ daher auch "'den Kern"' von$f$ und schreiben $ \ker f= (B,\iota)$ beziehungsweise kürzer: $ \ker f= B $ oder auch $\ker f = \iota$
\end{bem+df}
\begin{anm}
	Die Existenz von Kernen ist im Allgemeinen nicht gegeben
\end{anm}
\begin{bsp}
	In $R$-Mod ist der kategorielle Kern gegeben durch die Inklusion des gewöhnlichen Kerns:
	$$\begin{tikzcd}
	\ker f \arrow[hook]{r}{\iota} & A \arrow{r}{f} & A'\\C \arrow[swap]{ur}{g}\arrow[dashed]{u}{h}
	\end{tikzcd}$$
	$ f \circ g = 0 \Ra \im g \subseteq \ker f $ setze  $ h := g\big|^{\ker f}:  C \in \ker f$, dann ist $ \iota\circ h = g $ und $h$ ist eindeutig mit dieser Bedingung.
\end{bsp}
\begin{bem}\label{5.16}
	Sei $\mathcal{A} $ eine additive Kategorie, $ A,A' \in \ObA, f: A  \to A' , (\ker f, \iota) $ Kern von f. Dann ist $\iota$ ein Monomorphismus. 
\end{bem}
\begin{proof}
	Seien $ h_1,h_2: C \to \ker f $ mit $ \iota\circ h_1 = \iota \circ h_2 =:g \Ra f \circ g = f \circ \iota \circ h_1 = 0 $ Dann existiert ein eindeutig bestimmtes $h: C \to \ker f $ mit $ g = \iota\circ h \Ra h = h_1 = h_2. $
\end{proof}
\begin{bem+df}\label{5.17}
	Dual zum Kern definiert man den Kokern (Notation: $ \coker f$). Die Aussagen \ref{5.14}, \ref{5.16} gelten dual.
\end{bem+df}
\begin{df}\label{5.18}
	 Sei $\mathcal{A} $ eine additive Kategorie, $ A,A' \in \ObA, f: A \to A'$ 
	 \begin{enumerate}
	 \item[] $\im f := \ker(\coker f) $ heißt das \define{Bild\index{kategorielles Bile}} von $f$ 
	 \item[] $\coim f := \coker(\ker f) $ heißt das \define{Kobild\index{kategorielles Kobild}} von $f$.
	\end{enumerate}
\end{df}
\begin{anm}
	$\im f $ kommt mit einem Monomorphisus $\iota': \im f \to A' ,\, \coim f $ mit einem Epimorphismus $ q': A \to \coim f $.
\end{anm}
\begin{bsp}
	Sein $\mathcal{A} = R $-Mod ,$ f: A  \to A' $ $R$-Modulhomomorphismus. Dann ist 
	$$\im f = \ker\left(\QR{A'}{\im f},\quad  A' \to \QR{A'}{\im f}\right) = (\im f, \ \im f \hookrightarrow A')$$
	$$\coim f= \coker(\ker f, \ \ker f \to A) = \left(\QR{A}{\ker f},\quad  A \to \QR{A}{\ker f}\right)$$
\end{bsp}
\begin{bem}\label{5.20}
	Sei $\mathcal{A}$ eine additive Kategorie, $A,B\in \ObA$, $f:A\to B$, sodass $\ker f, \coker f, \im f, \coim f$ existieren ($\im f, \iota')$ Bild von $f$, ($\coim f, q'$) Kobild von $f$. Dann existiert ein eindeutig bestimmter Morphismus $\bar f: \coim f \to \im f$ mit $f= \iota' \circ \bar f \circ q'$
	$$\begin{tikzcd}
	A \arrow{r}{f} \arrow{d}{q'} & B \\
	\coim f \arrow[dashed]{r}{\bar f} & \im f \arrow{u}{\iota'}
	\end{tikzcd}$$
\end{bem}
\begin{proof}
	$$\begin{tikzcd}
	\ker f \arrow{r}{\iota} & A \arrow{d}{q'} \arrow{r}{f} & B \arrow{r}{q} & \coker f \\
	& \coim f = \coker \iota \arrow[dashed]{ur}{f'} \arrow[swap, dashed]{r}{\bar f} &\ker q = \im f \arrow{u}{\iota'}  & 
	\end{tikzcd}
	$$
	\begin{enumerate}
		\item Existenz: Wegen $f\circ \iota = 0$ existiert nach der Universellen Eigenschaft von $\coker$ ein $f': \coim f \to B$ mit $f= f' \circ q'$. Es ist $q \circ  f=0$, also $q \circ f' \circ q' = q \circ f = 0 = 0 \circ q'$. Da $q'$ ein Epimorphismus ist, folgt $q\circ f' = 0$. Nach der Universellen Eigenschaft des Kerns, existiert ein $\bar f: \coim f \to \im f$ mit $\iota' \circ \bar f = f'$, also $\iota' \circ \bar f \circ q' = f' \circ q' = f$.
		\item Eindeutigkeit: Sei $\tilde{f}: \coim f \to \im f$ mit $\iota' \circ \bar f\circ q' = f = \iota' \circ \tilde{f} \circ q'$, woraus, wegen $\iota'$ Monomorphismus zunächst $\bar f \circ q' = \tilde{f} \circ q'$ folgt und dann, wegen $q'$ Epimorphismus, $\bar f = \tilde{f}$.
	\end{enumerate}
\end{proof}
\begin{df}\label{5.21}
	Sei $\mathcal{A}$ eine additive Kategorie. $\mathcal{A}$ heißt \define{abelsche Kategorie\index{abelsche Kategorie}}, wenn gilt:
	\begin{enumerate}
		\item[(Ab1)] Jeder Morphismus in $\mathcal{A}$ hat Kern und Kokern
		\item[(Ab2)] (\define{Homomorphiesatz}). Für jeden Morphismus $f:A \to A'$ in $\A$ ist der induzierte Morphismus 
		$$\bar f : \coim f \to \im f$$
		ein Isomorphismus 
	\end{enumerate}
\end{df}
\begin{bsp}
	\begin{enumerate}[label = \alph*)]
		\item $R$-Mod ist eine abelsche Kategorie
		\item Die Kategorie der freien $\Z$-Moduln ist additiv, aber nicht abelsch: (Ab1) ist nicht erfüllt.
		\item Die Kategorie der abelschen topologischen Gruppen ist eine additive Kategorie, die (Ab1) erfüllt, aber nicht (Ab2): $\id: (\R, +) \to (\R, +)$ (links mit der diskreten Topologie und rechts mit der Standardtopologie) $\bar{\id} = \id$ ist kein Isomorphismus.
	\end{enumerate}
\end{bsp}
\begin{anm}
	Ist $\A$ eine abelsche Kategorie, dann ist auch $\A^\text{op}$ abelsche Kategorie (einziger nichttrivialer Punkt: Existenz endlicher Produkte, was jedoch aus \ref{5.11} folgt).
\end{anm}
\begin{sa}\label{5.23}
	Sei $\A$ eine abelsche Kategorie, $A,A' \in \ObA$, $f:A \to A'$ Mono- und Epimorphsimus. Dann ist $f$ ein Isomorphismus.
\end{sa}
\begin{proof}
	\begin{itemize}
		\item Da $f$ ein Monomorphismus ist, ist $(0, 0 \to A)$ ein Kern von $f$, denn: 
		$$ \begin{tikzcd}
		0 \arrow{r} & A \arrow{r}{f} & A' \\
		C \arrow[dashed, red]{u}\arrow{ur}{g} \arrow[swap, red, yshift = -1.5ex]{ur}{0}& & 
		\end{tikzcd} \quad f\circ g = 0 = f \circ 0 \overset{f \text{ Mono}}{\Longrightarrow} g=0
		$$
		\item $\coim f = \coker(0 \to A) = (A, \id_A)$, denn 
		$$\begin{tikzcd}
		0 \arrow{r} & A \arrow{dr}{g} \arrow{r}{\id_A} & A \arrow[red, dashed]{d}{g}\\
		& & C
		\end{tikzcd}
		$$
		Analog ist $\im f = (A', \id_{A'})$, also ist $\bar f = f $ ein Isomorphismus nach (Ab2).
	\end{itemize}
\end{proof}
\begin{bem}\label{5.24}
	Sei $\A$ eine abelsche Kategorie, $A,A' \in \ObA$, $f:A \to A'$. Dann gilt:
	\begin{enumerate}[label= \alph*)]
		\item $f$ Monomorphismus $\Lra \ker f =0$
		\item $f$ Epimorphismus $\Lra \coker f = 0$
	\end{enumerate}
\end{bem}
\begin{proof}
	Übungsaufgabe.
\end{proof}
\begin{df}\label{5.25}
	Sei $\A$ eine abelsche Kategorie, $A,A', A''\in \ObA$.
	$$\begin{tikzcd}
	A' \arrow{r}{f} & A \arrow{r}{g} & A''
	\end{tikzcd}$$
	heißt eine \define{exakte Folge\index{kategorielle exakte Folge}} $\Lra \im f \cong \ker g$ in dem Sinne, dass es einen Isomorphismus
	$\im f \overset{\alpha}{\longrightarrow} \ker g$ gibt, sodass das Diagramm
	$$\begin{tikzcd}
	& A & \\
	\im f \arrow{ur}{\iota'} \arrow{rr}{\alpha} & & \ker g \arrow{ul}{\iota}
	\end{tikzcd}$$
	kommutiert (wobei ($\ker g, \iota)$ Kern von $g$, ($\im f, \iota'$) Bild von $f$)
\end{df}
\begin{sa}\label{5.26}
	Sei $\A$ eine abelsche Kategorie. Dann gilt:
	\begin{enumerate}[label= \alph*)]
		\item In $\A$ gilt das Fünferlemma
		\item In $\A$ gilt das Schlangenlemma
		\item Eine Folge $\begin{tikzcd}
		M' \arrow{r}{f} & M \arrow{r}{g} & M'' \arrow{r} & 0
		\end{tikzcd}$ in $\A$ ist genau dann exakte, wenn für jedes Objekt $N\in \ObA$ die Folge abelscher Gruppen 
		$$\begin{tikzcd}
		0 \arrow{r}& \Hom_\A(M'', N) \arrow{r} & \Hom_\A(M,N) \arrow{r} & \Hom_\A(M', N)
		\end{tikzcd}$$
		exakte ist.
		\item Eine Folge $\begin{tikzcd}
		0 \arrow{r}&  N' \arrow{r} & N \arrow{r}& N''
		\end{tikzcd}$ in $\A$ ist genau dann exakte, wenn für jedes $M\in \ObA$ die Folge abelscher Gruppen
		$$\begin{tikzcd}
		0 \arrow{r}& \Hom_\A(M, N') \arrow{r} & \Hom_\A(M,N) \arrow{r} & \Hom_\A(M, N'')
		\end{tikzcd}$$
		exakt ist.
	\end{enumerate}
\end{sa}
\begin{proof}
	$a)$ Stacks-Project: 12.5.17, $b)$ 12.5.20\\
	$c), \, d)$ werden in 2.6 für $R$-Mod bewiesen.
\end{proof}
\begin{df}\label{5.27}
	Seien $\A$, $\mathcal{B}$ abelsche Kategorien, $F: \A \to \mathcal{B}$ ein additiver Funktor. $F$ heißt 
	\begin{enumerate}
		\item[] \define{exakt\index{exakter Funktor}} $\defi$ $F$ überführt kurze exakte Folgen in $\A$ in kurze exakte Folgen in $\mathcal{B}$
		\item[] \define{liksexakt\index{linksexakter Funktor}} $\defi$ Für jede exakte Folge $\begin{tikzcd}
		0 \arrow{r} & M' \arrow{r} & M \arrow{r} & M''
		\end{tikzcd}$ in $\A$ ist die Folge $$\begin{tikzcd}
		0 \arrow{r} & FM' \arrow{r} & FM \arrow{r} & FM''
		\end{tikzcd}$$ exakt
		\item[] \define{rechtsexakt\index{rechtsexakter Funktor}} $\defi$ Für jede exakte Folge $\begin{tikzcd}
		M' \arrow{r} & M \arrow{r} & M'' \arrow{r} & 0
		\end{tikzcd}$ in $\A$ ist die Folge
		$$\begin{tikzcd}
		FM' \arrow{r} & FM \arrow{r} & FM''  \arrow{r} & 0
		\end{tikzcd}$$
		exakt.
	\end{enumerate}
\end{df}
\begin{anm}
	$F$ ist exakt $\defi F$ ist links- und rechtsexakt $\Lra$ Für alle exakten Folgen $\begin{tikzcd}
	A' \arrow{r} & A \arrow{r} & A''
	\end{tikzcd}$ in $\mathcal{A}$ ist $\begin{tikzcd}
	FA' \arrow{r} & FA \arrow{r} & FA''
	\end{tikzcd}$ exakt (Übung)
\end{anm}
\begin{df}\label{5.28}
	Sei $\mathcal{A}$ eine abelsche Kategorie, $I,P \in \ObA$. $I$ heißt
	\begin{enumerate}
		\item[] 
		 \begin{minipage}[t]{0.7\textwidth}
		\define{injektiv\index{injektives Objekt}} $\defi$ Für jeden Monomorphismus $\iota: A \hookrightarrow B$ und jeden Morphismus $f: A \to I$ existiert ein Morphismus $g:B \to I$ mit $g\circ \iota = f$, d.h. $\iota_I^*: \Hom_\A(B,I) \to \Hom_\A(A,I)$ ist surjektiv.	
		\end{minipage}
		\begin{minipage}[t]{0.3\textwidth} 
			$$\begin{tikzcd}
			A \arrow[hook]{r}{\iota} \arrow{d}{f} & B\arrow[dashed]{dl}{g} \\
			I & 
			\end{tikzcd}$$
		\end{minipage}
	\end{enumerate}
	$P$ heißt
	\begin{enumerate}

	\item[] \begin{minipage}[t]{0.7\textwidth}
		\define{projektiv\index{projektives Objekt}} $\defi$ $P$ ist injektiv in $\A^\text{op}$, d.h. für jeden Epimorphismus $p: B\twoheadrightarrow A$ und jeden Morphismus $f:P \to A$ existiert ein Morphismus $g:P \to B$ mit $p \circ g = f$
	\end{minipage}
	\begin{minipage}[t]{0.3\textwidth} 
		$$\begin{tikzcd}
		& P\arrow[dashed, swap]{dl}{g} \arrow{d}{f} \\
		B \arrow[swap]{r}{p} & A
		\end{tikzcd}$$
	\end{minipage}
	\end{enumerate}
\end{df}
\begin{bem}\label{5.29}
	Sei $\A$ eine abelsche Kategorie, $I\in \ObA$. Dann sind äquivalent:
	\begin{enumerate}[label= \roman*)]
		\item $I$ ist injektiv
		\item Der Funktor $\Hom_\A(-, I): \A^\text{op} \to \Z$-Mod ist exakt
	\end{enumerate}
\end{bem}
\begin{proof}
	Nach \ref{5.26} $c)$ ist für alle exakten Folgen $\begin{tikzcd}
	A' \arrow{r} & A \arrow{r} & A'' \arrow{r} & 0
	\end{tikzcd}$ in $\A$ ist auch die Folge
	$$\begin{tikzcd}
	0 \arrow{r}& \Hom_\A(A'', I) \arrow{r} & \Hom_\A(A,I) \arrow{r} & \Hom_\A(A',I)
	\end{tikzcd}$$
	exakt. Somit genügt es zu zeigen, dass $I$ injektiv $\Lra$ Für alle exakten Folgen\\ $\begin{tikzcd}
	0 \arrow{r} & A' \arrow{r}{\iota} & A
	\end{tikzcd}$ in $\A$ ist 
	$$\begin{tikzcd}
	\Hom_\A(A,I) \arrow{r}{\iota_I^*} & \Hom_\A(A', I) \arrow{r} & 0
	\end{tikzcd}$$
	eine exakte Folge in $\Z$-Mod, d.h. $\iota_I^*:\Hom_\A(A,I) \longrightarrow \Hom_\A(A',I)$ ist surjektiv. Die Exaktheit von  $\begin{tikzcd}
	0 \arrow{r} & A \arrow{r}{\iota} & A
	\end{tikzcd}$ ist äquivalent dazu, dass $\iota$ ein Monomorphismus ist.
\end{proof}
\begin{bem}\label{5.30}
	Sei $\A$ eine abelsche Kategorie, $P\in \ObA$. Dann sind äquivalent:
	\begin{enumerate}[label= \roman*)]
		\item $P$ ist projektiv
		\item Der Funktor $\Hom_\A(P,-): \A \to \Z$-Mod ist exakt.
	\end{enumerate}
\end{bem}
\begin{df}\label{5.31}
	Seien $\mathcal{C}, \mathcal{D}$  (additive) Kategorien, $F: \mathcal{C} \to \mathcal{D}$, $G: \mathcal{D} \to \mathcal{C}$ (additive) Funktoren. Dann heißt $F$ \define{linksadjungiert\index{adjungierter Funktor}} zu $G$ (und $G$ \define{rechtsadjungiert} zu $F$) $\defi$ Es gibt eine natürliche Äquivalenz
	$$\Mor_\mathcal{C}(-, G -) \iso \Mor_\mathcal{D}( F -, -)$$
	von Bifunktoren $\mathcal{C}^\text{op} \times \mathcal{D} \to$ Mengen (bzw. $\mathcal{C}^\text{op} \times \mathcal{D} \to \Z$-Mod im additiven Fall).\\
	Notation: $F $ \rotatebox{90}{$\perp$} $G$
\end{df}
\begin{bsp}
	$F: \text{Mengen} \to K$-VR, $M\mapsto K^{(M)}$, $G: K$-VR $ \to$ Mengen der Vergissfunktor. Es ist $$\Mor_{\text{Mengen}}(M,V) \underset{\text{Bij.}}{\iso} \Mor_{K\text{-VR}}(K^{(M)} , V)$$
	für alle Mengen $M$ und $K$-VR, wobei die naheliegenden Diagramme kommutieren, d.h. wir haben eine natürliche Äquvalent.
	$$\Mor_\text{Mengen}(-, G-) \iso \Mor_{K\text{-VR}}(F -, -)$$
	also $F$ \rotatebox{90}{$\perp$} $G$.
\end{bsp}
\begin{sa}\label{5.33}
	Seien $\A, \mathcal{B}$ abelsche Kategorien, $F:\A \to \mathcal{B}, \, G: \mathcal{B} \to \A$ additive Funktoren mit $F$ \rotatebox{90}{$\perp$} $G$. Dann gilt:
	\begin{enumerate}[label= \alph*)]
		\item $F$ ist rechtsexakt
		\item Ist $F$ exakt, dann überführt $G$ injektive Objekte aus $\mathcal{B}$ in injektive Objekte aus $\A$.
		\item $G$ ist linksexakt
		\item Ist $G$ exakt, dann überführt $F$ projektive Objekte aus $\A$ in projektive Objekte aus $\mathcal{B}$.
	\end{enumerate}
\end{sa}
\begin{proof}
	\begin{enumerate}[label= \alph*)]
		\item Sei $\begin{tikzcd}
		A' \arrow{r} & A \arrow{r} & A'' \arrow{r} & 0
		\end{tikzcd}$ eine exakte Folge in $\A$. Nach \ref{5.26} $c)$ ist 
			$$\begin{tikzcd}
		0 \arrow{r}& \Hom_\A(A'', GB) \arrow{r} & \Hom_\A(A,GB) \arrow{r} & \Hom_\A(A',GB)
		\end{tikzcd}$$
		exakt für alle $B\in \text{Ob }\mathcal{B}$ und , da $F$  \rotatebox{90}{$\perp$} $G$ ist
		$$\begin{tikzcd}
		0 \arrow{r}& \Hom_\mathcal{B}(FA'', B) \arrow{r} & \Hom_\mathcal{B}(FA,B) \arrow{r} & \Hom_\mathcal{B}(FA',B)
		\end{tikzcd}$$
		exakt für alle $B\in \text{Ob }\mathcal{B}$. Damit ist nach \ref{5.26} $c)$
		$$\begin{tikzcd}
		FA' \arrow{r} & FA \arrow{r} & FA'' \arrow{r} & 0
		\end{tikzcd}$$
		exakt.
		\item Sei $I\in \text{Ob } \mathcal{B}$ injektiv. Es ist zu zeigen, dass $GI\in \ObA$ injektiv ist, d.h. der Funktor $\Hom_\A(-, GJ): \A^\text{op}\to \Z$-Mod ist exakt. Allerdings gilt $\Hom_\A(-,GJ ) \overset{\sim}{\Ra} \Hom_\mathcal{B}(F-,J)$ und letzterer ist exakt, da $F$ exakt und $I$ injektiv.
		
	\end{enumerate}
\end{proof}
\begin{df}\label{5.34}
	Seien $\mathcal{C}, \mathcal{D} $ Kategorien, $F: \mathcal{C} \to \mathcal{D} $ ein Funktor. $F$ heißt \define{volltreu\index{volltreuer Funktor}} $\defi $ Für alle $A,B \in \ObC $ ist die Abb $Mor_{\mathcal{C}}(A,B) \to Mor_{\mathcal{D}}(FA,FB), f \mapsto F(f) $ bijektiv.
\end{df}
\begin{sa}[Einbettungssatz von Freyd-Mitchell]\label{5.35}\index{Einbettungssatz von Freyd-Mitchell}
	Sei $\mathcal{A} $ eine abelsche Kategorie (dh. $\ObA$ ist eine Menge) Dann existiert ein Ring $R$ und ein volltreuer exakter Funktor $F: \mathcal{A} \to R$-Mod 
\end{sa}
\begin{anm}
	\begin{itemize}
		\item $F$ induziert eine Äquivalenz zwischen $\mathcal{A}$ und einer vollen Unterkategorie von $R-Mod $ ( das heißt $\mathcal{C} $ ist eine Unterkategorie von $R-Mod $ mit $ Hom_{\mathcal{C}}(A,B)= Hom_{R-Mod}(A,B) $ für alle $ A,B \in \ObC$)
		\item In $\mathcal{A} $ berechnete Kerne und Kokerne entsprechen über diese Äquivalenz Kernen und Kokernen in $R-Mod$. ( Achtung: injektive/projektive Objekte korrespondieren im Allgemeinen nicht zu injektiven/projektiven $R$-Moduln)
	\end{itemize}
\end{anm}
\newpage 
\subsection{Projektive und Injektive Moduln}
%hier fehlen überall die striche bei restklassen
\begin{sa}\label{6.1}
	Sei $\begin{tikzcd}
	0  \arrow{r} & N' \arrow{r}{f} & N \arrow{r}{g} & N'' 
	\end{tikzcd} $ eine Folge von $R$-Moduln. Dann sind äquivalent: 
	\begin{enumerate} [label= \roman*)]
		\item $\begin{tikzcd}
		0  \arrow{r} & N' \arrow{r}{f} & N \arrow{r}{g} & N'' 
		\end{tikzcd} $ ist exakt
		\item Für jeden $R$-Modul $M$ ist die Sequenz abelscher Gruppen  $$\begin{tikzcd}
		0  \arrow{r} & Hom_{R}(M,N') \arrow{r}{f_{*}^{M}} & Hom_{R}(M,N)  \arrow{r}{g_{*}^{M}} & Hom_{R}(M,N'') 
		\end{tikzcd} $$ ist exakt.
	\end{enumerate}
	insbesondere ist der kovariante Funktor $Hom_R(M,-): R-Mod \to \Z$-Mod linksexakt
\end{sa}
\begin{proof}
	$(i) \Ra (ii)$ Sei $\begin{tikzcd}
	0  \arrow{r} & N' \arrow{r}{f} & N \arrow{r}{g} & N'' 
	\end{tikzcd} $ exakt. 
	\begin{enumerate}
		\item Injektivität von $f_{*}^{M}: $ Sei $ \phi \in Hom_{R}(M,N')$ mit $f_{*}^{M}(\phi) = 0 \Ra f \circ \phi = 0$ Wegen $f$ injektiv, folgt $\phi = 0$, also $ \ker f_{*}^{M} = 0$
		\item $\im f_{*}^{M} = \ker g_{*}^{M},$ \\
		"$\subseteq$"' Es  ist $g_{*}^{M} \circ f_{*}^{M} = (g \circ f)_{*}^{M} = 0_{*}^{M} = 0$, also $\im f_{*}^{M} \subseteq \ker g_{*}^{M},$ \\
		"$\supseteq$ "' 
		%hier fehlt was
		Sei $\phi: M \to N $ mit $ \phi \in  g_{*}^{M} \Ra g \circ \phi = 0 \Ra \im \phi  \subseteq \ker g = \im f. $ Setze $\phi^{'}: M \to \im \phi \to \im f  \to N' \Ra \phi^{'} \in Hom_{R}(M,N') $ mit $ f \circ \phi^{'} = \phi \Ra \phi \in \im f_{*}^{M}$.
	\end{enumerate}
	$(ii) \Ra (i)$ Sei $\begin{tikzcd}
	0  \arrow{r} & Hom_{R}(M,N') \arrow{r}{f_{*}^{M}} & Hom_{R}(M,N)  \arrow{r}{g_{*}^{M}} & Hom_{R}(M,N'') \arrow{r} & 0
	\end{tikzcd} $ exakt für alle $R$-Moduln $M$.
	\begin{enumerate}
		\item $f$ injektiv: Setze $M := \ker f , \, \iota: \ker f \to N' $ Inklusion. Dann ist
		 $f_{*}^{M}(\iota)  = f \circ \iota = 0$. Und, da $f_{*}^{M}$ injektiv, ist $ \iota = 0 \Ra \ker f = 0$
		\item $\im f = \ker g: $ \\
		"$\subseteq$"' Setze $M := N' \Ra 0 = 0_{*}^{M}(id_{N'}) = (g_{*}^{M} \circ f_{*}^{M}) (id_{N'}) = ((g \circ f)_{*}^{M})(id_{N'}) = g \circ f \circ id_{N'} = g \circ f $ \\
		"$\supseteq$ "'Setze $M := \ker g , \, \iota: \ker g \to N \Ra g_{*}^{M} (\iota) = g \circ \iota = 0 \Ra i \in \im f_{*}^{M}$ Dann existiert ein $\phi: \ker g \to N' $ mit $ f \circ \phi = \iota. $ Somit: $ x \in \ker g \Ra x = \iota(x) = f(\phi(x)) \in \im f.$
	\end{enumerate}
\end{proof}
\begin{anm}
	Der kovariante Funktor $\Hom_R(M,-) $ ist im Allgemeinen nicht exakt.
\end{anm}
\begin{bsp}
	Sei $ R = \Z  ,M = \QR{\Z}{2\Z}. $ Wir betrachten die exakte Sequenz  \\ $$\begin{tikzcd}
	0  \arrow{r} & \Z \arrow{r} & \Z \arrow{r} & \QR{\Z}{2\Z} \arrow{r} & 0
	\end{tikzcd}$$ von $\Z$-Moduln  mit $f: \Z \to \Z, x \mapsto 2x, \pi $ kanonische projektion. Die Abbildung $ \pi_{*}^{M}: Hom_{\Z}\left(\QR{\Z}{2\Z}, \Z\right) \to Hom_{\Z}\left(\QR{\Z}{2\Z},\QR{\Z}{2\Z}\right) $ ist nicht surjektiv, denn:  Für $ \phi \in Hom_{\Z}(\QR{\Z}{2\Z}, \Z) $ gilt: 
	$$0 = \phi(0) = \phi( 1 +1) = \phi(2\cdot 1) = 2\phi(1)$$, also $ \phi(1)= 0$, das heißt $\phi = 0$. Insbesondere ist $\pi_{*}^{M}(\phi) = \pi_{*}^{M}(0) =0 \neq \id_{\QR{\Z}{2\Z}} . $ Mit anderen Worten $\QR{\Z}{2\Z}$ ist kein projektiver $\Z$-Modul. 
\end{bsp}
\begin{sa}\label{6.3}
	Sei $P$ ein $R$-Modul. Dann sind äquivalent:
	\begin{enumerate} [label= \roman*)]
		\item $P$ ist ein projektiver $R$-Modul 
		\item $\Hom_R(P,-): R$-Mod $\to \Z$-Mod ist exakt.
		\item Für jeden Epimorphisumus $ \pi: M \to N $ von $R$-Moduln und jeden Hom $ \phi: P \to N $ existiert ein Homomorphismus $\psi: P \to M $ mit $ \pi \circ \psi = \phi$
		$$\begin{tikzcd}
		& P \arrow[swap, dashed]{dl}{\psi} \arrow{d}{\phi} \\
		M \arrow{r}{\pi} & N
		\end{tikzcd}$$
		\item Jede kurze exakte Sequenz $\begin{tikzcd}
		0  \arrow{r} & L \arrow{r}{f} & M \arrow{r}{g} & P \arrow{r} & 0
		\end{tikzcd} $ von $R$-Moduln spaltet.
		\item Es gibt einen $R$-Modul $P'$, sodass $ P \oplus P' $ ein freier $R$-Modul ist (das heißt $P$ ist direkter Summand eines freien $R$-Moduls )
	\end{enumerate}
\end{sa}
\begin{proof}
	$(i) \defi (ii) \defi (iii) $ folgt aus Definition 5.30. \\
	$(iii) \Ra (iv) $ Sei $\begin{tikzcd}
	0  \arrow{r} & L \arrow{r} & M \arrow{r} & P \arrow{r} & 0
	\end{tikzcd} $
	%hier fehlt diagramm 2
	 eine kurze exakte Sequenz von $R$-Moduln. Nach $(iii)$ existiert zu dem Epimorphismus $ g: M \to P $ und dem Homomorphismus $\id_P: P \to P $ ein Homomorphismus $ \psi: P \to M $  mit $ g \circ \psi = \id_P$, das heißt die Sequenz spaltet. 
	 	$$\begin{tikzcd}
	 0 \arrow{r} & L \arrow{r}{f} & M \arrow{r}{g} & P \arrow{r} & 0 \\
	 & & & P \arrow[dashed]{ul}{\psi} \arrow[swap]{u}{\id} & 
	 \end{tikzcd}$$
	$(iv) \Ra (v) $ Es existiert ein freier $R$-Modul $F$ und ein Epimorphismus $ f: F \to P $. Wir erhalten eine exakte Sequenz $\begin{tikzcd}
	0  \arrow{r} & \ker f \arrow{r} & F \arrow{r} & P \arrow{r} & 0
	\end{tikzcd} $, diese spaltet nach $(iv)$ , das heißt $ F \simeq P \oplus \ker f $\\
	
	\begin{minipage}[t]{0.7\textwidth}
		$(v) \Ra (iii)$ Sei $ \pi: M \to N$ ein Epimorphismus von $R$-Moduln, $\phi: P \to N $ ein Homomorphimus. 
		Wegen $(v)$ existiert ein $R$-Modul $P' $ sodass $F:= P \oplus P' $ frei ist, Setze $$\phi': F \to N, \quad  (x,y) \mapsto \phi(x)$$
		Sei $ (b_i)_{i \in I} $ eine Basis von F, wähle für $i \in I $ jeweils ein $ z_i \in \pi^{-1}(\phi'(b_i)) .$ Durch $\psi': F \to M , b_i \mapsto z_i $ wird ein Homomorphismus definiert mit $ \pi \circ \psi' = \phi'.$ Setze
	\end{minipage}
	\begin{minipage}[t]{0.3\textwidth} 
		$$\begin{tikzcd}[column sep = large, row sep = large]
	P \oplus P' = F \arrow{dr}{\phi '} \arrow{d}{\psi'} \arrow{r} & P \arrow{d}{\phi} \arrow[dashed, bend right = 50]{l} \\
	M \arrow{r}{\pi} & N
	\end{tikzcd}$$
	\end{minipage}
 $$\psi: P \to M ,\quad  x \mapsto \psi'((x,0))$$ dann gilt für $x \in P: \pi(\psi(x)) = \pi(\psi'((x,0)))= \phi'((x,0)) = \phi(x) $ das heißt $\pi \circ \psi = \phi.$
\end{proof}
\begin{fo}\label{6.4}
	\begin{enumerate} [label= \alph*)]
		\item Jeder freie $R$-Modul ist ein projektiver R-Modul
		\item Jeder $R$-Modul ist ein Faktormodul eines projektiven $R$-Moduls. 
	\end{enumerate}
\end{fo}
\begin{proof}
	\begin{enumerate} [label= \alph*)]
		\item klar nach \ref{6.3}.
		\item da jeder $R$-Modul Faktormodul eines freien $R$-Moduls ist. 
	\end{enumerate}
\end{proof}
\begin{sa}\label{6.5}
	Sei $\begin{tikzcd}
	 M' \arrow{r}{f} & M \arrow{r}{g} & M'' \arrow{r} & 0
	\end{tikzcd} $ eine Sequenz von $R$-Moduln. Dann sind äquivalent:
	\begin{enumerate} [label= \roman*)]
		\item $\begin{tikzcd}
		M' \arrow{r}{f} & M \arrow{r}{g} & M'' \arrow{r} & 0
		\end{tikzcd} $ ist exakt.
		\item Für jeden $R$-Modul $N$ ist die Sequenz abelscher Gruppen: \\
		$\begin{tikzcd}
		0  \arrow{r} & Hom_{R}(M'',N) \arrow{r}{g_{N}^{*}} & Hom_{R}(M,N)  \arrow{r}{f_{N}^{*}} & Hom_{R}(M',N) \end{tikzcd}$ exakt.
	\end{enumerate}
	Insbesondere ist der kontravariante Funktor: $Hom_R(-,N): R-Mod^{op} \to \Z$-Mod  linksexakt.
\end{sa}
\begin{proof}
	Übungsaufgabe.
\end{proof}
\begin{anm}
	Der kontravariante Funktor $Hom_R(-,N) $ ist im Allgemeinen nicht exakt.
\end{anm}
\begin{bsp}
	Sei $ R= \Z , N = \Z $. Wir betrachten die exakte Sequenz \\
	 $$\begin{tikzcd}
	0  \arrow{r} & \Z \arrow{r}{f} & \Z \arrow{r}{\pi} & \QR{\Z}{2\Z} \arrow{r} & 0
	\end{tikzcd} $$ von $\Z$-Moduln mit $f: \Z \to \Z ,\; x \mapsto 2x$ und $ \pi $ der kanonischen Projektion. Die Abbildung $ f_{\Z}^{*} : Hom_{\Z}(\Z,\Z) \to Hom_{\Z}(\Z,\Z)$ ist nicht surjektiv, denn für alle $\phi \in Hom_{\Z}(\Z,\Z)$ ist $$ (f_{\Z}^{*}(\phi))(x) = (\phi \circ f)(x) = \phi(2x) =2 \phi(x) \in 2\Z$$
	insbesondere ist $ f_{\Z}^{*}(\phi) \neq id_{\Z}$. Mit anderen Worten: $\Z$ ist kein injektiver $\Z$-Modul.
\end{bsp}
\begin{sa}\label{6.7}
	Sei $Q$ ein $R$-Modul. Dann sind äquivalent:
	\begin{enumerate} [label= \roman*)]
		\item $Q$ ist ein injektiver $R$-Modul 
		\item $Hom_{R}(-,Q): R-$Mod $\to \Z-$Mod ist exakt.
		\item Für jeden Monomorphismus $\iota: L \to M $ von $R$-Moduln und jedem Homomorphismus $\phi: L \to Q $ exsistiert ein Homomorphismus $\psi: M \to Q $ von $R$-Moduln  mit $ \psi \circ \iota = \phi $
		$$\begin{tikzcd}
		L \arrow{r} {\iota} \arrow{d}& M \arrow[dashed]{dl}{\psi} \\
		Q' &
		\end{tikzcd}
		$$
		\item Jede kurze exakte Sequenz $\begin{tikzcd}
		0  \arrow{r} & Q \arrow{r} & M \arrow{r} & N \arrow{r} & 0
		\end{tikzcd} $ von $R$-Moduln spaltet.
	\end{enumerate}
\end{sa}
\begin{proof}
	$(i) \Lra (ii) \Lra (iii) $ folgt aus \ref{5.29} \\
	$(iii) \Ra (iv) $ Sei $\begin{tikzcd}
	0  \arrow{r} & L \arrow{r} & M \arrow{r} & P \arrow{r} & 0
	\end{tikzcd} $ eine exakte Sequenz von $R$-Moduln.
	Nach $(iii)$ existiert zum Monomorphismus $f: Q \to M $ von $R$-Moduln und zum Homomorphismus $ id_Q : Q \to Q $ ein Homomorphismus $\psi: M \to Q$ mit $ \psi \circ f = id_Q $. das heißt die Sequenz spaltet.
	$$\begin{tikzcd}
	0\arrow{r} & Q \arrow[swap]{d}{\id_Q} \arrow{r}{f} & M \arrow[dashed]{dl}{\psi} \arrow{r}{g} & N \arrow{r} & 0\\& Q & & &
	\end{tikzcd}
	$$ 
	$(iv) \Ra (iii) $ Sei $\iota : L \to M $ ein Monomorphismus, $ \phi: L \to Q $ ein Homomorphismus von $R$-Moduln. Setze 
 \begin{align*} S &:= \{ (\phi(x), - \iota(x))| \, x \in L\} \subseteq Q \oplus M & M' &:= \QR{(Q \oplus M)}{S},& N &:= \QR{M}{\im \iota}\end{align*} $\pi: M \to N$  kanonische Projektion. 
	\begin{enumerate}
		\item  Wir erhalten eine exakte Sequenz
		 $$\begin{tikzcd}[row sep = 0.1ex]	0  \arrow{r} & Q \arrow{r}{f} & M' \arrow{r}{g} & N \arrow{r} & 0 \\
		 & y \arrow[mapsto]{r} & \bar{(y,0)} & & \\
		 & & \bar{(y,z)} \arrow[mapsto]{r} & \pi(z) & 
		\end{tikzcd} $$
		 von $R$-Moduln. denn: 
			\begin{itemize}
			\item $g$ ist wohldefiniert, denn: $\pi \circ \iota = 0$
			\item $f$ ist injektiv, denn $ (y,0) = (0,0)  \Ra $ Es existiert ein $ x \in L $ mit $ y= \phi(x), 0 = - \iota(x)$. Wegen $\iota$ injektiv, folgt $ x = 0 \Ra y = \phi(0) = 0 $
			\item $g$ surjektiv, klar
			\item $\im f = \ker g: $\\ "'$\subseteq$"' klar, wegen $g \circ f = 0 $\\ 
			"'$\subseteq$"' Sei $\bar{(y,z)} \in \ker g \Ra \pi(z) = 0 \Ra z \in \im \iota,$ das heißt es existiert ein $ x \in L $  mit $ z = \iota(x) = -\iota(-x)$Dann gilt
			\begin{eqnarray*}
				 \bar{(y,z)} &=& \bar{(y, -\iota(-x))} = \bar{(y+\phi(x), 0)} +\bar{(\phi(-x), -\iota(-x))}\\
				 & =& \bar{(y+\phi(x), 0)} = f(y+ \phi(x)) \in \im f.
			\end{eqnarray*}
		\end{itemize}
	\item Wegen (iv) spaltet die Sequenz, das heißt es existiert ein R-Modulhomomorphisumus $ h: M' \to Q $ mit $h \circ f = id_Q$. Setze $$\psi: M \to Q, z \mapsto h((0,z))$$
	$\psi $ ist ein $R$-Modulhomomorphismus. Für $x \in L $ ist \begin{eqnarray*}
		(\psi \circ \iota)(x)& =& h((0,\iota(x))) = h((0, \iota(x))) + h(\phi(x),-\iota(x))\\
		& =& h((\phi(x),0)) = h(f(\phi(x))) = \phi(x)
		\end{eqnarray*}
	Also ist $\psi \circ \iota = \phi$
	
	\end{enumerate}
\end{proof}
\begin{bsp}
	Sei $K$ ein Körper und $V$ ein $K$-Vektorraum. Dann ist $V$ ein injektiver $K$-Modul, denn für jede exakte Folge $\begin{tikzcd}	0  \arrow{r} & V \arrow{r}{f} & M \arrow{r}{g} & N \arrow{r} & 0 \end{tikzcd}$ von $K$-Moduln, ist $N$ ein freier $K$-Modul, d.h. die Folge spaltet.
\end{bsp}
\begin{sa}[Baer-Kriterium]\label{6.9}\index{Baer-Kriterium} Sei $Q$ ein $R$-Modul. Dann sind äquivalent: 
	\begin{enumerate} [label= \roman*)]
		\item $Q$ ist ein injektiver $R$-Modul
		\item Für jedes Linksideal $I \subseteq R $ und jede $R$-lineare Abbildung $\phi: I \to Q $ existiert eine $R$-lineare Abbildung $\psi: R \to Q $ mit $ \psi\big|_{I} = \phi$. 
		$$\begin{tikzcd}
		I \arrow[hook]{r}\arrow[swap]{d}{\phi} & R\arrow[dashed]{dl}{\psi} \\Q &
		\end{tikzcd}$$
	\end{enumerate}
\end{sa}
\begin{proof}
	$(i) \Ra (ii) $ Betrachte Diagramm
	$$\begin{tikzcd}
	I \arrow[hook]{r}{\iota}\arrow[swap]{d}{\phi} & R\arrow[dashed]{dl}{\psi} \\Q &
	\end{tikzcd}$$  Da $Q$ injektiv, Exsistert ein $R$-Modulhomomorphismus $\psi: R \to Q $ mit $ \phi = \psi \circ \iota = \psi$. 
	$(ii) \Ra (i) $ Sei $\iota: L \to M $ ein Monomorphismus von $R$-Moduln, $\phi: L \to Q $.
	$$\begin{tikzcd}
	L \arrow{r}{\iota}\arrow[swap]{d}{\phi} & M\arrow[dashed]{dl}{?} \\Q &
	\end{tikzcd}$$ 
	 Ohne Einschränkung sei $L \subseteq M $ ein Untermodul, $\iota$ Inklusionsabbildung. 
	\begin{enumerate}
		\item Setze $\mathcal{X} := \{ (L',\phi')|\, L' \subseteq M \text{ Untermodul mit } L \subseteq L' , \phi': L' \to Q$ $R$-linear mit $\phi'\big|_L = \phi\}$. Dann ist $\mathcal{X} \neq \emptyset,$ denn: $ (L,\phi) \in \mathcal{X} $. Auf $\mathcal{X} $ ist die  Halbordung "'$\leq$"' durch $$ (L', \phi') \leq (L'',\phi'') \Lra L' \subseteq L'',\, \phi''\big|_{L'} = \phi' $$ erklärt. $\mathcal{X} $ ist induktiv geordnet bzgl "'$\leq$"', denn: 
		Sei $ (L_i,\phi_i)_{i \in I } $ eine totalgeordnete Familie von Elementen aus $\mathcal{X}. $ Setze $ L' := \bigcup_{i \in I} L_i$. $L'$ ist Untermodul von $M$ ( beachte: $a,b \in L' \Ra $ Es existieren $i,j$ mit $ a \in L_i, b \in L_j, $ohne Einschränkung: $ L_i \subseteq L_j  \Ra a+b \in L_j \subseteq L' $) und es ist $L \subseteq L'.$ Außerdem kann die $R$-lineare Abbildung $\phi':L' \to Q $ mit $\phi'|_{L_i} := \phi_i $ für alle $i \in I $ definieren. (wohldefiniert, denn: Für $i,j \in I$, ohne Einschränkung: $(L_i,\phi_i) \subseteq (L_j, \phi_j) $ ist $\phi_j|_{L_i} = \phi _i)$ $\Ra  (L',\phi') $ ist obere Schranke für die Familie$(L_i,\phi_i)_{i \in I}. $ Mit dem Zornshen Lemma folgt, dass ein maximales Element $(L',\phi')$ in $\mathcal{X}$ exsitiert.
		\item Behauptung: $L' = M $, denn: \\
		\begin{minipage}[t]{0.7\textwidth}
		Sei $x \in M $. Setze $I:= \{a \in R|\, ax \in L'\} \subseteq R\cdot I $ ist Linksideal in $R$, und die Abbildung $f:I \to Q, a \mapsto \phi'(ax) $ ist $R$-linear. Mit (ii) folgt, dass eine $R$-lineare Abbildung $g:R \to Q $ mit $ g|_I=f. $ Setze:
		\end{minipage}
		\begin{minipage}[t]{0.3\textwidth} 
			$$\begin{tikzcd}
			I \arrow[hook]{r} \arrow{d}{f}& R \arrow[dashed,red]{dl}{g}\\ Q &
			\end{tikzcd}$$
		\end{minipage}
	 $$\psi': L' \oplus R \to Q,\quad (y,a) \mapsto \phi'(y) + g(a) $$
	und 
	$$\pi: L' \oplus R \to M,\quad(y,a) \mapsto y +ax$$
		welche beide $r$-Modulhomomorphismen sind. Es ist $\psi'(\ker\pi) = 0$, denn für $(y,a) \in \ker\pi$ ist $ y+ax=0,$ also $ax=-y \in L'$, das heißt $a \in I,$ also $g(a)=f(a)=\phi'(ax)=\phi'(-y) = -\phi'(y) $und somit $\psi'(y,a) = \phi'(y)+g(a) = 0 \Ra \psi'$ induziert $R$-Modulhomomorphismus $$\QR{L' \oplus R}{\ker\pi} \to Q, \quad(y,a) \mapsto \phi'(y)+g(a)$$
		 Außerdem ist $$L'+Rx = \im\pi \simeq \QR{L' \oplus R}{\ker\pi} \quad \text{via}\quad y+ax \mapsto (y,a)$$
		 Wir erhalten den Homomorphismus $$\psi:L'+Rx \to Q \quad \text{mit} \quad\psi(y+ax) = \phi'(y)+g(a) $$ für alle $a \in R, y \in L' $, das heißt $\psi|_{L'} = \phi' \Ra (L',\phi') \leq (L'+Rx, \psi) $ Wegen $(L',\phi') $ maximal folgt dass $ L'=L'+Rx \Ra x \in L' $. Somit $M \subseteq L' \subseteq M, $ also $M = L'$.
	\end{enumerate}
\end{proof}
\begin{df}\label{6.10}
	Sei $A$ ein Integritätsbereich (kommutativer nullteilerfreier Ring), $M$ ein $A$-Modul. $M$ heißt \define{teilbar\index{teilbarer Modul}} $\defi$ Für alle $a\in A \backslash \{0\}$ ist $ aM=M. \Lra$ Für alle $x \in M, a \in A \backslash \{0\}$ existiert ein $y\in M $ mit $x=ay$.
\end{df}
\begin{bem}\label{6.11}
	Sei $A$ ein Integritätsbereich, $M$ ein injektiver $A$-Modul. Dann ist $M$ teilbar.
\end{bem}
\begin{proof}
	Sei $x \in M , a \in A \backslash \{0\}$. Wir betrachten die Abbildung $$\phi: Aa \to M, \quad ra \mapsto rx$$ $\phi$ ist wohldefiniert, denn: $ r_1a= r_2a \Ra (r_1-r_2)a =0 $ Da $A$ nullterilerfrei ist folgt $r_1=r_2 $. $ \phi$ ist $A$-linear, so folgt mit Satz \ref{6.9}, dass eine $A$-lineare Abbildung $\psi: A \to M $ mit $\psi|_{Aa} = \phi$. Setze $y:= \psi(1)$, dann ist $ x = \phi(a)=\psi(a) = \psi(a 1) = a\psi(1)=ay.$
\end{proof}
\begin{bem}\label{6.12}
	Sei $A$ ein Hauptidealring, $M$ ein $A$-Modul. Dann sind äquivalent: 
	\begin{enumerate} [label= \roman*)]
		\item $M$ injektiv 
		\item $M$ teilbar 
	\end{enumerate}
\end{bem}
\begin{proof}
	$(i) \Ra (ii) $ aus \ref{6.11} \\
	$(ii) \Ra (i) $ Sei $ I \subseteq A $ ein Ideal, $\phi: I \to M $ $A$ linear. Falls $ I = 0 $, dann wird $\phi $ durch die Nullabbildung nach $A$ fortgesetzt. Im Folgendem sein $ I \neq 0 $. Da $A$ ein Hauptidealring ist, existieren $ a\in A, a\neq 0 $ mit $I =Aa$. Setze $ x:= \phi(a) \Ra \phi(ra)=r\phi(a) = rx $ für alle $ r \in A$. Wegen (ii) existiert ein $y \in M $ mit $ x = ay $. Setze $$\psi: A \to M,\quad  r \mapsto ry$$
	Dann ist $\psi$ $A$-linear und $ \psi(ra)=  ray = rx = \phi(ra) $ für alle $r \in A $ das heißt $\psi|_{Aa} = \phi.$ Dann folgt aus 6.9 $M$ ist injektiv. 
\end{proof}
\begin{bsp}
	\begin{enumerate} [label= \alph*)]
		\item Sei $K$ ein Körper, $V$ ein $K$-VR $\Ra V$ ist teilbarer $K$-Modul, also injektiver $K$-Modul. Ist $\cha K = 0 $ dann ist $V$ teilbarer $\Z$-Modul, also injektiver $\Z$-Modul. 
		\item Faktormoduln teilbarer $\Z$-Moduln sind teilbar, somit sind Faktormoduln inketiver $\Z$-Moduln wieder injektive $\Z$-Moduln. 
		\item Nach (a) sind $\Q, \R $ injektive $\Z$-Moduln, nach (b) also auch $\QR{\Q}{\Z}, \QR{\R}{\Z}$
	\end{enumerate}
\end{bsp}
\begin{Ziel}
	injektive $R$-Moduln sind dierekte Faktoren von kofreien $R$-Moduln 
\end{Ziel} \\
\begin{anm}
	$M$ ein $\Z$-Modul. Dann ist $Hom_{\Z}(R,M)$ via $(a\phi)(r) = \phi(ra) $ ein $R$-Modul. (beachte: $b(a\phi)(r) = (a\phi)(rb) = \phi(rba) = ((ba)\phi)(r) $)
\end{anm}
\begin{bem}\label{6.14}
	Sei $M$ ein injektiver $\Z$-Modul. Dann ist $Hom_{\Z}(R,M)$ ein injektiver $R$-Modul. Insbesondere ist $R^{v} := Hom_{\Z}(R, \QR{\Q}{\Z}) $ ein injektiver $R$-Modul.
\end{bem}
\begin{proof}
	Sei $ I \subseteq R $ ein Linksideal, $\phi: I \to Hom_{\Z}(R,M) $ $R$-linear. Nach \ref{6.9}, genügt es zu zeigen: $\phi$ lässt sich auf $R$ fortsetzen. Setze $$f: I \to M ,\quad a \mapsto \phi(a)(1)$$ Dann ist $f $ ist $\Z$-linear und für $r \in R , a \in I $ gilt: $ f(ra) = \phi(ra)(1)= (r\phi(a))(1) = \phi(a)(1 r) = \phi(a)(r)$. Da $M$ ein injektiver $\Z$-Modul, existiert eine $\Z$-lineare Abbildung $g: R \to M $ mit $g\big|_I=f $. Wir setzen $\psi: R \to Hom_{\Z}(R,M), a \mapsto ag $. $\psi$ ist $R$-linear, und für $a\in I, r \in R $ ist $\psi(a)(r) =(ag)(r) = g(ra) = f(ra) = \phi(a)(r)$, das heißt $\psi\big|_{I}=\phi.$
\end{proof}
\begin{df}\label{6.15}
	Sei $M$ ein $R$-Modul. $M$ heißt \define{kofrei\index{kofreier Modul}} $\defi$ Es existiert eine Menge $ I$ mit $M \simeq (R^{v})^I = \prod_{i \in I} R^{v} $.
\end{df}
\begin{bem}\label{6.16}
	Sei $(M_i)_{i \in I} $eine Familie von $R$-Moduln. Dann gilt:
	\begin{enumerate} [label= \alph*)]
		\item $\bigoplus_{i \in I} M_i $ ist ein projektiver $R$-Modul $\Lra M_i $ projektive $R$-Moduln für alle $ i \in I $.
		\item $\prod_{i \in I} M_i $ ist ein injektiver R-Modul $\Lra$ $M_i $ ist injektiver $R$-Modul für alle $i  \in I $.
	\end{enumerate}
\end{bem}
\begin{proof}
	Übungsaufgaben.
\end{proof}
\begin{sa}\label{6.17}
	Sei $M$ ein kofreier $R$-Modul. dann ist $M$ ein injektiver $R$-Modul. 
\end{sa}
\begin{proof}
	folgt direkt aus \ref{6.16} und \ref{6.14}
\end{proof}
\begin{bem}\label{6.18}
	Sei $M$ ein $R$-Modul, $m \in M , m \neq 0 $. Dann existiert ein $R$-Modulhomomorphimus $\phi: M \to R^{v} $ mit $\phi(m) \neq 0.$
\end{bem}
\begin{proof}
	\begin{enumerate}
		\item Die Abbildung $$\theta: \Hom_{\Z}(M ,\QR{\Q}{\Z}) \to \Hom_{R}(M, R^{v}), \psi \mapsto (m \mapsto \phi_m: R \to \QR{\Q}{\Z}, r \mapsto \psi(rm))$$
		ist ein Homomorphismus von $\Z$-Moduln (tatsächlich sogar ein Isomorphismus).
		\item Ist $\psi: M \to \QR{\Q}{\Z} $ ein $\Z$-Modulhomomorphismus mit $ \psi(m) \neq 0$ dann ist $\theta(\psi)(m) = \phi_m \neq 0$ wegen $\phi_m(1)= \psi(m) \neq 0, $ das heißt: $\theta(\psi): M \to R^{v} $ ist ein $R$-Modulhomomorphismus mit $\theta(\psi)(m) \neq 0$
		\item Nach 2 genügt es zu zeigen: Es existiert ein $\Z$-Modulhomomorphimus $\psi: M \to \QR{\Q}{\Z} $ mit $ \psi(m) \neq 0. $ Setze $N := \langle m\rangle_{\Z}$. \\
		1.Fall: $N \simeq \QR{\Z}{n\Z} $ für ein $n \in N $. Setze
		$$\begin{tikzcd}[column sep = small, row sep = 0.1 ex]
		\tilde{\psi}: N \arrow{r}{\sim} & \QR{\Z}{n \Z} \arrow{r} & \QR{\Q}{\Z} \\
		& 1 \arrow[mapsto]{r} & \frac{1}{n} + \Z
		\end{tikzcd}$$
		 Dann ist $\psi^{~}(m)\neq 0 $ und, da $\QR{\Q}{\Z} $ injektiver $\Z$-Modul ist, setzt sich $\tilde{\psi} $ auf $M$ fort.\\
		2. Fall: $N \cong \Z$. Setze dann 
		$$\begin{tikzcd}[column sep = small, row sep = 0.1 ex]
		\tilde{\psi}: N \arrow{r}{\sim} & \Z \arrow{r} & \QR{\Q}{\Z} \\
		& 1 \arrow[mapsto]{r} & \frac{1}{2} + \Z
		\end{tikzcd}$$
		Dann ist $\tilde{\psi} (m) \neq 0$, also weiter wie in Fall 1.
	\end{enumerate}, 
\end{proof}
\begin{sa}\label{6.19}
	Jeder $R$-Modul ist Untermodul eines kofreien, also insbesondere eines injektiven, $R$-Moduls.
\end{sa}
\begin{proof}
	Sei $0\neq M$ ein $R$-Modul. Nach 6.18 existiert zu jedem $m\in M$ ein $R$-Modulhomomorphismus $\phi_m:M \to R^v$ mit $\phi_m(m) \neq 0$. Wir setzen
	$$f:M \longrightarrow \prod_{m\in M\backslash \{0\}} R^v, \quad x \mapsto ((\phi_m(x))_{m\in M \backslash \{0\}}$$
	Dann gilt
	\begin{itemize}
		\item $f$ ist ein $R$-Modulhomomorphismus
		\item $f$ ist injekitv, denn: Sei $x\in M$ mit $f(x) = 0$. Dann ist $\phi_m(x) = 0$ für alle $m\in M\backslash \{0\}$. Wäre $x\neq 0$, dann wäre $\phi_x(x) = 0$, Widerspruch!
	\end{itemize}
\end{proof}
\begin{fo}\label{6.20}
	Sei $Q$ ein $R$-Modul. Dann sind äquivalent:
	\begin{enumerate}[label= \roman*)]
		\item $Q$ ist injektiv
		\item Es gibt einen $R$-Modul $Q'$, sodass $Q \times Q'$ ein kofreier $R$-Modul ist (d.h. $Q$ ist direkter Faktor eines kofreien $R$-Moduls)
	\end{enumerate}
\end{fo}
\begin{proof}
	$i) \Ra ii)$ Nach \ref{6.19} existiert ein kofreier $R$-Modul $N$, sodass $Q$ Untermodul von $N$ ist. Die exakte Folge 
	$$\begin{tikzcd}
	0 \arrow{r} & Q \arrow{r} & N \arrow{r} & \QR{N}{Q} \arrow{r} & 0
	\end{tikzcd}$$
	spaltet nach \ref{6.7}, da $Q$ injektiv ist, d.h. $N \cong Q \oplus \QR{N}{Q}  = Q \times \QR{N}{Q}$\\
	$ii) \Ra i)$ Ist $Q\times Q'$ kofrei, dann ist nach 6.17 $Q \times Q'$ injektiv und nach 6.16 $Q$ injektiv.
\end{proof}
\newpage
\subsection{Komplexe}
\begin{center}
	\textbf{In diesem Abschnitt sei $\mathcal{A}$ stets eine abelsche Kategorie}
\end{center}
\begin{df}\label{7.1}
	Ein \define{Komplex\index{Komplex}} $A^{^\bullet}$ in $\mathcal{A}$ ist eine Familie $(A^i, d_i)_{i\in \Z}$ von Objekten $A^i \in \ObA$ und Morphismen $d_i: A^i \to A^{i+1}$ (\define{Differentiale\index{Differential}})
	$$\begin{tikzcd}
	\ldots \arrow{r} & A^{-1} \arrow{r}{d_{-1}}& A^0 \arrow{r}{d_0} & A^1 \arrow{r}{d_1} & A^2 \arrow{r} & \ldots
	\end{tikzcd}$$
	sodass $d_i \circ d_{i-1} =0$ für alle $i\in \Z$ gilt. Ein \define{Komplexhomomorphismus\index{Komplexhomomorphismus}}  $f:A^{^\bullet} \to B^{^\bullet}$ in einem Komplexe $B^{^\bullet}$ in $\mathcal{A}$ ist eine Familie $f=(f_i)_{i\in \Z}$ von Homomorphismen $f_i : A^i \to B^i$, sodass für alle $i\in \Z$ gilt:
	$$d_i \circ f_i = f_{i+1} \circ d_i$$
	d.h. das Diagramm 
	$$\begin{tikzcd}
	\ldots \arrow{r} & A^{i-1}\arrow{d}{f_{i-1}} \arrow{r}{d_{i-1}}& A^i \arrow{d}{f_i}\arrow{r}{d_i} & A^{i+1} \arrow{d}{f_{i+1}} \arrow{r}{d_{i+1}} & \ldots\\
	\ldots \arrow{r} & B^{i-1} \arrow[swap]{r}{d_{i-1}} & B^i \arrow[swap]{r}{d_i} & B^{i+1} \arrow{r} & \ldots
	\end{tikzcd}$$
	kommutiert.
\end{df}
\begin{anm}
	Komplexe in $\mathcal{A}$ zusammen mit Komplexhomomorphismenbilden bilden eine abelsche Kategorie (Kerne, Kokerne, endliche Produkte separat an jeder Stelle bilden).
\end{anm}
\begin{bem}\label{7.2}
	Sei $A^{^\bullet}$ ein Komplex in $\mathcal{A}$. Dann induzieren die Differentiale in natürlicher Weise Monomorphismen $\im d_{i-1} \longrightarrow \ker d_i$ für $i\in \Z$.
\end{bem}
\begin{proof}
	Wir betrachten das Diagramm
	$$\begin{tikzcd}[column sep = large]
	A^{i-1} \arrow[swap]{d}{q_{i-1}} \arrow{r}{d_{i-1}} & A^i \arrow{r}{d_i} & A^{i+1} \\
	\text{coim} \, d_{i-1} \arrow[swap]{r}{\bar{d}_{i-1}} & \im d_{i-1} \arrow{u}{k_{i-1}}\arrow[dashed]{r}{l_i} & \ker d_i \arrow[swap]{ul}{j_i}
	\end{tikzcd}$$
	(Es ist $k_{i-1}$ ein Mono-, $q_{i-1}$ ein Epi- und der durch den Homomorphiesatz induzierte Pfeil $\bar{d}_{i-1}$ ein Isomorphismus). Damit ist $0= d_i \circ d_{i-1} = d_i \circ k_{i-1} \circ \bar{d}_{i-1} \circ q_{i-1}$ und, da $q_{i-1}$ Epi, $d_{i-1}$ Iso, folgt $d_i \circ k_{i-1} =0$. Nach der Universellen Eigenschaft des Kerns existiert ein $l_i : \im d_{i-1} \to \ker d_i$ mit $k_{i-1} = j_i \circ l_i$. Nun ist $l_i$ ein Monomorphismus, da $k_{i-1} = j_i \circ l_i$ Monomorphismus.
\end{proof}
\begin{df}\label{7.3}
	Sei $A^{^\bullet}$ ein Komplex in $\mathcal{A}$.
	\begin{enumerate}
		\item[] $\mathcal{Z}^i(A^{^\bullet}):= \ker d_i$ \hfill (\define{$i$-Kozykel}\index{Kozykel})
		\item[] $\mathcal{B}^i(A^{^\bullet}) := \im d_{i-1}$ \hfill (\define{$i$-Koränder\index{Korand}})
		\item[] $\mathcal{H}^i(A^{^\bullet}) := \coker (\im d_{i-1} \to \ker d_i)$ \hfill (\define{$i$-te Kohomologie\index{Kohomologie}})\\
	\noindent\hspace*{14mm}	$ = \coker(\mathcal{B}^i(A^{^\bullet}) \to \mathcal{Z}^i(A^{^\bullet}))$
	\end{enumerate}
\end{df}
\begin{anm} 
	Ein Komplexhomomorphismus $f:A^{^\bullet} \to B^{^\bullet}$ induziert Homomorphismen
	 \begin{align*}
	\mathcal{Z}^i(f)&: Z^i(A^{^\bullet}) \to \mathcal{Z}^i(B^{^\bullet}),& \mathcal{B}^i(f)&: \mathcal{B}^iA^{^\bullet} \to \mathcal{B}^i (B^{^\bullet}),& \mathcal{H}^i(f)&: \mathcal{H}^i(A^{^\bullet}) \to \mathcal{H}^i(B^{^\bullet})
	\end{align*}
\end{anm}
\begin{sa}[Lange exakte Kohomologiefolge]\label{7.4}\index{Lange exakte Kohomologiefolge}
	Sei $$\begin{tikzcd}
	0 \arrow{r} & A^{^\bullet} \arrow{r} & B^{^\bullet} \arrow{r} & C^{^\bullet} \arrow{r} & 0
	\end{tikzcd}$$
	eine kurze exakte Folge von Komplexen in $\mathcal{A}$ (d.h. die Morphisemen sind Komplexhomomorphismen und für jedes $i\in \Z$ ist 
	$$\begin{tikzcd}
	0 \arrow{r} & A^i \arrow{r} & B^i \arrow{r} & C^i \arrow{r} & 0
	\end{tikzcd}$$
	exakt). Dann existiert eine natürliche lange exakte Folge 
	$$\begin{tikzcd}[column sep = small]
	\ldots \arrow{r} & \mathcal{H}^i(A^{^\bullet}) \arrow{r} & \mathcal{H}^i(B^{^\bullet}) \arrow{r} & \mathcal{H}^i(C^{^\bullet}) \arrow{r} & \mathcal{H}^{i+1}(A^{^\bullet}) \arrow{r}& \mathcal{H}^{i+1}(B^{^\bullet}) \arrow{r} & \mathcal{H}^{i+1}(C^{^\bullet}) \arrow{r}& \ldots
	\end{tikzcd}$$
\end{sa}
\begin{proof}[Beweisskizze]
	\begin{enumerate}
		\item $M^{^\bullet}$ ein Komplex in $\mathcal{A}$. Setze 
		$$Q^i(M^{^\bullet}) := \coker( \im d_{i-1} \to M^i) \quad \text{für } i\in \Z$$
		Dann induzieren die Differentiale natürliche Morphismen 
		$$\bar{d}_i: Q^i(M^{^\bullet}) \longrightarrow \mathcal{Z}^{i+1}(M^{^\bullet})$$
		mit $\ker \bar{d}_i = \mathcal{H}^i(M^{^\bullet})$ und $\coker(\bar{d}_i) = \mathcal{H}^{i+1}(M^{^\bullet})$
		\item Wir erhalten für $i\in \Z$ ein kommutatives Diagramm mit exakten Zeilen:
		$$\begin{tikzcd}
		& Q^i(A^{^\bullet}) \arrow{r}\arrow{d}{\bar{d}_i} & Q^i(B^{^\bullet}) \arrow{r}\arrow{d}{\bar{d}_i} & Q^i(C^{^\bullet}) \arrow{r}\arrow{d}{\bar{d}_i} & 0\\
		0 \arrow{r} & \mathcal{Z}^{i+1}(A^{^\bullet}) \arrow{r}  & \mathcal{Z}^{i+1}(B^{^\bullet})  \arrow{r} & \mathcal{Z}^{i+1}(C^{^\bullet}) &
		\end{tikzcd} $$
		\item Das Schlangenlemma liefert nach 1. für jedes $i\in \Z$ eine exakte Folge 
			$$\begin{tikzcd}[column sep = small]
		 \mathcal{H}^i(A^{^\bullet}) \arrow{r} & \mathcal{H}^i(B^{^\bullet}) \arrow{r} & \mathcal{H}^i(C^{^\bullet}) \arrow{r} & \mathcal{H}^{i+1}(A^{^\bullet}) \arrow{r}& \mathcal{H}^{i+1}(B^{^\bullet}) \arrow{r} & \mathcal{H}^{i+1}(C^{^\bullet})
		\end{tikzcd}$$
		Diese setzen sich zu einer langen exakten Folge aus der Behauptung zusammen.
	\end{enumerate}
\end{proof}
\begin{df}\label{7.5}
	Sei $A\in \ObA$. Eine \define{injektive Auflösung\index{Auflösung von Objekten}} von $A$ ist ein Komplex 
	$$I^{^\bullet}: \begin{tikzcd}
	I^0\arrow{r}{d_0} & I^1 \arrow{r}{d_1}& I^2\arrow{r} & \ldots
	\end{tikzcd}$$
	bestehend aus injektiven Objekten $I^i$ aus $\mathcal{A}$ mit $I^i=0$ für $i<0$ zusammen mit einem Morphismus $\epsilon : A\longrightarrow I^0$, so dass der \define{augmentierte Komplex\index{augmentiere Komplex}}
	$$\begin{tikzcd}
	0 \arrow{r}& A \arrow{r}{\epsilon} &I^0\arrow{r}{d_0} & I^1 \arrow{r}{d_1}& I^2\arrow{r} & \ldots
	\end{tikzcd}$$
	exakt ist (Notation: $A\longrightarrow I^{^\bullet}$ injektive Auflösung von $A$).\\
	Eine \define{projektive Auflösung} von $A$ ist eine injektive Auflösung von $A$ in $\mathcal{A}^\text{op}$, d.h. ein Komplex 
	$$P^{^\bullet}: \begin{tikzcd}
	\ldots \arrow{r} &P^{-2} \arrow{r} & P^{-1} \arrow{r} & P^0
	\end{tikzcd}$$
	aus projektiven Objekten $P^i$ aus $\mathcal{A}$ mit $P^i=0$ für $i>0$ zusammen mit einem Morphismus $\epsilon:P^0 \to A$, sodass der augmentierte Komplex
	$$\begin{tikzcd}
	\ldots \arrow{r} & P^{-2} \arrow{r} &P^{-1} \arrow{r} & P^0 \arrow{r}{\epsilon} & A \arrow{r} & 0
		\end{tikzcd}$$
	exakt ist (Notation: $P^{^\bullet} \longrightarrow A$ projektive Auflösung).
\end{df}
\begin{anm}
	Man schreibt in obiger Situation auch $P_i = P^{-i}$ und $\mathcal{H}_i(-) = \mathcal{H}^{-i}(-)$.
\end{anm}
\begin{df}\label{7.7}
	$\mathcal{A}$ hat
	\begin{enumerate}
		\item[] \define{genügend viele Injektive\index{genügend viele Injektive}}"' $\defi$ Für jedes $A\in \ObA$ existiert ein injektives Objekt $I\in \ObA$ und ein Monomorphismus $\iota:A \to I$.
		\item[]  \define{genügend viele Projektive\index{genügend viele Projektive}} $\defi \mathcal{A}^\text{op}$ hat genügend viele Injektive
	\end{enumerate}
\end{df}
\begin{bsp}
	$R$-Mod hat nach 6.19 genügend viele Injektive und nach 6.4 genügend viele Projektive.
\end{bsp}
\begin{bem}\label{7.8}
	Sei $A\in \ObA$. Dann gilt:
	\begin{enumerate}[label = \alph*)]
		\item Hat $\mathcal{A}$ genügend viele Injektive, dann hat $A$ eine injektive Auflösung
		\item Hat $\mathcal{A}$ genügend viele Projektive, dann hat $A$ eine projektive Auflösung
	\end{enumerate}
\end{bem}
\begin{proof}
	Es genüge $a)$ zu zeigen, $b)$ folgt dual.
	\begin{enumerate}
		\item Die Situation ist:
		$$\begin{tikzcd}
		0 \arrow{r} & A \arrow{r}{\epsilon}& I^0 \arrow{dr}{\pi_0} \arrow{rr}{d_0} & & I^1\arrow{dr}{\pi_1} \arrow{rr}{d_1} & & I^2 \\
		& & & M^0 \arrow[hook]{ur}{\iota_0} & & M^1 \arrow{ur}{\iota_1}  &		
		\end{tikzcd}$$
		Nach Voraussetzung existiert ein injektives Objekt $I^0 \in \ObA$ und ein Monomorphismus $\epsilon:A \to I^0$. Sei $\coker \epsilon= (M^0, \pi_0)$. Es existiert ein injektives Objekt $I^1$ und ein Monomorphismus $\iota_0:M^0 \hookrightarrow I_1$. Iteriere dieses Verfahren: $\coker(d_0) = (M^1, \pi_1)$, es existiert ein injektives Objekt $I^2$ und ein Monomorphismus $\iota_1:M^1\hookrightarrow I^2$, setze $d_1:= \iota_1 \circ \pi_1$.
		\item Exaktheit: bei $I^0$ gilt:
		$$\im \epsilon = \ker (\coker \epsilon) = \ker \pi_0 \underset{\text{Mono}}{\overset{\iota}{=}} \ker (\iota_0 \circ \pi_0) = \ker d_0$$
		analog bei den anderen Stellen
	\end{enumerate}
\end{proof}
\begin{sa}[Hufeisenlemma]\label{7.9}\index{Hufeisenlemma}
	$\mathcal{A}$ habe genügend viele Injektive. Gegeben sei ein Diagramm (Schwarz)
	$$\begin{tikzcd}
	& 0\arrow{d} &\textcolor{red}{0} \arrow[red]{d} & \textcolor{red}{0} \arrow[red]{d} & \\
	0 \arrow{r}& A'\arrow{d} \arrow{r} & {I'}^0\arrow[red]{d} \arrow{r} & {I'}^1 \arrow[red]{d} \arrow{r} & \ldots\\
	\textcolor{red}{0} \arrow[red]{r}& A\arrow{d} \arrow[red]{r}& \textcolor{red}{I^0} \arrow[red]{d}\arrow[red]{r} &\arrow[red]{d} \textcolor{red}{I^1} \arrow[red]{r} & \ldots\\
	0 \arrow{r} & A''\arrow{d} \arrow{r} & {I''}^0\arrow[red]{d} \arrow[red]{d} \arrow{r} & {I''}^1 \arrow[red]{d} \arrow{r} & \ldots\\
	& 0 & 0 & 0
	\end{tikzcd}$$
	in $\mathcal{A}$, wobei die linke Spalte exakt sei, $A'\to I'^{^\bullet}$ eine injektive Auflösung von $A'$, $A''\to {I''}^{^\bullet}$ eine injektive Auflösung von $A''$. Dann lässt sich das Daigramm so zu einem kommutativen Diagramm ergänzen \textcolor{red}{(rot)}, dass $A\to I^{^\bullet}$ eine injektive Auflösung von $A$ ist und die Spalten alle exakt sind.
\end{sa}
\begin{proof}
	in den Standardwerken über homologische Algebra (zumindest für $R$-Mod). Für den Beweis einer Verallgemeinerung in Kontext abelsche Kategorien siehe Stacks-Project 013P.
\end{proof}
\textbf{Frage:} in welchem Verhältnis stehen zwei injektive Auflösungen eines Objekts? 
\begin{df}\label{7.10}
	Seien $A^{\bullet}, B^{\bullet} $ Komplexe in $\mathcal{A}, f,g: A^{\bullet} \to B^{\bullet} $ Komplexhomomorphismen. $f,g $ heißen \define{homotop\index{homotope Komplexhomomomorphismen}} $\defi $ es existieren Homomorphismen $ s^{i}: A^{i+1} \to B^i $ für alle $i \in \Z $ mit $$ f_i  -g_i = d_{i-1} \circ s^{i-1} + s^i \circ d_i $$
	(Notation: $f \sim g$)
	%Diagramm Hier müssen doppel und Querpfeile ergänzt werden 
	$$\begin{tikzcd}[column sep = large, row sep = large]
	\ldots \arrow{r} & A^{i-1}\arrow[xshift = -0.8 ex, swap]{d} \arrow[xshift = 0.8 ex]{d} \arrow{r}& A^i \arrow[red, near end]{dl}{s^{i-1}} \arrow[xshift = -0.8 ex, swap]{d}{f_i} \arrow[xshift = 0.8ex]{d}{g_i} \arrow{r}{d_i} & A^{i+1} \arrow[red]{dl}{s^i} \arrow[xshift = -0.8 ex, swap]{d} \arrow[xshift = 0.8 ex]{d}\arrow{r} & \ldots\\
	\ldots \arrow{r} & B^{i-1} \arrow[swap]{r}{d_{i-1}} & B^i \arrow[swap]{r} & B^{i+1} \arrow{r} & \ldots
	\end{tikzcd}$$
\end{df}
\begin{anm}
	\begin{itemize}
		\item Homotopie von Komplexhomomorphismen ist eine Äquivalenzrelation.
		\item Sind $f,g: A^{\bullet} \to B^{\bullet} $ Komplexhomomorphismen mit $ f \sim g $ und $F: \mathcal{A} \to \mathcal{B} $ ein additiver Funktor von $\mathcal{A} $ in eine abelsche Kategorie $\mathcal{B} $, dann erhalten wir einen Komplexhomomorphismus $ Ff, Fg: FA^{\bullet} \to FB^{\bullet} $ mit$ Ff \sim Fg$.
	\end{itemize}
\end{anm}
\begin{bem}\label{7.11}
	Seien $A^{\bullet}, B^{\bullet} $ Komplexe in $\mathcal{A},\,  f,g: A^{\bullet} \to B^{\bullet} $ Komplexhomomorphismen mit $ f\sim g $. Dann gilt: $ \mathcal{H}^i(f) = \mathcal{H}^i(g): \mathcal{H}^i(A^{\bullet}) \to \mathcal{H}^i(B^{\bullet}) $
\end{bem}
\begin{proof}
	Wir setzen $h:= f-g: A^{\bullet} \to B^{\bullet} $. Offenbar genügt es zu zeigen: $\mathcal{H}^i(h) = 0 $ für alle $ i \in \Z$. 
	\begin{enumerate}
		\item Der Morphismus $ \mathcal{Z}^i(h): \mathcal{Z}^iA^{\bullet} \to \mathcal{Z}^iB^{\bullet} $ faktorisiert über $\BB^iB^{\bullet},$ denn: \\
		$$\begin{tikzcd}[row sep = large]
		\ker d_i = Z^iA^{\bullet} \arrow[dashed, blue]{d}{\ZZ^ih} \arrow[hook]{r}{\alpha_i} \arrow[dashed,red, bend right = 75]{dd}{\lambda_i}& A^i \arrow{d}{h_i}\arrow{rr}{d_i} & & A^{i+1} \arrow{d}{h_{i+1}} \\
		\ker d_i =\ZZ^iB^{\bullet} \arrow[hook]{r}{\beta_i} & B^i \arrow[swap]{rr}{d_i}\arrow[two heads]{dr}{\gamma_i} & & B^{i+1}  \\
		\im d_{i-1} = \BB^iB^{\bullet} = \ker \gamma_i \arrow{u}{\theta_i} \arrow[swap, hook]{ur}{\delta_i} &  &\coker d_{i-1} &
		\end{tikzcd}$$
		
		$Z^i(h) $ ist der eindutig bestimmte Morphismus $ \ZZ^iA^{\bullet} \to \ZZ^iB^{\bullet} $ mit $h_i \circ \alpha_i = \beta_i \circ \ZZ^ih $ (beachte: $d_i \circ h_i \circ \alpha_i = h_{i+1} \circ d_i \circ \alpha_i = 0$). Wegen $ f \sim g $ existiert eine Famillie $(si)_{i \in \Z} $ von Homomorphismen $s_i: A^{i+1} \to B^i $ mit $ h_i=f_i -g_i = d_{i -1} \circ s^{i-1} + s_i \circ d_i $ für alle $i \in \Z$. Dann ist
		$$h_i \circ \alpha_i = d_{i-1} \circ s^{i-1} \circ \alpha_i + s_i \circ \underbrace{d_i \circ \alpha_i}_{=0} = d_{i-1} \circ s^{i-1} \circ \alpha_i$$ Wegen $ \gamma_i \circ d_{i-1}= 0 $ ist $ \overbrace{\gamma_i \circ d_{i-1}}^{=0} \circ s^{i-1} \circ \alpha_i = 0, $ also $ \gamma_i \circ h_i \alpha_i = 0$ Aus der Univesellen Eigenschaft des Kerns von $\gamma_i $, existiert ein eindeutig bestimmtes $ \lambda_i : \ZZ^iA^{\bullet} \to \BB^iB^{\bullet} = \ker \gamma_i $ mit $ h_i \circ \alpha_i = \delta_i \circ \lambda_i $ und mit $\delta_i = \beta_i \circ \theta_i \Ra \beta_i \circ \theta_i \circ \lambda_i = h_i \circ \alpha_i = \beta_i \circ \ZZ^ih $ da $\beta_i$ ein Monomorphismus folgt: $ \theta_i \circ \lambda_i = \ZZ^ih.$
		\item $\HH^ih = 0$, denn betrachte die Situation:
		$$\begin{tikzcd}[row sep = large]
		\BB^iA^{\bullet} \arrow{d}{\BB^ih} \arrow{r}{\theta_i^{'}} & \ZZ^iA^{\bullet} \arrow{d}{\ZZ^ih} \arrow[blue]{dl}{\lambda_i} \arrow[two heads]{r}{\epsilon_i^{'}} & \HH^iA^{\bullet} = \coker \theta_i^{'} \arrow{d}{\HH^ih} & \\
		B^iB^{\bullet} \arrow{r}{\theta_i} & Z^iB^{\bullet} \arrow[swap, two heads]{r}{\epsilon_i} & \HH^iB^{\bullet} = \coker \theta_i& 
		\end{tikzcd}$$
		$\HH^ih$ ist der eindeutig bestimmte Morphismus $\HH^iA^{\bullet} \to \HH^iB^{\bullet} $ mit $\HH^ih \circ \epsilon_i^{'} = \epsilon_i \circ \ZZ^ih = \epsilon_i \circ \theta_i \circ \lambda_i = 0$, denn $\epsilon_i \circ \theta_i =0 $,somit $\HH^ih=0 $.
	\end{enumerate}
\end{proof}
\begin{df}\label{7.12}
	Seien $A^{\bullet}, B^{\bullet} $ Komplexe in $\mathcal{A}, f,g: A^{\bullet} \to B^{\bullet} $ Komplexhomomorphismen. $f$ heißt
	\begin{enumerate}
	\item[] \define{Homotopieäquivalenz\index{Homotopieäquivalenz}} $\defi $ es existiert ein $g: B^{\bullet} \to A^{\bullet} $ Komplexhomomorphismus mit $g \circ f \sim id_{A^{\bullet}} $ und $f \circ g \sim id_{B^{\bullet}}$
	\item[] \define{Quasiisomorphismus\index{Quasiisomorphismus}} $\defi $ Für alle $i \in \Z $ ist $\HH^if:  \HH^iA^{\bullet} \to \HH^iB^{\bullet} $ ein Isomorphismus.
\end{enumerate}
\end{df}
\begin{bem}\label{7.13}
	Seien $A^{\bullet}, B^{\bullet} $ Komplexe in $\mathcal{A}, f: A^{\bullet} \to B^{\bullet} $ Homotopieäquivalenz. Dann ist $f$ ein Quasiisomorphismus. 
\end{bem}
\begin{proof}
	Nach Vorraussetzung existiert ein $g: B^{\bullet} \to A^{\bullet} $ Komplexhomomorphismus mit $g \circ f \sim id_{A^{\bullet}} $ und $f \circ g \sim id_{B^{\bullet}}$. Dann ist
	$$\HH^i(g) \circ \HH^i(f) = \HH^i(g \circ f ) = \HH^i(id_{A^{\bullet}}) = id_{\HH^iA^{\bullet}}$$
	analog: $\HH^i(f) \circ \HH^i(g) = id_{\HH^iB^{\bullet}}$. Also ist  $\HH^i(f) $ ein Isomorphismus.
\end{proof}
\begin{anm}
	Nicht jeder Quasiisomorphismus ist eine Homotopieäquivalenz.
\end{anm}
\begin{sa}\label{7.14}
	Gegeben sei folgendes Diagramm von Komplexen in $\mathcal{A} $: \\
	%Diagramm
	$$\begin{tikzcd}
	0 \arrow{r} & A \arrow{d}{\phi} \arrow{r}{\epsilon}& E^0 \arrow[dashed,red]{d}{f_0}\arrow{r} & E^1 \arrow[dashed,red]{d}{f_1} \arrow{r} & \ldots\\
	0 \arrow{r} & B \arrow[swap]{r}{\eta} & I^0 \arrow[swap]{r} & I^1 \arrow{r} & \ldots
	\end{tikzcd}$$
	sodass gilt: \begin{itemize}
		\item die obere Zeile ist exakt
		\item Alle $I^i, i \geq 0,$ sind injektiv.
	\end{itemize}
	Dann existiert ein Komplexhomomorphismus $f: E^{\bullet} \to I^{\bullet}, $ der $\phi$ fortsetzt, in dem Sinne, dass $ f_0 \circ \epsilon = \eta \circ \phi $ ist. Ist $g:E^{\bullet} \to I^{\bullet} $ ein weiterer solcher Komplexhomomorphismus, dann ist $g \sim f$.
\end{sa}
\begin{proof}[Beweisskizze für die Existenz von f:] 
	\begin{enumerate}
		\item Wir konstruieren zunächst $f_0$. Situation: 
		%Diagramm 
		$$\begin{tikzcd}
		 A \arrow{d}{\phi} \arrow{r}{\epsilon}& E^0 \arrow[dashed,red]{d}{f_0} \\
		 B \arrow[swap]{r}{\eta} & I^0 
		\end{tikzcd}$$
		Da $I^{0} $ injektiv und $\epsilon $ ein Monomorphismus, existiert ein $f_0: E^{0} \to I^{0} $, sodass $\eta \circ \phi = f_0 \circ \epsilon.$
		\item Konstruktion von $f_1$: Situation: \\
		%Diagramm
		$$\begin{tikzcd}[row sep = large, column sep = large]
		A \arrow{ddd}{\phi} \arrow{r}{\epsilon} & E^0 \arrow{ddd}{f_0}\arrow[two heads]{dr}{\pi_0}\arrow{rrr}{d_0} & & & E^1\arrow[red, dashed]{ddd}{f_1} \\
		& & \coker \epsilon\arrow[dashed, blue]{d}{\tilde{f}_0} \arrow[blue, dashed, hook]{urr}{\iota_0}\\
		& & \coker \eta\arrow[dashed, blue, hook]{drr}{{\iota_0}'} & & \\
		B\arrow{r}{\eta} & I^0\arrow[two heads]{ur}{{\pi_0}'}\arrow[swap]{rrr}{{d_0}'} & & &I^1
		\end{tikzcd}$$
		Wegen der Kommutativität vom linken Rechteck, also
		$$ \pi_0^{'} \circ f_0 \circ \epsilon = \pi_0^{'} \circ \eta \circ \phi = 0 $$ existiert ein eindeutig bestimmtes $ \tilde{f}_0: \coker\epsilon \to \coker \eta$, sodass das linke Trapez kommutiert. Da $ d_0 \circ \epsilon = 0$ und $ d_0^{'} \circ \eta = 0, $ existieren nach der Universellen Eigenschaft des Kokerns eindeutig bestimmte $ \iota_0: \coker\epsilon \to E^{1}, \, \iota_0^{'}: \coker \eta \to I^{1}$, sodass das obere und untere Dreieck kommutieren.\\ \emph{Behauptung:} $\iota_0$ ist ein Monomorphismus, denn: \begin{eqnarray*}
			\coker \epsilon &=& \im \pi_0 \simeq \coim \pi_0 = \coker(\underbrace{\ker \pi_0}_{=\im \epsilon}) = \coker(\im\epsilon)\\ &\simeq& \coker(\ker d_0)
		 =\coim(d_0) \simeq \im d_0
		\end{eqnarray*}
		was aus dem Homomorphiesatz und der Exaktheit bei $E^{0}$ folgt. Nun verifiziert man, dass 
		$$\begin{tikzcd}
		 \coker \epsilon \arrow{rr}{\sim} \arrow{dr}{\iota_0} && \im d_0 \arrow[hook]{dl}\\
		 & E^1 &
		\end{tikzcd}$$
		kommutiert, das heißt, dass $\iota_0$ ein Monomorphismus ist. Da $I^1 $ injektiv und $\iota_0$ ein Monomorphismus, existiert ein $f_1: E^1 \to I^1$, sodass auch das rechte Trapez kommutiert. Also ist $f_1 \circ d_0 = d_0^{'} \circ f_0 $. 
		\item Iteriere das Verfahren.
	\end{enumerate}
\end{proof}
\begin{fo}\label{7.15}
	Sei $A \in \ObA, \,\epsilon: A \to I^{\bullet}, \,\eta: A \to J^{\bullet}$ injektive Auflösungen von $A$. Dann existiert eine Homotopieäquivalenz $f:I^{\bullet} \to J^{\bullet} $ mit $ f_0 \circ \epsilon = \eta$. Diese ist eindeutig bestimmt bis auf Homotopie.
\end{fo}
\begin{proof}
	Wir betrachten das Diagramm von Komplexen: \\
	%Diagramm 
	$$\begin{tikzcd}[column sep = large, row sep = large]
	\tilde{I^{\bullet}}: &0 \arrow{r} & A \arrow[xshift= -0.8 ex, swap]{d}{\id_A} \arrow{r}{\epsilon}& I^0 \arrow[dashed,red, swap, xshift= -0.8 ex]{d}{f_0}\arrow{r} & I^1 \arrow[dashed,red,xshift= -0.8 ex, swap ]{d}{f_1} \arrow{r} & \ldots\\
	\tilde{J^{\bullet}}:& 0 \arrow{r} & A \arrow[xshift=0.8 ex]{u}\arrow[swap]{r}{\eta} & J^0 \arrow[dashed, blue, xshift = 0.8 ex, swap]{u}{g_0} \arrow[swap]{r} & J^1 \arrow[dashed, blue, swap, xshift=0.8 ex]{u}{g_1} \arrow{r} & \ldots
	\end{tikzcd}$$
	mit exakten Zeilen. Nach \ref{7.14} existiert ein Komplexhomomorphismus $ f: I^{\bullet} \to J^{\bullet} $, der $\id_A $ fortsetzt und es existiert ein Komplexhomomorphismus $g: J^{\bullet} \to I^{\bullet}$, der $\id_A $ fortsetzt. Dann ist aber auch $g \circ f: I^{\bullet} \to I^{\bullet} $ eine Fortsetzung von $id_A$, ebenso wie $id_{J^{\bullet}}$. Aus der Eindeutigkeit in \ref{7.14} ist $ g \circ f \sim id_{I^{\bullet}}.$ Analog ist $f \circ g \sim id_{J^{\bullet}}.$ Somit folgt, dass $f$ eine Homotopieäquivalenz ist. Die Eindeutigkeitsaussage folgt aus \ref{7.14}.
\end{proof}
\begin{fo}
	Sei $I^{\bullet} $ ein exakter Komplex von injektiven Objekten in $\mathcal{A} $ mit $ I^i = 0 $ für $ i \ll 0$. Dann ist $0^{\bullet} \to I^{\bullet} $ eine Homotopieäquivalenz. 
\end{fo}
\begin{proof}
	Ohne Einschränkung ist $ I^i = 0 $ für $ i < 0 $ (durch Verschiebung des Komplexes) Dann sind $ 0^{\bullet}, I^{\bullet} $ injektive Auflösungen von 0. Aus \ref{7.15} folgt: $0^{\bullet} \to I^{\bullet} $ist eine Homotopieäquivalenz.
\end{proof} 
\newpage
\subsection{Abgeleitete Funktoren}
\begin{center}
	\textbf{In diesem Abschnitt sei $\mathcal{A}$ eine abelsche Kategorie mit genügend vielen Injektive, $\mathcal{B}$ eine abelsche Kategorie und $F:\mathcal{A} \to \BB$ ein linksexakter Funktor}
\end{center}
\begin{bem+df}\label{8.1}
	Für $i\in \N_0$ und jedes Objekt $A\in \ObA$ fixieren wir eine injektive Auflösung $A\to I^\bullet$ von $A$ und setzen 
	$$R^iF(A):= \HH^i(FI^\bullet)$$
	Ist $\phi:A \to A'$ ein Morphismus in $\mathcal{A}$ und sind $A\to I^\bullet$, $A'\to I'^\bullet$ injektive Auflösungen von $A,A'$, dann existiert ein bis auf Homotopie eindeutiger Komplexhomomorphismus $f:I^\bullet \to I'^\bullet$, der $\phi$ fortsetzt. Wir setzen 
	$$R^iF(\phi) := \HH^i(Ff).$$
	Auf diese Weise wird $R^iF:\A \to \BB$ zu einem additiven Funktor. Wird auf dieselbe Art und Weise mit einer anderen Wahl von injektiven Auflösungen ein Funktor $\widehat{R}^iF:\A\to \BB$ konstruiert, dann sind $R^iF(A)$ und $\hat R^iF(A)$ kanonisch isomroph für alle $A\in \ObA$, und es gibt eine natürliche Äquivalenz $R^iF\overset{\sim}{\Ra} \hat R^iF$. $R^iF$ heißt der \define{i-te rechtsabgeleitete Funktor\index{rechtsabgeleiteter Funktor}}
\end{bem+df}
\begin{proof}
	\begin{itemize}
		\item Wohldefiniertheit von $R^iF(\phi)$: Ist $g:I^\bullet\to I'^\bullet$ eine weitere Fortsetzung von $\phi$, dann ist $f \sim g$ nach \ref{7.14} und somit $Ff\sim Fg$. Mit \ref{7.11} folgt $\HH^i(Ff) = \HH^i(Fg)$ für alle $i\geq 0$.
		\item $R^iF$ ist ein Funktor, denn für $\phi:A \to A', \, \psi:A' \to A''$ mit Fortsetzungen $f:I^\bullet\to I'^\bullet, \, g:I'^\bullet \to I''^\bullet$ auf injektiven Auflösungen $A\to I^\bullet, \, A'\to I^\bullet, \, A'' \to I''^\bullet$ von $A,A',A'''$ ist $g\circ f:I^\bullet\to I''^\bullet$ ein Fortsetzung von $\psi \circ \phi:A \to A''$, also 
		\begin{eqnarray*}
			(R^iF)(\psi \circ \phi) &=& \HH^i(F(g\circ f)) = \HH^i(Fg\circ Ff) = \HH^iFg \circ \HH^iFf\\
			& =& R^iF(\psi) \circ R^iF(\phi)
		\end{eqnarray*}
		\item $R^iF$ ist additiv, denn sind $\phi:A\to A', \, \psi :A \to A'$ mit Fortsetzungen $f:I \to I'^\bullet, \, g:I^\bullet\to I'^\bullet$ auf injektiven Auflösungen $A\to I^\bullet, \, A' \to I'^\bullet$ von $A, A'$, so ist $f+g :I^\bullet\to I'^\bullet$ eine Fortsetzung von $\phi + \psi:A\to A'$. Der Rest ist klar, da $F, \,\HH^i$ additiv.
		\item Unabhängigkeit von der Wahl der Auflösungen im obigen Sinne: Für $A\in \ObA$ sei $A\xrightarrow{\epsilon} I_A^\bullet$ die injektive Auflösung, mit der $R^iF$ berechnet wird und $A\xrightarrow{\eta} J_A^\bullet$ die injektive Auflösung von $A$, mit der $\hat R^i F$ berechnet wird. Nach \ref{7.15} existiert eine Homotopieäquivalenz $h_A:I_A^\bullet\to J_A^\bullet$ existiert mit $h_A^0 \circ \epsilon = \eta$. Wir definieren
		$$t_A^i:R^iF(A) = \HH^i(FI_A^\bullet)\longrightarrow \hat R^iF(A) = \HH^i(FJ_A^\bullet) \quad \text{mit} \quad t_A^i:= \HH^iFh_A$$
		Dann ist $t^i_A$ nach \ref{7.13} ein Isomorphismus, da $Fh_A$ eine Homotopieäquivalenz ist. Durch $t^i=(t_A^i)_{A\in \ObA}:R^iF\Ra \hat R^iF$ ist eine natürliche Äquivalenz gegeben, denn für alle $A,A'\in \ObA, \, \phi:A \to A'$ kommutiert das Diagramm
		$$\begin{tikzcd}[row sep = large]
		R^iF(A) \arrow{r}{t_A^i} \arrow[swap]{d}{\HH^iFf_I} & \hat R^iF(A) \arrow{d}{\HH^iFf_J}\\
		R^iF(A') \arrow[swap]{r}{t^i_{A'}} & \hat R^iF(A')
		\end{tikzcd}$$
		(wobei $f_I:I_A^\bullet\to J_A^\bullet, \, f_J:J_A^\bullet \to J_A'^\bullet$ Fortsetzungen von $\phi$ sind), denn $f_J \circ h_A, \, h_{A'} \circ f_I: I_A^\bullet \to J_A'^{\bullet}$ sind beides Fortsetzungen von $\phi=\phi \circ \id_A = \id_{A'} \circ \phi$, somit $f_J \circ h_A \sim h_A' \circ f_I$, also $Ff_J \circ Fh_A \sim Fh_{A'} \circ Ff_I$ und damit $\HH^iFf_J \circ t^i_A = t^i_{A'} \circ \HH^iFf_I$.
	\end{itemize}
\end{proof}
\begin{bem}\label{8.2}
	Es gilt:
	\begin{enumerate}[label= \alph*)]
		\item $R^0F=F$
		\item Ist $F$ exakt, dann ist $R^iF=0$ für alle $i >0$.
	\end{enumerate}
\end{bem}
\begin{proof}
	Für $A\in \ObA$ ist $R^iF(A) = \HH^i(FI^\bullet)$, wobei $A\xrightarrow{\epsilon} I^\bullet$ eine injektive Auflösung von $A$ ist. Wir wissen: Ist $\begin{tikzcd}
	0 \arrow{r} & A \arrow{r}{\epsilon} & I^0 \arrow{r}{d_0} & I^1
	\end{tikzcd}$ exakt, dann ist ohnehin $\begin{tikzcd}
	0 \arrow{r} & FA \arrow{r}{F\epsilon} & FI^0 \arrow{r}{Fd_0} & FI^1
	\end{tikzcd}$ exakt.
	$$\begin{tikzcd}
	FI^\bullet: & \ldots \arrow{r} & 0 \arrow{r} & 0 \arrow{r}& FI^0 \arrow{r}{Fd_0} & FI^1\arrow{r}{Fd_1}& FI^2\arrow{r} & \ldots
	\end{tikzcd}$$
	mit \begin{eqnarray*}
		R^0F(A) &=& \HH^0(FI^\bullet) = \coker(0 \ra \ker Fd_0) = \ker Fd_0
		= \im F\epsilon\\
		&=& \coim F\epsilon = FA
	\end{eqnarray*}
	Falls $F$ exakt, dann ist 
	$$\begin{tikzcd}
	0\arrow{r} & FA \arrow{r}{F\epsilon} & FI^0 \arrow{r}{Fd_0} & FI^1\arrow{r}{Fd_1}& \ldots
	\end{tikzcd}$$
	exakt, also $R^iF(A) = \HH^i(FI^\bullet) = 0$ für $i>0$.
\end{proof}
\begin{sa}\label{8.3}
	Sei $\begin{tikzcd}
	0\arrow{r} & A'\arrow{r} & A \arrow{r} & A''\arrow{r} & 0
	\end{tikzcd}$ eine exakte Folge in $\A$. Dann existieren natürliche Morphismen 
	$$\delta^i:R^iF(A'') \longrightarrow R^{i+1}F(A') \quad \text{für alle } i\geq 0$$
	sodass die Folge 
	$$\begin{tikzcd}
	0 \arrow{r} & FA'\arrow{r} & FA \arrow{r} & FA''\\
	\arrow{r} & R^1FA' \arrow{r} & R^1FA \arrow{r} & R^1FA''\\
	\vdots & \vdots & \vdots & \vdots \\
	\arrow{r} & R^iFA' \arrow{r} & R^iFA \arrow{r} & R^iFA''\\
	\arrow{r}{\delta^i} & R^{i+1}FA' \arrow{r} & R^{i+1}FA \arrow{r} & R^{i+1}FA''\\
	\vdots & \vdots & \vdots & \vdots \\
	\end{tikzcd}$$
	exakt ist. Ist 
	$$\begin{tikzcd}
	0 \arrow{r} & A' \arrow{d}\arrow{r} & A \arrow{d}\arrow{r} & A'' \arrow{d}\arrow{r} & 0\\
	0 \arrow{r} & B'\arrow{r} & B \arrow{r} & B'' \arrow{r} & 0
	\end{tikzcd}$$
	ein kommutatives Diagramm, wobei die untere Zeile exakt ist, so kommutiert für alle $i\geq 0$ das Diagramm
	$$\begin{tikzcd}
	R^iF(A'') \arrow{r}{\delta^i} \arrow{d} & R^{i+1}F(A') \arrow{d} \\
	R^iF(B'') \arrow[swap]{r}{\delta^i} & R^{i+1}F(B')
	\end{tikzcd}$$
\end{sa}
\begin{proof}[Beweisskizze]
	Nach dem Hufeisenlemma existieren kompatible injektive Auflösungen $A'\to I'^\bullet, \, A\to I^\bullet, \, A''\to I''^\bullet$ in dem Sinne, dass
	$$\begin{tikzcd}
	0 \arrow{r} & I'^\bullet\arrow{r} & I^\bullet\arrow{r} & I''^\bullet \arrow{r} & 0
	\end{tikzcd}$$
	eine exakte Folge von Komplexen ist. $I'^i$ ist injektiv für alle $i\geq 0$, also spaltet die Folge
	$$ \begin{tikzcd}
	0 \arrow{r} & I'^i\arrow{r} & I^i\arrow{r} &I''^i \arrow{r} & 0
	\end{tikzcd}$$
	(wobei Spaltung in abelschen Kategorien analog zu $R$-Mod definiert ist und analoge Resultate gelten). Dann ist 
	$I^i= I'^i \oplus I''^i$ und, da $F$ additiv, $F(I^i) = F(I'^i) \oplus F(I''^i)$, womit
	$$\begin{tikzcd}
	0 \arrow{r} & FI'^i \arrow{r} & FI^i \arrow{r} & FI''^i \arrow{r} & 0
	\end{tikzcd}$$ exakt ist. Insbesondere existiert ein exakte Folge von Komplexen 
	$$\begin{tikzcd}
	0 \arrow{r} & FI'^\bullet\arrow{r} & FI^\bullet\arrow{r} & FI''^\bullet \arrow{r} & 0
	\end{tikzcd}$$
	woraus wir eine lange exakte Kohomologiefolge erhalten, was die Behauptung liefert.
\end{proof}
\begin{df}\label{8.4}
	Sei $A\in \ObA$. Dann heißt $A$ \define{$F$-azylisch\index{azyklische Objekte}} $\defi R^iF(A) = 0$ für alle $i \geq 1$.
\end{df}
\begin{bem}\label{8.5}
	Ist $A\in \ObA$ injektiv, dann ist $A$ $F$-azyklisch.
\end{bem}
\begin{proof}
	Offenbar ist $\begin{tikzcd}
	A \arrow{r}{\id} & (A \arrow{r} & 0 \arrow{r} & 0 \arrow{r} & \ldots)
	\end{tikzcd}$ eine injektive Auflösung von $A$. Damit ist 
	$$R^iF(A) = \HH^i(\begin{tikzcd}
	FA \arrow{r} & 0 \arrow{r} & 0\arrow{r} & \ldots)
	\end{tikzcd} =0 \quad \text{für }i\geq 1$$
	woraus die Behauptung folgt.
\end{proof}
\begin{sa}\label{8.6}
	Sei $A\to J^\bullet$ eine Auflösung von $A$ durch $F$-azyklische Objekte, d.h. $J^\bullet$ ist ein Komplex mit $J^i=0$ für $i<0$ und $J^i$ $F$-azyklisch für $i\geq 0$, sodass der augmentierte Komplex $$\begin{tikzcd}
	0 \arrow{r} & A \arrow{r} & J^1 \arrow{r} & J^2 \arrow{r} & \ldots
	\end{tikzcd}$$
	exakt ist. Dann gibt es einen kanonischen Isomorphismus $R^iF(A) \cong \HH^i(FJ^\bullet)$ für alle $i\geq 0$.
\end{sa}
\begin{proof}
	(für $R$-Mod in S.Lang "'Algebra"')
\end{proof}
\begin{anm}
	Die Theorie der Linksableitung rechtsexakter Funktoren lässt sich analog entwickeln: Sei $\A$ eine abelsche Kategorie mit genügend vielen Projektiven, $\BB$ eine abelsche Kategorie, $F:\A\to \BB$ ein rechtsexakter Funktor. Wir wählen für jedes $A\in \ObA$ eine projektive Auflösung $P_\bullet \to A$ und setzen 
	$$L_iF(A):= \HH_i(FP_\bullet)$$
	Rest analog.
\end{anm}
\newpage
\subsection{$\delta$-Funktoren}
\begin{center}
	\textbf{In Folgenden seien $\A, \BB$ abelsche Kategorien}
\end{center}
\begin{df}\label{9.1}
	Ein \define{$\delta$-Funktor\index{$\delta$-Funktor}} $H=(H^n)_{n\geq 0}$ ist eine Familie additiver Funktoren $H^n:\A \to \BB$ zusammen mit Homomorphismen $\delta:H^n(C) \to H^{n+1}(A)$ für alle $n\geq 0$ und jede kurze exakte Folge $\begin{tikzcd}
	0\arrow{r} & A \arrow{r} & B\arrow{r} & C \arrow{r} & 0
	\end{tikzcd}$, sodass gilt:
	\begin{enumerate}
		\item[(D1)] $\delta$ ist funktoriell, d.h. ist 
		$$\begin{tikzcd}
		0\arrow{r} & A\arrow{d} \arrow{r} & B\arrow{r} \arrow{d}& C\arrow{d} \arrow{r} & 0\\
		0\arrow{r} & A' \arrow{r} & B'\arrow{r} & C' \arrow{r} & 0
		\end{tikzcd} $$
		ein kommutatives Diagramm in $\A$ mit exakten Zeilen, dann kommutiert 
		$$\begin{tikzcd}
		H^n(C) \arrow{d}\arrow{r}{\delta} & H^{n+1}(A) \arrow{d} \\
		H^n(C') \arrow{r}{\delta} & H^{n+1}(A')
		\end{tikzcd}$$
		in $\BB$ für alle $n\geq 0$
		\item[(D2)] Für jede kurze exakte Folge $\begin{tikzcd}
		0\arrow{r} & A \arrow{r} & B\arrow{r} & C \arrow{r} & 0
		\end{tikzcd}$ in $\AA$ ist die lange exakte Folge 
		$$\begin{tikzcd}
		\ldots \arrow{r} & H^n(A) \arrow{r} & H^n(B) \arrow{r}& H^n(C) \arrow{r}{\delta} & H^{n+1}(A) \arrow{r} & \ldots
		\end{tikzcd}$$
		exakt in $\BB$.
	\end{enumerate}
\end{df}
\begin{bsp}
	$\A$ habe genügend viele Injektive, $F:\A \to \BB$ linksexakt. Dann ist $H:=(R^nF)_{n\geq 0}$ ein $\delta$-Funktor nach \ref{8.3}
\end{bsp}
\begin{df}\label{9.3}
	Sei $H=(H^n)_{n\geq 0}: \A \to \BB $ ein $\delta$-Funktor. H heißt \define{universell\index{universeller Funktor}} $\defi $ Für jeden $\delta$-Funktor $H'=(H^{'n})_{n\geq 0}: \A \to \BB $ setzt sich jede natürliche Transformation $ f^{\circ}: H^{\circ} \Ra H^{'\circ} $ eindeutig zu einem Homomorphismus von $\delta$-Funktoren fort, d.h. zu einer Familie $f=(f^{n})_{n \geq 0}$ von natürlichen Transformationen $f^{n}: H^{n} \Ra H^{'n} $ die auf naheliegende Weise mit den $\delta^{'}$ verträglich sind.
\end{df}
\begin{bem}\label{9.4}
	Sind $F,G$ universelle $\delta$-Funktoren mit $F^{0} = G^{0}$, dann gibt es eine kanonische natürliche Äquivalenz von $\delta$-Funktoren $ F \overset{\sim}{\Ra} G.$
\end{bem}
\begin{proof}
	$id: F^{0} \Ra G^{0} $ setzt sich fort zu einem Homomorphismus $\Phi: (\Phi_n)_{n \geq 0}, \Phi_n: F_n \Ra G_n $ von $\delta$-Funktoren. $id: G^{0} \Ra F^{0} $ setzt sich fort zu einem Homomorphismus $\Psi = (\Psi_n)_{n \geq 0}, \Psi_n: G_n\Ra F_n $von $\delta$-Funktoren. $\Psi \circ \Phi := (\Psi_n \circ \Phi_n)_{n \geq 0}$ ist eine Fortsetzung von $id: F^{0} \Ra F^{0}.$ Aus der Eindeutigekeit in der Universellen Eigenschaft folgt $\Psi \circ \Phi = id_F.$ Analog $ \Phi \circ \Psi = id_G$
\end{proof}
\begin{df}\label{9.5}
	Sei $ F: \A \to \BB $ ein additiver Funktor.  F heißt \define{auslöschbar} $\defi $ Für jedes $A \in \ObA $ existiert ein $A' \in \ObA $ und ein Monomorphismus $ u: A \hookrightarrow A'$ mit $ F(u)= 0 $. 
\end{df}
\begin{sa}\label{9.6}
	Sei $H=(H^n)_{n\geq 0}: \A \to \BB $ ein $\delta$-Funktor, sodass $H^n $ auslöschbar für alle $ n \geq 1$. Dann ist $H$ universell. 
\end{sa}
\begin{proof}[Beweisskizze]
	Sei $H^{'}=(H^{'n})_{n\geq 0}: \A \to \BB $ ein $\delta$-Funktor, $f^{0}: H^{0} \Ra H^{'n}$ eine natürliche Transformation. Wir konstruieren die natürliche Transformation $f^n: H^n \Ra H^{'n} $ die mit $\delta$ kommutieren, per Induktion nach n. Seien $f^{0},\dots f^{n}$ bereits konstruiert. Sei $A \in \ObA $. Da $H^{n+1} $ auslöschbar, gibt es eine exakte Folge:\\ $\begin{tikzcd}
	0\arrow{r} & A \arrow{r}{u} & A^{'}\arrow{r} & B \arrow{r} & 0
	\end{tikzcd}$ mit $H^{n+1}(n) = 0$. So erhalten wir Diagramm: 
	$$\begin{tikzcd}
	H^{n}A^{'}\arrow{d} \arrow{r} & H^{n}B\arrow{r}{\delta} \arrow{d}& H^{n+1}A \arrow[dashed]{d} \arrow{r} & 0\\
	H^{'n}A' \arrow{r} & h^{'n}B\arrow{r}{\delta^{'}} & H^{'n+1}A
	\end{tikzcd} $$
	konstruiere mit den üblichen Argumenten einen Morphismus $f^{n+1}_A : H^{n+1}A \to H^{'n+1}A, $ der mit den $\delta$'s vertauscht, und so dass $f^{n+1} = (f^{n+1}_A)_{A \in \ObA}:H^{n+1} \Ra H^{'n+1} $ eine natürliche Transformation ist. 
\end{proof}
\begin{fo}\label{9.7}
	Habe $\A $ genügend viele Injektive, $F: \A \to \BB $ linksexakter Funktor. Dann ist $(R^{n}F)_{n \geq 0}$ ein universeller $\delta$-Funktor. 
\end{fo}
\begin{proof}
	Nach \ref{9.6} genügt es zu zeigen: $R^{n}F$ ist auslöschbar für alle $ n \geq 1$. Sein $n \geq 1, A \in \ObA $. Dann esistiert ein injektives Objekt $I \in \ObA$ und ein Monomorphismus $u: A \hookrightarrow I $. 
	$\Ra R^{n}F(u): R^{n}F(A) \to R^{n}F(I) = 0 $nach \ref{8.5} ist der Nullmorphismus für $n \geq 1$.
\end{proof}
\newpage
\subsection{Ext und Erweiterungen}
\begin{df}
	Seien $M,N$ $R$-Moduln. Wir setzen: $$Ext^{n}_R(M,N)\index{Ext-Funktor} := R^{n}Hom_R(M,-)(N), \quad \text{für }n\geq 0 $$
	Explizit: wähle eine injektive Auflösung $N \to I^{\bullet} $ von $N$, dann ist $$Ext^{n}_R(M,N)=H^{n}(Hom_R(M,I^{\bullet}))$$
\end{df}
\begin{sa}\label{10.2}
	Seien $M,N$ $R$-Moduln. Dann gibt es kanonische Isomorphismen $$Ext^{n}_R(M,N) \simeq R^{n}Hom_R(-,N)(M)$$ für alle $ n \geq 0$, insbesondere kann $Ext^{n}_R(M,N)$ auch über eine projektive Auflösung $P_{\bullet} \to M $ von $M$ betrachtet werden via $Ext^{n}_R(M,N)=H^{n}(Hom_R(P_{\bullet},N))$.
\end{sa}
\begin{proof}
	Sei $N$ fixiert. \\
	\begin{enumerate}
		\item Die Familie kontravarianter Funktoren $(R^{n}Hom_R(-,N))_{n \geq 0}$ ist ein kontravarianter universeller $\delta$-Funktor mit $ R^{0}Hom_R(-,N)= Hom_R(-,N)$ (analoge Aussage zu \ref{9.7}). Denn: $(R^{n}Hom_R(-,N))_{n \geq 0}$ ist ein  kontravarianter $\delta$-Funktor nach der kontravaritanten version von \ref{8.3}. Es bleibt noch zu zeigen: $(R^{n}Hom_R(N,-))_{n \geq 0}$ ist universell. Dafür genügt es nach \ref{9.6} zu zeigen, dass $R^{n}Hom_R(-,N)$ koauslöschbar für alle $n \geq 1$. Sei $M$ ein $R$-Modul. Dann existiert ein projektiver $R$-Modul $P$ und ein Epimorphismus $u: P \to M $
		%epi
		So erhält man: $R^{n}Hom_R(-,N)(u) : R^{n}Hom_R(-,N)(M) \to R^{n}Hom_R(-,N)(P). P $ ist $Hom_R(-,N)$-zyklisch, da ( $\dots 0 \to 0 \to P) \overset{id_P}{\to} P $ eine projektive Auflösung von $P$ ist und $R^{n}Hom_R(-,N)(P) =H^{n}(Hom_R(P,N) \to 0 \to 0 \to \dots ) =0 $ für $ n \geq 1$ ist. Daraus folgt: $R^{n}Hom_R(-,N)(n) = 0,$ das heißt $R^{n}Hom_R(-,N) $ ist koauslöschbar für $n \geq 1 $. 
		\item Wir setzen: $$F^{n}: R-Mod \to \Z-Mod, M \mapsto R^{n}Hom_R(M,-)(N) = H^{n}(Hom_R(M,I^{\bullet}))$$ wobei $N \to I^{\bullet} $ eine injektive Auflösung von $N$ ist. Dann ist $(F^{n})_{n \geq 0} $ ebenfalls ein kontravarianter universeller $\delta$-Funktor mit $F^{0} = Hom_R(-,N),$ da:
		$(F_n)_{n \geq 0} $ ist ein kontravarianter $\delta$-Funktor, denn:
		\begin{itemize}
			\item $F^{n}$ ist kontravarianter additver Funktor: klar
			\item Sei $\begin{tikzcd}
			0\arrow{r} & M' \arrow{r} & M \arrow{r} & M'' \arrow{r} & 0
			\end{tikzcd}$ eine exakte Sequenz von $R$-Moduln, so erhält man eine Sequenz von Komplexen:$$\begin{tikzcd}
			0\arrow{r} & Hom_R(M'',I^{\bullet}) \arrow{r} & Hom_R(M,I^{\bullet})\arrow{r} & Hom_R(M', I^{\bullet}) \arrow{r} & 0
			\end{tikzcd}$$ (beachte: $Hom_R(-,I)$ exakt für injektive $R$-Moduln I). Die lange exakte Kohomologiefolge liefert die Behauptung. 
		\end{itemize}
		$(F_n)_{n \geq 0} $ ist ein universell: nach \ref{9.6} genpgt zu zeigen, dass $F_n$ koauslöschbar für alle $n \geq 1$. sei $M$ ein $R$-Modul. Dann existiert ein projektiver $R$-Modul $P$ und ein Epimorphismus $u: P \to M $ 
		%epi
		Damit erhält man $F^{n}(u): F^{n}(M) = R^{n}Hom_R(m,-)(N) \to R^{n}Hom_R(P,-)(N) = F^{n}(P). $ Wegen der Projektivität von $P$ ist $Hom_R(p,-) $exakt und deshalb ist $R^{n}Hom_R(P,-)= 0 $ für $n \geq 1$ Daraus folgt $F^{n}(u) = 0$, das heißt $F_n $ ist koauslöschbar für $ n \geq 1$. 
		\item nach 1 und 2 sind $(R^{n}Hom_R(-,N))_{n \geq 0} $ und $ (F^{n})_{n \geq 0} $ beides kontravariante $\delta$-Funktoren, mit $ R^{0}Hom_R(-,N) = Hom_R(-,N)= F^{0}. 
		$ Mit \ref{9.4} folgt, dass es für alle $R$-Moduln M eine kanonische Isomorphie: 
		$$R^{n}Hom_R(-,N)(M) \simeq F(M) = R^{n}Hom_R(M,-)(N) = Ext^{n}_R(M,N)$$
	\end{enumerate}
\end{proof}
\begin{sa}\label{10.3}
	Sei $A$ ein Hauptidealring, $M,N$, seinen $R$-Moduln. Dann gilt: $Ext^{n}_A(M,N) =0 $
	 für alle $ n \geq 2$.
\end{sa}
\begin{proof}
	\begin{enumerate}
		\item Wir konstruieren eine injektive Auflösung von $N$. Es existiert ein injektiver $A$-Modul $I^{0}, $ und ein Monomorphismus $\epsilon: N \hookrightarrow I^{0}.$ 
		$$\begin{tikzcd}
		0\arrow{r} & N \arrow{r}{\epsilon} & I^{0} \arrow{r} & I^1 \arrow{r} & 0 
		\end{tikzcd}$$
		$I^{0} $ injektiv, dann folgt mit \ref{6.11} $I^{0} $ ist teilbar $\Ra \coker \epsilon = \QR{I^0}{im \epsilon } $ teilbar, damit folt aus \ref{6.12} und da A ein HIR, dass der $\coker \epsilon $ injektiv ist.  Setze $I^1 := \coker \epsilon$, dann ist $\begin{tikzcd}
		N \arrow{r}{\epsilon} & I^{0} \arrow{r} & I^1 \arrow{r} & 0 \arrow{r}  & 0 \dots
		\end{tikzcd}$ eine injektive Auflösung von $N$. 
		\item Für $n \geq 2 $ ist $ Ext^{n}_A(M,N) = H^{n}(Hom_R(M,J^{\bullet})) = 0.$
		
	\end{enumerate}
\end{proof}
\begin{bem+df}
	Seien $M,N $ $R$-Moduln. \\
	$ \mathcal{E}(M,N) = \{exakte Sequenzen \begin{tikzcd}
	0\arrow{r} & N \arrow{r} & E\arrow{r} & M \arrow{r} & 0
	\end{tikzcd} von R-Moduln\}. $ Wir definieren auf $\mathcal{E}(M,N) $ eine Relation "$\sim$" wie folgt: \\
	$\begin{tikzcd}
	0\arrow{r} & N \arrow{r} & E\arrow{r} & M \arrow{r} & 0
	\end{tikzcd} \sim \begin{tikzcd}
	0\arrow{r} & N \arrow{r} & E'\arrow{r} & M \arrow{r} & 0
	\end{tikzcd}$ 
	$\defi$ Es exsistiert ein Homomorphismus $\alpha:E \to E'$, sodass: 
	$$\begin{tikzcd}
	0\arrow{r} & N\arrow{d}{id} \arrow{r} & E\arrow{r} \arrow{d}{\alpha}& M\arrow{d}{id} \arrow{r} & 0\\
	0\arrow{r} & N \arrow{r} & E'\arrow{r} & M \arrow{r} & 0
	\end{tikzcd} $$ 
	kommutiert (nach Fünferlemma ist $\alpha$ ein Isomorphismus). $"\sim"$ ist eine Äquivalenzrelation auf $\mathcal{E}(M,N)$, wir setzen: $E(M,N) := \QR{\mathcal{E}(M,N)}{\sim}$. \\
	$E(M,N)$ enthält ein ausgezeichentes Element, die Äquivalenzklasse der spaletenden exakten Sequenzen $\begin{tikzcd}
	0\arrow{r} & N \arrow{r} & N \oplus M \arrow{r} & M \arrow{r} & 0
	\end{tikzcd}$.
\end{bem+df}