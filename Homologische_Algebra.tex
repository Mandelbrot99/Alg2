\newpage
\section{Homologische Algebra}
\begin{center}
	In diesem Kapitel sei R stets ein Ring
\end{center}
\setcounter{subsection}{3}
\subsection{Kategorien}
\begin{df}
	Eine Kategorie $ \mathcal{C} $ besteht aus 
	\begin{itemize}
		\item einer Klasse $\ObC$ von "'Objekten"'
			einer Menge $Mor_{\mathcal{C}}(A,B) $ von "'Morphismen"' für alle $A,B \in \ObC$
		\item einer Verknüpfung $\circ : \Mor_{\mathcal{C}}(B,C) \times \Mor_{\mathcal{C}}(A,B) \to Mor_{\mathcal{C}}(A,C) $ für alle $A,B,C \in \ObC$
	\end{itemize}
	wobei folgende Axiome gelten:
	\begin{enumerate}
		\item[$(K1)$] $Mor_{\mathcal{C}}(A,B) \cap Mor_{\mathcal{C}}(A',B') = \emptyset$, falls $A \neq A'$ oder $B \neq B'$
		\item[$(K2)$] Für alle $A,B,C,D \in \ObC, f \in Mor_{\mathcal{C}}(A,B), g \in Mor_{\mathcal{C}}(B,C), h \in Mor_{\mathcal{C}}(C,D)$ gilt:
		$$ h \circ (g \circ f) = (h \circ g) \circ f \qquad (\text{Assoziativität})$$
		\item[$(K3)$] für jedes $ A \in \ObC$ existiert ein Morphismus $id_A \in Mor_{\mathcal{C}}(A,A)$, sodass für alle $B \in \ObC, f \in Mor_{\mathcal{C}}(A,B), g \in Mor_{\mathcal{C}}(B,A)$ gilt: $$f \circ id_A = f, \ id_A \circ g = g $$.
	\end{enumerate}
\end{df}
\begin{anm}
	\begin{itemize}
		\item Man sagt "Klasse"' statt Menge, um Paradoxien, wie " die Menge aller Mengen" zu vermeiden.
		\item Trotzdem schreiben wir $A \in \ObC$ um zu sagen dass $A$ zu $\ObC$ gehört (und werden $\ObC$ im Folgenden wie eine Menge behandeln).
		\item In den folgenden Abschnitten werden wir mengentheoretische Probleme ignorieren und häufig von Mengen sprechen auch wenn es sich nur um Klassen handelt.
		\item Für $f \in \Mor_{\mathcal{C}}(A,B)$ schreiben wir auch $f: A \to B $. $A$ heißt "'Quelle"' und $B$ heißt "'Ziel"' von $f$; wegen $(K1)$ sind diese eindeutig bestimmt.
		\item für $A \in \ObC$ ist $id_A$ eindeutig bestimmt (analoges Argument wie bei Monoiden: $id_A = id_A^{'} \circ id_A = id_A^{'}$)
	\end{itemize}
\end{anm}
\begin{bsp}
	\begin{itemize}
		\item Mengen: Kategorie der Mengen mit Abbildungen von Mengen als Morphismen
		\item Ringe: Kategorie der Ringe mit Ringhomomorphismen als Morphismen
		\item $R$-Mod: Kategorie der $R$-(Links)-Moduln mit $R$-Modulhomomorphismen als Morphismen
		\item Top: Kategorie der topologischen Räume mit stetigen Abbildungen als Morphismen
		\item $\ObC=\{\ast\}, Mor_{\mathcal{C}}(\ast,\ast) := M $, wobei $M$ Monoid, $\circ = $ Verknüpfung in $M$.
	\end{itemize}
\end{bsp}
\begin{df}
	Sei $\mathcal{C}$  eine Kategorie. Die zu $\mathcal{C}$ "duale Kategorie'" ($\mathcal{C}^{op}$) ist die Kategorie mit: 
	\begin{itemize}
		\item $\text{Ob}(\mathcal{C}^{op}) = \ObC$, $Mor_{\mathcal{C}^{op}}(A,B) := Mor_{\mathcal{C}}(B,A)$ für $A,B \in \text{Ob}(\mathcal{C}^{op}) = \ObC$
		\item $\circ_{op}: Mor_{\mathcal{C}^{op}}(A,B) \times Mor_{\mathcal{C}^{op}}(B,C) \to Mor_{\mathcal{C}^{op}}(A,C) $ mit $(f,g) \mapsto f \circ g $ für $A,B,C \in \ObC$
	\end{itemize}
\end{df}
\begin{anm}
	\begin{itemize}
		\item  Anschaulich: Übergang von $\mathcal{C}$ zu  $\mathcal{C}^{op} \ \is $ Pfeile umdrehen
		\item $(\mathcal{C}^{op})^{op} = \mathcal{C}$
	\end{itemize}
\end{anm}
\begin{df}
	Seien $\mathcal{C}, \mathcal{D} $ Kategorien. Ein "'(kovarianter) Funktor"' $F: \mathcal{C} \to \mathcal{D} $ besteht aus einer Abbildung $$ \ObC \to \text{Ob}(\mathcal{D}), \quad A \mapsto FA$$
	und Abbildungen: $$ Mor_{\mathcal{C}}(A,B) \to Mor_{\mathcal{D}}(FA,FB), \quad f \mapsto F(f) $$ für alle $A,B \in \ObC$, sodass gilt:
	\begin{enumerate}
		\item[(F1)] $F(g \circ f) = F(g) \circ F(f) $ für alle $ f \in Mor_{\mathcal{C}}(A,B), g\in Mor_{\mathcal{C}}(B,C), \ A,B,C \in \ObC$
		\item[(F2)] $F(id_A) =id_{FA} $ für alle $A \in \ObC.$
	\end{enumerate}
\end{df}
\begin{bsp}
	\begin{enumerate} [label= \alph*)]
		\item Vergiss-Funktoren, zum Beispiel: $R$-Mod $\to $ Mengen, $R$-Mod $\to \Z$-Mod, ...
		\item Sei $\mathcal{C}$ eine Kategorie $\Ra$ Jedes Objekt $X \in \ObC$ induziert einen Funktor $$Mor_{\mathcal{C}}(X,-): \mathcal{C} \to \text{Mengen}, \quad A \mapsto Mor_{\mathcal{C}}(X,A) $$
		Für $f \in Mor_{\mathcal{C}}(A,B) $ ist hierbei $f_{\ast}^{X} := Mor_{\mathcal{C}}(X,-)(f)$ gegeben durch $$ f_{\ast}^{X}: Mor_{\mathcal{C}}(X,A) \to Mor_{\mathcal{C}}(X,B), \quad g \mapsto f \circ g\qquad \begin{tikzcd}
		X \arrow{r}{g} \arrow[swap]{rd}{f_*^X(g)} & A \arrow{d}{f} \\ & B
		\end{tikzcd} $$
		\item Sei $M \in R$-Mod $\Ra Hom_R(M,-): R$-Mod $\to \Z$-Mod, $N \mapsto Hom_R(M,N) $ ist ein Funktor.
	\end{enumerate}
\end{bsp}
\begin{df}
	Seien $\mathcal{C}, \mathcal{D}$ Kategorien. Ein "'(kontavarianter) Funktor"' $F$ von $\mathcal{C}$ nach $\mathcal{D}$ ist ein Funktor $F: \mathcal{C}^{op} \to \mathcal{D} $, das heißt besteht aus einer Abbildung $$ \ObC \to \text{Ob}(\mathcal{D}), \quad A \mapsto FA$$
	und Abbildungen: $$ Mor_{\mathcal{C}}(A,B) \to Mor_{\mathcal{D}}(FB,FA), \quad f \mapsto F(f) $$ für alle $A,B \in \ObC$, sodass gilt:
	\begin{enumerate}
		\item[(F1')] $F(g \circ f) = F(f) \circ F(g) $ für alle $ f \in Mor_{\mathcal{C}}(A,B), g\in Mor_{\mathcal{C}}(B,C), \ A,B,C \in Ob\mathcal{C}$
		\item[(F2')] $F(id_A) =id_{FA} $ für alle $A \in \ObC.$
	\end{enumerate}
\end{df}
\begin{bsp}
		\begin{enumerate} [label= \alph*)]
		\item Sei $\mathcal{C}$ eine Kategorie $\Ra$ Jedes Objekt $Y \in \ObC$ induziert einen kontravarianten Funktor $$Mor_{\mathcal{C}}(-,Y): \mathcal{C} \to \text{Mengen}, \quad A \mapsto Mor_{\mathcal{C}}(A,Y) $$
		Für $f \in Mor_{\mathcal{C}}(A,B) $ ist hierbei $f_{Y}^{\ast} := Mor_{\mathcal{C}}(-,Y)(f)$ gegeben durch $$ f_{Y}^{\ast}: Mor_{\mathcal{C}}(B,Y) \to Mor_{\mathcal{C}}(A,Y), \quad g \mapsto g \circ f\qquad \begin{tikzcd}
		A \arrow{r}{f_Y^*(g)} \arrow[swap]{d}{f} & Y \\B \arrow[swap]{ur}{g}
		\end{tikzcd}$$
		\item Sei $N \in R$-Mod $\Ra Hom_R(-,N): R$-Mod $\to \Z$-Mod, $M \mapsto Hom_R(M,N) $ ist ein kontavarianter Funktor.
	\end{enumerate}
\end{bsp}
\begin{anm}
	\begin{itemize}
		\item Sind $F: \mathcal{C} \to \mathcal{D}, G: \mathcal{D} \to \mathcal{E} $ Funktoren, so ist auf naheliegende Weise der Funktor $G \circ F : \mathcal{C} \to \mathcal{E}$ definiert.
		\item Unter Funktoren werden kommutative Diagramme auf kommutative Diagramme abgebildet.
	\end{itemize}
\end{anm}
\begin{df}
	Seien $\mathcal{C},\mathcal{D} $ Kategorien. "'Das Produkt"' \,$\mathcal{C} \times \mathcal{D} $ ist diejenige Kategorie mit $ Ob(\mathcal{C} \times \mathcal{D}) = Ob(\mathcal{C}) \times \text{Ob}(\mathcal{D}) $ und $ Mor_{\mathcal{C} \times \mathcal{D}}((A_1,B_1),(A_2,B_2)) = Mor_{\mathcal{C}}(A_1,A_2) \times Mor_{\mathcal{D}}(B_1,B_2)$ und "komponentenweisen $\circ"$.
\end{df}
\begin{df}
	Seien $\mathcal{C},\mathcal{D}, \mathcal{E}$ Kategorien. Ein "Bifunktor" \ $F$ "von $\mathcal{C}$ kreuz $\mathcal{D} $ nach $\mathcal{E}$ "' ist ein Funktor $F: \mathcal{C} \times \mathcal{D} \to \mathcal{E}$
\end{df}
\begin{bsp}
	\begin{enumerate}  [label= \alph*)]
		\item $\bigoplus$: $R$-Mod $\times R$-Mod $\to R$-Mod, $(M,N) \to M \bigoplus N $ ist ein Bifunktor
		\item Sei $\mathcal{C} $ eine Kategorie $\Ra \mathcal{C}^{op} \times \mathcal{C} \to \text{Mengen}, (M,N) \mapsto  Mor_{\mathcal{C}}(M,N) $ ist ein Bifunktor.
	\end{enumerate}
\end{bsp}
\begin{df}
	Sei $\mathcal{C}$ eine Kategorie, $ A,B \in \ObC, f: A \to B $ $f$ heißt
	\begin{enumerate}
		\item[] "'Monomorphismus"' $\defi$ Für alle $C \in \ObC,\, g_1,g_2: C \to A $ gilt: $f \circ g_1 = f \circ g_2 \Ra g_1 = g_2$ $\Lra$ Für alle $C \in \ObC$ ist $f_{\ast}^{C}: Mor_{\mathcal{C}}(C,A) \to Mor_{\mathcal{C}}(C,B)$ injektiv.
		\item[] "'Epimorphismus"' $\defi$ Für alle $C \in \ObC, \,g_1,g_2: B \to C $ gilt: $g_1 \circ f = g_2 \circ f \Ra g_1 = g_2$ $\Lra$ Für alle $C \in \ObC$ ist $f_{C}^{\ast}: Mor_{\mathcal{C}}(B,C) \to Mor_{\mathcal{C}}(A,C)$ injektiv.
		\item[] "'Isomorphismus"' $\defi$ Es existiert ein $g:B \to A $ mit $ f\circ g = id_B $ und $ g \circ f = id_A.$
	\end{enumerate}
\end{df}
\begin{anm}
	In der Situation von 4.11 gilt:
	\begin{itemize}
		\item $f$ Monomorphismus in $\mathcal{C}$ $\Lra$ $f$ Epimorphismus in $\mathcal{C}^{op}.$
		\item $f$ Isomorphismus in $\mathcal{C} \Lra  f $ ist Isomorphismus in $\mathcal{C}^{op}.$
		\item Ist $f$ ein Isomorphismus und $g:B \to A $ mit $ f \circ g = id_B $ und $g \circ f = id_A$, dann ist $g$ ein eindeutig bestimmt (und wird mit $f^{-1} $ bezeichnet), denn sind $g_1,g_2: B \to A$ mit dieser Eigenschaft $\Ra g_1 = g_1 \circ id_B = g_1 \circ (f \circ g_2) =(g_1 \circ f) \circ g_2 = id_A \circ g_2 = g_2.$
		\item In Mengen ist $f$ Monomorphismus $\Lra f $ injektv, $f$ Epimorphismus $Lra f$ surjektiv, $f$ Isomorphismus $\Lra f $ bijektiv. Im Allgemeinen ist dies für Kategorien, in denen die Morphismen Abbildungen sind, jedoch falsch (vgl. Bsp. 4.13)
	\end{itemize}
\end{anm}
\begin{bem}
	Sei $\mathcal{C}$ eine Kategorie, $A,B \in \ObC, f:A \to B $ ein Isomorphismus. Dann ist $f$ ein Monomorphismus und Ein Epimorphismus.
	\begin{proof}
		Seien $ C \in \ObC , g_1,g_2:C \to A $ mit $ f \circ g_1 = f \circ g_2 \Ra f^{-1} \circ (f \circ g_1) = f^{-1} \circ (f \circ g_2) \Ra (f^{-1} \circ f) \circ g_1 = (f^{-1} \circ f) \circ g_2 \Ra g_1=g_2 \Ra f $ Monomorphimus. Analog wird gezeigt dass $f$ ein Epimorphimus.
		%hier fehlen die geschweiften klammern 
	\end{proof}
\end{bem}
\begin{anm}
	Die Umkehrung von 4.12 ist im Allgemeinen falsch, siehe nächstes Beispiel.
\end{anm}
\begin{bsp}
	\begin{enumerate} [label= \alph*)]
		\item  Sei $\mathcal{C} = Top  $ die Kategorie der Topologischen Räume mit stetigen Abbildungen. Wir betrachten $id: (\R, \text{diskrete Topologie}) \to (\R, \text{Standardtopologie}).$ Diese ist eine stetige Abbildung, ein Monomorphismus sowie ein Epimorphismus, jedoch kein Isomorphismus (Nicht hömöomorph, da kein stetiges Inverses)
		\item Sei $\mathcal{C} = Ringe, f:\Z \to \Q $ Inklusion. $f$ ist ein  Monomorphismus und ein Epimorphimus (Achtung, denn: Für $g_1,g_2: \Q \to R$ Ringhomomorphismus ist ein Ring $R$ mit $g_1 \circ f = g_2 \circ f, $ das heißt $ g_1\big|_{\Z} = g_2\big|_{\Z} $ folgt $ g_1 = g_2 $
		 \begin{minipage}[t]{0.7\textwidth}
		 	wegen der Universellen Eigenschaft von $\Q$ als
		 	Quotientenkörper von $\Z$), aber kein Isomorphismus. 
		Insbesondere ist ein Epimorphismus in $\mathcal{C} $ im obigen Sinne  ("kategorieller Epimorphismus") nicht dasselbe wie ein surjektiver Ringhomomorphismus.
		\end{minipage}
		\begin{minipage}[t]{0.3\textwidth} 
			$$\begin{tikzcd}
			\Z \arrow[hook]{r}{f} & \Q \arrow[xshift = 0.7ex]{d}{g_2} \arrow[xshift = -0.7ex, swap]{d}{g_1}\\ & R
			\end{tikzcd}$$
		\end{minipage}
	\end{enumerate}
\end{bsp}
\begin{df}
	Seien $\mathcal{C}, \mathcal{D}$ Kategorien, $F,G:\mathcal{C} \to \mathcal{D}$ Funktoren. Eine "'natürliche Transformation"' $t$ von $F$ nach $G$ ($t: F \Ra G$) ist eine Familie $(t_A)_{A\in \ObC}$ von Morphismen $t_A\in \Mor_{\mathcal{D}}(FA, GA)$, sodass 
	$$\begin{tikzcd}
	FA \arrow{r}{t_a} \arrow[swap]{d}{F(f)}& GA\arrow{d}{G(f)}\\
	FB \arrow{r}{t_B} & GB
	\end{tikzcd}$$
	für alle $A,B \in \ObC, \, f:A\to B$ kommutiert. Man sagt häufig auch $t_A: FA \to GA$ ist natürlich in $A$.
\end{df}
\begin{bsp}
	\begin{enumerate}[label= \alph*)]
		\item Sei $\mathcal{C}$ eine Kategorie, $A,B\in \ObC$, $f:A\to B$. Dann ist 
		$$f^* = (f^*_Y)_{Y\in \ObC} : \Mor_{\mathcal{C}}(B, -) \Ra \Mor_{\mathcal{C}}(A, -)$$
		eine natürliche Transformation von Funktoren $\mathcal{C} \to \text{Mengen}$, denn für $Y_1, Y_2\in \ObC$, $g:Y_1 \to Y_2$ kommutiert das Diagramm:
		$$\begin{tikzcd}
		\Mor_{\mathcal{C}}(B, Y_1) \arrow{r}{f_{Y_1}^*}\arrow[swap]{d}{g_*^B} & \Mor_{\mathcal{C}}(A, Y_1)\arrow{d}{g_*^A} \\
		\Mor_{\mathcal{C}}(B, Y_2) \arrow{r}{f_{Y_2}^*} & \Mor_{\mathcal{C}}(B, Y_2)
		\end{tikzcd}$$
		denn: Für $\phi:B\to Y_1$ ist 
		$$(g_*^A \circ f_{Y_1}^*)(\phi) = g_*^A(\phi \circ f) = g \circ \phi \circ f = f^*_{Y_2}(g\circ \phi) = (f_{Y_2}^* \circ g_*^B) (\phi)$$
		\item Sei $K$-VR die Kategorie der $K$-Vektorräume über einem festen Körper $K$ (mit linearen Abbildungen als Morphismen). Für $V\in K$-VR sei $V^*:= \Hom_K(V,K)$ der Dualraum. Die kanonische Abbildung $\phi_v:V \to V^{**}, \; w \mapsto \phi_v(w):V^* \to K, \, \psi \mapsto \psi(w)$ ist natürlich in $V$, denn für $V,W \in K$-VR, eine lineare Abbildung $f:V \to W$ kommutiert das Diagramm
		$$\begin{tikzcd}
		V \arrow{r}{\phi_v}\arrow{d}{f} & V^{**}  \arrow{d}{f^{**}} \\W \arrow{r}{\phi_w} & W^{**}
		\end{tikzcd}$$
		mit $f^{**}: V^{**} \to W^{**}, \, (\phi: V^* \to K) \mapsto f^{**}(\phi) : W^* \to K, \, \psi \mapsto \phi(\underbrace{\psi \circ f}_{\in V^*})$, d.h. $\phi:id_V \Ra \_^{**}$ ist eine natürliche Transformation von $id:K$-VR$ \to K$-VR nach $\_^{**}:K$-VR$ \to K$-VR.
	\end{enumerate}
\end{bsp}
\begin{df}
	Seien $\mathcal{C}, \mathcal{D}$ Kategorien, $F,G: \mathcal{C} \to \mathcal{D}$ Funktoren, $t:F\Ra G$ eine natürliche Transformation. $t$ heißt "'natürliche Äquivalenz"' $\defi$ Für alle $A\in \ObC$ ist $t_A:FA \to GA$ ein Isomorphismus. (Notation $t: F\overset{\sim}{\Ra} G$)
\end{df}
\begin{anm}
	Ist $t:F\to G$ eine natürliche Äquivalenz, dann existiert eine natürliche Äquivalenz $t^{-1}: G \overset{\sim}{\Ra} F$ via $t_A^{-1} = (t_A)^{-1} : GA \to FA$ 
\end{anm}
\begin{bsp}
	Bezeichne $K$-VR$_{<\infty}$ die Kategorie der endlichdiimensionalen $K$-VR. Dann ist die natürliche Transformation $\phi:\id \Ra \_^{**}$ aus Beispiel 4.15 eine natürliche Äquivalenz.
\end{bsp}
\begin{df}
	Seien $\mathcal{C}, \mathcal{D}$ Kategorien, $F:\mathcal{C} \to \mathcal{D}$ ein Funktor. $F$ heißt "'Kategorienäquivalenz"' $\defi$ Es existiert ein Funktor $G: \mathcal{D} \to \mathcal{C}$ und natürliche Äquivalenzen $F \circ G \overset{\sim}{\Ra} \id_{\mathcal{D}}$, $G\circ F \overset{\sim}{\Ra} \id_{\mathcal{C}}$
\end{df}
\begin{bsp}
	Der Funktor $\_^*: K$-VR$_{<\infty} \to (K$-VR$_{<\infty})^\text{op}, \; v \mapsto V^*$ ist eine Kategorienäquivalenz, denn mit $\_^{\overset{\sim}{*}}: (K$-VR$_{<\infty})^\text{op} \to K$-VR$_{<\infty}, \; W \mapsto W^*$ gilt offenbar $\_^{\overset{\sim}{*}} \circ \_^* = \_^{**}$, und $\phi:\id \overset{\sim}{\Ra}  \_^{**}$ ist eine natürliche Äquivalenz, analog andersherum (d.h. die Kategorie $K$-VR$_{< \infty}$ ist selbstdual).
\end{bsp}
\begin{sa}[Yoneda-Lemma]
	Sei $\mathcal{C}$ eine Kategorie, $A\in \ObC$, $F:\mathcal{C} \to \text{Mengen}$ ein Funktor. Dann gibt es eine Bijektion 
	\begin{eqnarray*}
	\Phi: \{\text{natürliche Transformationen }t:\Mor_{\mathcal{C}}(A, -) \Ra F\} & \to & F(A) \\
	t & \mapsto & t_a(\id_A)
	\end{eqnarray*}
\end{sa}
\begin{proof}
	\begin{enumerate}
		\item Sei $a\in F(A)$. Wir definieren $S^a: \Mor_{\mathcal{C}}(A,-) \Ra F$ als $s^a= (s^a_B)_{B\in \ObC}$ mit 
		$$s_B^a := F(\phi)(a) \quad \text{für } \phi \in \Mor_{\mathcal{C}}(A,B)$$
		$s^a$ ist eine natürliche Transformation, denn für $B,C\in \ObC, \, f:B \to C$ kommutiert
		$$\begin{tikzcd}
		\Mor_{\mathcal{C}}(A,B) \arrow{r}{s_B^a}\arrow[swap]{d}{f_*^A} & F(B) \arrow{d}{F(f)}\\
		\Mor_{\mathcal{C}}(A,C) \arrow{r}{s_C^a} & F(C)
		\end{tikzcd}$$
		denn:
		\begin{eqnarray*}(F(f) \circ s_B^a)(\phi) &=& F(f) (s_B^a(\phi)) = F(f)(F(\phi)(a)) = F(f\circ \phi)(a) \\
			&=& F(f_*^A(\phi))(a) = s_C^a(f_*^A(\phi))
	\end{eqnarray*}
	\item Setze \begin{eqnarray*}
	\Psi:F(A) &\to& \{\text{natürliche Transformationen }t:\Mor_{\mathcal{C}}(A, -) \Ra F\}\\
	a & \mapsto & s^a
	\end{eqnarray*}
	Dann sind $\Phi, \Psi$ invers zueinander, denn: Für $a\in F(A)$, $t:\Mor_{\mathcal{C}}(A,-) \Ra F$ gilt
	$$(\Phi \circ \Psi)(a) = \Phi(s^a) = s_A^a(\id_A) = F(\id_A)(a) = \id_{FA}(a) = a$$
	und 
	$$(\Psi \circ \Phi)(t) = \Psi(t_A(\id_A))$$
	und für $B\in \ObC, \, \phi \in \Mor_{\mathcal{C}}(A,B)$ gilt wegen der Kommutativität von 
	$$\begin{tikzcd}
	\Mor_{\mathcal{C}}(A,A) \arrow{r}{t_A}\arrow[swap]{d}{\phi_*^A} & F(A) \arrow{d}{F(\phi)}\\
	\Mor_{\mathcal{C}}(A,B) \arrow{r}{t_B} & F(B)
	\end{tikzcd}$$
	$$(\Psi(t_A(\id_A)))_B(\phi) = s^{t_A(\id_A)}_B(\phi)= F(\phi)(t_A(\id_A)) = t_B(\phi_*^A(\id_A)) = t_B(\phi) $$
	d.h. $(\Psi \circ \Phi)(t) = t$
	\end{enumerate}
\end{proof}
\begin{fo}
	Sei $\mathcal{C}$ eine Kategorie, $A,B \in \ObC$. Dann ist die Abbildung 
	\begin{eqnarray*}
	 \Psi: \Mor_{\mathcal{C}}(B,A) & \longrightarrow & \{\text{natürliche Transformationen } \Mor_{\mathcal{C}}(A,-) \Ra \Mor_{\mathcal{C}}(B,-) \}\\
	 \psi:B \to A & \mapsto & \psi^*: \Mor_{\mathcal{C}}(A,-) \to \Mor_{\mathcal{C}}(B,-)
 	\end{eqnarray*}
 bijektiv.
\end{fo}
\begin{proof}
	Wende 4.20 auf $F= \Mor_{\mathcal{C}}(B,-)$ and. In der Notation des Beweises von 4.20 ist $\Psi(\psi) = s^\psi = (s_C^\psi)_{C\in \ObC}$, wobei für $C\in \ObC, \, \phi \in \Mor_{\mathcal{C}}(A,C)$ gilt: 
	$$(s_C^\psi)(\phi) = \Mor_{\mathcal{C}}(B,-)(\phi)(\psi) = \phi_*^B(\psi) = \phi \circ \psi = \psi^*_C(\phi)$$
	d.h. $\Psi(\psi) = \psi^*$.
\end{proof}
\begin{anm}
	\begin{itemize}
		\item Folgerung 4.21 liefert einen sogenannten volltreuen Funktor $\mathcal{C}^\text{op} \to \text{Funk}(\mathcal{C}, \text{Mengen})$, wobei $A\mapsto \Mor_{\mathcal{C}}(A,-)$, wobei $\text{Funk}(\mathcal{C}, \text{Mengen})$ die Funktorkategorie von $\mathcal{C}$ nach Mengen bezeichnet (Objekte sind Funktoren: $\mathcal{C} \to $ Mengen, und Morphismen die natürlichen Transformationen) ("'Yoneda-Einbettung"')
		\item Folgerung 4.21 liefert insbesondere eine Verallgemeinerung des Satzes von Caley aus der Gruppentheorie: Für eine Gruppe $G$ ist $G \hookrightarrow S(G), \; g \mapsto \tau_G$ (Linkstranslation mit $g\in G$) ein injektiver Gruppehomomorphismus. Wende 4.21 an auf: \begin{itemize}
			\item $\mathcal{C} = $ Kategorie mit $\ObC = \{ \cdot  \}$, $\Mor_{\mathcal{C}}( \cdot , \cdot ) = G$
			\item $A=B= \cdot $
		\end{itemize}
	und erhalte eine Bijektion 
	\begin{eqnarray*}
		G= \Mor_{\mathcal{C}}( \cdot, \cdot )  & \longrightarrow & \{\text{natürliche Transformationen } \Mor_{\mathcal{C}}( \cdot, -)  \Ra \Mor_{\mathcal{C}}(\cdot, -) \}\\
		g & \mapsto & g^*: \Mor_{\mathcal{C}}( \cdot, -) \Ra \Mor_{\mathcal{C}}( \cdot, -) \is \tau_G
	\end{eqnarray*}
	\end{itemize}
\end{anm}
\newpage
\subsection{Abelsche Kategorien}
\begin{df}
	Sei $\mathcal{C}$ eine Kategorie, $A\in \ObC$. $A$ heißt
	\begin{enumerate}
		\item[] "'Anfangsobjekt"' $\defi$ Für alle $M\in \ObC$ ist $\Mor_{\mathcal{C}}(A,M)$ einelementig
		\item[] "'Endobjekt"' $\defi$ Für alle $M\in \ObC$ ist $\Mor_{\mathcal{C}}(M,A)$ einelementig
	\end{enumerate}
\end{df}
\begin{anm}
	Falls sie existieren, sind Anfangs- und Endobjekte eindeutig bestimmt bis auf eindeutigen Isomorphismus (denn: Sind $A_1, A_2$ Anfangsobjekte, dann ist $\Mor_{\mathcal{C}}(A_1, A_2) = \{\alpha\}$, $\Mor_{\mathcal{C}}(A_2, A_1) = \{\beta\}$, $\Mor_{\mathcal{C}}(A_1, A_1) = \{\id_{A_1}\}$ und analog $\Mor_{\mathcal{C}}(A_2, A_2) = \{\id_{A_2}\}$, insbesondere ist $\beta \circ \alpha = \id_{A_1}, \, \alpha \circ \beta = \id_{A_2}$).
\end{anm}
\begin{df}
	Sei $\mathcal{C}$ eine Kategorie. $0\in \ObC$ heißt "'Nullobjekt"' $\defi 0$ ist sowohl Anfangs- als auch Endobjekt. Existiert in $\mathcal{C}$ ein Nullobjekt $0$, so enthält $\Mor_{\mathcal{C}}(A,B)$ für alle $A,B \in \ObC$ einen ausgezeichnetes Element, den "'Nullmorphismus"' $ A \to 0 \to B$
\end{df}
\begin{anm}
	Der Nullmorphismus in $\Mor_{\mathcal{C}}(A,B)$ ist unabhängig von der Wahl des Nullobjekts:
	$$\begin{tikzcd}
	A \arrow{r} \arrow{dr} & 0\arrow{d} \arrow{r} & B \\
	& \tilde{0} \arrow{ur} &
	\end{tikzcd}$$
\end{anm}
\begin{bsp}
	\begin{enumerate}[label=\alph*)]
		\item In Mengen ist $\emptyset$ ein Anfangsobjekt, jede einelementige Menge ist ein Endobjekt, insbesondere existiert in Mengen kein Nullobjekt
		\item in Ringe ist $\Z$ ein Anfangsobjekt, und der Nullring ist ein Endobjekt. In Ringe existiert ebenfalls ein Nullobjekt
		\item In $R$-Mod ist der Nullmodul ein Nullobjekt.
	\end{enumerate}
\end{bsp}
\begin{df}
	Sei $\mathcal{C}$ eine Kategorie. $(A_i)_{i \in I} $ eine Familie von Objjekten aus $\mathcal{C}$. Ein Produkt $(A, (p_i)_{i \in I})$ von $(A_i)_{i \in I}$ ist ein Objekt $A\in \mathcal{C} $ zusammen mit Morphismen $p_i:A \to A_i $, sodass für alle $B \in \ObC $die Abbildung $$ Mor_C(A,B) \to \prod_{i \in I} Mor_C(B,A_i) , \quad f \mapsto (p_i \circ f) _{i \in I} $$ bijektiv ist, das heißt für jede Familie $(f_i)_{i \in I}$ von Morphismen $f_i: B \to A_i $ existiert ein eindeutig bestimmtes $f:B \to A $ mit $ f_i = p_i \circ f $ für alle $ i \in I$.
\end{df}
\begin{bem}
	Sei $\mathcal{C}$ eine Kategorie, $(A_i)_{i \in I}$ eine Familie von Objekten aus $\mathcal{C}$, $(A, (p_i)_{i \in I}), (A', (p_i')_{i \in I}),$ Produkte von $(A_i)_{i \in I}. $Dann existiert ein eindeutig bestimmter Isomorphismus $ f:A \to A' $, sodass für alle $i \in I $ gilt: $p_i' \circ f = p_i$
	(kurz: $A,A'$ sind kanonisch isomorph. Wir sprechen daher oft von \textbf{dem } Produkt und schreiben $ A = \prod_{i \in I} A_i $
	%hier fehlt noch ein Diagramm (1.1)
\end{bem}
\begin{proof}
	\begin{enumerate} 
		\item Wir wenden die Universelle Eigenschaft auf das Produkt $(A', (p_i')_{i \in I}), B = A, f_i = p_i \Ra $ Wir erhalten einen eindeutig bestimmten $ f: A \to A' $ mit $ p_i' \circ f = p_i $ für alle $ i \in I$. Analog: Wenden die Universelle Eigenschaft auf das Produkt $(A, (p_i)_{i \in I}), B = A', f_i= p_i' \Ra $ Es existiert genau ein $ g: A' \to A $ mit $p_i \circ g = p_i'$ für alle $i \in I $.
		\item $ g \circ f = id_A, f \circ g = id_{A'} $ ( d.h $f$ ist ein Isomorphismus ) denn: Für alle $ i \in I $ ist $p_i \circ ( g \circ f ) = ( p_i \circ g) \circ f =  p_i' \circ f = p_i $ wenden die Universelle Eigenschaft auf das Produkt$ (A, (p_i)_{i \in I}), B= A, f_i = p_i $ an:
		Es existiert genau ein $ h: A \to A $ mit $ p_i \circ h = p_i $ für alle $ i \in I $ ( nämlich $ h = id_A $) Somit ist $ id_A = g \circ f.$ Analog:  $ f \circ g = id_{A'}$.
	\end{enumerate}
\end{proof}
\begin{bsp}
	\begin{enumerate} [label=\alph*)]
		\item In Mengen ist das Produkt das kartesische Produkt.
		\item In $R$-Mod ist das Produkt das direkte Produkt.
		\item In der Kategorie der endlichen abelschen Gruppen existiert kein Produkt der Familie $( \QR{\Z}{n\Z})_{n \in \N}$ (Übung)
	\end{enumerate}
\end{bsp}
\begin{bem+df}
	Sei $\mathcal{C} $ eine Kategorie, $ (A_i)_{i \in I}$ eine Familie von Objekten  aus $ \mathcal{C}$. Ein Koprodukt $(A, (q_i)_{i \in I})$ von $(A_i)_{i \in I}$ ist ein Objekt $ A \in \ObC $ zusammen mit Morphismen $ q_i: A_i \to A$ , sodass $(A, (q_i)_{i \in I})$ Ein Produkt von $(A_i)_{i \in I } $ in $ \mathcal{C}^{op} $ ist, das heißt für alle $ B \in \ObC $ ist die Abbildung $$ Mor_C(A,B) \to \prod_{i \in I} Mor_C(A_i,B) , \quad f \mapsto (f \circ q_i) _{i \in I} $$ bijektiv ist. Falls es existiert ist ein Koprodukt von $(A_i)_{i \in I }$ eindeutig bestimmt bis auf Isomorphie (analog 5.5). Wir sprechen dan von \textbf{dem} Koprodukt und schreiben $ A = \bigoplus_{i \in I} A_i $
	%koprodukt rein
\end{bem+df}
\begin{bsp}
	\begin{enumerate} [label=\alph*)]
		\item In Mengen ist das Koprodukt die disjunkte Vereinigung.
		\item in $R$-Mod ist das Koprodukt die direkte Summe.
		\item In der Kategorie der Gruppen existiert ein Koprodukt, das sogenannte freie Produkt ( siehe Zettel Algebra 1)
	\end{enumerate}
\end{bsp}
\begin{df}
	Sei $\mathcal{A} $ eine Kategorie. $\mathcal{A} $ heißt "'additiv"', denn gilt:
	\begin{enumerate}
		\item[($K1$)] $\mathcal{A}$ hat ein Nullobjekt,
		\item[($K2$)] In $\mathcal{A}$ existieren endliche Produkte
		\item[($K3$)] Für alle $A,B \in \ObA$ trägt $Mor_{\mathcal{A}}(A,B) $ die Struktur einer abelschen Gruppe mit dem Nullmorphismus als neutrales Element, sodass für alle $ A,B,C \in \ObA $ die Verknüpfung:
		 $$ Mor_{\mathcal{A}}(B,C) \times  Mor_{\mathcal{A}}(A,B) \to Mor_{\mathcal{A}}(A,C)$$ bilinear ist.
	\end{enumerate}
\end{df}
\begin{anm}
	In einer additiven Kategorie $\mathcal{A} $ schreiben wir auch $Hom_{\mathcal{A}}$ für $Mor_{\mathcal{A}}. $
\end{anm}
\begin{bsp}
	\begin{enumerate} [label=\alph*)]
		\item $R$-Mod ist eine additive Kategorie
		\item Ringe sind keine additive Kategorie (kein Nullobjekt, vgl 5.3(b)).
	\end{enumerate}
\end{bsp}
\begin{sa}
	Sei $\mathcal{A} $ eine Kategorie, $ A_1,A_2 \in \ObA$ ,$(A_1 \times A_2, (p_1,p_2)) $ Produkt von $ A_1 \times A_2 $, $ i_1: A_1 \to A_1 \times A_2 $ sei via der Univesellen Eigenschaft gegeben durch $id_{A_1}: A_1 \to A_1 , 0: A_1 \to A_2$. Analog sei $ i_2: A_2 \to A_1 \times A_2 $ sei via der Univesellen Eigenschaft gegeben durch $ 0: A_2 \to A_1, id: A_2 \to A_2$. Dann ist $(A_1 \times A_2, (i_1,i_2)) $ ein Koprodukt von $A_1,A_2$ in $\mathcal{A}$.
\end{sa}
\begin{proof}
	\begin{enumerate}
		\item Behauptung: $i_1 \circ p_1 + i_2 \circ p_2: A_1 \times A_2 \to A_1 \times A_2 $ stimmt mit $id_{A_1 \times A_2} $ überein. Denn: Es ist $p_1 \circ (i_1 \circ p_1 + i_2 \circ p_2) = p_1 \circ i_1 \circ p_1 + p_1 \circ i_2 \circ p_2 = p_1 = p_1 \circ id_{A_1 \times A_2}$ 
			%Klammern drunter 
		Analog: $p_2 \circ (i_1 \circ p_1 + i_2 \circ p_2) = p_2 = p_2 \circ id_{A_1 \times A_2}$ $\overset{UE}{\Ra} i_1 \circ p_1 + i_2 \circ p_2 = id_{A_1 \times A_2}$.
		\item Universelle Eigenschaft des Koprodukts:
		 Sei $ B \in \ObA $, $ f_1: A_1 \to B, f_2: A_2 \to B $ Existenz: Wir setzen $f:= f_1 \circ p_1 + f_2 \circ p_2: A_1 \times A_2 \to B  \Ra f \circ i_1 = f_1 \circ p_1 \circ i_1 + f_2 \circ p_2 \circ i_1 = f_1.$
		 % KLammern 
		Analog: $f \circ i_2 = f_2.$
		Eindeutigkeit: Sei $f': A_1 \times A_2 \to B $ mit $f' \circ i_1 = f_1 , f' \circ i_1 = f_2. \Ra f' = f' \circ (i_1 \circ p_1 + i_2 \circ p_2 ) = f' \circ i_1  \circ p_1 + f' \circ  i_2 \circ p_2 = f_1 \circ p_1 + f_2 \circ p_2 = f $
	\end{enumerate}
\end{proof}
\begin{fo}
	Sei $\mathcal{A} $ eine Additive Kategorie. Dann existieren in $\mathcal{A} $  endliche Koprodukte. 
\end{fo}
\begin{df}
	Seien $ \mathcal{A}, \mathcal{B} $ additve Kategorien, $F: \mathcal{A} \to \mathcal{B}$ Funktor. \\
	 $F$ heißt "'additiv "' $\defi $ für alle $ A, A' \in \ObA $ ist eine Abbildung: $$ Mor_{\mathcal{A}}(A,A') \to  Mor_{\mathcal{B}}(FA,FA'), \quad f \mapsto F(f) $$ ein Homomorphismus abelscher Gruppen.
\end{df}
\begin{anm}
	$F$ additiv $\Ra  F(A \bigoplus A' ) = F(A) \bigoplus F(A') $ (Übungen)
\end{anm}
\begin{bem+df}
	Sei $ \mathcal{A}$  eine additive Kategorie, $ A,A' \in \ObA, f: A \to A' $. Ein Kern $(B,i) $ von $f$ ist ein Objekt $ B \in \ObA$ zusammen mit einem Morphismus $ i: B \to A $, sodass $ f \circ i = 0 $ ist und für alle $ C \in$Ob$\mathcal{A}$  die Abbildung: $$Hom_{\mathcal{A}}(C,B) \to \{g \in Hom_{\mathcal{A}}(C,A) | f \circ g = 0 \}, \quad h \mapsto i \circ h $$ bijektiv ist, das heißt für alle $g:C \to A $ mit $ f \circ g = 0 $ existiert ein eindeutig bestimmter Morphismus $ h: C \to B $ mit $ g = i \circ h$:
	$$\begin{tikzcd}
	B \arrow{r}{i} \arrow[swap]{d}{h} & A \\C \arrow[swap]{ur}{g}
	\end{tikzcd}$$
	%hier ist noch was nicht ganz fertig 2.1
	Ist $(B',i') $ ein weiterer Kern von f, dann existiert ein eindeutig bestimmete Isomorphismus $ \alpha: B \to B' $mit $ i = i' \circ \alpha$:
	%Diagramm 2.2
	Wir nennen $(B,i) $ daher auch \textbf{den} Kern von$f$ und schreiben $ \ker f= (B,i)$ beziehungsweise kürzer: $ \ker f= B $.
\end{bem+df}
\begin{anm}
	Die Existenz von Kernen ist im Allgemeinen nicht gegeben
\end{anm}
\begin{bsp}
	In $R$-Mod ist der Kategorielle Kern gegeben durch die Inklusion des gewphnlichen Kerns:
	$$\begin{tikzcd}
	\ker f \arrow{r}{i} \arrow[swap]{d}{h} & A \\C \arrow[swap]{ur}{g}
	\end{tikzcd}$$
	%hier ist auch was noch nicht fertig 2.3
	$ f \circ g = 0 \Ra \im g \subseteq \ker f $ setze  $ h := g :  C \in \ker f$, dann ist $ i\circ h = g $ und $h$ ist eindeutig mit dieser Bedingung.
\end{bsp}
\begin{bem}
	Sei $\mathcal{A} $ eine additive Kategorie, $ A,A' \in \ObA, f: A  \to A' , (\ker f, i ) $ Kern von f. Dann ist i ein Monomorphismus. 
\end{bem}
\begin{proof}
	Seien $ h_1,h_2: C \to \ker f $ mit $ i \circ h_1 = i \circ h_2 =:g \Ra f \circ g = f \circ i \circ h_1 = 0 \Ra $ Es existiert ein eindeeutig bestimmtes $h: C \to \ker f $ mit $ g = i \circ h \Ra h = h_1 - h_2. $
\end{proof}
\begin{bem+df}
	Dual zum Kern definiert man den Kokern (Notation:$ \coker f$). Die Aussage 5.14, 5.16 gelten dual.
\end{bem+df}
\begin{df}
	 Sei $\mathcal{A} $ eine additive Kategorie, $ A,A' \in \ObA, f: A \to A'$ $\im f := \ker(\coker f) $ heißt das "'Bild"' von $f$ 
	 $\coim f := \coker(\ker f) $ heißt das "'Kobild"' von $f$.
\end{df}
\begin{anm}
	$\im f $ kommt mit einem Monomorphisus $i': \im f \to A' , \coim f $ mit einem Epimorphismus $ q': A \to \coim f $.
\end{anm}
\begin{bsp}
	Sein $\mathcal{A} = R $-Mod ,$ f: A  \to A' $ $R$-Modulhomomorphismus. Dann ist $\im f = \ker(\QR{A'}{\im f}, A' \to \QR{A'}{\im f}) = (\im f, \im f \to A'), \coim f= \coker(\ker f, \ker f \to A) = (\QR{A}{\ker f}, A \to \QR{A}{\ker f}).$
\end{bsp}