%---------Umgebungen---------
\newtheorem{df}{Definition}[subsection]
\newtheorem{sa}[df]{Satz}
\newtheorem{lem}[df]{Lemma}
\newtheorem{fo}[df]{Folgerung}
\newtheorem{bem}[df]{Bemerkung}
\newtheorem{bem+df}[df]{Bemerkung + Definition}
\newtheorem{sa+df}[df]{Satz + Definition}
\newtheorem{fo+df}[df]{Folgerung + Definition}
\newtheoremstyle{bsp_bem}{5pt}{3pt}{}{}{\bfseries}{:}{ }{}
\theoremstyle{bsp_bem}
\newtheorem{bsp}[df]{Beispiel}
\newtheorem{alg}[df]{Algorithmus}
\newenvironment{anm}{\textbf{Anmerkung:}}{\medskip}
\newenvironment{Ziel}{\textbf{Ziel}}{\medskip}

%---------Layout---------
\newcommand{\charp}[2][Endo]{\ensuremath{\chi_#2^\text{char}}}
\newcommand{\QR}[2]{
	\raisebox{0.8ex}{\ensuremath{#1}}
	\ensuremath{\mkern-3mu}\big/\ensuremath{\mkern-1mu}
	\raisebox{-0.8ex}{\ensuremath{#2}}}
\newcommand{\is}{\ensuremath{\, \widehat{=} \,}}
\newcommand{\minp}[2][Endo]{\ensuremath{\chi_#2^\text{min}}}
\newcommand{\ad}[2]{\ensuremath{#2^\text{ad}}}
\newcommand{\defi}{\ensuremath{\overset{\text{Def}}{\Leftrightarrow}}}
\newcommand{\norm}{\mathrel{\text{$\unlhd$}}\!}
\newcommand{\iso}{\ensuremath{\overset{\sim}{\longrightarrow}}}
%------griechische Buchstaben-----
\renewcommand{\epsilon}{\varepsilon}
\renewcommand{\phi}{\varphi}
\renewcommand{\rho}{\varrho}

%----------Operatoren-------
\DeclareMathOperator{\sgn}{sgn}
\DeclareMathOperator{\id}{\text{id}}
\DeclareMathOperator{\im}{\text{im}\,}
\DeclareMathOperator{\ord}{\text{ord}\,}
\DeclareMathOperator{\Quot}{\text{Quot}}
\DeclareMathOperator{\cha}{\text{char}\,}
\DeclareMathOperator{\Hom}{\text{Hom}}
\DeclareMathOperator{\coker}{coker}
%------wichtige Befehle----
\newcommand{\N}{\mathbb{N}}
\newcommand{\Z}{\mathbb{Z}}
\newcommand{\Q}{\mathbb{Q}}
\newcommand{\R}{\mathbb{R}}
\newcommand{\K}{\mathbb{K}}
\newcommand{\C}{\mathbb{C}}
\newcommand{\F}{\mathbb{F}}
\def\bign#1{\mathclose{\hbox{$\left#1\vbox to8.5\p@{}\right.\n@space$}}\mathopen{}}

%---------Pfeile-------
\newcommand{\ra}{\rightarrow}
\newcommand{\la}{\leftarrow}
\newcommand{\Ra}{\Rightarrow}
\newcommand{\La}{\Leftarrow}
\newcommand{\Lra}{\Leftrightarrow}

%--------Ableitungen----------
\renewcommand{\bar}{\overline}

\begin{comment}
Beispielschablonen für Diagramme etc.



Dreieck z.B. Homomorphiesatz TODO: Anpassung der Bezeichnungen


$$\begin{tikzcd}
G \arrow[swap]{dr}{\pi} \arrow{rr}{\phi} & & G'\\
& G/N \arrow[swap,dashed]{ru}{\overline{\phi}}&
\end{tikzcd}$$

%inklusionsumkehrende Bijektion mit Array:
\begin{eqnarray*}
\Phi:\{\text{Untergruppen von } G/H\} &\longrightarrow &\{\text{Untergruppen } V \text{ von } G \text{ mit } V\supseteq H\}\\
U & \mapsto & \pi^{-1}(U)
\end{eqnarray*}



exakte Sequenz, geht auch inline



$\begin{tikzcd}
\ldots \arrow{r}{a}& A \arrow{r}{b}& B \arrow{r}{c} & C \arrow{r} & \ldots
\end{tikzcd}$


kommutatives Diagramm aus exakten Sequenzen


$$\begin{tikzcd}
\ldots \arrow{r}{a}& A \arrow{r}{b}\arrow{d}{b'}& B \arrow{r}{c} \arrow{d}{c'}& C \arrow{r}\arrow{d}{d'} & \ldots\\
\ldots \arrow{r}{d}& D \arrow{r}{e}& E \arrow{r}{f} & F \arrow{r}{g} & \ldots
\end{tikzcd}$$

spaltende Sequenz
$\begin{tikzcd}
\ldots \arrow{r}{a}& A \arrow{r}{b}& B \arrow{r}{c} & C \arrow[bend right = 50]{l}{\phi} \arrow{r} & \ldots
\end{tikzcd}$

Schlangenlemma

\begin{tikzpicture}[>=triangle 60]
\matrix[matrix of math nodes,column sep={60pt,between origins},row
sep={60pt,between origins},nodes={asymmetrical rectangle}] (s)
{
&|[name=ka]| \ker f &|[name=kb]| \ker g &|[name=kc]| \ker h \\
%
&|[name=A]| A' &|[name=B]| B' &|[name=C]| C' &|[name=01]| 0 \\
%
|[name=02]| 0 &|[name=A']| A &|[name=B']| B &|[name=C']| C \\
%
&|[name=ca]| \coker f &|[name=cb]| \coker g &|[name=cc]| \coker h \\
};
\draw[->] (ka) edge (A)
(kb) edge (B)
(kc) edge (C)
(A) edge (B)
(B) edge node[auto] {\(p\)} (C)
(C) edge (01)
(A) edge node[auto] {\(f\)} (A')
(B) edge node[auto] {\(g\)} (B')
(C) edge node[auto] {\(h\)} (C')
(02) edge (A')
(A') edge node[auto] {\(i\)} (B')
(B') edge (C')
(A') edge (ca)
(B') edge (cb)
(C') edge (cc)
;
\draw[->,gray] (ka) edge (kb)
(kb) edge (kc)
(ca) edge (cb)
(cb) edge (cc)
;
\draw[->,gray,rounded corners] (kc) -| node[auto,text=black,pos=.7]
{\(\partial\)} ($(01.east)+(.5,0)$) |- ($(B)!.35!(B')$) -|
($(02.west)+(-.5,0)$) |- (ca);
\end{tikzpicture}

kommutatives Diagramm für Produkt/Koprodukt



$\begin{tikzcd}
& A \arrow[dd] \arrow[rdd, bend left] \arrow[ldd, bend right] &  \\
& & \\
B \arrow[r] & C & D \arrow[l]
\end{tikzcd}$

\end{comment}