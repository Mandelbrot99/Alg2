\newpage
\section{Moduln}
\begin{center}
	\textbf{In dieser Vorlesung steht die Bezeichnung "'Ring"' stets für einen (nicht notwendig kommutativen) Ring mit 1. In diesem Kapitel sei $R$ ein Ring.}
\end{center}
\subsection{Grundlagen über Moduln}
\begin{df}
	Ein "'$R$-Linksmodul"' ist eine abelsche Gruppe $(M, +)$ zusammen mit einer Abbildung $R\times M \to M, \, (a,x) \mapsto ax$ (skalare Multiplikation), sodass für alle $a,b\in R, \, x,y\in M$ gilt:
	\begin{enumerate}[label= \alph*)]
		\item $a(x+y) = ax + ay$
		\item $(a+b)x = ax+bx$
		\item $a(bx) = (ab)x$
		\item $1x = x$
	\end{enumerate}
Ein "'$R$-Rechtsmodul"' ist eine abelsche Gruppe $(M, +)$ zusammen mit einer Abbildung $M\times R \to M, \, (x,a) \mapsto xa$, sodass für alle $a,b\in R, \, x,y\in M$ gilt:
\begin{enumerate}
	\item[$a')$] $(x+y)a = xa+ yb$
	\item[$b')$] $x(a+b) = xa + xb$
	\item[$c')$] $x(ab) = (xa)b$
	\item[$d')$] $x1=x$
\end{enumerate}
\end{df}
\begin{anm}
	Es bezeichne $R^\text{op}$ den zu $R$ entgegengesetzten Ring, d.h. eine Menge $R$ mit derselbern Addition, sowie der Multiplikation $a \cdot_\text{op}b := b \cdot a$. Ist $M$ ein $R-$Rechtsmodul, dann wird $M$ durch $ax:= xa$ zu einem $R^\text{op}$-Linksmodul, denn es gilt 
	$$a(bx) = (bx)a = (xb)a = x(ba) = (ba)x = (a \cdot_\text{op} b)x \quad \text{für alle } a,b\in R, \, x,a\in M$$
	Analog anders herum. Im Folgenden betrachten wir in der Regel nur $R$-Linksmoduln, und unter einem $R$-Modul verstehen wir einen $R$-Linksmodul
	\begin{itemize}
		\item Forderung $a)$ impliziert, dass für alle $a\in R$ die Abbildung 
		$$l_a: M \to M, \quad x\mapsto ax$$
		zum Ring $\text{End}(M)$ aller Gruppenhomomorphismen $M\to M$ gehört.
		$$(\text{mit } (f+g)(x) := f(x) + g(x), \, (f\cdot g) := (f\circ g)(x) = f(g(x))$$für $f,g\in \text{End}(M), \, x\in M)$. Nach $b)-d)$ ist die Abbildung $\phi:R\to \text{End}(M), \, a\mapsto l_a$ ein Ringhomomorphismus. Umgekehrt macht jeder Ringhomomorphismus $\phi:R\to \text{End}(M)$ eine abelsche Gruppe $(M,+)$ zu einem $R$-Modul via $ax:= \phi(a)(x)$
		\item Für alle $x\in M$ ist $0x=0, \, (-1)x=-x$, und für alle $a\in R$ ist $a0=0$ 
	\end{itemize}
\end{anm}
\begin{bsp}
	\begin{enumerate}[label=\alph*)]
		\item Ist $K$ ein Körper, dann sind $K$-Moduln die $K$-Vektorräume.
		\item Jede abelsche Gruppe $G$ ist ein $\Z$-Modul via
		$$\Z\times G\to G, \quad (n,x) \mapsto nx:= \begin{cases} \underbrace{x+ \ldots x}_{\text{n-mal}}& n>0 \\
		0 & n=0\\
		-(\underbrace{x+ \ldots + x}_{\text{(-n)-mal}}) & n<0
		\end{cases}$$
		Für jeden Ring $R$ gibt es genau einen Ringhomomorphismus $\Z\to R$ (analog zur Algebra 1), insbesondere gibt es für jede abelsche Gruppe $G$ genau einen Ringhomomorphismus $\Z\to \text{End}(G)$, d.h. genau eine Struktur als $\Z-$Modul, sodass die Moduladdition mit der gegebenen Addition auf $G$ überein einstimmt (nämlich obige).
	\end{enumerate}
\end{bsp}
\begin{df}
	Seien $M,M'$ $R$-Moduln, $\phi:M \to M'$. Dann heißt $\phi$ "$R$-Modul-\\homomorphismus" ($R$-linear), wenn für alle $x,y\in M, \, a,b\in R$ gilt: 
	\begin{enumerate}[label=\alph*)]
		\item $\phi(x+y) = \phi(x) + \phi(y)$
		\item $\phi(ax) = a\phi(x)$
	\end{enumerate}
	$\text{Hom}_R(M, M')$ bezeichne die Menge der $R$-Modulhomomorphismen von $M$ nach $M'$.
\end{df}
\begin{anm}
	$\text{Hom}_R(M,M')$ ist eine abelsche Gruppe bezüglich $(f+g)(x) := f(X) + g(x)$ für $f,g\in \text{Hom}_R(M, M'), \, x\in M$
\end{anm}
\begin{bsp}
	Sei $M$ ein $R$-Modul, $\phi\in \text{Hom}_R(M, M)=: \text{End}_R(M) \subseteq \text{End}_\Z(M) = \text{End}(M)$. Den Polynomring $R[X]$ kann man wie über kommutativen Ringen definieren, allerdings ist die Einsetzungsabbildung 
	$$R[X] \to R, \quad \sum_{i=0}^na_i X^i\mapsto \sum_{i=0}^na_ib^i, \quad \text{für ein } b\in R$$
	im Allgemeinen kein Ringhomomorphismus ("'$X$ vertauscht mit Elementen aus $R$, $b$ im Allgemeinen nicht"'). Die Abbildung 
	$$\Psi:R[X] \to \text{End}(M), \quad \sum_{i=0}^n a_i X^i \mapsto \sum_{i=0}^n a_i \phi^i$$
	ist ein Ringhomomorphismus, da $\phi$ $R$-linear ist. Somit wird $M$ zum $R[X]$-Modul.
\end{bsp}
\begin{df}
	Seien $M,M'$ $R$-Moduln, $\phi:M\to M'$ $R$-linear. $\phi$ heißt 
	\begin{enumerate}
		\item[] "'Monomorphismus"' $\defi \phi$ ist injektiv (Notation: $M \hookrightarrow M'$)
		\item[] "'Epimorphismus"' $\defi \phi$ ist surjektiv (Notation: $M \twoheadrightarrow M'$)
		\item[] "'Isomorphismus"' $\defi \phi$ ist bijektiv (Notation: $M \overset{\sim}{\ra} M'$)
	\end{enumerate}
	Existiert ein Isomorphismus zwischen $M,M'$, so heißen $M,M'$ "'isomorph"' (Notation: $M \cong M'$)
\end{df}
\begin{anm}
	Ist $\phi$ ein Isomorphismus, dann ist $\phi^{-1}$ ein Isomorphismus.
\end{anm}
\begin{bem}
	Seien $M,M'$ $R$-Moduln. Dann gilt:
	\begin{enumerate}[label=\alph*)]
		\item $R$ kommutativ $\Ra \text{Hom}_R(M, M')$ ist ein $R$-Modul via $(a \phi)(x):= a\phi(x)$ für $a\in R, \, \phi \in \text{Hom}_R(M, M'), \, x\in M$.
		\item $\text{End}_R(M) = \text{Hom}_R(M, M)$ ist ein Unterring von $\text{End}(M) = \text{End}_\Z(M)$.
		\item Die Abbildung $\Phi:\text{Hom}_R(R, M)\to M, \; \phi \mapsto \phi(1)$ ist ein Isomorphismus von abelschen Gruppen (hierbei ist $R$ auf natürliche Weise ein $R$-Linksmodul). Ist $R$ kommutativ, so ist $\Phi$ ein Isomorphismus von $R$-Moduln.
		\item $\text{End}_R(R)\cong R^\text{op}$
	\end{enumerate}
\end{bem}
\begin{proof}
	\begin{enumerate}[label=\alph*)]
		\item Beachte: Für $a\in R, \, \phi\in \text{Hom}_R(M, M')$ ist $a\phi$ wieder $R$-linear, denn für $a,b\in R, \, x\in M$ ist $(a\phi)(bx) = a\phi(bx) = ab \phi(x) = ba \phi(x) = b (a\phi)(x)$
		\item Nachrechnen.
		\item Eine Umkehrabbildung zu $\Phi$ ist gegeben durch 
		$$\Psi:M \to \text{Hom}_R(R, M), \quad m \mapsto(\phi:R\to M, \, a \mapsto am)$$
		\item Nach Aussage $c)$ haben wir sofort einen Isomorphismus: $\Phi:\text{End}_R(R) \to R, \, \phi\mapsto \phi(1)$ von abelschen Gruppen. Es ist 
		\begin{eqnarray*}\Phi(\phi \psi) &=& (\phi \psi)(1) = \phi(\psi(1)) = \phi( \psi(1) \cdot 1) = \psi(1) \phi(1)\\
		&=& \phi(1) \cdot_{\text{op}}  \psi(1) = \Phi(\phi) \cdot_\text{op} \Phi(\psi)
		\end{eqnarray*}
	\end{enumerate}
\end{proof}
\begin{df}
	Sei $M$ ein $R$-Modul, $N\subseteq M$. $N$ heißt $R$-Untermodul von $M$, wenn gilt:
	\begin{enumerate}[label= \alph*)]
		\item $0\in N$
		\item $x+y\in N$ für alle $x,y\in N$
		\item $ax\in N$ für alle $a\in R, x\in N$
	\end{enumerate}
\end{df}
\begin{bsp}
	\begin{enumerate}[label=\alph*)]
		\item Betrachte $R$ als $R$-Linksmodul. Dann sind die Untermodul von $R$ genau die Linksideale in $R$ (analog: Rechtsideale für $R$ als $R$-Rechtsmodul).
		\item Ist $M$ ein $R$-Modul, dann sind $\{0\}$ (meist als $0$ geschrieben) und $M\subseteq M$ die trivialen Untermoduln. Ist $(M_i)_{i\in I}$ eine Familie von Untermoduln von $M$, dann ist $\bigcap_{i\in I} M_I\subseteq M$ ein Untermodul, sowie $\sum_{i\in I} M_i = \{\sum_{i\in I} x_i|\, x_i \in M_i, \, x_i=0 \text{ für fast alle } i\in I\}$
		\item Sind $M, M'$ $R$-Moduln, $\phi\in \text{Hom}_R(M, M'), \, N\subseteq M$ ein Untermodul, $N' \subseteq M'$ ein Untermodul, dann sind $\phi(N) \subseteq M'$ und $\phi^{-1}(N') \subseteq M$ Untermoduln. 
		\begin{enumerate}
			\item[] $\im \phi:= \phi(M)$ heißt das "'Bild"' von $\phi$
			\item[] $\ker \phi:= \phi^{-1}(\{0\})$ heißt der "'Kern"' von $\phi$
		\end{enumerate}
	Es gilt: $\phi$ ist injektiv $\Lra \ker \phi =0$ und $\phi$ surjektiv $\Lra \im \phi = M'$
	\end{enumerate}
\end{bsp}
\begin{bem+df}
	Sei $M$ ein $R$-Modul, $N\subseteq M$ ein Untermodul. Dann ist die Faktorgruppe $\QR{M}{N}$ via $a(x+N) = ax+N, \, a\in R, \,x\in M$ ein $R$-Modul,der "'Faktormodul"' von $M$ nach $N$. Die kanonische Abbildung $\pi:M \to \QR{M}{N}, \; m \mapsto m+N$ ist ein Modulepimorphismus mit $\ker \pi = N$.
\end{bem+df}
\begin{bsp}
	Sei $I\subseteq R$ ein Linksideal, $M$ ein $R$-Modul. Dann ist 
	$$IM:= \left\{\sum_{i=1}^n a_i x_i|\, n\in \N, \, a_i \in I, \, x_i \in M\right\}\subseteq M$$
	ein Untermodul von $M$. Ist $I$ ein zweiseitiges Ideal, dann ist $\QR{R}{I}$ ein Ring (beachte: Die Zweiseitigkeit von $I$ geht ein bei der Wohldefiniertheit der Multiplikation
	$$\QR{R}{I} \times \QR{R}{I} \longrightarrow \QR{R}{I}, \quad (a+I, b+I) \mapsto ab+I$$
	$\QR{M}{IM}$ ist ein $\QR{R}{I}$-Modul vermöge
	$$(a+I)(x+M) := ax+IM, \quad a\in R, \, x\in M$$ 
\end{bsp}
\begin{center}
	\emph{Die nächsten Sätze zeigt man wie für Gruppen ($K$-VR,...)}
\end{center}
\ 
\begin{sa}
	Seien $M, M^{’} $ $R$-Moduln, $N \subseteq M $ ein Untermodul, $ \pi: M \to \QR{M}{N} $ die kanonische Projektion, $\phi: M \to M^{’} $ $R$-Modulhomomorphismus. Dann sind äquivalent: 
\begin{enumerate}[label= \roman*)]
	\item $N \subseteq ker\phi$
	\item Es ex. genau ein Modulhomomorphismus $\overline{\phi}: \QR{M}{N} \to M^{’} $ mit $\overline{\phi} \circ \pi = \phi:$
	$$\begin{tikzcd}
	M \arrow[swap]{dr}{\pi} \arrow{rr}{\phi} & & N\\
	& \QR{M}{N} \arrow[swap,dashed]{ru}{\overline{\phi}}&
	\end{tikzcd}$$
	\end{enumerate}
\end{sa}
\begin{sa}[Homomorphiesatz] Seien $M, M^{’} $ $R$-Moduln,  $\phi: M \to M^{’} $ ein $R$-Modulhomomorphismus. Dann existiert ein $R$-Modulisomorphismus $\overline{\phi}: \QR{M}{ker\phi} \overset{\sim}{\longrightarrow} \im\phi $ mit $\overline{\phi}(x+ker\phi)=\phi(x) $ für alle $ x \in M$.
\end{sa}
\begin{sa}
	(Isomorphiesätze) Sein $M$ ein $R$-Modul, $N_1, N_2 \subseteq M$ Untermoduln. Dann gilt:
	\begin{enumerate}[label=\alph*)]
		\item Die Abbildung$$\QR{N_1}{N_1 \cap N_2} \overset{\sim}{\longrightarrow} \QR{(N_1+N_2)}{N_2} \qquad x + N_1\cap N_2  \mapsto x +N_2$$ ist ein Isomorphismus.
		\item Ist $ N_2 \subseteq N_1$, so ist
		 $$\QR{M/N_2}{M/N_1} \overset{\sim}{\longrightarrow} \QR{M}{N_1}\qquad (x+N_2)+\QR{N_1}{N_2} \mapsto x+N_1$$
		 ein Isomorphismus.
	\end{enumerate}
\end{sa}
\begin{sa}
	Sei $M$ ein $R$-Modul, $N \subseteq M$ ein Untermodul, $ \pi: M \to \QR{M}{N} $ die kanonische Projektion. Dann gibt es eine Bijektion 
	\begin{eqnarray*}
		\{\text{Untermoduln} \ M^{’} \text{von} \  M \text{mit} \ N \subseteq M^{’} \} &\longrightarrow &\{\text{Untermoduln } \text{ von } \QR{M}{N}\}\\
		M^{’}& \mapsto & \pi(M^{’}) \\
		\pi^{-1}(L) &\rotatebox[]{180}{$\mapsto$} & L % Pfeil muss anders rum
	\end{eqnarray*}
	die inklusionserhaltend ist.
\end{sa}
\begin{bem+df}
	Sei $(M_i)_{i \in I}$ eine Familie von $R$-Moduln. Dann gilt: $\prod_{i \in I} M_i$ ist ein $R$-Modul mit komponentenweiser Addition und skalarer Multiplikation und heißt das "'direkte Produkt"' der $M_{i}$. Die Projektionsabbildungen $p_{j}: \prod_{i \in I} M_i \to M_{j} $ mit $(m_i)_{i \in I} \mapsto m_j $ sind $R$-Modulhomomorphismen.
\end{bem+df}
\begin{sa}[Universelle Eingenschaft des Produkts] Sei $(M_i)_{i \in I}$ eine Familie von $R$-Moduln. Dann gilt: Für jeden $R$-Modul $M$ ist die Abbildung $$ \Hom_{R}(M,\prod_{i \in I} M_i) \to \prod_{i \in I} Hom_{R}(M,M_i) \qquad \phi \mapsto (p_i \circ \phi)_{i \in I}$$
	eine Bijektion, d.h. für jede Familie $(\phi_i)_{i \in I}$ von $R$-Modulhomomorphismen $\phi_i: M \to M_i $ ex. genau ein $R$-Modulhomomorphismus $\phi: M \to \prod_{i \in I} M_i$ mit $ p_i \circ \phi = \phi_i$ für alle $i \in I$ (nämlich der durch $\phi(x) := ((\phi_i(x))_{i\in I}) $
\end{sa}
\begin{df}
	Sei $(M_i)_{i \in I}$ eine Familie von $R$-Moduln. Der Untermodul $$ \bigoplus_{i \in I} M_i := \{(m_i)_{i \in I} \in \prod_{i \in I} M_i \ | \text{ fast alle }  m_i =0\} \subseteq \prod_{i \in I} M_i $$ 
	heißt die "'direkte Summe"' der $M_i$. Die Inklusionsabbildungen $$q_j: M_j \to \bigoplus_{i \in I} M_i, \quad x \mapsto (x_i)_{i \in I} \quad \text{mit} \quad x_i=\begin{cases} x & i = j \\ 0 & \text{sonst.} \end{cases}$$ sind $R$-Modulhomomorphismen.
\end{df}
\begin{anm}
	Ist $I$ endlich, dann ist $\bigoplus_{i \in I}M_i = \prod_{i \in I} M_i$.
\end{anm}
\begin{sa}[Universelle Eingenschaft der Summe] Sei $(M_i)_{i \in I}$ eine Familie von $R$-Moduln. Dann gilt: Für jeden $R$-Modul $M$ ist die Abbildung $$ Hom_{R}(\bigoplus_{i \in I} M_i,M) \to \prod_{i \in I} Hom_{R}(M_i,M) \quad \text{mit} \quad \psi \mapsto (\psi \circ q_i)_{i \in I}$$
	eine Bijektion, d.h. für jede Familie $(\psi_i)_{i \in I}$ von $R$-Modulhomomorphismen $\psi_i: M_i \to M $ ex. genau ein $R$-Modulhomomorphismus $\psi: \bigoplus_{i \in I}M_i \to M$ mit $ \psi \circ q_i = \psi_i$ für alle $i \in I$ (nämlich der durch $\psi((m_i)_{i \in I}) := \sum_{i\in I} \psi_i(m_i)$ definierte).
\end{sa}
\begin{anm}
	Sei $I$ eine Indexmenge, $M$ ein $R$-Modul. Dann ist:\\
	 \begin{align*}
	 M^I &:= \prod_{i \in I}M, & M^{(I)} &:= \bigoplus_{i \in I}M, & M^{r} &:= M^{\{1,..,r\}}=M^{(\{1,..,r\})}
	 \end{align*}
\end{anm}
\begin{bem}
	Sei $M$ ein $R$-Modul, $(M_i)_{i \in I}$ eine Familie von Untermoduln von M. Dann erhalten wir (aus der Universellen Eigenschaft von $\bigoplus$ mit $\psi_i:M_i\hookrightarrow M$ Inklusionsabbildung) einen $R$-Modulhomomorphismus $$ \psi:\bigoplus_{i \in I} M_i \to M, \quad (m_i)_{i \in I} \mapsto \sum_{i\in I}m_i \quad \text{mit} \quad\im\psi = \sum_{i\in I}M_i $$
	Ist $\psi$ injekitv, so heißt die Summe $\sum_{i\in I}M_i $ "'direkt"', und wir schreiben auch $\bigoplus_{i \in I}M_i $ für $ \sum_{i\in I}M_i$.
\end{bem}
\begin{anm} In der Situation von 1.19 gilt:
	\begin{itemize}
		\item $\sum_{i\in I}M_i $ direkt $\Longleftrightarrow \sum_{i\in J}M_i $ \ direkt für alle Teilmengen $J \subseteq I$ 
		\item $M_1+M_2 = M_1 \bigoplus M_2 \Longleftrightarrow M_1 \cap M_2 =0$
	\end{itemize}
\end{anm}
\begin{df}
	Sei $M$ ein $R$-Modul und sei $x \in M$. Die Abbildung $f_{x}:R \to M, a \mapsto ax$ ist ein $R$-Modulhomomorphismus, das Linksideal $$ ann_{R}(x) := \ker f_x = \{a \in R \ | \  ax=0\}$$ heißt der "'Annulator" von $x$. Das Bild $\im f_x = Rx=\{ax \ | \ a \in R\} $ heißt der von $x$ erzeugte Untermodul von M. Allgemeiner heißt für eine Teilmenge $X \subseteq M$ $$ RX := \langle X\rangle_{R} := \sum_{x\in X} Rx = \im(R^{(X)} \to M) = \bigcap_{X \subseteq N \subseteq M \atop N \text{Untermodul} \text{ mit } X \subseteq N}N $$
	Der von X erzeugte Untermodul von M.
\end{df}
\begin{df}
	Sei $M$ ein $R$-Modul, $(x_i)_{i \in I} $ Familie von Elementen aus $M$, $\psi:R^{(I)} \to M, (a_i)_{i \in I} \mapsto \sum_{i\in I}a_ix_i$.
	$(x_i)_{i \in I} $ heißt 
	\begin{enumerate}
		\item[] "'Erzeugendensystem"' von $M$ mit $R$ $\defi$  $\psi $ surjektiv $\Longleftrightarrow$ $M$ stimmt mit dem von $(x_i)_{i \in I} $ erzeugten Untermodul überein
		\item[]	"'linear abhängig"' $\Longleftrightarrow$ $\psi $ injektiv 
		\item[] "'Basis"' von $M$ über $R$ $\Longleftrightarrow$ $\psi$ bijektiv 
	\end{enumerate}
	$M$ heißt
	\begin{enumerate}
		\item[]	 "'endlich erzeugt"' $\Longleftrightarrow$ $M$ besitzt ein endliches Erzeugendensystem 
		\item[] "'frei"' $\Longleftrightarrow$ $M$ besitzt eine Basis
	\end{enumerate}
\end{df}
\begin{anm}
	\begin{itemize}
		\item Ist $R=K$ ein Körper, so sind alle $K$-Moduln frei (LA1)
		\item Im allgemeinen ist dies jedoch falsch: $\QR{\Z}{2\Z} $ ist eine abelsche Gruppe (=$\Z$ Modul), die nicht frei als $\Z$-Modul ist.
		\item Jeder $R$-Modul $M$ ist Faktormodul eines freien $R$-Moduls, denn: $$ R^{(M)} \to M, (a_x)_{ x \in M} \mapsto \sum_{x\in M}a_{x}x \quad \text{ist surjektiv}.$$
		\item Basen eines freien $R$-Moduls können unterschiedliche Länge haben.
	\end{itemize}
\end{anm}
\begin{sa}
	Sei $A$ ein kommutativer Ring, $A \neq 0$, $n_1,n_2 \in N $. Dann gilt: \\
	$$ A^{n_1} \simeq A^{n_2} \Longleftrightarrow n_1 =n_2 $$
\end{sa}
\begin{proof}
	Vorüberlegung: nach Algebra 1, 4.18 ex in $A$ ein maximales Ideal $J$ . Sei $n \in \N$. Dann ist $\QR{A^{n}}{JA^{n}}$ ein $\QR{A}{J}$-Modul (vgl Beispiel 1.10) und $\QR{A}{J}$ ist ein Körper. Die Abbildung $\QR{A^{n}}{JA^{n}} \to (\QR{A}{J})^{n}, (x_1,..,x_n) +JA^{n} \mapsto (x_1+J,...,x_n+J) $ ist ein Isomorphismus von $\QR{A}{J}$-Moduln, d.h. $\QR{A^{n}}{JA^{n}} \simeq (\QR{A}{J})^{n} $ ist ein n-dimensionaler $\QR{A}{J}$-Vektorraum. Aus $ A^{n_1} \simeq A^{n_2}$ folgt $\QR{A^{n_1}}{JA^{n_1}} \simeq \QR{A^{n_2}}{JA^{n_2}}$, also $ (\QR{A}{J})^{n_1} \simeq (\QR{A}{J})^{n_2}$ (als $\QR{A}{J}$-Vektorraum)
\end{proof}
\begin{df}
	Sei $A$ ein kommutativer Ring, $M$ ein freier $A$-Modul mit endlicher Basis. Die Kardinalität dieser Basis heißt der "Rang" von M (unabhängig von der Wahl einer endlichen Basis nach 1.22)
\end{df}
\newpage
\subsection{Exakte Folgen}
\begin{df}
	Eine "’exakte Folge (exakte Sequenz) "’ von $R$-Moduln ist eine Familie $(f_i)_{i \in I}4$ von $R$-Modulhomomorphismen $f_i: M_i \to M_{i+1} $ für ein (endliches oder unendliches) Intervall $I \in \Z $, sodass: $$ \im f_i = \ker f_{i+1} \quad \text{für alle } i \in I \text{mit} i+1 \in I $$ gilt. \\
	Schreibweise: \quad  $\begin{tikzcd}
	\ldots \arrow{r}{}& M_{i-1} \arrow{r}{f_{i-1}}& M_i \arrow{r}{f_i} & M_{i+1} \arrow{r} & \ldots
	\end{tikzcd}$.	Eine exakte Folge der Form: $$\begin{tikzcd}
	0  \arrow{r} & M^{’} \arrow{r}{f} & M \arrow{r}{g} & M^{''} \arrow{r} & 0
	\end{tikzcd}  \quad (\ast) $$
	heißt eine "'kurze exakte Folge"' (hierbei sind die äußeren Abbildungen die Nullabbildungen). Die Exaktheit von $ (\ast)$ bedeutet explizit:
	\begin{itemize}
		\item $f$ injektiv
		\item $g$ surjektiv
		\item $\im f = \ker g $.
	\end{itemize}
\end{df}
\begin{anm}
	\begin{itemize}
		\item Seien $M,N$ $R$-Moduln und $ f:M \to N $ ein $R$-Modulhomomorphismus. Falls $f$ injektiv, dann ist $\begin{tikzcd}
		0  \arrow{r} & M \arrow{r}{f} & N \arrow{r}{g} & \QR{N}{\im f} \arrow{r} & 0
		\end{tikzcd} $ exakt. \\
		falls $f$ surjektiv, so ist $\begin{tikzcd}
		0  \arrow{r} & \ker f \arrow{r} & M \arrow{r}{f} & N \arrow{r} & 0
		\end{tikzcd} $ exakt.
		\item Ist $\begin{tikzcd}
		0  \arrow{r} & M^{’} \arrow{r}{f} & M \arrow{r}{g} & M^{''} \arrow{r} & 0
		\end{tikzcd} $ eine exakte Folge von $R$-Moduln, und setzen wir $N:= \ker g$, so induziert g einen Isomorphismus $\overline{g}: \QR{M}{N} \overset{\sim}{\longrightarrow}  M^{''}$, und $f$ beschränkt sich zu einem Isomorphismus $f: M^{’} \overset{\sim}{\longrightarrow} N$. 
		(d.h. $$\begin{tikzcd}
		0 \arrow{r}& M^{'} \arrow{r}{f}\arrow{d}[swap]{\sim}{f}& M \arrow{r}{g} \arrow[equal]{d} & M^{''} \arrow[swap]{r} & 0\\
		0 \arrow{r}& N \arrow[hook]{r}{e}& M \arrow{r}{f} & \arrow{u}[swap]{\sim}\QR{M}{N} \arrow{r} & 0
		\end{tikzcd}$$
		ist ein kommutatives Diagramm mit exakten Zeilen.)
		\item Ist $\begin{tikzcd}
		0  \arrow{r} & M_i \arrow{r} & M_i^{'} \arrow{r} & M_i^{''} \arrow{r} & 0
		\end{tikzcd}, \; i \in I$ eine Familie exakter Folgen von $R$-Moduln, dann sind auch die Folgen 
		$$ \begin{tikzcd}
		\prod_{i\in I} M_i^{'} \arrow{r}& \prod_{i\in I} M_i \arrow{r}& \prod_{i\in I} M_i^{''}
		\end{tikzcd}$$ 
		sowie
		$$\begin{tikzcd}
		\bigoplus_{i\in I} M_i^{'} \arrow{r} &\bigoplus_{i\in I} M_i \arrow{r}& \bigoplus_{i\in I} M_i^{''}
		\end{tikzcd}$$
		(mit der komponentenweisen Abbildungen) exakt.\\
	\end{itemize}
\end{anm}
\begin{sa}
	Sei $\begin{tikzcd}
	0  \arrow{r} & M^{’} \arrow{r}{f} & M \arrow{r}{g}\arrow[bend right = 50,swap, dashed]{l}{t} & M^{''}\arrow[bend right = 50,swap, dashed]{l}{s} \arrow{r} & 0
	\end{tikzcd}$ eine kurze exakte Sequenz von $R$-Moduln. Dann sind äquivalent:
	\begin{enumerate}[label= \roman*)]
		\item Es gibt ein Untermodul $N'\subseteq M$ mit $M= \ker g \oplus N'$
		\item Es gibt einen $R$-Moduolhomomorphismus $s:M''\to M$ mit $g \circ s = \id_{M''}$
		\item Es existiert ein $R$-Modulhomomorphismus $t:M \to M'$ mit $t\circ f = \id_{M'}$
	\end{enumerate}
	Ist eine dieser äquivalenten Bedingungen erfüllt, sagt man, das die kurze exakte Sequenz "'spaltet"'. In diesem Fall gilt: $M \cong M' \oplus M''$. Der Homomorphismus $s$ heißt ein "'Schnitt"' von $g$.
\end{sa}
\begin{proof}
	$i) \Ra ii)$ Sei $N'\subseteq M$ ein Untermodul mit $M= \ker g \oplus N'$. Dann ist $N'\cap \ker g = 0$. Dann ist $g\big|_{N'} : N' \to M''$ injektiv. Außerdem gilt: $M''= g(M) = g(N')$, also ist $G\big|_{N'} : N' \overset{\sim}{\longrightarrow} M''$ ein Isomorphismus. Setze $s:M'' \to N' \hookrightarrow M$. Dann ist $s$ ein $R$-Modulhomomorphismus mit $g\circ s = \id_{M''}$. Außerdem ist $M= \ker g \oplus N' = \ker g \oplus \im s = \im f \oplus \im s = f(M') \oplus s(M'') \underset{f,s\text{ inj}}{\cong} M' \oplus M''$\\
	$ii) \Ra iii)$ Sei $s:M'' \to M$ ein Modulhomomorphismus mit $g\circ s= \id_{M''}$. Sei $h:f(M') \to M'$ invers zu $f\big|^{f(M)}: M' \overset{\sim}{\longrightarrow} f(M')$. Für $m\in M$ ist 
	$$g\circ (\id_M - s\circ g))(m) = g(m) - g\circ (s\circ g)(m) = g(m) - ((\underbrace{g\circ s}_{=\id_{M''}}) \circ g)(m) = 0$$
	Also ist $(\id_{M} - s\circ g)(m) \in \ker g = \im f$. Wir setzen $t:M \xrightarrow{\id_M - s\circ g} f(M') \xrightarrow{h} M'$, welcher ein $R$-Modulhomomorphisus ist mit 
	$$t\circ f = h\circ (\id_M - s\circ g)\circ f = \underbrace{h\circ \id_M \circ f}_{=\id_{M'}} - h\circ s \circ \underbrace{g\circ f}_{=0} = \id_{M'}$$
	$iii)\Ra i)$ Setze $t:M\to M'$ ein Modulhomomorphismus mit $t\circ f = \id_{M'}$. Setze $N':= \ker t$. Für $m\in M$ ist $m=\id_M(m) = \underbrace{(\id_M- f\circ t)(m)}_{\in \ker t} + \underbrace{(f\circ t)(m)}_{\in \im f}$, also ist $M= N' + \im f$. Sei außerdem $m\in N' \cap \im f$. Dann existiert ein $m'\in M'$ mit $m= f(m')$, somit ist
	$$0 =t(m) = (t\circ f)(m') = \id_{M'}(m') = m'$$
	also auch $m=0$. Damit ist $M= N' \oplus \im f$.
\end{proof}
\begin{sa}
	Sei$\begin{tikzcd}
	0  \arrow{r} & M^{’} \arrow{r}{f} & M \arrow{r}{g} & M^{''} \arrow{r} & 0
	\end{tikzcd} $ eine exakte Sequenz von $R$-Moduln, $M''$ ein freier $R$-Modul. Dan spaltet die obige Folge.
\end{sa}
\begin{proof}
	Sei also $(v_i)_{i\in I}$ eine Basis von $M''$. Wähle für alle $i\in I$ ein $m_i\in M$ mit $g(m_i) = v_i$ (beachte: $g$ ist surjektiv). Sei $s:M'' = \bigoplus_{i\in I} Rv_i \to M$ der durch die Vorgabe $s(v_i) = m_i$ induzierte Modulhomomorphismus (existiert nach der UE von $\bigoplus$). Es ist 
	$$(g\circ s) (v_i) = g(m_i) = v_i, \quad \forall i\in I$$
	Also ist $g\circ s = \id_{M''}$
\end{proof}
\begin{fo}
	Sei $\begin{tikzcd}
	0  \arrow{r} & M^{’} \arrow{r} & M \arrow{r} & M^{''} \arrow{r} & 0
	\end{tikzcd} $ eine kurze exakte Sequenz von $R$-Moduln, $M', M''$ freie $R$-Moduln. Dann ist auch $M$ frei.
\end{fo}
\begin{proof}
	Nach Voraussetzung ist $M' \cong R^{(I)}$, $M'' \cong R^{(J)}$. Nach 1.2.3 spaltet die Folge, also ist 
	$$M  \cong M'\oplus M'' \cong R^{(I)} \oplus R^{(J)} \cong R^{(I \overset{\cdot}{\cup} J)}$$ und damit auch frei.
\end{proof}
\begin{anm}
	Ist $R$ kommutativ, und haben $M,M'$ endliche Basen, dann zeigt der Beweis: 
	$$\text{rang}(M) = \text{rang}(M') + \text{rang}(M'')$$
\end{anm}
\begin{bem}
	Sei $\begin{tikzcd}
	0  \arrow{r} & M^{’} \arrow{r}{f} & M \arrow{r}{g} & M^{''} \arrow{r} & 0
	\end{tikzcd} $ ein kurze exakte Sequenz von $R$-Moduln. Dann gilt: 
	\begin{enumerate}[label= \alph*)]
		\item Ist $M$ endlich erzeugt, dann ist $M''$ endlich erzeugt.
		\item Sind $M', \, M''$ endlich erzeugt, dann ist $M$ endlich erzeugt.
	\end{enumerate}
\end{bem}
\begin{proof}
	\begin{enumerate}[label=\alph*)]
		\item Ist $M$ endlich erzeugt, dann existiert ein $n\in \N$ und ein Epimorphismus $\phi:R^n \to M$. Dann ist $g\circ \phi:R^n \to M''$ ebenfalls ein Epimorphismus, also ist $M''$ endlich erzeugt.
		\item Sei $(x_1, \ldots, x_r)$ ein Erzeugendensystem von $M'$, $(y_1, \ldots, y_s)$ ein Erzeugendensystem von $M''$. Da $g$ surjektiv, exitieren $z_1, \ldots, z_s\in M$ mit $g(z_i) = y_i$ für $i=1, \ldots, s$. \\
		Behauptung: $f(x_1), \ldots, f(x_r), z_1, \ldots, z_s$ ist ein Erzeugendensystem von $M$, denn sei $m\in M$. Dann exsitieren $a_1, \ldots, a_s\in R$ mit $g(m) = \sum_{i=1}^s a_i y_i = \sum_{i=1}^s a_i g(z_i) = g(\sum_{i=1}^s a_i z_i)$. Damit ist $m- \sum_{i=1}^s a_i z_i \in \ker g = \im f$. Also existiert ein $v\in M'$, etwa $v= \sum_{i=1}^r b_i x_i$ mit $f(v) = m - \sum_{i=1}^s a_i z_i $. Also ist 
		$$m=f(v) + \sum_{i=1}^sa_i z_i = \sum_{i=1}^r b_i f(x_i) + \sum_{i=1}^s a_i z_i$$ 
	\end{enumerate}
\end{proof}
\begin{anm}
	Aus $M$ endlich erzeugt, folgt im Allgemeinen nicht, dass $M'$ endlich erzeugt ist.
\end{anm}
\begin{bsp}
	Sei $K$ ein Körper, $R=K[X_1, X_2, \ldots]$. Dann ist $R$ als $R$-Modul offensichtlich endlich erzeugt (von 1). Setze $I:= \{f\in R \ | \ \text{konstanter Term von } f \text{ ist } =0\}$. Dann ist $I$ ein Ideal in $R$, aber $I$ ist nicht endlich erzeugt als $R$-Modul, denn angenommen ex existieren $f_1, \ldots, f_r\in I$ mit $I = \sum_{i=1}^rRf_i$. Dann existiert ein $n\in \N$, sodass $f_1, \ldots, f_r\in K[X_1, \ldots, X_n] \subseteq R$. \\
	Problem: $X_{n+1} \notin I$, denn andernfalls wäre $X^{n+1} = a_1 f_1 + \ldots + a_rf_r$ mit $a_1, \ldots, a_r \in R$ und setze $X_1 = \ldots = X_n = 0, \ X_{n+1} = 1$, also $1=0$ Widerspruch!		
\end{bsp}
\begin{bem}
	Seien $M_1, \ldots, M_r$ $R$-Moduln. Dann sind äquivalent:
	\begin{enumerate}[label=\roman*)]
		\item $M= \bigoplus_{i=1}^r M_i$ ist endlich erzeugt.
		\item $M_1, \ldots, M_r$ sind endlich erzeugt.
	\end{enumerate}
\end{bem}
\begin{proof}
	Es genügt, die Behaptung für $r=2$ zu zeigen (Rest induktiv). Wir haben kurze exakte Folgen $$\begin{tikzcd}
	0  \arrow{r} & M_1 \arrow{r}{f} & M_1 \oplus M_2 \arrow{r}{g} & M_2 \arrow{r} & 0
	\end{tikzcd} $$
	 und
	$$\begin{tikzcd}
	0  \arrow{r} & M_2 \arrow{r}{f} & M_1 \oplus M_2 \arrow{r}{g} & M_1 \arrow{r} & 0
	\end{tikzcd} $$
	Damit folgt die Behauptung aus 2.5
\end{proof}
\begin{anm}
	Ist $M= \bigoplus_{i\in I} M_I$ mit $|I| = \infty, \ M_i \neq 0$ für alle $i\in I$, dann ist $M$ nicht endlich erzeugt, dann für $x_1, \ldots, x_s\in M$ existiert ein $J\subsetneq I$ mit $ x_1, \ldots, x_s \in \bigoplus_{j\in J} M_j$, also $\sum_{i=1}^s R_i \subseteq \bigoplus_{j\in J} M_j\subsetneq \bigoplus_{i\in I} M_i$
\end{anm}
\begin{bem}[Fünferlemma] Ist ein kommutatives Diagramm von $R$-Modulhomomorphismen mit exakten Zeilen 
	$$\begin{tikzcd}
	M_1 \arrow{r} \arrow{d}{\phi_1}& M_2 \arrow{r}\arrow{d}{\phi_2}& M_3 \arrow{r} \arrow{d}{\phi_3} & M_4 \arrow{r} \arrow{d}{\phi_4} & M_5 \arrow{d}{\phi_5}\\
	N_1 \arrow{r}& N_2 \arrow{r}& N_3 \arrow{r} & N_4 \arrow{r} & N_5
	\end{tikzcd}$$
	gegeben und $\phi_1$ surjektiv, $\phi_2, \phi_4$ Isomorphismen, $\phi_5$ injektiv. Dann ist $\phi_3$ ein Isomorphismus.
\end{bem}
\begin{proof}
	Diagrammjagd (Übungen).
\end{proof}
\begin{anm}
	Wir meist in der Situation $M_1 = N_1 = M_5 = N_5$ angewandt.
\end{anm}
\begin{bem}[Schlangenlemma]
	Sei folgendes Diagramm von $R$ Modulhomomorphismen mit exakten Zeilen gegeben:
	$$\begin{tikzcd}
& M^{'} \arrow{r}{f}\arrow{d}{\phi'}& M \arrow{r}{f} \arrow{d}{\phi} & M^{''} \arrow{d}{\phi''} \arrow{r}& 0\\
	0 \arrow{r}& N' \arrow{r}{g'}& N \arrow{r}{g} & \arrow{u}[swap]{\sim}N''
	\end{tikzcd}$$
	Dann existiert eine exakte Sequenz von $R$-Moduln 
	$$\begin{tikzcd}
	\ker \phi'  \arrow{r} & \ker \phi \arrow{r}{f} & \ker \phi'' \arrow{r}{\delta} & \coker \phi'  \arrow{r} &  \coker \phi \arrow{r} & \coker \phi''
	\end{tikzcd} $$
	wobei $\delta$ die sogenannte Übergangabbildung ist (Konstruktion siehe Beweis) und $f',f,g',g$ induziert sind. Ist $f'$ injektiv, dann ist auch $\ker \phi' \longrightarrow \ker \phi$ injektiv. Ist $g$ surjektiv, dann auch $\coker \phi \longrightarrow \coker \phi''$
	
\end{bem}
\begin{proof} 
	Betrachte
	\begin{center}
		\begin{tikzpicture}[>=triangle 60]
		\matrix[matrix of math nodes,column sep={60pt,between origins},row
		sep={60pt,between origins},nodes={asymmetrical rectangle}] (s)
		{
			&|[name=ka]| \ker \phi' &|[name=kb]| \ker \phi &|[name=kc]| \ker \phi'' \\
			%
			&|[name=A]| M' &|[name=B]| M &|[name=C]| M'' &|[name=01]| 0 \\
			%
			|[name=02]| 0 &|[name=A']| N' &|[name=B']| N &|[name=C']| N'' \\
			%
			&|[name=ca]| \coker \phi' &|[name=cb]| \coker \phi &|[name=cc]| \coker \phi'' \\
		};
		\draw[->] (ka) edge (A)
		(kb) edge (B)
		(kc) edge (C)
		(A) edge node[auto] {\(f'\)}(B)
		(B) edge node[auto] {\(f\)} (C)
		(C) edge (01)
		(A) edge node[auto] {\(\phi'\)} (A')
		(B) edge node[auto] {\(\phi\)} (B')
		(C) edge node[auto] {\(\phi''\)} (C')
		(02) edge (A')
		(A') edge node[auto] {\(g'\)} (B')
		(B') edge node[auto] {\(g\)} (C')
		(A') edge (ca)
		(B') edge (cb)
		(C') edge (cc)
		;
		\draw[->,gray] (ka) edge (kb)
		(kb) edge (kc)
		(ca) edge (cb)
		(cb) edge (cc)
		;
		\draw[->,red,rounded corners] (kc) -| node[auto,text=black,pos=.7]
		{\(\delta\)} ($(01.east)+(.5,0)$) |- ($(B)!.35!(B')$) -|
		($(02.west)+(-.5,0)$) |- (ca);
		\end{tikzpicture}
	\end{center}
	Konstruktion von $\delta:$ Sei $m'' \in \ker \phi'' \subseteq M''$. Da $f$ surjektiv, existiert ein $m\in M$ mit $m'' = f(m)$. Setze $n:= \phi(m)$. Dann ist $g(n) = g(\phi(m)) = \phi''(f(m)) = \phi''(m'') = 0$. Dann ist $n\in \ker g = \im g'$. Also existiert ein $n'\in N'$ mit $g'(n') = n$ ($n'$ ist eindeutig bestimmt wegen $g'$ injektiv.) Setze $\delta(m''):= n' + \im \phi'$\\
	Wohldefiniertheit von $\delta$: Sei $\tilde{m} \in M$ mit $m'' = f(\tilde{m})$. Dann ist $(\tilde{m}) = f(m)$, also $\tilde{m} - m \in \ker f= \im f'$. Damit existiert ein $m'\in M'$ mit $\tilde{m} - m = f'(m')$. Also ist
	$$\tilde{n}:= \phi(\tilde{m}) = \phi(m+ f'(m')) = \underbrace{\phi(m)}_{=n} + \phi(f'(m')) = g'(n') + g'(\phi'(m')) = g'(\underbrace{n' + \phi'(m')}_{:= \tilde{n}'}) $$
	Damit ist $\tilde{n}' + \im \phi' = n' + \im \phi'$, Rest ist Übungsaufgabe.
\end{proof}