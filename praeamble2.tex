\usepackage[margin=1in]{geometry}
\usepackage[colorlinks,
pdfpagelabels,
pdfstartview = FitH,
bookmarksopen = true,
bookmarksnumbered = true,
linkcolor = black,
plainpages = false,
hypertexnames = false,
citecolor = black] {hyperref}
\usepackage{amsmath,amssymb,amsthm,mathrsfs,amsfonts} 
\usepackage[ngerman]{babel}
\usepackage{tikz-cd}
\usepackage{amstext}
\usepackage{mathtools}
\usepackage{color}
\usepackage[utf8]{inputenc}
\usepackage{pgfplots}
\usepackage{tikz-cd}
\usepackage[autooneside=false,headsepline,markcase=noupper]{scrlayer-scrpage}
\usepackage{blindtext}
\usepackage{enumitem}
\usepackage{verbatim}
\usepackage{MnSymbol}



%------------Seitendesign--------

\setlength{\parindent}{0em}
\setlength{\textheight}{21cm}
\setlength{\textwidth}{15cm}
\setlength{\oddsidemargin}{1cm}
\setlength{\evensidemargin}{0.0cm}
\setlength{\topmargin}{0cm}
\setlength{\footskip}{1cm}
\setlength{\headheight}{1.1\baselineskip}


%----------Kopfzeile-------------
\automark[subsection]{section}
\pagestyle{scrheadings}
\clearscrheadfoot
\ofoot[\pagemark]{\pagemark}
\ohead{  \ifstr{\rightbotmark}{\leftmark}{}{\rightbotmark}}
\ihead{\leftmark}


\makeatletter
% Damit die letzte (\botmark) statt der ersten (\firstmark) Marke auf
% einer Seite für die "rechte Marke" genommen wird:
\providecommand*{\rightbotmark}{\expandafter\@rightmark\botmark\@empty\@empty}
\makeatother

